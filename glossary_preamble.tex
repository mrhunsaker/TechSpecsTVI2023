% Enhanced Professional Glossary definitions for TechSpecsTVI project
% Include this file in your main document preamble after \usepackage{glossaries}
% Use \makeglossaries after including this file

\newglossaryentry{3dprinter}{
	name={3D printer},
	description={A device that creates three-dimensional objects from digital files by depositing material layer by layer. In assistive technology, 3D printers create tactile models, adaptive devices, and custom prosthetics. Examples include printing raised maps for geography lessons or custom grips for tools},
	seealso={3dprinting,tactilegraphics,assistivetechnology}
}

\newglossaryentry{3dprinting}{
	name={3D printing},
	description={The additive manufacturing process of creating three-dimensional objects from digital files by depositing material layer by layer. Widely used in assistive technology for creating tactile learning materials, adaptive equipment, and personalized devices that meet individual accessibility needs},
	seealso={3dprinter,tactilegraphics,assistivetechnology}
}

\newglossaryentry{504plan}{
	name={504 plan},
	description={A formal document outlining accommodations and services for students with disabilities under Section 504 of the Rehabilitation Act of 1973. Unlike IEPs, 504 plans do not require specialized instruction but ensure equal access to educational programs through reasonable accommodations},
	seealso={IEP,ada,reasonableaccommodation,FAPE}
}

\newglossaryentry{AAC}{
	name={AAC},
	description={Augmentative and Alternative Communication - methods used to supplement or replace speech and writing for individuals with communication impairments. Includes low-tech options like communication boards and high-tech solutions like speech-generating devices},
	seealso={SGD,communicationboard,communication,ASL}
}

\newglossaryentry{accessibility}{
	name={Accessibility},
	description={The design of products, devices, services, or environments to be usable by people with disabilities. This includes physical accessibility (ramps, wide doorways), digital accessibility (screen reader compatibility, keyboard navigation), and programmatic accessibility (policy and procedural accommodations)},
	seealso={universaldesign,digitalaccessibility,webaccessibility,assistivetechnology}
}

\newglossaryentry{accessibilityaudit}{
	name={Accessibility Audit},
	description={A systematic evaluation of a product, service, or environment to identify barriers for people with disabilities. Includes manual testing, automated scanning tools, user testing with people with disabilities, and compliance verification against standards like WCAG or Section 508},
	seealso={WCAG,section508,usabilityTesting,VPAT}
}

\newglossaryentry{accessiblematerials}{
	name={Accessible Materials},
	description={Educational and informational content designed or adapted to be usable by people with disabilities. Examples include braille textbooks, large print materials, audio recordings, tactile graphics, and digitally formatted texts compatible with assistive technology},
	seealso={braille,largeprint,audiobook,tactilegraphics,instructionalmaterials,nimas}
}

\newglossaryentry{accessiblemath}{
	name={Accessible Math},
	description={Mathematical content presented in formats that can be accessed by students with disabilities, particularly those with visual impairments. Includes braille math notation, MathML for screen readers, tactile graphics, and audio descriptions of equations and geometric figures},
	seealso={Nemeth,MathML,mathOCR,tactilegraphics,formulas}
}

\newglossaryentry{adaptivestrategies}{
	name={Adaptive Strategies},
	description={Systematic approaches and techniques developed by individuals with disabilities to accomplish tasks using alternative methods. May include compensatory behaviors, environmental modifications, or specialized tools to achieve functional independence},
	seealso={assistivetechnology,independence,independentlivingskills}
}

\newglossaryentry{ada}{
	name={ADA},
	description={Americans with Disabilities Act - US civil rights law enacted in 1990 that prohibits discrimination based on disability and requires reasonable accommodations in employment, public accommodations, transportation, and telecommunications},
	seealso={504plan,reasonableaccommodation,section508}
}

\newglossaryentry{ADHD}{
	name={ADHD},
	description={Attention-Deficit/Hyperactivity Disorder - a neurodevelopmental condition affecting attention, hyperactivity, and impulse control. Assistive technology for ADHD includes focus apps, time management tools, noise-canceling headphones, and organizational software},
	seealso={learningdisability,cognitiveaccessibility,apps}
}

\newglossaryentry{AI}{
	name={AI},
	description={},
	see={artificialintelligence}
}

\newglossaryentry{alttext}{
	name={Alt Text},
	description={Alternative text descriptions for images that provide meaningful information about visual content to screen reader users. Alt text should be concise yet descriptive, focusing on the purpose and content of the image rather than literal visual details},
	seealso={screenreader,imagesandmedia,webaccessibility,WCAG}
}

\newglossaryentry{api}{
	name={API},
	description={Application Programming Interface - a set of protocols and tools that allow different software applications to communicate with each other. In accessibility, APIs enable assistive technologies to interact with operating systems and applications to access content and functionality},
	seealso={assistivetechnology,operatingsystem,software}
}

\newglossaryentry{apps}{
	name={Apps},
	description={Software applications designed for mobile devices or computers. Accessibility-focused apps include screen readers, magnification software, communication aids, navigation tools, and specialized educational applications for students with disabilities},
	seealso={software,smartphone,tablet,assistivetechnology}
}

\newglossaryentry{ARIA}{
	name={ARIA},
	description={Accessible Rich Internet Applications - a set of HTML attributes that provide semantic information to assistive technologies for complex web applications. ARIA labels, roles, and properties help make dynamic content and UI controls accessible to screen readers},
	seealso={html,webaccessibility,screenreader,WCAG}
}

\newglossaryentry{artificialintelligence}{
	name={Artificial Intelligence},
	description={The simulation of human intelligence processes by machines, including learning, reasoning, and self-correction. In assistive technology, AI enhances features like object recognition, scene description, predictive text input, and personalized learning adaptations},
	seealso={machinelearning,ML,objectrecognition,NLP,speechrecognition}
}

\newglossaryentry{ASL}{
	name={ASL},
	description={American Sign Language - a visual language using hand shapes, facial expressions, and body movements for communication within the Deaf community. Technology support includes video relay services, sign language recognition software, and educational apps},
	seealso={videorelayservice,communication,AAC}
}

\newglossaryentry{assistivetechnology}{
	name={Assistive Technology},
	description={Any item, piece of equipment, software program, or product system used to increase, maintain, or improve functional capabilities of individuals with disabilities. Defined by the Assistive Technology Act as encompassing both devices and services},
	seealso={AT,ATassessment,ATservices,accessibility,universaldesign}
}

\newglossaryentry{AT}{
	name={AT},
	description={},
	see={assistivetechnology}
}

\newglossaryentry{ATassessment}{
	name={Assitive Technology (AT) Assessment},
	description={A comprehensive evaluation process to determine appropriate assistive technology solutions for an individual. Involves analyzing user needs, environmental factors, tasks, and available tools using frameworks like SETT to make evidence-based recommendations},
	seealso={settframework,assistivetechnology,ATservices}
}

\newglossaryentry{ATservices}{
	name={Assitive Technology (AT) Services},
	description={Support activities that directly assist individuals with disabilities in the selection, acquisition, or use of assistive technology devices. Includes evaluation, customization, training, technical support, maintenance, and coordination with other services},
	seealso={assistivetechnology,ATassessment,troubleshooting}
}

\newglossaryentry{audio}{
	name={Audio},
	description={Sound-based information delivery methods used in assistive technology. Includes text-to-speech output, audio descriptions of visual content, auditory navigation cues, and sound-based feedback systems that provide non-visual access to information},
	seealso={texttospeech,audiodescription,auditoryfeedback,audiobook}
}

\newglossaryentry{audiobook}{
	name={audiobook},
	description={Digital audio recordings of books that provide accessible reading alternatives for people with visual impairments, learning disabilities, or physical disabilities that make holding books difficult. Often include navigation features and variable playback speeds},
	seealso={daisy,bookshare,printdisability,accessiblematerials}
}

\newglossaryentry{audiodescription}{
	name={Audio Description},
	description={Narrated descriptions of visual elements in videos, movies, or live performances inserted during natural pauses in dialogue. Provides access to visual information for viewers who are blind or have low vision},
	seealso={captioning,imagesandmedia,visualimpairment}
}

\newglossaryentry{auditoryfeedback}{
	name={Auditory feedback},
	description={Sound-based responses from technology that provide information about system status, user actions, and content. Examples include screen reader speech, system beeps, earcons (audio icons), and spoken confirmations of user input},
	seealso={earcons,screenreader,texttospeech,sonification}
}

\newglossaryentry{auditoryprocessingdisorder}{
	name={auditory processing disorder},
	description={A condition where the brain has difficulty processing auditory information, despite normal hearing ability. Assistive technology includes FM systems, noise reduction headphones, visual supports, and specialized listening devices},
	seealso={FM,hearingaid,learningdisability}
}

\newglossaryentry{autism}{
	name={autism},
	description={A neurodevelopmental condition characterized by differences in social communication, interaction patterns, and restricted interests. Assistive technology includes communication apps, sensory regulation tools, visual schedules, and social skills software},
	seealso={AAC,communication,apps,cognitiveaccessibility}
}

\newglossaryentry{bana}{
	name={BANA},
	description={Braille Authority of North America - the organization responsible for developing and maintaining braille codes and standards in the United States and Canada. BANA publishes guidelines for literary braille, Nemeth Code mathematics, and music braille},
	seealso={braille,Nemeth,musicbraille}
}

\newglossaryentry{biometrics}{
	name={biometrics},
	description={Biological measurements and characteristics used for identification and access control. In assistive technology, biometric systems like fingerprint readers, voice recognition, and eye tracking provide alternative authentication methods for users with mobility limitations},
	seealso={speechrecognition,eyetracking,voicecontrol}
}
\newglossaryentry{blind}{
	name={blind},
	description={A term used to describe individuals with severe visual impairments who have little or no functional vision. Assistive technology for blind users includes braille displays, screen readers, audio description, and tactile graphics to access information and communicate effectively},
	seealso={visualimpairment,braille,screenreader,tactilegraphics}
}

\newglossaryentry{bluetooth}{
	name={Bluetooth},
	description={Short-range wireless communication technology that enables devices to connect without cables. In assistive technology, Bluetooth connects braille displays to computers, hearing aids to smartphones, and enables wireless control of environmental systems},
	seealso={brailledisplay,hearingaid,environmentalcontrol}
}

\newglossaryentry{BMML}{
	name={BMML},
	description={Braille Music Markup Language - a digital format for encoding music notation that can be converted to braille music notation or other accessible formats, enabling musicians who are blind to access and share musical scores digitally},
	seealso={musicbraille,MusicXML,music}
}

\newglossaryentry{bookshare}{
	name={Bookshare},
	description={A digital library service operated by Benetech that provides accessible books for people with print disabilities. Members can download books in various formats including DAISY audio, braille, and large print, with over 800,000 titles available},
	seealso={daisy,audiobook,printdisability,accessiblematerials}
}

\newglossaryentry{braille}{
	name={braille},
	description={A tactile writing system using patterns of raised dots arranged in cells of up to six dots each, invented by Louis Braille in 1824. Each cell can represent letters, numbers, punctuation, or special symbols, enabling people who are blind to read and write through touch},
	seealso={brailledisplay,brailleembosser,brailleliteracy,tactile,Nemeth}
}

\newglossaryentry{Braille Ready Format}{
	name={braille ready format},
	description={A file format containing braille text using ASCII characters that represent braille dots, ready for embossing or display on braille devices. Files use .brf or .brl extensions and can be sent directly to braille embossers without additional formatting},
	seealso={brailleembosser,brailledisplay,braille}
}

\newglossaryentry{brailledisplay}{
	name={braille display},
	description={An electronic device with a row of braille cells that use pins to form braille characters, allowing users to read screen content tactilely. Displays typically have 12-80 cells and include navigation buttons for cursor routing and scrolling through text},
	seealso={refreshablebrailledisplay,RTD,braille,cursorrouting,screenreader}
}

\newglossaryentry{brailleeducation}{
	name={braille education},
	description={Instruction in reading and writing braille, including literary braille, mathematics (Nemeth Code), and specialized codes for music and science. Essential for developing literacy skills in students who are blind or have severe visual impairments},
	seealso={brailleliteracy,Nemeth,musicbraille,TVI}
}

\newglossaryentry{brailleembosser}{
	name={braille embosser},
	description={A specialized printer that creates braille text on paper by pressing dots into the surface from the underside. Embossers can produce single or double-sided braille and often support multiple paper sizes and braille formats},
	seealso={braille,Braille Ready Format}
}

\newglossaryentry{brailleliteracy}{
	name={braille literacy},
	description={The ability to read and write braille fluently. Research shows braille literacy is strongly correlated with academic achievement and employment success for people who are blind, making it essential for educational and career opportunities},
	seealso={braille,brailleeducation,digitalliteracy}
}

\newglossaryentry{BYOD}{
	name={BYOD},
	description={Bring Your Own Device - policies allowing students and employees to use personal devices for educational or work purposes. In special education, BYOD can provide access to familiar assistive technology while requiring consideration of privacy, security, and compatibility issues},
	seealso={assistivetechnology,tablet,smartphone}
}

\newglossaryentry{captioning}{
	name={captioning},
	description={Text display of spoken dialogue and sound effects in video content, essential for accessibility for deaf and hard of hearing viewers. Includes closed captions (can be turned on/off) and open captions (permanently displayed)},
	seealso={audiodescription,CART,imagesandmedia}
}

\newglossaryentry{CART}{
	name={CART},
	description={Communication Access Realtime Translation - live transcription services that provide real-time text display of spoken communication using specialized software and trained operators. Used in classrooms, meetings, and events for deaf and hard of hearing participants},
	seealso={captioning,speechrecognition}
}

\newglossaryentry{CCTV}{
	name={CCTV},
	description={Closed-Circuit Television, in assistive technology referring to video magnification systems that use cameras and displays to enlarge text and objects for users with low vision. Modern CCTV systems offer variable magnification, contrast enhancement, and color options},
	seealso={videomagnifier,magnification,lowvision}
}

\newglossaryentry{charts}{
	name={charts},
	description={Visual representations of data and information that require alternative accessible formats for users with visual impairments. Accessible versions include tactile graphics, data tables with clear structure, sonification, and detailed text descriptions},
	seealso={datavisualization,tactilegraphics,sonification,alttext}
}

\newglossaryentry{cloudcomputing}{
	name={cloud computing},
	description={Delivery of computing services over the internet, including storage, processing, and software applications. Cloud-based assistive technology provides device-independent access to accessibility tools and enables synchronization across multiple devices},
	seealso={software,apps,assistivetechnology}
}

\newglossaryentry{cognitiveaccessibility}{
	name={cognitive accessibility},
	description={Design considerations for people with cognitive disabilities, including simplified interfaces, clear navigation, consistent layouts, plain language, and support for memory, attention, and processing difficulties},
	seealso={intellectualdisability,cognitiveload,UDL,usabilityTesting}
}

\newglossaryentry{cognitiveload}{
	name={cognitive load},
	description={The amount of mental effort and working memory required to complete a task. In assistive technology design, reducing cognitive load through intuitive interfaces and efficient navigation is crucial for user success and reduced fatigue},
	seealso={cognitiveaccessibility,userexperience,usabilityTesting}
}

\newglossaryentry{communication}{
	name={communication},
	description={The exchange of information through various means. Assistive communication technologies include speech-generating devices, communication boards, sign language interpreters, and apps that facilitate expression for people with speech or language disabilities},
	seealso={AAC,SGD,communicationboard,ASL}
}

\newglossaryentry{communicationboard}{
	name={communication board},
	description={A low-tech AAC device displaying symbols, pictures, or words that users can point to for communication. Boards can be paper-based or electronic and are often customized to individual vocabulary needs and communication contexts},
	seealso={AAC,SGD,communication}
}

\newglossaryentry{cpu}{
	name={CPU},
	description={Central Processing Unit - the main processor of a computer that executes instructions. CPU performance is critical for smooth operation of assistive technology, especially resource-intensive applications like screen readers with complex speech synthesis},
	seealso={processor,ram,hardware,screenreader}
}

\newglossaryentry{CSE}{
	name={CSE},
	description={Committee on Special Education - the multidisciplinary team responsible for determining eligibility for special education services, developing IEPs, and making placement decisions for students with disabilities in New York State},
	seealso={IEP,MDT,FAPE}
}

\newglossaryentry{cursorrouting}{
	name={cursor routing},
	description={A feature on braille displays that allows users to move the computer cursor to specific positions by pressing routing buttons above each braille cell. Essential for efficient editing and navigation of text documents},
	seealso={brailledisplay,navigation,keyboardnavigation}
}

\newglossaryentry{cvi}{
	name={CVI},
	description={Cortical Visual Impairment - a visual impairment caused by damage to the visual processing areas of the brain rather than the eyes themselves. The leading cause of visual impairment in children, requiring specialized educational approaches and environmental modifications},
	seealso={visualimpairment,TVI}
}

\newglossaryentry{dailylivingaids}{
	name={daily living aids},
	description={Assistive devices that help people with disabilities perform everyday tasks independently. Examples include talking scales, large-button phones, medication organizers with audio reminders, tactile measuring tools, and adaptive kitchen equipment},
	seealso={assistivetechnology,independence,independentlivingskills}
}

\newglossaryentry{daisy}{
	name={DAISY},
	description={Digital Accessible Information System - an international standard for creating navigable, accessible digital books. DAISY books include features like bookmarks, heading navigation, and synchronized text and audio, developed by the DAISY Consortium},
	seealso={audiobook,ebooks,bookshare,accessiblematerials}
}

\newglossaryentry{datavisualization}{
	name={data visualization},
	description={The presentation of data in visual formats like charts, graphs, and infographics. Accessible data visualization requires alternative representations including data tables, tactile graphics, sonification, and comprehensive text descriptions},
	seealso={charts,tactilegraphics,sonification,alttext}
}

\newglossaryentry{deafblind}{
	name={deaf-blind},
	description={A condition involving both hearing and vision loss that significantly impacts communication, access to information, and mobility. Specialized assistive technology includes tactile communication devices, vibrating alerts, and combined audio-tactile interfaces},
	seealso={tactile,haptic,communication}
}

\newglossaryentry{digitalaccessibility}{
	name={digital accessibility},
	description={The design of digital technology (websites, apps, documents) to be usable by people with disabilities. Includes features like keyboard navigation, screen reader compatibility, sufficient color contrast, and clear content structure},
	seealso={webaccessibility,accessibility,screenreader,keyboardnavigation}
}

\newglossaryentry{digitaldivide}{
	name={digital divide},
	description={The gap between individuals who have access to modern digital technology and those who do not. For people with disabilities, this includes barriers to accessible technology, training, and affordable assistive technology solutions},
	seealso={digitalliteracy,equitableaccess,assistivetechnology}
}

\newglossaryentry{digitalliteracy}{
	name={digital literacy},
	description={The ability to effectively use digital technologies for communication, learning, and work. For users of assistive technology, digital literacy includes proficiency with accessibility features and adaptive software alongside general computer skills},
	seealso={brailleliteracy,assistivetechnology,technology}
}
\newglossaryentry{disability}{
	name={disability},
	description={A physical or mental condition that limits a person's movements, senses, or activities. Disabilities can be permanent or temporary and vary widely in type and severity, requiring different accommodations and assistive technologies to support functional independence},
	seealso={accessibility,reasonableaccommodation,universaldesign}
}
\newglossaryentry{dolphinsupernova}{
	name={Dolphin SuperNova},
	description={A screen magnification and reading software designed for users with low vision. It provides features like magnification, color inversion, text-to-speech, and braille support, enabling users to access digital content more easily},
	seealso={screenmagnification,lowvision,assistivetechnology}
}

\newglossaryentry{documentstructure}{
	name={document structure},
	description={The organization and markup of documents using headings, lists, tables, and other elements to create logical navigation paths for assistive technology. Proper structure enables screen reader users to navigate efficiently using headings and landmarks},
	seealso={html,navigation,screenreader,ARIA}
}

\newglossaryentry{DRM}{
	name={DRM},
	description={Digital Rights Management - technologies that control access to copyrighted digital content. DRM can create barriers to accessibility by preventing content from being converted to accessible formats or used with assistive technology},
	seealso={ebooks,accessiblematerials,fileformats}
}

\newglossaryentry{dyscalculia}{
	name={dyscalculia},
	description={A specific learning disability affecting mathematical reasoning and number sense. Assistive technology includes talking calculators, math apps with visual supports, graphing software with audio feedback, and number line tools},
	seealso={learningdisability,accessiblemath,graphingcalculator}
}

\newglossaryentry{dyslexia}{
	name={dyslexia},
	description={A specific learning disability affecting reading and language processing skills. Assistive technology includes text-to-speech software, word prediction programs, reading comprehension apps, and specialized fonts designed for improved readability},
	seealso={learningdisability,texttospeech,wordprediction,fonts}
}

\newglossaryentry{earcons}{
	name={earcons},
	description={Brief audio signals used in user interfaces to convey information non-visually. Earcons use abstract sounds or musical phrases to represent system states, notifications, or interface elements, supporting auditory navigation for screen reader users},
	seealso={auditoryfeedback,sonification,screenreader}
}

\newglossaryentry{ebooks}{
	name={e-books},
	description={Electronic books in digital formats that can be read on computers, tablets, or dedicated e-readers. Accessible e-books support features like adjustable text size, text-to-speech, high contrast, and screen reader compatibility},
	seealso={EPUB,daisy,audiobook,bookshare,tablet}
}

\newglossaryentry{educationalequity}{
	name={educational equity},
	description={The principle that all students should have access to the resources, opportunities, and educational experiences necessary for success, regardless of disability. Requires removing barriers and providing appropriate accommodations and supports},
	seealso={equitableaccess,FAPE,UDL,inclusivedesign}
}

\newglossaryentry{emergencyresponse}{
	name={emergency response},
	description={Systems and procedures for handling emergency situations that must be accessible to people with disabilities. Includes visual and tactile fire alarms, accessible evacuation routes, and communication systems for emergency notifications},
	seealso={safety,haptic,auditoryfeedback}
}

\newglossaryentry{environmentalcontrol}{
	name={environmental control},
	description={Technology systems that allow individuals with mobility limitations to control their physical environment, including lights, temperature, door locks, and appliances through switches, voice commands, or mobile apps},
	seealso={smarthome,switch,voicecontrol,IoT}
}

\newglossaryentry{EPUB}{
	name={EPUB},
	description={A standardized e-book format that supports reflowable text, multimedia content, and accessibility features. EPUB3 includes enhanced support for screen readers, navigation, and MathML, making it a preferred format for accessible digital books},
	seealso={ebooks,MathML,screenreader,fileformats}
}

\newglossaryentry{equitableaccess}{
	name={equitable access},
	description={Providing fair and impartial access to opportunities and resources, ensuring that people with disabilities can participate fully in education, employment, and community life through appropriate accommodations and universal design principles},
	seealso={educationalequity,universaldesign,reasonableaccommodation}
}

\newglossaryentry{ergonomics}{
	name={ergonomics},
	description={The science of designing equipment and environments to fit human capabilities and limitations. In assistive technology, ergonomic considerations prevent injury and reduce fatigue through proper positioning, interface design, and adaptive equipment},
	seealso={assistivetechnology,userexperience,safety}
}

\newglossaryentry{eyetracking}{
	name={eye tracking},
	description={Technology that detects and follows eye movements to enable computer control through gaze. Used as an alternative input method for individuals with severe motor disabilities who cannot use traditional keyboards or pointing devices},
	seealso={switch,switchaccess,biometrics}
}

\newglossaryentry{FAPE}{
	name={FAPE},
	description={Free Appropriate Public Education - a right guaranteed under IDEA ensuring that students with disabilities receive special education and related services at public expense, meeting their unique needs and preparing them for further education and employment},
	seealso={IEP,504plan,LRE,educationalequity}
}

\newglossaryentry{fileformats}{
	name={file formats},
	description={Different methods of encoding and storing digital information, with varying levels of accessibility. Accessible formats include structured HTML, properly tagged PDFs, EPUB3, and DAISY, while inaccessible formats include image-only PDFs and unstructured documents},
	seealso={pdf,EPUB,html,daisy,XML}
}

\newglossaryentry{FM}{
	name={FM system},
	description={Frequency Modulation assistive listening devices that transmit sound wirelessly from a microphone worn by a speaker directly to a receiver worn by a listener, reducing background noise and improving speech clarity for hearing aid users},
	seealso={hearingaid,auditoryprocessingdisorder,bluetooth}
}

\newglossaryentry{fonts}{
	name={fonts},
	description={Typeface designs that significantly affect readability for users with visual impairments. Accessible fonts feature clear character distinctions, adequate spacing, and good contrast. Examples of accessible fonts include Arial, Helvetica, and specially designed fonts like OpenDyslexic},
	seealso={typography,textformatting,dyslexia,largeprint}
}

\newglossaryentry{formulas}{
	name={formulas},
	description={Mathematical and scientific expressions that require specialized accessible formats. Accessible formula presentation includes MathML for screen readers, braille math notation (Nemeth Code), large print, and detailed verbal descriptions of mathematical relationships},
	seealso={accessiblemath,MathML,Nemeth,stem}
}
\newglossaryentry{freedomscientificfusion}{
	name={Freedom Scientific Fusion},
	description={A combined screen reader and magnification software designed for users with low vision and blindness. Fusion integrates JAWS screen reader capabilities with ZoomText magnification features, providing a comprehensive accessibility solution for Windows users},
	seealso={screenreader,screenmagnification,jaws,zoomtext}
}

\newglossaryentry{functionalindependence}{
	name={functional independence},
	description={The ability to perform daily living activities without assistance. Assistive technology promotes functional independence by providing tools and adaptations that enable users to complete tasks like communication, mobility, and self-care independently},
	seealso={independence,assistivetechnology,independentlivingskills}
}

\newglossaryentry{screenmagnification}{
	name={screen magnification},
	description={Assistive technology that enlarges text and images on computer screens to make them more readable for users with low vision. Screen magnifiers can adjust magnification levels, contrast, and color settings to enhance visibility without distorting content},
	seealso={lowvision,zoomtext,dolphinsupernova,assistivetechnology}
}

\newglossaryentry{gps}{
	name={GPS},
	description={Global Positioning System - satellite-based navigation technology adapted for users with visual impairments through audio announcements, haptic feedback, and integration with mobility apps that provide detailed navigation instructions and environmental information},
	seealso={navigation,mobility,indoornavigation,apps}
}

\newglossaryentry{graphingcalculator}{
	name={graphing calculator},
	description={Advanced calculators that display mathematical functions visually. Accessible versions include talking graphing calculators with speech output, large display models, and software applications with screen reader support for mathematical computation and graphing},
	seealso={accessiblemath,dyscalculia,interactivegraphingenvironments}
}

\newglossaryentry{guidedog}{
	name={guide dog},
	description={A specially trained dog that assists people who are blind or have low vision with navigation and obstacle avoidance. Guide dogs provide increased mobility, independence, and safety while requiring ongoing care and partnership between handler and dog},
	seealso={mobility,whitecane,navigation,independence}
}

\newglossaryentry{haptic}{
	name={haptic},
	description={Related to the sense of touch and tactile feedback. Haptic technology provides information through touch sensations, including vibration patterns, force feedback, and tactile displays that convey spatial and textural information non-visually},
	seealso={tactile,hapticfeedback,brailledisplay,deafblind}
}

\newglossaryentry{hapticfeedback}{
	name={haptic feedback},
	description={Touch-based feedback through vibration, force, or texture that provides information about user interactions and system states. Examples include smartphone vibrations for notifications and tactile feedback in touchscreen interfaces},
	seealso={haptic,tactile,auditoryfeedback,smartphone}
}

\newglossaryentry{hardware}{
	name={hardware},
	description={Physical components of computer and assistive technology systems, including processors, memory, input devices, and specialized equipment like braille displays, switch interfaces, and eye-tracking systems that enable access for users with disabilities},
	seealso={software,cpu,ram,assistivetechnology}
}

\newglossaryentry{hearingaid}{
	name={hearing aid},
	description={Electronic devices worn in or behind the ear that amplify and process sound for people with hearing loss. Modern hearing aids include features like Bluetooth connectivity, directional microphones, and smartphone app controls},
	seealso={FM,bluetooth,auditoryprocessingdisorder}
}

\newglossaryentry{html}{
	name={HTML},
	description={HyperText Markup Language - the standard markup language for creating web pages. Properly structured HTML with semantic elements, headings, and ARIA labels is essential for creating accessible web content that works with screen readers and other assistive technologies},
	seealso={webaccessibility,ARIA,documentstructure,screenreader}
}

\newglossaryentry{IEP}{
	name={IEP},
	description={Individualized Education Program - a legal document that outlines special education services, goals, and accommodations for students with disabilities ages 3-21. Developed by a multidisciplinary team including parents, teachers, and related service providers},
	seealso={504plan,FAPE,LRE,CSE,MDT,transition}
}

\newglossaryentry{imagesandmedia}{
	name={images and media},
	description={Visual and multimedia content including photos, graphics, videos, and audio that require alternative accessible formats. Accessibility features include alt text for images, captions for videos, audio descriptions, and transcripts for audio content},
	seealso={alttext,captioning,audiodescription,presentations}
}

\newglossaryentry{inclusivedesign}{
	name={inclusive design},
	description={A design methodology that considers the needs of people with disabilities from the beginning of the design process, creating products and environments that are usable by the widest range of people without requiring specialized adaptations},
	seealso={universaldesign,UDL,accessibility,userexperience}
}

\newglossaryentry{independence}{
	name={independence},
	description={The ability to perform tasks and make decisions without assistance from others. Assistive technology promotes independence by providing alternative methods for accessing information, communicating, and controlling the environment},
	seealso={independentlivingskills,assistivetechnology,dailylivingaids,selfdetermination}
}

\newglossaryentry{independentlivingskills}{
	name={independent living skills},
	description={Abilities needed to live independently in the community, including cooking, cleaning, money management, transportation, and communication. Often supported by assistive technology such as talking appliances, smartphone apps, and adaptive tools},
	seealso={independence,dailylivingaids,assistivetechnology,transition}
}

\newglossaryentry{indoornavigation}{
	name={indoor navigation},
	description={Technology systems that provide navigation assistance inside buildings where GPS signals are unavailable. Uses techniques like Bluetooth beacons, Wi-Fi positioning, and smartphone apps to guide users with detailed audio directions},
	seealso={navigation,gps,mobility,bluetooth}
}

\newglossaryentry{instructionalmaterials}{
	name={instructional materials},
	description={Educational content including textbooks, worksheets, presentations, and multimedia used in teaching. Accessible instructional materials are designed or adapted to be usable by students with disabilities through formats like braille, large print, audio, and digital text},
	seealso={accessiblematerials,nimas,nimac,UDL}
}

\newglossaryentry{intellectualdisability}{
	name={intellectual disability},
	description={A disability characterized by significant limitations in intellectual functioning and adaptive behavior. Assistive technology includes simplified interfaces, visual supports, task prompting systems, and communication aids designed for cognitive accessibility},
	seealso={cognitiveaccessibility,AAC,apps}
}

\newglossaryentry{interactivegraphingenvironments}{
	name={interactive graphing environments},
	description={Software applications that allow users to create, manipulate, and analyze mathematical graphs and data visualizations. Accessible versions provide audio feedback, keyboard navigation, and tactile output for users who cannot see visual displays},
	seealso={accessiblemath,graphingcalculator,datavisualization}
}

\newglossaryentry{IoT}{
	name={IoT},
	description={Internet of Things - interconnected devices that communicate over networks to provide automated services. In assistive technology, IoT enables smart home systems, environmental controls, health monitoring devices, and location-based services for independence},
	seealso={smarthome,environmentalcontrol,cloudcomputing}
}

\newglossaryentry{jaws}{
	name={JAWS},
	description={Job Access With Speech - a commercial screen reader software developed by Freedom Scientific. One of the most widely used screen readers worldwide, providing speech and braille output for Windows computers with extensive customization options},
	seealso={screenreader,nvda,narrator,brailledisplay}
}

\newglossaryentry{JSON}{
	name={JSON},
	description={JavaScript Object Notation - a lightweight data interchange format that is easy for humans to read and write, and easy for machines to parse and generate. Used in web APIs and data storage, JSON can be structured to support accessibility features in applications},
	seealso={XML,api,webaccessibility}
}

\newglossaryentry{keyboardnavigation}{
	name={keyboard navigation},
	description={The ability to operate software and websites using only keyboard input, essential for users who cannot use pointing devices. Requires proper focus management, logical tab order, and keyboard shortcuts for all interactive elements},
	seealso={webaccessibility,screenreader,switch,switchaccess}
}

\newglossaryentry{laptop}{
	name={laptop},
	description={A portable computer that serves as the primary computing platform for many students using assistive technology. Modern laptops include built-in accessibility features and can run specialized software like screen readers, magnification programs, and communication aids},
	seealso={assistivetechnology,screenreader,magnification,tablet}
}

\newglossaryentry{largeprint}{
	name={large print},
	description={Text and materials presented in enlarged formats to improve readability for people with low vision. Standard large print uses 16-18 point fonts, while some users require even larger sizes depending on their visual needs},
	seealso={lowvision,fonts,magnification,accessiblematerials}
}

\newglossaryentry{latency}{
	name={latency},
	description={The delay between user input and system response. Low latency is critical for screen reader users who rely on immediate audio feedback for efficient navigation and interaction with computer interfaces},
	seealso={screenreader,auditoryfeedback,userexperience}
}

\newglossaryentry{LateX}{
	name={LaTeX},
	description={A document preparation system used for high-quality typesetting, particularly for mathematical, scientific, and technical documents. LaTeX can generate accessible PDF outputs and convert to braille-ready formats when properly structured},
	seealso={pdf,accessiblemath,stem,documentstructure}
}

\newglossaryentry{learningdisability}{
	name={learning disability},
	description={A neurological condition that affects how individuals process, store, or respond to information. Assistive technology includes text-to-speech software, graphic organizers, word prediction programs, and specialized educational apps},
	seealso={dyslexia,dyscalculia,ADHD,texttospeech,wordprediction}
}

\newglossaryentry{LLM}{
	name={LLM},
	description={Large Language Model - AI systems trained on vast amounts of text data to understand and generate human language. In accessibility, LLMs power features like automatic image description, content summarization, and intelligent document conversion},
	seealso={artificialintelligence,NLP,objectrecognition,alttext}
}

\newglossaryentry{LRE}{
	name={LRE},
	description={Least Restrictive Environment - IDEA principle requiring students with disabilities to be educated with non-disabled peers to the maximum extent appropriate, with special education services provided in general education settings when possible},
	seealso={IEP,FAPE,inclusivedesign}
}

\newglossaryentry{lowvision}{
	name={low vision},
	description={A visual impairment that cannot be fully corrected with glasses, contacts, surgery, or medication, but still allows for some functional vision. Low vision aids include magnification devices, high contrast materials, and lighting modifications},
	seealso={visualimpairment,magnification,CCTV,videomagnifier,largeprint}
}

\newglossaryentry{machinelearning}{
	name={machine learning},
	description={A subset of artificial intelligence that enables computers to learn and improve from experience without explicit programming. In assistive technology, machine learning powers OCR, speech recognition, object identification, and personalized user interfaces},
	seealso={ML,artificialintelligence,OCR,speechrecognition,objectrecognition}
}

\newglossaryentry{magnification}{
	name={magnification},
	description={Technology that enlarges visual content for users with low vision. Includes software magnification (screen magnifiers), hardware devices (video magnifiers), and optical magnification tools with features like high contrast and color enhancement},
	seealso={lowvision,CCTV,videomagnifier,zoom,largeprint}
}

\newglossaryentry{markdown}{
	name={Markdown},
	description={A lightweight markup language using plain text formatting that can be easily converted to HTML, PDF, and other formats. Markdown's simple syntax makes it accessible for creating structured documents that convert well to braille and other accessible formats},
	seealso={html,documentstructure,fileformats}
}

\newglossaryentry{MathML}{
	name={MathML},
	description={Mathematical Markup Language - a markup language for describing mathematical notation that enables screen readers to speak mathematical expressions meaningfully. Supported by modern browsers and assistive technology for accessible math presentation},
	seealso={accessiblemath,screenreader,formulas,EPUB,html}
}

\newglossaryentry{mathOCR}{
	name={math OCR},
	description={Optical Character Recognition specialized for mathematical notation that can convert images of equations into accessible formats like MathML or LaTeX. Enables conversion of printed mathematical content into screen reader accessible formats},
	seealso={OCR,accessiblemath,MathML,LateX}
}

\newglossaryentry{MDT}{
	name={MDT},
	description={Multidisciplinary Team - a group of professionals from different disciplines who collaborate to assess student needs, develop intervention plans, and make educational decisions. Team composition varies based on student needs and may include teachers, therapists, psychologists, and medical professionals},
	seealso={IEP,CSE,ATassessment}
}

\newglossaryentry{ML}{
	name={ML},
	description={},
	see={machinelearning}
}

\newglossaryentry{mobility}{
	name={mobility},
	description={The ability to move around and navigate environments safely and independently. Assistive mobility technologies include white canes, guide dogs, GPS navigation apps, indoor navigation systems, and environmental awareness tools},
	seealso={navigation,whitecane,guidedog,gps,OM,mobilitytraining}
}

\newglossaryentry{mobilitytraining}{
	name={mobility training},
	description={Systematic instruction in safe and efficient travel skills for people with visual impairments, including cane techniques, route planning, street crossing, and use of public transportation. Provided by certified orientation and mobility specialists},
	seealso={OM,mobility,whitecane,TVI}
}

\newglossaryentry{music}{
	name={music},
	description={Audio art form requiring specialized accessibility considerations for musicians and music students with disabilities. Accessible music resources include braille music notation, large print scores, audio recordings, and adaptive musical instruments},
	seealso={musicbraille,MusicXML,BMML,stem}
}

\newglossaryentry{musicbraille}{
	name={music braille},
	description={A specialized braille code system for representing musical notation tactilely, enabling musicians who are blind to read and write music. Uses unique symbols for notes, rhythms, dynamics, and other musical elements},
	seealso={braille,music,MusicXML,BMML,bana}
}

\newglossaryentry{MusicXML}{
	name={MusicXML},
	description={A digital format for representing musical scores that can be converted to various accessible formats including braille music, large print, and audio playback. Enables sharing of musical notation across different software platforms},
	seealso={musicbraille,music,BMML,fileformats}
}

\newglossaryentry{narrator}{
	name={Narrator},
	description={Microsoft's built-in screen reader for Windows operating systems. Provides basic text-to-speech functionality and keyboard navigation, included free with Windows to ensure basic accessibility without additional software purchases},
	seealso={screenreader,jaws,nvda,operatingsystem}
}

\newglossaryentry{naturaluserinterface}{
	name={natural user interface},
	description={Computing interfaces that enable users to interact through natural behaviors like speech, gestures, and touch rather than traditional input devices. Particularly beneficial for users with motor disabilities who may find conventional interfaces challenging},
	seealso={voicecontrol,speechrecognition,eyetracking,tablet}
}

\newglossaryentry{navigation}{
	name={navigation},
	description={The process of finding one's way through physical or digital environments. Assistive navigation technologies include GPS systems with audio directions, indoor navigation apps, tactile maps, and screen reader navigation commands for digital content},
	seealso={mobility,gps,indoornavigation,screenreader,documentstructure}
}

\newglossaryentry{Nemeth}{
	name={Nemeth Code},
	description={A braille code used for representing mathematical and scientific notation. Developed by Dr. Abraham Nemeth, it allows people who are blind to read and write complex mathematical expressions in braille, including equations, formulas, and graphs},
	seealso={braille,accessiblemath,formulas,stem,bana}
}

\newglossaryentry{neuralnetwork}{
	name={neural network},
	description={A computational model inspired by biological neural networks that can learn patterns in data. In assistive technology, neural networks power image recognition, speech processing, predictive text, and other AI-driven accessibility features},
	seealso={artificialintelligence,machinelearning,objectrecognition,speechrecognition}
}

\newglossaryentry{nimac}{
	name={NIMAC},
	description={National Instructional Materials Access Center - a repository established under IDEA that houses electronic files of textbooks in NIMAS format, making them available to students with print disabilities through authorized users},
	seealso={nimas,instructionalmaterials,printdisability,accessiblematerials}
}

\newglossaryentry{nimas}{
	name={NIMAS},
	description={National Instructional Materials Accessibility Standard - a US technical standard for creating accessible instructional materials. Requires publishers to provide electronic files in specific XML format that can be converted to accessible formats like braille and audio},
	seealso={nimac,instructionalmaterials,XML,accessiblematerials}
}

\newglossaryentry{nlg}{
	name={NLG},
	description={Natural Language Generation - AI technology that converts structured data into human-readable text. In accessibility, NLG creates automatic descriptions of charts, graphs, and other visual content for screen reader users},
	seealso={NLP,artificialintelligence,alttext,datavisualization}
}

\newglossaryentry{NLP}{
	name={NLP},
	description={Natural Language Processing - AI technology that enables computers to understand, interpret, and generate human language. Used in assistive technology for speech recognition, text analysis, and intelligent content adaptation},
	seealso={nlg,artificialintelligence,speechrecognition,LLM}
}

\newglossaryentry{notetaker}{
	name={notetaker},
	description={Electronic braille device combining braille display functionality with note-taking capabilities, file storage, and basic computing functions. Examples include BrailleNote and Focus notetakers that serve as portable workstations for braille users},
	seealso={brailledisplay,braille,assistivetechnology}
}

\newglossaryentry{nvda}{
	name={NVDA},
	description={NonVisual Desktop Access - a free, open-source screen reader for Windows developed by NV Access. Provides comprehensive screen reading functionality with speech and braille support, widely used due to its no-cost availability},
	seealso={screenreader,jaws,narrator,opensource}
}

\newglossaryentry{objectrecognition}{
	name={object recognition},
	description={AI technology that identifies and classifies objects in images or real-time video feeds. Assistive applications include smartphone apps that describe surroundings, identify currency, read text in images, and navigate environments},
	seealso={artificialintelligence,OCR,alttext,apps,smartphone}
}

\newglossaryentry{OCR}{
	name={OCR},
	description={},
	see={opticalcharacterrecognition}
}

\newglossaryentry{officesuite}{
	name={office suite},
	description={Integrated software applications for document creation, editing, and presentation. Microsoft Office and similar suites include accessibility features like screen reader support, keyboard navigation, and tools for creating accessible documents},
	seealso={software,presentations,pdf,accessibility}
}

\newglossaryentry{OM}{
	name={O\&M},
	description={},
	see={orientationmobility}
}

\newglossaryentry{OMR}{
	name={OMR},
	description={Optical Music Recognition - technology that converts images of musical notation into digital music formats like MusicXML, enabling conversion of printed sheet music into accessible formats such as braille music or audio playback},
	seealso={OCR,music,MusicXML,musicbraille}
}

\newglossaryentry{onscreenkeyboard}{
	name={on-screen keyboard},
	description={Virtual keyboard displayed on computer screens that can be operated with pointing devices, eye tracking, or switch access. Provides text input alternatives for users with motor disabilities who cannot use physical keyboards},
	seealso={switch,switchaccess,eyetracking,tablet}
}

\newglossaryentry{opensource}{
	name={open source},
	description={Software development model where source code is freely available for use, modification, and distribution. Open source assistive technology like NVDA screen reader provides cost-effective accessibility solutions and community-driven development},
	seealso={nvda,software,digitaldivide}
}

\newglossaryentry{operatingsystem}{
	name={operating system},
	description={System software that manages computer hardware and provides services for applications. Modern operating systems like Windows, macOS, and Linux include built-in accessibility features such as screen readers, magnification, and voice recognition},
	seealso={software,hardware,screenreader,magnification,voicecontrol}
}

\newglossaryentry{opticalcharacterrecognition}{
	name={optical character recognition},
	description={The electronic conversion of images containing text into machine-encoded text. Critical for making scanned documents, PDFs, and photographs containing text accessible to screen reader users and searchable},
	seealso={scannersoftware,pdf,mathOCR,OMR,machinelearning}
}

\newglossaryentry{orientation}{
	name={orientation},
	description={The ability to understand one's position and relationship to the environment. For people with visual impairments, orientation skills are developed through training and supported by assistive technology like talking compasses and GPS systems},
	seealso={orientationmobility,mobility,gps,situationalawareness}
}

\newglossaryentry{orientationmobility}{
	name={orientation \& mobility},
	description={Systematic instruction that teaches people with visual impairments to travel safely and independently. Includes techniques for using white canes, understanding traffic patterns, navigating with technology, and developing spatial awareness skills},
	seealso={mobilitytraining,whitecane,TVI,mobility,orientation}
}

\newglossaryentry{PBIS}{
	name={PBIS},
	description={Positive Behavioral Interventions and Supports - an evidence-based framework for improving school climate and student behavior through proactive strategies, data-driven decision making, and multi-tiered support systems},
	seealso={RTI,UDL}
}

\newglossaryentry{pdf}{
	name={PDF},
	description={Portable Document Format - a file format that preserves document formatting across different systems. PDFs can be accessible when properly tagged with structure and alternative text, but many PDFs lack accessibility features and require conversion},
	seealso={documentstructure,alttext,OCR,fileformats,LateX}
}

\newglossaryentry{personcenteredplanning}{
	name={person-centered planning},
	description={A collaborative planning process that focuses on individual strengths, preferences, and goals to develop support plans. Emphasizes self-determination and includes the person with disabilities as the primary decision-maker in their services},
	seealso={selfdetermination,IEP,transition}
}

\newglossaryentry{personalizelearning}{
	name={personalized learning},
	description={Educational approaches that customize learning experiences based on individual student needs, interests, and abilities. Technology enables adaptive content delivery, alternative assessment methods, and individualized accommodation provision},
	seealso={UDL,artificialintelligence,assistivetechnology}
}

\newglossaryentry{presentations}{
	name={presentations},
	description={Visual displays of information typically using slides and multimedia. Accessible presentations include alternative text for images, clear structure, sufficient color contrast, and compatibility with screen readers and other assistive technologies},
	seealso={alttext,imagesandmedia,officesuite,screenreader}
}

\newglossaryentry{printdisability}{
	name={print disability},
	description={Any disability that prevents effective reading of conventional printed material. Includes blindness, low vision, physical disabilities affecting book handling, and learning disabilities. Qualifies individuals for accessible book services},
	seealso={bookshare,daisy,audiobook,accessiblematerials,nimac}
}

\newglossaryentry{processor}{
	name={processor},
	description={},
	see={cpu}
}

\newglossaryentry{programminglanguages}{
	name={programming languages},
	description={Formal languages used to create software applications, including accessibility-focused programs. Popular languages for assistive technology development include Python, JavaScript, C++, and Java, each offering different capabilities for accessibility solutions},
	seealso={software,api,assistivetechnology}
}

\newglossaryentry{ram}{
	name={RAM},
	description={Random Access Memory - temporary computer memory that stores data currently being used. Adequate RAM is crucial for smooth operation of assistive technology, as screen readers and other accessibility software can be memory-intensive},
	seealso={cpu,hardware,screenreader,latency}
}

\newglossaryentry{reasonableaccommodation}{
	name={reasonable accommodation},
	description={Modifications or adjustments to policies, practices, or environments that enable people with disabilities to participate equally without causing undue hardship. Required under ADA and other disability rights legislation},
	seealso={ada,504plan,IEP,equitableaccess}
}

\newglossaryentry{refreshablebrailledisplay}{
	name={refreshable braille display},
	description={},
	see={brailledisplay}
}

\newglossaryentry{relatedservices}{
	name={related services},
	description={Support services required to assist students with disabilities to benefit from special education, including speech therapy, occupational therapy, physical therapy, transportation, and assistive technology services as defined by IDEA},
	seealso={IEP,ATservices,FAPE}
}

\newglossaryentry{RTD}{
	name={RTD},
	description={},
	see={brailledisplay}
}

\newglossaryentry{RTI}{
	name={RTI},
	description={Response to Intervention - a multi-tiered educational approach that provides increasingly intensive interventions based on student response to instruction. Used for early identification of learning difficulties and prevention of academic failure},
	seealso={PBIS,UDL,learningdisability}
}

\newglossaryentry{safety}{
	name={safety},
	description={Protection from harm or danger, which is a critical consideration in assistive technology design. Safety features include audio warnings, obstacle detection, emergency communication systems, and fail-safe mechanisms in mobility and navigation devices},
	seealso={emergencyresponse,mobility,ergonomics}
}

\newglossaryentry{scannersoftware}{
	name={scanner software},
	description={Applications that convert printed text to digital format using scanners or smartphone cameras, often including OCR capabilities and text-to-speech output. Essential tools for accessing printed materials for users with visual impairments},
	seealso={OCR,texttospeech,apps,smartphone}
}

\newglossaryentry{screenreader}{
	name={screen reader},
	description={Software that reads aloud text and interface elements displayed on a computer screen, providing auditory access to digital content for users who are blind or have low vision. Examples include JAWS, NVDA, and VoiceOver},
	seealso={jaws,nvda,VoiceOver,narrator,TalkBack,texttospeech,brailledisplay}
}

\newglossaryentry{section508}{
	name={Section 508},
	description={A 1998 amendment to the US Rehabilitation Act requiring federal agencies to make electronic and information technology accessible to people with disabilities. Establishes accessibility standards for government websites, software, and digital content},
	seealso={WCAG,ada,VPAT,webaccessibility}
}

\newglossaryentry{selfdetermination}{
	name={self-determination},
	description={The combination of skills, knowledge, and beliefs that enable individuals to engage in goal-directed, self-regulated, autonomous behavior. Critical for successful outcomes for people with disabilities in education, employment, and independent living},
	seealso={independence,personcenteredplanning,transition}
}

\newglossaryentry{settframework}{
	name={SETT framework},
	description={Student, Environment, Tasks, and Tools - a comprehensive framework developed by Joy Zabala for assistive technology assessment and selection. Considers the student's needs, environmental factors, required tasks, and appropriate tools for successful outcomes},
	seealso={ATassessment,assistivetechnology}
}

\newglossaryentry{SGD}{
	name={SGD},
	description={Speech-Generating Device - electronic assistive technology that produces spoken language for individuals with complex communication needs. Ranges from simple message devices to sophisticated computer-based systems with dynamic vocabulary},
	seealso={AAC,communication,communicationboard,speechsynthesis}
}

\newglossaryentry{situationalawareness}{
	name={situational awareness},
	description={The perception and understanding of environmental elements and events with respect to time and space. Assistive technology enhances situational awareness through audio cues, haptic feedback, and environmental sensors for users with sensory impairments},
	seealso={orientation,mobility,safety,objectrecognition}
}

\newglossaryentry{smartcane}{
	name={smart cane},
	description={Advanced mobility aids that combine traditional white cane functions with electronic sensors, GPS navigation, haptic feedback, and smartphone connectivity to provide enhanced environmental information and navigation assistance},
	seealso={whitecane,mobility,gps,haptic,smartphone}
}

\newglossaryentry{smarthome}{
	name={smart home},
	description={Residential technology systems that automate and control home functions through internet-connected devices. Particularly beneficial for people with disabilities, enabling voice control, automated routines, and remote access to home systems},
	seealso={IoT,environmentalcontrol,voicecontrol,independence}
}

\newglossaryentry{smartphone}{
	name={smartphone},
	description={Mobile computing devices with advanced accessibility features built into operating systems. Include screen readers, magnification, voice control, and specialized apps for navigation, object recognition, and communication support},
	seealso={apps,screenreader,objectrecognition,scannersoftware,tablet}
}

\newglossaryentry{software}{
	name={software},
	description={Computer programs and applications, including specialized assistive technology software like screen readers, magnification programs, communication aids, and accessibility utilities that enable access to digital content and computer functions},
	seealso={hardware,assistivetechnology,apps,screenreader,opensource}
}

\newglossaryentry{sonification}{
	name={sonification},
	description={The use of non-speech audio to convey information, particularly useful for representing data patterns, trends, and relationships to users who cannot see visual displays. Examples include audio graphs and musical representations of statistical data},
	seealso={earcons,datavisualization,charts,auditoryfeedback}
}

\newglossaryentry{speechrecognition}{
	name={speech recognition},
	description={Technology that converts spoken language into text or computer commands. Enables hands-free computer operation for users with motor disabilities and provides alternative input methods for those who cannot use keyboards},
	seealso={voicecontrol,CART,biometrics,naturaluserinterface}
}

\newglossaryentry{speechsynthesis}{
	name={speech synthesis},
	description={Technology that converts written text into spoken audio output using computer-generated voices. Essential component of screen readers and other assistive technologies, with options for different voices, speeds, and pronunciation settings},
	seealso={texttospeech,tts,screenreader,SGD}
}

\newglossaryentry{stem}{
	name={STEM},
	description={Science, Technology, Engineering, and Mathematics - educational disciplines that require specialized accessibility considerations for students with disabilities, including accessible lab equipment, tactile graphics for diagrams, and alternative formats for mathematical content},
	seealso={accessiblemath,Nemeth,tactilegraphics,LateX,music}
}

\newglossaryentry{switch}{
	name={switch},
	description={Alternative input devices activated by minimal movement, pressure, or proximity, used by individuals with motor disabilities to control computers, communication devices, and environmental systems. Types include sip-and-puff, proximity, and pressure switches},
	seealso={switchaccess,eyetracking,environmentalcontrol}
}

\newglossaryentry{switchaccess}{
	name={switch access},
	description={A method of operating computers and devices using alternative switches instead of standard keyboards or mice. Includes scanning interfaces that highlight options sequentially for switch activation by users with severe motor limitations},
	seealso={switch,onscreenkeyboard,keyboardnavigation}
}

\newglossaryentry{tablet}{
	name={tablet},
	description={Portable touchscreen computing device that offers accessible interfaces through built-in features like VoiceOver (iOS) or TalkBack (Android), voice recognition, and specialized accessibility apps for users with various disabilities},
	seealso={smartphone,apps,VoiceOver,TalkBack,ebooks,BYOD}
}

\newglossaryentry{tactile}{
	name={tactile},
	description={Related to the sense of touch and physical sensation. Tactile technologies include braille displays, tactile graphics, haptic feedback systems, and raised-surface materials that provide information through touch rather than vision},
	seealso={braille,tactilegraphics,haptic,deafblind}
}

\newglossaryentry{tactilegraphics}{
	name={tactile graphics},
	description={Raised images, diagrams, and maps that can be explored through touch, providing access to visual information for people who are blind. Created through various methods including embossing, 3D printing, and specialized tactile production techniques},
	seealso={tactile,3dprinting,charts,datavisualization,stem}
}

\newglossaryentry{TalkBack}{
	name={TalkBack},
	description={Google's built-in screen reader for Android devices that provides spoken feedback and gesture-based navigation for users who are blind or have low vision. Includes features like explore-by-touch and reading controls},
	seealso={screenreader,tablet,smartphone,VoiceOver}
}

\newglossaryentry{technology}{
	name={technology},
	description={The application of scientific knowledge for practical purposes, including assistive technology that enhances functional capabilities for people with disabilities. Encompasses both hardware devices and software applications designed for accessibility},
	seealso={assistivetechnology,digitalliteracy,digitaldivide}
}

\newglossaryentry{textformatting}{
	name={text formatting},
	description={The visual presentation and structure of text including fonts, sizes, colors, and layout. Accessible text formatting considers readability for users with visual impairments and compatibility with screen readers through semantic markup},
	seealso={fonts,typography,documentstructure,screenreader}
}

\newglossaryentry{texttospeech}{
	name={text-to-speech},
	description={},
	see={tts}
}

\newglossaryentry{transition}{
	name={transition},
	description={The coordinated set of activities designed to promote movement from school to post-school environments, including higher education, employment, and independent living. Required IEP component for students 16 and older},
	seealso={IEP,independentlivingskills,selfdetermination,personcenteredplanning}
}

\newglossaryentry{troubleshooting}{
	name={troubleshooting},
	description={The systematic process of identifying and resolving problems with assistive technology through diagnostic steps, testing solutions, and technical support. Critical skill for maintaining effective use of accessibility equipment and software},
	seealso={ATservices,assistivetechnology,hardware,software}
}

\newglossaryentry{tts}{
	name={TTS},
	description={Text-to-Speech - technology that converts written text into spoken audio output using computer-generated voices. Essential for screen readers and other accessibility applications that provide auditory access to written information},
	seealso={speechsynthesis,screenreader,audio,dyslexia}
}

\newglossaryentry{TVI}{
	name={TVI},
	description={Teacher of Students with Visual Impairments - a specialized educator certified to work with students who are blind or have low vision. Provides instruction in braille, assistive technology, orientation and mobility, and the expanded core curriculum},
	seealso={brailleeducation,orientationmobility,assistivetechnology,cvi}
}

\newglossaryentry{typography}{
	name={typography},
	description={The art and technique of arranging type to make written language legible, readable, and appealing. In accessibility, typography considerations include font choice, size, spacing, and contrast to ensure readability for users with visual impairments},
	seealso={fonts,textformatting,dyslexia,largeprint}
}

\newglossaryentry{UDL}{
	name={UDL},
	description={Universal Design for Learning - an educational framework based on neuroscience research that provides multiple means of representation, engagement, and expression to accommodate learner variability and reduce barriers to learning},
	seealso={universaldesign,inclusivedesign,educationalequity,personalizelearning,cognitiveaccessibility}
}

\newglossaryentry{universaldesign}{
	name={universal design},
	description={The design of products and environments to be usable by all people, to the greatest extent possible, without the need for adaptation or specialized design. Based on seven principles including equitable use and flexibility},
	seealso={UDL,inclusivedesign,accessibility,equitableaccess}
}

\newglossaryentry{usabilityTesting}{
	name={usability testing},
	description={Evaluation methods that involve observing users with disabilities as they interact with products or systems to identify barriers and usability issues. Essential for developing truly accessible technology solutions},
	seealso={accessibilityaudit,userexperience,cognitiveaccessibility}
}

\newglossaryentry{usb}{
	name={USB},
	description={Universal Serial Bus - a standard interface for connecting devices to computers. In assistive technology, USB connections enable braille displays, adaptive keyboards, switches, and other accessibility devices to interface with computers},
	seealso={brailledisplay,switch,hardware}
}

\newglossaryentry{userexperience}{
	name={user experience},
	description={The overall experience of a person using a product or system, including usability, accessibility, and satisfaction. UX design for assistive technology requires understanding diverse user needs and interaction patterns},
	seealso={usabilityTesting,inclusivedesign,cognitiveload,ergonomics}
}

\newglossaryentry{videomagnifier}{
	name={video magnifier},
	description={An electronic device that uses a camera and display screen to magnify text and objects for users with low vision. Features typically include variable magnification levels, high contrast modes, color enhancement, and both near and distance viewing options},
	seealso={CCTV,magnification,lowvision,zoom}
}

\newglossaryentry{videorelayservice}{
	name={video relay service},
	description={A telecommunications service that allows deaf and hard of hearing individuals to communicate with hearing people through video calls with sign language interpreters. Provides functional equivalence to voice telephone services},
	seealso={ASL,communication}
}

\newglossaryentry{virtualreality}{
	name={virtual reality},
	description={Immersive computer-generated environments that can provide alternative experiences for users with disabilities, including virtual travel, social interaction, job training, and therapeutic applications. Requires accessibility considerations for diverse users},
	seealso={technology,apps,accessibility}
}

\newglossaryentry{visualimpairment}{
	name={visual impairment},
	description={A condition that affects vision and cannot be corrected with standard glasses or contact lenses. Ranges from low vision to total blindness and requires adaptive techniques, assistive technology, or environmental modifications for accessing visual information},
	seealso={lowvision,cvi,TVI,braille,screenreader}
}

\newglossaryentry{voicecontrol}{
	name={voice control},
	description={Technology that allows users to operate devices and software through spoken commands. Particularly beneficial for individuals with motor disabilities who cannot use traditional input methods like keyboards and mice},
	seealso={speechrecognition,naturaluserinterface,smarthome,environmentalcontrol}
}

\newglossaryentry{VoiceOver}{
	name={VoiceOver},
	description={Apple's built-in screen reader for macOS and iOS devices that provides spoken descriptions and gesture-based navigation for users who are blind or have low vision. Includes features like rotor navigation and braille display support},
	seealso={screenreader,tablet,smartphone,TalkBack,brailledisplay}
}

\newglossaryentry{VPAT}{
	name={VPAT},
	description={Voluntary Product Accessibility Template - a standardized document that evaluates how accessible technology products are in accordance with Section 508 standards. Used by organizations to assess technology procurement decisions},
	seealso={section508,WCAG,accessibilityaudit}
}

\newglossaryentry{WCAG}{
	name={WCAG},
	description={Web Content Accessibility Guidelines - international standards developed by the W3C for making web content accessible to people with disabilities. Current version is WCAG 2.1 with principles of Perceivable, Operable, Understandable, and Robust (POUR)},
	seealso={webaccessibility,section508,VPAT,alttext,ARIA}
}

\newglossaryentry{webaccessibility}{
	name={web accessibility},
	description={The practice of designing and developing websites and web applications to be usable by people with disabilities. Includes features like keyboard navigation, screen reader compatibility, sufficient color contrast, and clear content structure},
	seealso={WCAG,digitalaccessibility,html,ARIA,keyboardnavigation,screenreader}
}

\newglossaryentry{webaim}{
	name={WebAIM},
	description={Web Accessibility In Mind - a nonprofit organization that provides web accessibility resources, training, and evaluation tools. Known for developing accessibility evaluation tools like WAVE and conducting annual screen reader user surveys},
	seealso={webaccessibility,WCAG,accessibilityaudit}
}

\newglossaryentry{whitecane}{
	name={white cane},
	description={A mobility tool used by people who are blind or have low vision to detect obstacles, drop-offs, and surface changes while traveling. Recognized internationally as a symbol of blindness and provides legal protection for users},
	seealso={mobility,guidedog,smartcane,orientationmobility,TVI}
}

\newglossaryentry{wordprocessing}{
	name={word processing},
	description={The creation, editing, formatting, and printing of text documents using software applications. Accessible word processing includes features like screen reader compatibility, alternative text for images, and support for braille displays},
	seealso={officesuite,documentstructure,screenreader,textformatting}
}
\newglossaryentry{wordprediction}{
	name={word prediction},
	description={Software feature that suggests complete words or phrases as users type, reducing keystrokes and supporting individuals with spelling difficulties, motor limitations, or language processing challenges. Often includes voice output options},
	seealso={dyslexia,learningdisability,speechsynthesis}
}

\newglossaryentry{XML}{
	name={XML},
	description={eXtensible Markup Language - a markup language for encoding documents in a format that is both human-readable and machine-readable. Used in accessible document formats like NIMAS files and structured content that can be converted to multiple accessible formats},
	seealso={html,nimas,documentstructure,fileformats,JSON}
}
\newglossaryentry{accessibilitysoftware}{
	name={accessibility software},
	description={Software applications designed to improve access to digital content for people with disabilities. Includes screen readers, magnification tools, speech recognition software, and other assistive technologies that enhance usability and functionality},
	seealso={screenreader,magnification,assistivetechnology,software}
}

\newglossaryentry{zoom}{
	name={zoom},
	description={The ability to magnify visual content on screens or through optical devices. Software zoom features in operating systems and applications allow users with low vision to enlarge text and images, while optical zoom refers to magnification through lenses},
	seealso={magnification,videomagnifier,lowvision,operatingsystem}
}

\newglossaryentry{zoomtext}{
	name={ZoomText},
	description={A screen magnification software program designed for users with low vision. Provides customizable magnification levels, color enhancements, and screen reading capabilities to improve accessibility of digital content},
	seealso={magnification,lowvision,screenreader,accessibilitysoftware}
}
