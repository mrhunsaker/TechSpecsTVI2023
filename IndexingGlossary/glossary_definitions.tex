% --- Additional WCAG and accessibility glossary entries ---
% Understandable (WCAG principle)
\newglossaryentry{understandable}{
	name={understandable},
	type={concept},
	description={One of the four WCAG principles: information and the operation of the user interface must be understandable.}
}

% Robust (WCAG principle)
\newglossaryentry{robust}{
	name={robust},
	type={concept},
	description={One of the four WCAG principles: content must be robust enough to be interpreted reliably by a wide variety of user agents, including assistive technologies.}
}

% ARIA (Accessible Rich Internet Applications)
\newglossaryentry{aria}{
	name={ARIA},
	type={standard},
	description={Accessible Rich Internet Applications—W3C specification for adding accessibility information to web content and applications, especially dynamic content and advanced user interface controls developed with Ajax, HTML, JavaScript, and related technologies.}
}

% PDF/UA (alias for pdfua)
\newglossaryentry{pdf/ua}{
	name={PDF/UA},
	type={standard},
	description={See \gls{pdfua}. Alias for case-insensitive referencing.}
}
% --- Aliases and missing entries for glossary consistency ---
% Alias for WCAG (uppercase)
\newglossaryentry{WCAG}{
	name={WCAG},
	type={standard},
	description={See \gls{wcag}. Alias for case-insensitive referencing.}
}

% Perceivable (WCAG principle)
\newglossaryentry{perceivable}{
	name={perceivable},
	type={concept},
	description={One of the four WCAG principles: information and user interface components must be presentable to users in ways they can perceive.}
}

% Operable (WCAG principle)
\newglossaryentry{operable}{
	name={operable},
	type={concept},
	description={One of the four WCAG principles: user interface components and navigation must be operable by all users.}
}
% Generated glossary definitions for LaTeX

% Include this file in your LaTeX preamble after \usepackage{glossaries}

% Use \gls{key} in your text to reference terms



\newglossaryentry{3dprinter}{
	name={3D printer},
	type={device},
	description={A device that creates three-dimensional objects from digital files, used in assistive technology to create tactile models and adaptive devices}
}

\newglossaryentry{3dprinting}{
	name={3D printing},
	type={process},
	description={The process of creating three-dimensional objects from digital files, widely used in assistive technology for creating tactile learning materials}
}

\newglossaryentry{accessibility}{
	name={accessibility},
	type={process},
	description={The design of products, devices, services, or environments to be usable by people with disabilities}
}

\newglossaryentry{accessiblematerials}{
	name={accessible materials},
	type={process},
	description={Instructional or curricular content produced or remediated so learners with disabilities can perceive, navigate, and interact equivalently (semantic structure, headings, alt text, captions, transcripts, MathML, \gls{braille}, proper color contrast). See also: \gls{bornaccessible}, \gls{instructionalmaterials}, \gls{equitableaccess}.}
}

\newglossaryentry{accessiblemath}{
	name={accessible math},
	type={process},
	description={Mathematical content encoded with semantic markup (e.g., \gls{mathml}), linear braille (Nemeth / UEB math), speech rules, tactile or audio graph alternatives, enabling non-visual navigation by element (term, fraction, superscript). See also: \gls{mathocr}, \gls{braille}, \gls{alttext}.}
}

\newglossaryentry{ada}{
	name={ada},
	type={standard},
	description={Americans with Disabilities Act - US civil rights law prohibiting discrimination based on disability}
}

\newglossaryentry{ai}{
	name={AI},
	type={process},
	description={Artificial Intelligence — computational techniques (including machine learning, large language models, computer vision, and natural language processing) that enable pattern recognition, prediction, generation, and decision support. In accessibility contexts AI powers real‑time OCR, scene and object description, math/handwriting recognition, adaptive user interfaces, predictive text, and semantic summarization. Modern deployments increasingly leverage on‑device NPUs for low‑latency, privacy‑preserving inference (reducing reliance on cloud round‑trips) and enable concurrent assistive workloads (screen reading, magnification, captioning) without saturating CPU resources. See also: \gls{machinelearning}, \gls{ocr}, \gls{processor}, \gls{latency}.}
}

\newglossaryentry{alttext}{
	name={alt text},
	type={process},
	description={Textual alternative that conveys the essential purpose or information of non-text content (images, graphics, UI icons). Enables comprehension for screen reader and braille users; complex visuals may require extended descriptions. See also: \gls{tactilegraphics}, \gls{datavisualization}, \gls{equitableaccess}.}
}

\newglossaryentry{api}{
	name={api},
	type={process},
	description={Application Programming Interface - protocols that enable software accessibility features}
}

\newglossaryentry{apps}{
	name={apps},
	type={process},
	description={Software applications (desktop, mobile, web) that may expose accessibility APIs for assistive technology (\gls{screenreader}, magnification, switch control). Inclusive apps implement semantic UI elements, keyboard access, and correct focus management.}
}

\newglossaryentry{artificialintelligence}{
	name={artificial intelligence},
	type={process},
	description={See \gls{ai}. Alias retained for indexing continuity.}
}

\newglossaryentry{assistivetechnology}{
	name={assistive technology},
	type={process},
	description={Any item, piece of equipment, or system used to increase, maintain, or improve functional capabilities of individuals with disabilities}
}

\newglossaryentry{audio}{
	name={audio},
	type={modality},
	description={Acoustic output channel leveraged for speech, earcons, spatial cues, and sonification. Critical to non-visual workflows (e.g., \gls{screenreader} speech, \gls{spatialaudio}, \gls{informationalaudiocue}). See also: \gls{texttospeech}, \gls{sonification}, \gls{haptic}.}
}

\newglossaryentry{audiobook}{
	name={audiobook},
	type={modality},
	description={Digital audio recordings of books, providing accessible reading alternatives}
}

\newglossaryentry{auditoryfeedback}{
	name={auditory feedback},
	type={modality},
	description={Sound-based responses that provide information about system status and user actions}
}

\newglossaryentry{bana}{
	name={BANA},
	type={standard},
	description={Braille Authority of North America - organization that promotes braille literacy and standardization}
}

\newglossaryentry{bluetooth}{
	name={bluetooth},
	type={standard},
	description={Wireless communication technology for short-range connections}
}

\newglossaryentry{bmml}{
	name={BMML},
	type={standard},
	description={Braille Music Markup Language (if used): an XML-like or structured representation intended to encode music braille semantics for transformation or validation. (Confirm formal spec; placeholder if internal.) See also: \gls{musicbraille}.}
}

\newglossaryentry{bookshare}{
	name={Bookshare},
	type={standard},
	description={An accessible online library providing DAISY, EPUB, and BRF books to qualified print‑disabled users; supports multi-modal reading (speech + braille). See also: \gls{brf}, \gls{accessiblematerials}.}
}

\newglossaryentry{braille}{
	name={braille},
	type={modality},
	description={A tactile writing system using raised dots that allows people who are blind to read and write through touch}
}

\newglossaryentry{brailledisplay}{
	name={braille display},
	type={device},
	description={A tactile electronic device that displays braille characters by raising and lowering pins, allowing blind users to read screen content}
}

\newglossaryentry{brailleeducation}{
	name={braille education},
	type={process},
	description={Instructional methodologies and curricula for teaching braille literacy, incorporating tactile discrimination, code knowledge (UEB, Nemeth), and technology integration (displays, embossers). See also: \gls{braille}, \gls{ueb}.}
}

\newglossaryentry{brailleembosser}{
	name={braille embosser},
	type={device},
	description={A printer that creates braille text on paper by embossing raised dots, essential for producing hard-copy braille materials}
}

\newglossaryentry{brailleliteracy}{
	name={braille literacy},
	type={process},
	description={The ability to read and write braille, essential for academic success of blind students}
}

\newglossaryentry{charts}{
	name={charts},
	type={process},
	description={Structured visual data representations (bar, line, scatter, etc.) requiring accessible alternatives: textual summaries, data tables, sonification, or tactile graphics. See also: \gls{tactilegraphics}, \gls{sonification}, \gls{alttext}.}
}

\newglossaryentry{cognitiveload}{
	name={cognitive load},
	type={process},
	description={The amount of mental effort required to complete a task, which can be increased by poorly designed interfaces}
}

\newglossaryentry{communication}{
	name={communication},
	type={process},
	description={Exchange of information through modalities (speech, text, braille, AAC). Accessibility ensures channel equivalence (captioning, braille displays, alternative input). See also: \gls{assistivetechnology}, \gls{equitableaccess}.}
}

\newglossaryentry{cpu}{
	name={CPU},
	type={device},
	description={Central Processing Unit - the main processor of a computer, critical for smooth assistive technology operation}
}

\newglossaryentry{cvi}{
	name={CVI},
	type={concept},
	description={Cortical Visual Impairment - a visual impairment caused by damage to the visual pathways in the brain}
}

\newglossaryentry{dailylivingaids}{
	name={daily living aids},
	type={device},
	description={Assistive devices that help people with disabilities perform everyday tasks independently}
}

\newglossaryentry{daisy}{
	name={DAISY},
	type={standard},
	description={Digital Accessible Information System - a standard for creating accessible digital talking books}
}

\newglossaryentry{datavisualization}{
	name={data visualization},
	type={process},
	description={The presentation of data in visual formats, requiring alternative accessible representations}
}

\newglossaryentry{digitalaccessibility}{
	name={digital accessibility},
	type={process},
	description={The design of digital technology to be usable by people with disabilities}
}

\newglossaryentry{digitalliteracy}{
	name={digital literacy},
	type={process},
	description={Ability to locate, evaluate, create, and communicate information using digital tools—paired with assistive tech proficiency (screen reader navigation, accessible document authoring). See also: \gls{accessiblematerials}.}
}

\newglossaryentry{documentstructure}{
	name={document structure},
	type={process},
	description={The organization and markup of documents to ensure they are navigable by assistive technology}
}

\newglossaryentry{ebooks}{
	name={e-books},
	type={format},
	description={Digital publications (EPUB, DAISY, tagged PDF) supporting reflow, structured navigation, and multi-modal output (speech/braille). Accessibility depends on semantic tagging and alt text. See also: \gls{pdf}, \gls{accessiblematerials}.}
}

\newglossaryentry{educationalequity}{
	name={educational equity},
	type={concept},
	description={The principle that all students, including those with disabilities, should have equal access to educational opportunities and resources}
}

\newglossaryentry{equitableaccess}{
	name={equitable access},
	type={process},
	description={Condition where learners have comparable opportunity, efficiency, and independence regardless of disability—measured by parity of latency, comprehension, task completion, and cost barriers. See also: \gls{accessiblematerials}, \gls{hardware}, \gls{bornaccessible}.}
}

\newglossaryentry{fileformats}{
	name={file formats},
	type={format},
	description={Different ways of encoding and storing digital information, with varying levels of accessibility}
}

\newglossaryentry{fonts}{
	name={fonts},
	type={concept},
	description={Typeface designs that affect readability for users with visual impairments}
}

\newglossaryentry{formulas}{
	name={formulas},
	type={process},
	description={Mathematical expressions requiring semantic representation (e.g., \gls{mathml}) for accurate speech, braille (Nemeth/UEB), and tactile/sonified interpretation. See also: \gls{accessiblemath}, \gls{mathocr}.}
}

\newglossaryentry{gps}{
	name={GPS},
	type={standard},
	description={Global Positioning System - satellite-based navigation technology adapted for use by people with visual impairments}
}

\newglossaryentry{graphingcalculator}{
	name={graphing calculator},
	type={device},
	description={Device or software computing and visualizing mathematical functions; accessible implementations provide sonified traces, braille/tactile graphs, and keyboard navigation. See also: \gls{tactilegraphics}, \gls{accessiblemath}.}
}

\newglossaryentry{haptic}{
	name={haptic},
	type={modality},
	description={Relating to touch-based feedback (vibration, force, pin arrays) conveying spatial, directional, or confirmation cues. Complements auditory channels. See also: \gls{hapticchannel}, \gls{tactilegraphicsdisplay}.}
}

\newglossaryentry{hapticfeedback}{
	name={haptic feedback},
	type={modality},
	description={Touch-based feedback through vibration or force, providing non-visual information to users}
}

\newglossaryentry{hardware}{
	name={hardware},
	type={concept},
	description={Physical computing components (CPU, memory, storage, display, GPU/NPU, audio subsystem) whose performance characteristics (latency, throughput) directly impact assistive tech responsiveness. See also: \gls{processor}, \gls{ram}, \gls{latency}.}
}

\newglossaryentry{html}{
	name={html},
	type={standard},
	description={HyperText Markup Language - the standard markup language for creating accessible web pages}
}

\newglossaryentry{imagesandmedia}{
	name={images and media},
	type={process},
	description={Non-text assets (photos, diagrams, video, audio) requiring alternative representations: alt text, captions, transcripts, tactile or data sonification. See also: \gls{alttext}, \gls{tactilegraphics}.}
}

\newglossaryentry{inclusivedesign}{
	name={inclusive design},
	type={process},
	description={Design approach considering human diversity (ability, language, context) from ideation through delivery; reduces retrofit remediation. See also: \gls{bornaccessible}, \gls{equitableaccess}.}
}

\newglossaryentry{independence}{
	name={independence},
	type={concept},
	description={The ability to perform tasks and make decisions without assistance, supported by appropriate assistive technology}
}

\newglossaryentry{independentlivingskills}{
	name={independent living skills},
	type={process},
	description={Abilities needed to live independently, often supported by assistive technology}
}

\newglossaryentry{indoornavigation}{
	name={indoor navigation},
	type={process},
	description={Technology systems that provide navigation assistance inside buildings where GPS is not available}
}

\newglossaryentry{instructionalmaterials}{
	name={instructional materials},
	type={process},
	description={Curricular content (textbooks, assignments, assessments) that must be produced in accessible formats (EPUB, tagged PDF, BRF) to ensure learning parity. See also: \gls{accessiblematerials}.}
}

\newglossaryentry{interactivegraphingenvironments}{
	name={interactive graphing environments},
	type={process},
	description={Software platforms enabling dynamic exploration of mathematical plots (zoom, trace, derivative). Accessible versions expose keyboard focus, provide sonification/tactile export, and describe axes and data points.}
}

\newglossaryentry{jaws}{
	name={jaws},
	type={device},
	description={Job Access With Speech - a commercial screen reader software developed by Freedom Scientific}
}

\newglossaryentry{laptop}{
	name={laptop},
	type={device},
	description={A portable computer that, when properly configured, serves as the primary computing platform for students using assistive technology}
}

\newglossaryentry{latency}{
	name={latency},
	type={concept},
	description={The delay between user input and system response, particularly critical for screen reader users who rely on immediate audio feedback}
}

\newglossaryentry{latex}{
	name={LaTeX},
	type={format},
	description={A document preparation system used for high-quality typesetting, particularly useful for mathematical and scientific documents}
}

\newglossaryentry{llm}{
	name={LLM},
	type={process},
	description={Large Language Model: a neural network trained on large corpora to generate and interpret text; supports summarization, code assist, alternative text drafts—requiring human verification for accuracy and bias. See also: \gls{ai}, \gls{machinelearning}.}
}

\newglossaryentry{machinelearning}{
	name={machine learning},
	type={process},
	description={Subset of \gls{ai} using statistical models (e.g., neural networks) to learn patterns from data; powers OCR, speech recognition, predictive text, and adaptive interfaces. See also: \gls{neuralnetwork}, \gls{llm}.}
}

\newglossaryentry{magnification}{
	name={magnification},
	type={modality},
	description={Technology that enlarges visual content for users with low vision}
}

\newglossaryentry{markdown}{
	name={Markdown},
	type={format},
	description={A lightweight markup language that can be easily converted to accessible formats including braille}
}

\newglossaryentry{mathml}{
	name={MathML},
	type={standard},
	description={Mathematical Markup Language - a markup language for describing mathematical notation in a screen reader accessible way}
}

\newglossaryentry{mathocr}{
	name={math OCR},
	type={process},
	description={Specialized recognition pipeline converting images/PDF math expressions into LaTeX and \gls{mathml} with structural semantics for speech and braille. See also: \gls{accessiblemath}, \gls{ocr}.}
}

\newglossaryentry{mobility}{
	name={mobility},
	type={concept},
	description={Ability to navigate physical or virtual environments; supported by orientation \& mobility training, GPS, indoor navigation beacons, haptic/spatial audio cues. See also: \gls{orientationmobility}.}
}

\newglossaryentry{music}{
	name={music},
	type={modality},
	description={Auditory art form requiring accessible notation conversions (music braille), tactile diagrams (fingering, rhythm), or descriptive audio for theoretical concepts. See also: \gls{musicbraille}, \gls{tactilegraphics}.}
}

\newglossaryentry{musicbraille}{
	name={music braille},
	type={modality},
	description={A braille code system for representing musical notation tactilely}
}

\newglossaryentry{musicxml}{
	name={MusicXML},
	type={format},
	description={A digital format for representing musical notation that can be converted to accessible formats}
}

\newglossaryentry{narrator}{
	name={narrator},
	type={device},
	description={Microsoft's built-in screen reader for Windows operating systems}
}

\newglossaryentry{navigation}{
	name={navigation},
	type={process},
	description={Efficient movement through digital or physical space using structural markers (headings, landmarks, ARIA roles) or spatial/haptic cues. For screen readers: heading, link, form, table navigation paradigms. See also: \gls{layoutanalysis}.}
}

\newglossaryentry{neuralnetwork}{
	name={neural network},
	type={process},
	description={Computational architecture of layered nodes enabling pattern learning (vision, language, audio). Basis for OCR, speech synthesis, captioning, and LLMs. See also: \gls{machinelearning}, \gls{llm}.}
}

\newglossaryentry{nimac}{
	name={NIMAC},
	type={standard},
	description={National Instructional Materials Access Center—repository supplying NIMAS source files to produce accessible formats (braille, audio, EPUB) for qualified students. See also: \gls{nimas}, \gls{accessiblematerials}.}
}

\newglossaryentry{nimas}{
	name={NIMAS},
	type={standard},
	description={National Instructional Materials Accessibility Standard - a US standard for creating accessible instructional materials}
}

\newglossaryentry{nlg}{
	name={NLG},
	type={process},
	description={Natural Language Generation: automated production of human-readable text (summaries, alt text drafts) from structured data or models; requires human review for factual accuracy. See also: \gls{llm}, \gls{ai}.}
}

\newglossaryentry{notetaker}{
	name={notetaker},
	type={device},
	description={Braille notetaker device combining braille display with note-taking and computing functions}
}

\newglossaryentry{nvda}{
	name={nvda},
	type={device},
	description={NonVisual Desktop Access - a free, open-source screen reader for Windows}
}

\newglossaryentry{ocr}{
	name={OCR},
	type={process},
	description={Optical Character Recognition - technology that converts images of text into machine-readable and screen reader accessible text}
}

\newglossaryentry{officesuite}{
	name={office suite},
	type={process},
	description={Microsoft Office suite applications with accessibility features for document creation and editing}
}

\newglossaryentry{omr}{
	name={OMR},
	type={process},
	description={Optical Music Recognition: converting scanned music notation into structured symbolic representations (MusicXML) for braille or audible rendering. See also: \gls{musicbraille}.}
}

% --- Alias entries for capitalized keys used in chapters ---
% These provide backward compatibility for \gls{MusicXML} and \gls{OMR}
% while keeping the canonical data in the lowercase entries (musicxml, omr).
\newglossaryentry{MusicXML}{
	alias=musicxml,
	name={MusicXML},
	description={}
}
\newglossaryentry{OMR}{
	alias=omr,
	name={OMR},
	description={}
}

% ---- Added alias entries and supporting base entries ----
% Aliases for capitalization variants
\newglossaryentry{AI}{
	alias=ai,
	name={AI},
	description={}
}
\newglossaryentry{MathML}{
	alias=mathml,
	name={MathML},
	description={}
}
\newglossaryentry{BrailleReadyFormat}{
	alias=brf,
	name={Braille Ready Format},
	description={}
}
\newglossaryentry{earcons}{
	alias=audio,
	name={earcons},
	description={}
}

% Base entries for platform screen readers plus their alias forms
\newglossaryentry{voiceover}{
	name={VoiceOver},
	type={device},
	description={Apple's built-in screen reader for macOS, iOS, and iPadOS}
}
\newglossaryentry{VoiceOver}{
	alias=voiceover,
	name={VoiceOver},
	description={}
}
\newglossaryentry{talkback}{
	name={TalkBack},
	type={device},
	description={Android's built-in screen reader providing gesture-driven navigation and braille support}
}
\newglossaryentry{TalkBack}{
	alias=talkback,
	name={TalkBack},
	description={}
}

\newglossaryentry{opensource}{
	name={open source},
	type={process},
	description={Software licensed to allow inspection, modification, and redistribution of source code; fosters accessibility innovation (community add-ons, braille translation engines). See also: \gls{liblouis} if defined.}
}

\newglossaryentry{operatingsystem}{
	name={operating system},
	type={process},
	description={System software that manages computer hardware and provides services for applications, with built-in accessibility features}
}

\newglossaryentry{opticalcharacterrecognition}{
	name={optical character recognition},
	type={process},
	description={Process converting images or scans into machine-encoded text enabling search, semantic tagging, and conversion to accessible formats (HTML, EPUB, braille). See also: \gls{postocrremediation}, \gls{layoutanalysis}.}
}

\newglossaryentry{orientation}{
	name={orientation},
	type={concept},
	description={Awareness of position, environment layout, and relative direction—foundation for safe independent mobility using auditory, tactile, and sometimes computer vision cues. See also: \gls{mobility}.}
}

\newglossaryentry{orientationmobility}{
	name={orientation \& mobility},
	type={process},
	description={Instruction and strategies that build environmental awareness and independent travel skills for individuals with visual impairments; integrates canes, GPS, indoor beacons, haptic/spatial audio. See also: \gls{mobility}.}
}

\newglossaryentry{pdf}{
	name={PDF},
	type={format},
	description={Portable Document Format - a file format that can be made accessible through proper tagging and structure}
}

\newglossaryentry{presentations}{
	name={presentations},
	type={process},
	description={Slide or media-based instructional materials requiring accessible design: structured heading order, descriptive link text, alt text for visuals, high color contrast, readable speaker notes.}
}

\newglossaryentry{processor}{
	name={processor},
	type={concept},
	description={A general term for compute elements (e.g., CPU, GPU, NPU / AI accelerator, and sometimes DSPs) that execute instructions and parallel workloads. For assistive technology performance, modern heterogeneous processors pair high‑performance and efficiency CPU cores with integrated graphics and dedicated NPUs. This division of labor reduces input→speech and magnification latency by: (1) offloading AI inference (OCR, scene description, speech synthesis, noise suppression) to NPUs, (2) reserving CPU cycles for event handling and accessibility tree updates, and (3) leveraging integrated GPUs for high‑refresh, low‑blur magnified rendering. Key attributes affecting accessibility include single‑thread responsiveness (input echo), memory controller bandwidth (braille / speech buffer refill), sustained thermal performance (preventing throttling spikes), and native support for low‑power, always‑on assistive services. See also: \gls{cpu}, \gls{latency}, \gls{ai}, \gls{ram}.}
}

\newglossaryentry{programminglanguages}{
	name={programming languages},
	type={process},
	description={Computer languages used to create software, including accessibility-focused applications}
}

\newglossaryentry{ram}{
	name={RAM},
	type={device},
	description={Random Access Memory - computer memory that significantly impacts screen reader performance and responsiveness}
}

\newglossaryentry{safety}{
	name={safety},
	type={concept},
	description={Protection from physical, digital, or cognitive harm in learning and technology contexts (secure navigation apps, privacy-preserving AI, avoidance of motion sickness triggers).}
}

\newglossaryentry{screenreader}{
	name={auditory access tool},
	type={device},
	description={Software that reads screen content aloud and provides keyboard navigation for users who are blind or have low vision}
}

\newglossaryentry{section508}{
	name={section 508},
	type={standard},
	description={A US law requiring federal agencies to make electronic and information technology accessible}
}

\newglossaryentry{settframework}{
	name={SETT framework},
	type={process},
	description={Student, Environment, Tasks, and Tools - a framework for assistive technology assessment and selection}
}

\newglossaryentry{situationalawareness}{
	name={situational awareness},
	type={concept},
	description={The perception and understanding of one's environment, supported by assistive technology for users with visual impairments}
}

\newglossaryentry{software}{
	name={software},
	type={process},
	description={Executable code and associated resources implementing functionality; accessible software adheres to semantic UI, keyboard operability, low-latency feedback, and assistive API integration.}
}

\newglossaryentry{sonification}{
	name={sonification},
	type={modality},
	description={The use of sound to convey information, particularly useful for representing data to users who are blind}
}

\newglossaryentry{speechsynthesis}{
	name={speech synthesis},
	type={process},
	description={Algorithmic generation of natural-sounding speech from text (parametric, neural TTS) powering \gls{screenreader} output and self-voicing applications. See also: \gls{tts}.}
}

\newglossaryentry{stem}{
	name={STEM},
	type={concept},
	description={Science, Technology, Engineering, and Mathematics - educational disciplines that require specialized accessibility considerations}
}

\newglossaryentry{tablet}{
	name={tablet},
	type={device},
	description={A portable touchscreen computing device that offers accessible interfaces and specialized apps for users with visual impairments}
}

\newglossaryentry{tactile}{
	name={tactile},
	type={modality},
	description={Relating to touch perception; tactile modalities (braille cells, raised lines, pin displays) convey spatial or textual information non-visually. See also: \gls{tactilegraphics}, \gls{tactilegraphicsdisplay}.}
}

\newglossaryentry{tactilegraphics}{
	name={tactile graphics},
	type={modality},
	description={Raised images and diagrams that can be read through touch, providing visual information to users who are blind}
}

\newglossaryentry{technology}{
	name={technology},
	type={concept},
	description={Tools, systems, and processes (hardware + software) engineered to solve problems; in accessibility, technology augments perception, interaction, and cognition.}
}

\newglossaryentry{textformatting}{
	name={text formatting},
	type={process},
	description={Arrangement of textual elements (fonts, spacing, emphasis, lists) affecting readability, parsing by assistive tech, and braille translation fidelity. See also: \gls{typography}.}
}

\newglossaryentry{texttospeech}{
	name={text-to-speech},
	type={process},
	description={Conversion of written text into audible speech (TTS) enabling non-visual reading; quality factors: latency, intelligibility, prosody, multilingual support. See also: \gls{tts}, \gls{speechsynthesis}.}
}

\newglossaryentry{troubleshooting}{
	name={troubleshooting},
	type={process},
	description={The process of identifying and resolving problems with assistive technology}
}

\newglossaryentry{tts}{
	name={tts},
	type={process},
	description={Text-to-Speech - technology that converts written text into spoken audio output}
}

\newglossaryentry{typography}{
	name={typography},
	type={concept},
	description={The art and technique of arranging type to make written language legible and appealing}
}

\newglossaryentry{usb}{
	name={usb},
	type={standard},
	description={Universal Serial Bus - standard for connecting devices to computers}
}

\newglossaryentry{videomagnifier}{
	name={video magnifier},
	type={device},
	description={A device that uses a camera and display to magnify text and objects for users with low vision}
}

\newglossaryentry{visualimpairment}{
	name={visual impairment},
	type={concept},
	description={A condition affecting vision that requires adaptive techniques or assistive technology for accessing information}
}

\newglossaryentry{wcag}{
	name={WCAG},
	type={standard},
	description={Web Content Accessibility Guidelines - international standards for making web content accessible to people with disabilities}
}

\newglossaryentry{webaccessibility}{
	name={web accessibility},
	type={process},
	description={The practice of making websites usable by people with disabilities}
}

\newglossaryentry{webaim}{
	name={WebAIM},
	type={standard},
	description={WebAIM (Web Accessibility In Mind)—organization providing research (e.g., screen reader surveys), training, and tools advancing inclusive web design.}
}

\newglossaryentry{xml}{
	name={xml},
	type={format},
	description={eXtensible Markup Language - markup language for encoding documents}
}

\newglossaryentry{zoom}{
	name={zoom},
	type={process},
	description={Magnification of visual content (OS-level or application) to increase perceivability; must preserve reflow, contrast, and avoid horizontal scrolling. Distinct from conferencing platform “Zoom.”}
}

% ================== Newly Added Entries (OCR, Gaming, Braille, Devices) ==================

\newglossaryentry{layoutanalysis}{
	name={layout analysis},
	type={process},
	description={Computational segmentation of a page or document into regions (text blocks, columns, tables, figures, formulas) to restore logical reading order and support accessible tagging}
}

\newglossaryentry{icr}{
	name={ICR},
	type={process},
	description={Intelligent Character Recognition; handwriting-oriented extension of OCR that applies machine learning to variable glyph shapes}
}

\newglossaryentry{binarization}{
	name={binarization},
	type={process},
	description={Image pre-processing step converting grayscale or color images to a two-tone (binary) form to improve OCR accuracy under noise or low contrast}
}

\newglossaryentry{semantictagging}{
	name={semantic tagging},
	type={process},
	description={Assignment of structural roles (headings, lists, tables, figures, math) enabling assistive technologies to provide navigable, meaningful output}
}

\newglossaryentry{pdfua}{
	name={PDF/UA},
	type={standard},
	description={ISO 14289 accessibility standard for PDF requiring a tagged structure tree, correct reading order, alt text for non-text content, and valid semantics}
}

\newglossaryentry{postocrremediation}{
	name={post-OCR remediation},
	type={process},
	description={Human or semi-automated correction and structural enhancement of raw OCR output (adding headings, tables, MathML, alt text) to meet accessibility standards}
}

\newglossaryentry{bornaccessible}{
	name={born accessible},
	type={process},
	description={Content authored with proper semantics, structure, and alternative text from inception, minimizing downstream remediation effort}
}

\newglossaryentry{confidencescore}{
	name={confidence score},
	type={process},
	description={A probability-like metric emitted by recognition or AI models (OCR, math OCR, ASR) indicating estimated correctness to prioritize human review}
}

\newglossaryentry{readingorder}{
	name={reading order},
	type={process},
	description={The logical sequence in which content should be presented by assistive technology, often reconstructed via layout analysis from physical page coordinates}
}

\newglossaryentry{spatialaudio}{
	name={spatial (binaural) audio},
	type={modality},
	description={Audio rendering technique (often using HRTFs) that simulates 3D positioning of sound sources to convey direction and distance non-visually}
}

\newglossaryentry{informationalaudiocue}{
	name={informational audio cue},
	type={modality},
	description={Short distinct sound conveying interface or game state (focus change, objective proximity, error, confirmation)}
}

\newglossaryentry{selfvoicinggame}{
	name={self-voicing game},
	type={process},
	description={A game that includes its own text-to-speech/narration layer rather than relying solely on an external screen reader}
}

\newglossaryentry{hapticchannel}{
	name={haptic channel},
	type={modality},
	description={Vibration or tactile output pathway delivering directional, timing, or urgency information to supplement or replace visual cues}
}

\newglossaryentry{audioonlygame}{
	name={audio-only game},
	type={modality},
	description={A game designed to be playable primarily or exclusively through auditory (and sometimes haptic) feedback}
}

\newglossaryentry{hybridcasual}{
	name={hybrid casual},
	type={process},
	description={Game design style combining low initial complexity with layered depth; often reduces sensory and cognitive barriers}
}

\newglossaryentry{adaptivedifficulty}{
	name={adaptive difficulty},
	type={process},
	description={Dynamic adjustment of challenge parameters based on real-time player performance metrics to maintain accessibility and engagement}
}

\newglossaryentry{accessiblemod}{
	name={accessible mod},
	type={process},
	description={Community or developer modification that adds accessibility features (audio cues, narration, remapping, contrast adjustments) to an existing title}
}

\newglossaryentry{tactileadaptation}{
	name={tactile adaptation},
	type={process},
	description={Physical modification (e.g., Braille labels, textured overlays, raised borders) enabling non-visual play of mainstream tabletop or device interfaces}
}

\newglossaryentry{ueb}{
	name={UEB},
	type={standard},
	description={Unified English Braille—standardized literary and technical braille code improving international consistency}
}

\newglossaryentry{brf}{
	name={BRF},
	type={format},
	description={Braille Ready Format: plain text encoding of braille cells for distribution to embossers and refreshable displays}
}

\newglossaryentry{pef}{
	name={PEF},
	type={format},
	description={Portable Embosser Format: XML-based structured braille format preserving pagination and formatting across embossers}
}

\newglossaryentry{translationtable}{
	name={translation table},
	type={process},
	description={Rule set mapping print characters, contractions, and context-sensitive patterns to braille cells for a given language or code}
}

\newglossaryentry{backtranslation}{
	name={back-translation},
	type={process},
	description={Process of converting braille back into print for quality assurance or instructional review}
}

\newglossaryentry{contractedbraille}{
	name={contracted (Grade 2) braille},
	type={process},
	description={Braille employing standardized contractions to shorten common words or letter groups for reading efficiency}
}

\newglossaryentry{uebmathtechnical}{
	name={UEB math/technical},
	type={standard},
	description={UEB extensions supporting technical and mathematical notation when Nemeth Code is not used}
}

\newglossaryentry{sixkeyinput}{
	name={six-key input},
	type={process},
	description={Direct braille entry method using six (or eight) keyboard keys corresponding to braille dots 1–6 (±7–8)}
}

\newglossaryentry{multilinebrailledisplay}{
	name={multi-line braille display},
	type={device},
	description={Braille hardware presenting multiple lines (or a large matrix) of cells, enabling page-like spatial navigation}
}

\newglossaryentry{tactilegraphicsdisplay}{
	name={tactile graphics display},
	type={device},
	description={High-density refreshable pin matrix rendering diagrams, charts, and spatial layouts for non-visual exploration}
}

\newglossaryentry{ebrf}{
	name={eBRF},
	type={format},
	description={Enhanced electronic Braille Ready Format variant supporting additional structural or tactile graphic metadata}
}

\newglossaryentry{annotationworkflow}{
	name={annotation workflow},
	type={process},
	description={Structured sequence for marking, labeling, or commenting on braille/tactile content to support study and assessment}
}

\newglossaryentry{opencreativecanvas}{
	name={open creative canvas},
	type={process},
	description={Free-form tactile or hybrid environment for drawing, sketching, and exploratory spatial prototyping}
}

\newglossaryentry{realtimecollaboration}{
	name={real-time collaboration},
	type={process},
	description={Simultaneous shared editing or tactile diagram updates across multiple devices or users}
}

\newglossaryentry{hybriddisplayarchitecture}{
	name={hybrid display architecture},
	type={concept},
	description={Combined tactile graphics surface plus a dedicated braille text line enabling simultaneous spatial and linear reading}
}

\newglossaryentry{pindensity}{
	name={pin density},
	type={concept},
	description={Number of refreshable pins per unit area on a tactile display affecting resolution and detail granularity}
}

\newglossaryentry{brailleaccessworkspace}{
	name={Braille Access Workspace},
	type={process},
	description={An integrated environment (iOS context) providing braille-first document editing, app launching, math entry, and command mapping}
}

\newglossaryentry{accessibilityactivity}{
	name={Activity (accessibility profile)},
	type={process},
	description={Configurable per-app profile controlling verbosity, speech rate, rotor/command availability for optimized braille or speech workflows}
}

\newglossaryentry{commandremapping}{
	name={command remapping},
	type={process},
	description={User-driven reassignment of gestures, chords, or keys to accessibility or system actions for efficiency and personalization}
}

\newglossaryentry{nemethentry}{
	name={Nemeth entry},
	type={process},
	description={Direct inline entry of Nemeth mathematical braille code within an editor or workspace without mode switching}
}

\newglossaryentry{settingsexport}{
	name={settings export},
	type={process},
	description={Mechanism for packaging and sharing accessibility preferences (verbosity, gestures, braille tables) across devices or users}
}

\newglossaryentry{contextsensitivegesture}{
	name={context-sensitive gesture},
	type={process},
	description={Input whose action adapts to the current mode or focus (e.g., text editing vs navigation) to reduce gesture overload}
}

\newglossaryentry{macrochaining}{
	name={macro chaining},
	type={process},
	description={Sequential execution of multiple mapped operations triggered by a single chord, gesture, or key binding}
}

\newglossaryentry{brailletableswitching}{
	name={braille table switching},
	type={process},
	description={On-the-fly change of active braille translation or contraction table to support multilingual or code-specific workflows}
}

% === AUTO-GENERATED GLOSSARY ENTRIES START (user requested: interpoint, 3D printing alias, filament, CCTV, wearable) ===

\newglossaryentry{interpoint}{
  name={interpoint (braille)},
  type={process},
  description={A braille embossing technique that produces braille on both sides of a sheet, aligning dots so they do not interfere; enables duplex (two‑sided) braille production to save paper and reduce bulk, but requires compatible embossers and careful translation/pagination to avoid dot collision.}
}

\newglossaryentry{3DPrinting}{
  alias=3dprinting,
  name={3D printing},
  type={process},
  description={Alias for \gls{3dprinting}: additive manufacturing processes (e.g., FDM, SLA) used to create tactile models and adaptive devices from digital designs; often used in accessibility contexts to produce manipulatives, tactile maps, and custom mounts.}
}

\newglossaryentry{filament}{
  name={filament (3D printing material)},
  type={concept},
  description={Thermoplastic feedstock (commonly PLA, ABS, PETG) used in FDM 3D printers; filament diameter (e.g., 1.75 mm) and material properties affect surface finish, tactile detail, strength, and durability of printed tactile learning materials.}
}

\newglossaryentry{cctv}{
  alias=videomagnifier,
  name={CCTV},
  type={device},
  description={Closed-Circuit Television: historically used term for video magnifier systems that project a camera-captured image to a display for magnification; in assistive contexts, synonymous with electronic video magnifiers (see \gls{videomagnifier}).}
}

\newglossaryentry{wearable}{
  name={wearable},
  type={device},
  description={Body‑worn computing devices (smartwatches, haptic belts, wearable cameras, smart glasses) that provide accessible cues (haptic, audio, spatial) or sensing for navigation, object recognition, and context‑aware assistance.}
}

% === AUTO-GENERATED GLOSSARY ENTRIES END
