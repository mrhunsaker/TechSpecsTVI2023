\chapter{Accessible Fonts}\label{appx66}
\begin{raggedright}
\textbf{Accessibility Note:} This appendix provides an overview and demonstration of accessible fonts, focusing on their strengths and weaknesses for readers with visual impairments or reading disabilities. The structure and content have been enhanced for clarity, navigation, and accessibility, including context for each font and sample.

Accessible typography is crucial for individuals with visual impairments, such as low vision or reading disabilities like dyslexia. Accessible fonts are designed to be easy to read by a diverse audience, including people with visual impairments. The use of accessible typefaces like Atkinson Hyperlegible and APHont can significantly improve the legibility and readability of text for people with visual impairments. These typefaces have features like increased letter spacing, bold outlines, higher contrast ratios, and wider characters, which make them easier to read. The Section 508 Standards \cite{Section508} and other regulations require sans-serif fonts in certain places, and typography choices have a huge impact on accessibility \cite{AccessiBe}. By using accessible typography, textual information can be made accessible to all users, irrespective of their abilities or disabilities \cite{ReadabilityGroup, HubSpot}.

A variety of accessible font options are available. The following sections present each font with a summary of its accessibility features, strengths, and weaknesses, followed by a demonstration of its readability \cite{EmptyBoxNote}.
\end{raggedright}


\section{Atkinson Hyperlegible Font}\label{sec:atkinson-hyperlegible}
\noindent
\textbf{Context:} Atkinson Hyperlegible is a proportional sans-serif font designed for maximum legibility for readers with low vision. Below are its strengths, weaknesses, and sample renderings.

\emph{Developed by \href{https://brailleinstitute.org/freefont}{the Braille Institute}}
\begin{raggedright}
\emph{Strengths:} Designed specifically for low vision, excellent character differentiation (e.g., distinct 'l', 'I', '1' and '0', 'O'). Very open apertures.

\emph{Weaknesses:} May appear wider than standard fonts, potentially leading to less text per line.

\subsection{Regular}\label{subsec:atkinson-regular}
\noindent\textbf{Sample (Regular):}
\FontSample{\atkinsonhyperlegiblefont}

\subsection{Italic}\label{subsec:atkinson-italic}
\noindent\textbf{Sample (Italic):}
\FontSample{{\atkinsonhyperlegiblefont\itshape}}
\end{raggedright}


\pagebreak
\section{Atkinson Hyperlegible Mono}\label{sec:atkinson-mono}
\noindent
\textbf{Context:} Atkinson Hyperlegible Mono is a monospaced variant, useful for coding and data tables where predictable character width is important.

\emph{Developed by \href{https://brailleinstitute.org/freefont}{the Braille Institute} (Monospaced variant)}
\begin{raggedright}
\emph{Strengths:} Retains the excellent character distinctiveness of its proportional counterpart. Monospaced nature can aid users who benefit from predictable character widths, especially for code or data.

\emph{Weaknesses:} Monospaced fonts generally take up more horizontal space, potentially reducing reading speed for some users in long prose.

\subsection{Regular}\label{subsec:mono-regular}
\noindent\textbf{Sample (Regular):}
\FontSample{\atkinsonmonofont}

\subsection{Italic}\label{subsec:mono-italic}
\noindent\textbf{Sample (Italic):}
\FontSample{{\atkinsonmonofont\itshape}}
\end{raggedright}


\pagebreak
\section{Atkinson Hyperlegible Next}\label{troubleAtkinsonNext}
\noindent
\textbf{Context:} Atkinson Hyperlegible Next is a modernized version of the original, with refinements for digital readability.

\emph{Developed by \href{https://brailleinstitute.org/freefont}{the Braille Institute} (Next variant)}
\begin{raggedright}
\emph{Strengths:} Similar accessibility features to the original Hyperlegible, with potential refinements for modern digital contexts.

\emph{Weaknesses:} Similar to the original Hyperlegible, its design choices for accessibility might make it less compact than other fonts.

\subsection{Regular}
\noindent\textbf{Sample (Regular):}
\FontSample{\atkinsonnextfont}

\subsection{Italic}
\noindent\textbf{Sample (Italic):}
\FontSample{{\atkinsonnextfont\itshape}}
\end{raggedright}


\pagebreak
\section{APHont}\label{trouble7}
\noindent
\textbf{Context:} APHont is a sans-serif font designed by the American Printing House for the Blind for large print and low vision accessibility.

\emph{Developed by the \href{https://www.aph.org/resources/large-print-guidelines/}{American Printing House for the Blind}, \href{https://www.aph.org/resources/large-print-guidelines/}{font download here}}
\begin{raggedright}
\emph{Strengths:} Explicitly designed for large print and low vision. Features clear, wide characters and good letter spacing.

\emph{Weaknesses:} May appear somewhat informal due to its design. Its wide characters consume more space, which can be an issue for page count or display on smaller screens.

\subsection{Regular}
\noindent\textbf{Sample (Regular):}
\FontSample{\aphontfont}

\subsection{Italic}
\noindent\textbf{Sample (Italic):}
\FontSample{{\aphontfont\itshape}}
\end{raggedright}


\pagebreak
\section{Comic Sans MS}\label{trouble9}
\noindent
\textbf{Context:} Comic Sans MS is an informal sans-serif font sometimes recommended for dyslexic readers, but its use is debated.

\begin{raggedright}
\emph{Strengths:} Often cited as beneficial for some dyslexic readers due to its informal, handwritten-like characters, which can reduce visual crowding and increase character differentiation.

\emph{Weaknesses:} Highly controversial and often considered unprofessional for formal documents. Its informal appearance can detract from the content's perceived seriousness. Not universally preferred for accessibility.

\subsection{Regular}
\noindent\textbf{Sample (Regular):}
\FontSample{\comicsansfont}

\subsection{Italic}
\noindent\textbf{Sample (Italic):}
\FontSample{{\comicsansfont\itshape}}
\end{raggedright}


\pagebreak
\hypertarget{trouble10}{}\section[JetBrains Mono]{JetBrains Mono \cite{JetBrainsMono}}\label{trouble10}
\noindent
\textbf{Context:} JetBrains Mono is a monospaced font designed for programming, with strong character differentiation and ligature support.

\begin{raggedright}
\emph{Strengths:} Excellent for code readability due to clear differentiation of similar characters (e.g., '0' and 'O', 'l', 'I', '1'). Good open apertures and designed for extended screen reading. Excellent ligature support for programming.

\emph{Weaknesses:} As a monospaced font, it's not ideal for long blocks of prose due to its uniform character width, which can make reading less fluid. Subscript characters might appear as empty boxes if the font isn't fully supported by your LaTeX setup for these symbols.

\subsection{Regular}
\noindent\textbf{Sample (Regular):}
\FontSample{\jetbrainsmonofont}

\subsection{Italic}
\noindent\textbf{Sample (Italic):}
\FontSample{{\jetbrainsmonofont\itshape}}
\end{raggedright}




\pagebreak
\section{OpenDyslexic}\label{sec:opendyslexic}
\noindent
\textbf{Context:} OpenDyslexic is a free, open-source font designed to address common reading errors for people with dyslexia.

\emph{Developed by Abelardo Gonzalez, designed for dyslexia. Free and open source: \url{https://opendyslexic.org/}}
\begin{raggedright}
\emph{Strengths:} Specifically designed to aid dyslexic readers with unique letterforms that prevent common reading errors (e.g., weighted bottoms to prevent letter reversal, distinct letter shapes).

\emph{Weaknesses:} Its highly distinctive letterforms can be distracting for non-dyslexic readers. Not everyone with dyslexia finds it beneficial, as accessibility needs vary greatly.

\subsection{Regular}
\noindent\textbf{Sample (Regular):}
\FontSample{\opendyslexicfont}

\subsection{Italic}
\noindent\textbf{Sample (Italic):}
\FontSample{{\opendyslexicfont\itshape}}
\end{raggedright}


\pagebreak
\section{Tiresias}\label{sec:tiresias}
\noindent
\textbf{Context:} Tiresias is a sans-serif font developed for visually impaired users, widely used in public information systems.

\emph{Designed for visually impaired users. Free for non-commercial use: \url{https://www.tiresias.org/fonts/}}
\begin{raggedright}
\emph{Strengths:} Developed with input from visually impaired individuals. Features include clear, unambiguous characters and generous spacing. Often used in broadcasting and public information systems.

\emph{Weaknesses:} Its design might feel somewhat dated or less aesthetically pleasing to some users compared to more modern sans-serif fonts.

\subsection{Regular}
\noindent\textbf{Sample (Regular):}
\FontSample{\tiresiasfont}

\subsection{Italic}
\noindent\textbf{Sample (Italic):}
\FontSample{{\tiresiasfont\itshape}}
\end{raggedright}


\pagebreak
\section{Verdana}\label{sec:verdana}
\noindent
\textbf{Context:} Verdana is a widely used sans-serif font designed for screen readability, pre-installed on many systems.

\emph{Widely recognized for readability on screens. Pre-installed on many systems.}
\begin{raggedright}
\emph{Strengths:} Designed by Microsoft for on-screen readability, featuring a large x-height, distinct characters, and generous letter spacing. Very common and generally well-supported.

\emph{Weaknesses:} Can appear somewhat wide and takes up more space than some other fonts, which might be a concern for print layouts or limited screen real estate.

\subsection{Regular}
\noindent\textbf{Sample (Regular):}
\FontSample{\verdanafont}

\subsection{Italic}
\noindent\textbf{Sample (Italic):}
\FontSample{{\verdanafont\itshape}}
\end{raggedright}




\pagebreak
\section{Arial / Arial Unicode MS}\label{sec:arial}
\noindent
\textbf{Context:} Arial is a neutral, sans-serif font commonly recommended for accessibility and widely available on most systems.

\emph{Commonly recommended for accessibility due to clarity and sans-serif design. Pre-installed on many systems.}
\begin{raggedright}
\emph{Strengths:} A widely available, neutral sans-serif font. Its familiarity and broad character set (especially Arial Unicode MS) make it a safe default for general accessibility.

\emph{Weaknesses:} While clear, some characters (like 'I', 'l', and '1') can be less distinct than in fonts specifically designed for accessibility. Its uniform appearance can make it less engaging for some readers.

\subsection{Regular}
\noindent\textbf{Sample (Regular):}
\FontSample{\arialfont}

\subsection{Italic}
\noindent\textbf{Sample (Italic):}
\FontSample{{\arialfont\itshape}}
\end{raggedright}


\pagebreak
\section{Calibri}\label{sec:calibri}
\noindent
\textbf{Context:} Calibri is a modern sans-serif font, the default in many Microsoft products, designed for digital readability.

\emph{Microsoft’s default sans-serif, designed for screen readability. Pre-installed on many systems.}
\begin{raggedright}
\emph{Strengths:} Modern, clean sans-serif with good readability, particularly on digital screens. Relatively open apertures and a friendly aesthetic.

\emph{Weaknesses:} Similar to Arial, its general-purpose design might mean that character differentiation for problematic pairs is not as optimized as in specialized accessible fonts.

\subsection{Regular}
\noindent\textbf{Sample (Regular):}
\FontSample{\calibrifont}

\subsection{Italic}
\noindent\textbf{Sample (Italic):}
\FontSample{{\calibrifont\itshape}}
\end{raggedright}


\pagebreak
\section{Century Gothic}\label{sec:centurygothic}
\noindent
\textbf{Context:} Century Gothic is a geometric sans-serif font, often recommended for clarity and used in Microsoft Office.

\emph{Recommended by some accessibility guidelines for its clarity. Common in Microsoft Office.}
\begin{raggedright}
\emph{Strengths:} A geometric sans-serif font with a clean, open design and good legibility. Its distinct letterforms can be helpful for some readers.

\emph{Weaknesses:} Its perfectly circular 'o' and generally wider characters can sometimes make it less space-efficient. The simple geometric shapes might not provide enough distinctiveness for some problematic character pairs.

\subsection{Regular}
\noindent\textbf{Sample (Regular):}
\FontSample{\centurygothicfont}

\subsection{Italic}
\noindent\textbf{Sample (Italic):}
\FontSample{{\centurygothicfont\itshape}}
\end{raggedright}

% Bibliography
\section{References}
\addcontentsline{toc}{section}{References}
\begin{thebibliography}{99}
\bibitem{Section508} Section508.gov. (n.d.). Understanding Accessible Fonts and Typography for Section 508 Compliance. \url{https://blog.hubspot.com/website/accessibility-fonts}. Retrieved January 12, 2024.
\bibitem{AccessiBe} accessiBe. (2023, May 14). How to Choose ADA-Compliant Fonts in 2024: A Complete Guide. accessiBe BLOG. \url{https://accessibe.com/blog/knowledgebase/ada-compliant-fonts}.
\bibitem{ReadabilityGroup} The Readability Group. (2020, August 14). A Guide to Understanding What Makes a Typeface Accessible and How to Make Informed Decisions. Medium. \url{https://medium.com/the-readability-group/a-guide-to-understanding-what-makes-a-typeface-accessible-and-how-to-make-informed-decisions-9e5c0b9040a0}. Retrieved January 12, 2024.
\bibitem{HubSpot} HubSpot. (n.d.). Accessibility Fonts: How to Choose the Right Typeface for Your Website. \url{https://blog.hubspot.com/website/accessibility-fonts}. Retrieved January 12, 2024.
\bibitem{EmptyBoxNote} Empty box characters mean a particular font does not contain that type of character. This is seen for all of the fonts below except JetBrains Mono with subscript characters.
\bibitem{JetBrainsMono} JetBrains. Font created for monospace/coding needs. \url{https://www.jetbrains.com/lp/mono/}.
\end{thebibliography}
