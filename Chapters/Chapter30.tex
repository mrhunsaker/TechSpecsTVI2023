\chapter{iOS 26 VoiceOver and Braille Enhancements}

iOS 26 introduces significant upgrades over iOS 18 for VoiceOver and Braille Screen Input users, focusing on expanded workspace capabilities, greater customization, and improved efficiency for both on-screen and physical braille devices\supercite{nelowvision2025,myvision2025,applevisVO2024}.

\section{~~Overview of New and Enhanced Features}

\subsection{High-Level Feature Comparison}
Table~\ref{tab:feature-comparison} summarizes the key differences between iOS 18 and iOS 26 for VoiceOver and Braille Screen Input.

\footnotesize
\begin{longtblr}[
	caption = {New and Enhanced VoiceOver \& Braille Features in iOS 26 vs iOS 18},
	label = {tab:feature-comparison},
	note = {Summary of major accessibility feature enhancements between iOS 18 and iOS 26 for VoiceOver and braille users.},
]{
	colspec = {X[l] X[l] X[l] X[l]},
	rowhead = 1,
	row{1} = {font=\normalfont},
	hlines,
	stretch = 1.5
}
Feature               & iOS 18           & iOS 26              & Enhancement Description                                                      \\
Braille Access / Workspace     & No dedicated workspace    & Integrated braille workspace & Note-taking, Nemeth math support, open BRF/TXT, app launching\supercite{myvision2025} \\
Expanded Gestures              & Basic navigation/entry    & Context-sensitive gestures   & Adds more navigation and control gestures\supercite{applevisBSI2024}                  \\
Customizable Activities        & Limited                   & App-specific Activities      & Adjust verbosity, speech rate, hide rotor actions per context                         \\
App/Action Remapping           & Partially locked gestures & Fully remappable commands    & Map nearly all gestures, keys to system/app actions\supercite{hks2025}                \\
Quick Launch (Braille Display) & Not available             & Direct app/system launch     & Open apps/books from display keys                                                     \\
Nemeth Code/Math               & Unsupported               & On-device Nemeth entry       & Type math directly\supercite{appleSupportBSI2025}                                     \\
Braille Output Settings        & Limited                   & Enhanced output control      & On-the-fly table switching, context-sensitive reading                                 \\
Share Accessibility Settings   & Not available             & AirDrop/QR code transfer     & Share VoiceOver/Braille profiles                                                      \\
Hide Rotor Actions             & Fixed set                 & Customizable rotor options   & Remove unused actions                                                                 \\
Voice Control + VoiceOver      & Not simultaneous          & Fully concurrent             & Use both without conflict\supercite{applevisVO2024}                                   \\
Personal Voice Setup           & 30 phrases                & 10 phrases + iCloud sync     & Faster voice creation                                                                 \\
Settings Backup/Export         & Not available             & Backup to file               & Restore or share accessibility settings                                               \\
\end{longtblr}
\normalsize

\section{~~Specific Commands and Gestures}

Table~\ref{tab:commands} lists specific on-screen and braille hardware commands in iOS 26, including newly added or enhanced gestures from iOS 18.

\footnotesize
\begin{longtblr}[
	caption = {VoiceOver \& Braille Screen Input Commands (iOS 26)},
	label = {tab:commands},
	note = {Representative set of core and newly enhanced braille and on-screen commands in iOS 26.},
]{
	colspec = {X[l] X[l] X[l]},
	rowhead = 1,
	row{1} = {font=\normalfont},
	hlines,
	stretch = 1.5
}
Command / Gesture & Type                           & Action / Notes                                                           \\
Enter Braille Screen Input & On-screen (double-tap top+bottom edges) & Quick workspace entry\supercite{appleSupportBSI2025}                              \\
Cycle spelling suggestions & Swipe up/down (1 finger)                & Browse autocorrect options                                                        \\
Move to new line           & Swipe right (2 fingers)                 & Insert line break                                                                 \\
Cycle input/command mode   & Swipe left/right (3 fingers)            & Switch modes inside workspace                                                     \\
Reverse dot positions      & Settings toggle                         & Swap 6-dot layout                                                                 \\
Switch braille table       & Swipe up (2 fingers)                    & Change language/table instantly                                                   \\
Pan left/right on display  & Dots 2+Space / 5+Space                  & Move braille content window\supercite{appleSupportCommands2024}                   \\
Select All                 & Dots 2-3-5-6+Space                      & Highlight all text                                                                \\
Cut / Copy / Paste         & Braille chords                          & Standard text actions (cut: 1-3-4-6+Space, copy: 1-4+Space, paste: 1-2-3-6+Space) \\
Undo / Redo                & Dots 1-3-5-6+Space / 2-3-4-6+Space      & Editing convenience                                                               \\
Quick Launch app           & Braille display shortcut                & Open app/system function instantly                                                \\
Hide rotor action          & Settings                                & Remove unused rotor entries                                                       \\
Reader/Live Text           & Safari/Share sheet button               & Activate large print/speech for page\supercite{nelowvision2025}                   \\
\end{longtblr}
\normalsize

\section{~~Braille Access / Workspace in Detail}

The \emph{Braille Access Workspace} in iOS 26 is a dedicated, full-screen environment for braille-based productivity\supercite{convergeaccessibility2025}. It integrates:

\begin{itemize}
	\item \textbf{Note-taking and editing}: Create/edit BRF or TXT documents with full cursor routing and text navigation.
	\item \textbf{Nemeth math entry}: Inline math creation and review.
	\item \textbf{Quick Launch}: Open apps or files without leaving the workspace.
	\item \textbf{Customizable Commands}: Almost any VoiceOver or text action can be mapped to a key/gesture.
	\item \textbf{File Handling}: Save, export, or share documents directly.
	\item \textbf{Settings Portability}: Export/import workspace and braille settings between devices.
\end{itemize}

\footnotesize
\begin{longtblr}[
	caption = {Common Braille Access / Workspace Commands},
	label = {tab:workspace},
	note = {Frequent productivity and navigation commands within the Braille Access workspace environment.},
]{
	colspec = {X[l] X[l] X[l]},
	rowhead = 1,
	row{1} = {font=\normalfont},
	hlines,
	stretch = 1.5
}
Command          & Gesture / Key      & Function                     \\
Enter Workspace           & Double-tap top+bottom edges & Launch braille environment            \\
Exit Workspace            & Two-finger scrub            & Return to prior context               \\
Switch entry/command mode & 3-finger swipe left/right   & Toggle between modes                  \\
Translate contracted text & 2-finger swipe down         & Perform immediate braille translation \\
Search / Find             & Dots 1-2-4+Space            & Search within a document              \\
Heading navigation        & Assigned chord              & Jump between headings in document/web \\
Dictation                 & Dots 1+5+6+Space            & Start voice dictation in field        \\
\end{longtblr}
\normalsize

\noindent
The workspace also supports macro chaining, where one chord can perform multiple actions (e.g., save, exit, and open the Files app), and adaptive gestures that behave differently depending on context.

\section{~~Conclusion}

Across three major domains—feature set, commands, and workspace integration—iOS 26 delivers a more powerful and customizable braille and VoiceOver experience, closing gaps left in iOS 18.
