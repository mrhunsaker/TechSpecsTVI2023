\chapter{Comprehensive Accessibility Audit of Office Suites for Screen Reader Accessibility}
\label{chap:office-suite-audit}

This chapter provides a comprehensive guide to conducting a full accessibility audit on Microsoft Office, LibreOffice, and Google Suite tools (Docs, Sheets, etc.), with a specific focus on screen reader accessibility. It covers the fundamental principles and relevant standards, built-in accessibility features, commercial and open-source audit tools across macOS, Windows, and Linux, and detailed strategies for remediating common accessibility failures.

\section{Fundamental Principles and Relevant Accessibility Standards}
\label{sec:office-principles-standards}

A thorough accessibility audit begins with a strong understanding of the underlying principles and standards that govern digital accessibility. For office documents and web-based productivity tools, the following are paramount:

\subsection{Web Content Accessibility Guidelines (WCAG)}
WCAG\footnote{\url{https://www.w3.org/WAI/standards-guidelines/wcag/}} are developed by the World Wide Web Consortium (W3C) and provide a widely accepted international standard for web content accessibility. While primarily for web, its principles are broadly applied to digital documents and software. WCAG is structured around four core principles (POUR):
\begin{itemize}
    \item \textbf{Perceivable}: Information and user interface components must be presentable to users in ways they can perceive. This includes providing text alternatives for non-text content (images, videos), ensuring content can be presented in different ways (e.g., simpler layout) without losing information, and making it easier for users to see and hear content (e.g., sufficient color contrast).
    \item \textbf{Operable}: User interface components and navigation must be operable. This covers keyboard accessibility for all functionality, providing users enough time to read and use content, avoiding content that causes seizures, and ensuring clear and consistent navigation.
    \item \textbf{Understandable}: Information and the operation of user interface must be understandable. This involves making text readable and understandable, making web pages appear and operate in predictable ways, and helping users avoid and correct mistakes.
    \item \textbf{Robust}: Content must be robust enough that it can be interpreted reliably by a wide variety of user agents, including assistive technologies. This means maximizing compatibility with current and future user agents.
\end{itemize}
Specific WCAG success criteria relevant to documents and screen reader accessibility include:
\begin{itemize}
    \item \textbf{1.1.1 Non-text Content}: All non-text content (images, charts, graphs) needs a text alternative (e.g., alt text).
    \item \textbf{1.3.1 Info and Relationships}: Information, structure, and relationships conveyed through presentation can be programmatically determined or are available in text. This includes proper use of headings, lists, and tables.
    \item \textbf{1.4.3 Contrast (Minimum)}: Visual presentation of text and images of text has a contrast ratio of at least 4.5:1.
    \item \textbf{2.1.1 Keyboard}: All functionality is available via a keyboard interface.
    \item \textbf{2.4.4 Link Purpose (In Context)}: The purpose of each link can be determined from the link text alone or from the link text together with its programmatically determined link context.
    \item \textbf{4.1.2 Name, Role, Value}: For all user interface components (including form elements, links, and components generated by scripts), the name and role can be programmatically determined; states, properties, and values that can be set by the user can be programmatically set; and notification of changes to these items is available to user agents, including assistive technologies. This is where ARIA becomes particularly relevant for web-based applications.
\end{itemize}

\subsection{Section 508 of the Rehabilitation Act}
Section 508\footnote{\url{https://www.section508.gov/}} is a U.S. federal law that requires federal agencies to make their electronic and information technology (EIT) accessible to people with disabilities. The refreshed Section 508 standards (2017) harmonize with WCAG 2.0 Level AA, making WCAG the de facto standard for federal agencies. It applies directly to documents, spreadsheets, and presentations used by federal agencies, emphasizing compatibility with assistive technologies like screen readers.

\subsection{Accessible Rich Internet Applications (ARIA)}
ARIA\footnote{\url{https://www.w3.org/WAI/ARIA/}} is a set of attributes that can be added to HTML elements to improve the accessibility of web content, particularly dynamic content and custom user interface components that are not natively accessible. While ARIA is primarily for web applications, its concepts are crucial for understanding how screen readers interpret complex web-based interfaces like Google Workspace, which often rely on advanced JavaScript and custom controls. ARIA roles, states, and properties provide semantic meaning to elements that screen readers can then convey to users.

\section{Built-in Accessibility Features and Checkers}
\label{sec:office-built-in-checkers}

Many modern office suites include built-in tools to assist users in creating accessible content. These are often the first line of defense in an accessibility audit.

\subsection{Microsoft Office (Word, Excel, PowerPoint)}
Microsoft Office applications include a robust \textbf{Accessibility Checker}\footnote{\url{https://support.microsoft.com/en-us/office/improve-accessibility-with-the-accessibility-checker-a78ab7ee-b363-4f7e-874d-b6a444954c6a}} that can identify common accessibility issues.
\begin{itemize}
    \item \textbf{Access}: Found under the "Review" tab in Word, Excel, PowerPoint, Outlook, and OneNote. Select "Check Accessibility."
    \item \textbf{Issues Detected}: It identifies errors (content difficult or impossible for people with disabilities to access), warnings (content that might be difficult to understand), and tips (content that could be presented better for accessibility). Specific checks include:
        \begin{itemize}
            \item Missing alternative text for images, charts, and shapes.
            \item Insufficient color contrast.
            \item Simple table structures (e.g., avoiding merged/split cells, ensuring clear headers).
            \item Meaningful sheet names in Excel.
            \item Logical reading order in PowerPoint (via the "Reading Order Pane" in the Selection Pane).
            \item Use of proper heading styles in Word.
            \item Accessibility of multimedia (captions).
        \end{itemize}
    \item \textbf{Remediation Guidance}: The checker often provides "how-to-fix" recommendations directly within the interface.
    \item \textbf{Operating Systems}: Available on Windows and macOS versions of Microsoft Office.
\end{itemize}

\subsection{Google Workspace (Docs, Sheets, Slides)}
Google Workspace tools are web-based and are designed with screen reader compatibility in mind. While they emphasize best practices and collaboration with screen readers, they also offer some checking capabilities.
\begin{itemize}
    \item \textbf{Built-in Optimization}: Google Docs, Sheets, and Slides are optimized to work with popular screen readers (e.g., ChromeVox, NVDA, JAWS, VoiceOver). Features include extensive keyboard shortcuts\footnote{\url{https://support.google.com/docs/answer/179738}}, braille display support, and voice typing.
    \item \textbf{Accessibility Checker for Sheets Add-on}\footnote{\url{https://workspace.google.com/marketplace/app/accessibility_checker_for_sheets/544321949511}}: A Google Workspace Marketplace add-on specifically for Google Sheets. This GUI tool checks for common issues within spreadsheets.
    \item \textbf{General Best Practices}: Google's help documentation emphasizes creating accessible documents through proper use of:
        \begin{itemize}
            \item Headings and styles (Docs, Slides).
            \item Alternative text for images\footnote{\url{https://support.google.com/docs/answer/6248388}}.
            \item Accessible fonts and sufficient color contrast.
            \item Descriptive link text.
        \end{itemize}
    \item \textbf{Operating Systems}: Web-based, so compatible across all OS with a modern web browser (Windows, macOS, Linux).
\end{itemize}

\subsection{LibreOffice (Writer, Calc, Impress)}
LibreOffice, a free and open-source office suite, also offers accessibility features, though its built-in checking capabilities are less comprehensive than Microsoft Office.
\begin{itemize}
    \item \textbf{General Accessibility}: LibreOffice supports keyboard navigation\footnote{\url{https://help.libreoffice.org/latest/en-US/text/sbasic/shared/01040300.html}}, uses standard accessibility APIs where available (e.g., AT-SPI on Linux, MSAA/UIA on Windows), and allows for the use of styles and alt text.
    \item \textbf{AccessODF Extension}\footnote{\url{https://extensions.libreoffice.org/en/extensions/show/accessodf}}: An extension for LibreOffice Writer (GUI, Java-based). It provides an "Accessibility Check" feature that can detect issues like insufficient color contrast, missing language identification, and improper use of headings (e.g., bolding text instead of applying heading styles). It also offers repair suggestions.
    \item \textbf{Linux Challenges}: A significant consideration for LibreOffice on Linux is the fragmentation and occasional instability of the Linux accessibility stack (GTK/QT updates, lack of a universal accessibility API). This can sometimes lead to screen reader compatibility issues that are beyond the control of the LibreOffice application itself.
    \item \textbf{Operating Systems}: Available on Windows, macOS, and Linux.
\end{itemize}

\section{External Accessibility Audit Tools}
\label{sec:office-external-tools}

Beyond built-in checkers, a variety of commercial and open-source tools can provide more in-depth analysis for both desktop and web-based office suite content.

\subsection{Tools for Desktop Applications (Microsoft Office, LibreOffice)}

For auditing the underlying accessibility of desktop applications and the documents they produce, specialized tools are often required.

\subsubsection{Commercial Tools}
\begin{itemize}
    \item \textbf{Accessibility Insights for Windows}\footnote{\url{https://accessibilityinsights.io/docs/windows/overview/}}:
        \begin{itemize}
            \item \textbf{Type}: GUI desktop application.
            \item \textbf{OS}: Windows.
            \item \textbf{Detected Issues}: Uses Windows UI Automation (UIA) to inspect applications. Offers "Live Inspect" to view UIA properties, "FastPass" (automated checks for 60+ common accessibility requirements, visual helper for keyboard tab stops, focus traps, and logical order), and a "Color Contrast Analyzer." It helps identify missing names/labels, incorrect roles, keyboard navigation issues, and insufficient contrast.
            \item \textbf{Usage}: Installs as a standard Windows application. Point it at the application or document you wish to inspect.
        \end{itemize}
    \item \textbf{Apple's Accessibility Inspector (Xcode)}\footnote{\url{https://developer.apple.com/library/archive/documentation/Accessibility/Conceptual/AccessibilityMacOSX/OSXAXTestingApps.html}}:
        \begin{itemize}
            \item \textbf{Type}: GUI desktop application, part of Xcode (Apple's IDE).
            \item \textbf{OS}: macOS.
            \item \textbf{Detected Issues}: Primarily for developers testing macOS and iOS applications. It can inspect UI elements for accessibility attributes (labels, hints, values, traits), color contrast, and dynamic type adjustments. It can simulate VoiceOver interactions. While not directly for auditing existing Office documents in a user context, it's invaluable for developers ensuring the accessibility of the *application itself* or for inspecting how an accessible document's elements are exposed to macOS accessibility APIs.
            \item \textbf{Usage}: Launch from Xcode. Select the target application or window.
        \end{itemize}
\end{itemize}

\subsubsection{Open-Source Tools}
\begin{itemize}
    \item \textbf{Color Contrast Analyzer (CCA)}\footnote{\url{https://www.tpgi.com/color-contrast-checker/}}:
        \begin{itemize}
            \item \textbf{Type}: GUI desktop application.
            \item \textbf{OS}: Windows, macOS.
            \item \textbf{Detected Issues}: Focuses specifically on color contrast ratios between foreground and background colors, ensuring compliance with WCAG 2.x guidelines. While not a screen reader specific issue, poor contrast significantly impacts users with low vision, who often use screen readers for text-to-speech.
            \item \textbf{Usage}: Eyedropper tool to select colors from any part of the screen.
        \end{itemize}
\end{itemize}

\subsection{Tools for Web-based Applications (Google Workspace)}

Given that Google Workspace operates in a web browser, standard web accessibility audit tools are highly applicable.

\subsubsection{Commercial Tools}
\begin{itemize}
    \item \textbf{Deque axe DevTools}\footnote{\url{https://www.deque.com/axe/}}:
        \begin{itemize}
            \item \textbf{Type}: Browser Extension (GUI) and Command-Line Interface (CLI - `axe-core` library).
            \item \textbf{OS}: Cross-platform (runs in Chrome, Firefox, Edge browsers).
            \item \textbf{Detected Issues}: Widely regarded as a leading accessibility engine. Detects a wide range of WCAG 2.x Level A and AA violations related to screen reader accessibility, including missing alt text, invalid ARIA usage, insufficient color contrast, keyboard accessibility issues, improper heading structure, and missing form labels. Provides detailed explanations and links to remediation guidance.
            \item \textbf{Usage}: Browser extension integrates into developer tools. CLI version can be integrated into automated testing pipelines.
        \end{itemize}
    \item \textbf{ARC Toolkit (TPGi)}\footnote{\url{https://www.tpgi.com/arc-toolkit/}}:
        \begin{itemize}
            \item \textbf{Type}: Browser Extension (GUI).
            \item \textbf{OS}: Cross-platform (Chrome).
            \item \textbf{Detected Issues}: Checks for WCAG 2.1 Level A/AA conformance, including code-level issues, color contrast, and common screen reader barriers. Provides detailed insights and remediation techniques.
            \item \textbf{Usage}: Installs as a browser extension, activated from the browser toolbar or developer tools.
        \end{itemize}
\end{itemize}

\subsubsection{Open-Source Tools}
\begin{itemize}
    \item \textbf{Lighthouse (Google)}\footnote{\url{https://developers.google.com/web/tools/lighthouse/}}:
        \begin{itemize}
            \item \textbf{Type}: Built into Chrome DevTools (GUI) and CLI.
            \item \textbf{OS}: Cross-platform (runs within Chrome browser).
            \item \textbf{Detected Issues}: Provides an accessibility score and flags issues based on WCAG principles and ARIA best practices. Checks for alt text, discernible names for links/buttons, proper heading structure, language attributes, table headers, video captions, unique IDs, and contrast.
            \item \textbf{Usage}: Open Chrome DevTools, navigate to the "Lighthouse" tab, and generate a report. CLI version for automated testing.
        \end{itemize}
    \item \textbf{WAVE (WebAIM)}\footnote{\url{https://wave.webaim.org/}}:
        \begin{itemize}
            \item \textbf{Type}: Browser Extension (GUI) and Online Tool.
            \item \textbf{OS}: Cross-platform (Chrome, Firefox, Edge browser extensions; online tool for any browser).
            \item \textbf{Detected Issues}: Provides visual feedback by injecting icons and indicators directly into the web page. Identifies errors, contrast errors, alerts, features, structural elements, and ARIA usage. Maps issues to WCAG 2.x and Section 508. Can detect issues in elements that might be hidden or complex for screen readers.
            \item \textbf{Usage}: Install the browser extension or paste a URL into the online tool.
        \end{itemize}
    \item \textbf{Pa11y}\footnote{\url{https://pa11y.org/}}:
        \begin{itemize}
            \item \textbf{Type}: CLI tool, also has dashboards and CI/CD integrations.
            \item \textbf{OS}: Cross-platform (Node.js).
            \item \textbf{Detected Issues}: Automates web accessibility testing. Uses HTML CodeSniffer and Axe to detect WCAG issues. Ideal for integrating into Continuous Integration/Continuous Deployment (CI/CD) pipelines to catch issues early.
            \item \textbf{Usage}: Run from the command line, specifying URLs to audit. Outputs in various formats (CLI, CSV, HTML, JSON).
        \end{itemize}
    \item \textbf{ANDI (Accessible Name & Description Inspector)}\footnote{\url{https://www.ssa.gov/accessibility/andi/help/howtouse.html}}:
        \begin{itemize}
            \item \textbf{Type}: Browser Bookmarklet (GUI overlay).
            \item \textbf{OS}: Cross-platform (works in any modern browser).
            \item \textbf{Detected Issues}: Provides a visual overlay that highlights accessibility issues and information. Useful for manual inspection, revealing accessible names and descriptions of elements, checking heading hierarchy, links/buttons, images (alt text), contrast, and keyboard accessibility (tab key). Offers practical suggestions.
            \item \textbf{Usage}: Drag a bookmarklet to your browser's toolbar, then click it on any web page.
        \end{itemize}
    \item \textbf{IBM Equal Access Accessibility Checker}\footnote{\url{https://www.ibm.com/able/toolkit/develop/checker/}}:
        \begin{itemize}
            \item \textbf{Type}: Browser Extension (GUI) and Node.js package (CLI).
            \item \textbf{OS}: Cross-platform (Chrome, Firefox browser extensions; Node.js for CLI).
            \item \textbf{Detected Issues}: Uses IBM's robust accessibility rule engine to detect WCAG violations, "needs review" items, and recommendations. Features include checking ARIA roles, visual keyboard tab order, and live updates as you navigate the page.
            \item \textbf{Usage}: Install as a browser extension, or use the CLI for automated testing.
        \end{itemize}
\end{itemize}

\subsection{Manual Testing Tools (Screen Readers and Automation Drivers)}

Automated tools catch many issues, but manual testing with screen readers is indispensable for evaluating the user experience, logical flow, and complex interactions that automated tools often miss.

\begin{itemize}
    \item \textbf{Screen Readers (Assistive Technologies)}:
        \begin{itemize}
            \item \textbf{JAWS (Job Access With Speech)}\footnote{\url{https://www.freedomscientific.com/products/software/jaws/}}: Commercial, Windows. A widely used and powerful screen reader.
            \item \textbf{NVDA (NonVisual Desktop Access)}\footnote{\url{https://www.nvaccess.org/}}: Free, open-source, Windows. Excellent for comprehensive testing and very popular.
            \item \textbf{VoiceOver}\footnote{\url{https://support.apple.com/en-us/HT206176}}: Built-in to macOS and iOS. Essential for testing accessibility on Apple platforms.
            \item \textbf{Orca}\footnote{\url{https://help.gnome.org/users/orca/stable/}}: Free, open-source, Linux (primarily for GNOME desktop environment).
            \item \textbf{ChromeVox}\footnote{\url{https://support.google.com/chromebooks/answer/7031755}}: Built-in to Chrome OS, also available as a Chrome browser extension for Windows/macOS. Useful for testing Google Workspace.
            \item \textbf{Usage}: Install and learn the basic commands. Use them to navigate and interact with documents and web applications as a screen reader user would, listening for clarity, context, and logical flow.
        \end{itemize}
    \item \textbf{Screen Reader Automation Drivers}:
        \begin{itemize}
            \item \textbf{Guidepup}\footnote{\url{https://www.guidepup.dev/}}:
                \begin{itemize}
                    \item \textbf{Type}: JavaScript API (CLI/scriptable).
                    \item \textbf{OS}: macOS (VoiceOver), Windows (NVDA).
                    \item \textbf{Usage}: Allows developers to write automated tests that control screen readers, verifying announcements, focus changes, and user flows. This is crucial for integrating screen reader testing into CI/CD.
                \end{itemize}
        \end{itemize}
\end{itemize}

\section{Common Screen Reader Accessibility Failures and Remediation}
\label{sec:office-remediation}

This section outlines prevalent accessibility issues found in office documents and their effective remediation using native application features and best practices.

\subsection{Missing or Incorrect Headings}
\begin{itemize}
    \item \textbf{Problem}: Screen readers rely on proper heading structures ($\texttt{H1}, \texttt{H2}, \texttt{H3}$, etc.) to provide navigation and outline the document's hierarchy. Using bold text instead of actual heading styles, or skipping heading levels (e.g., $\texttt{H1}$ directly to $\texttt{H3}$), disorients users and makes navigation difficult.
    \item \textbf{Remediation}:
        \begin{itemize}
            \item \textbf{Microsoft Office/LibreOffice Writer/Google Docs}: Always use the application's built-in heading styles (e.g., Home tab $\rightarrow$ Styles in Word; Styles menu in LibreOffice/Google Docs).
            \item Ensure headings are used in a logical, hierarchical order, starting with a single $\texttt{H1}$ for the main title, followed by $\texttt{H2}$ for major sections, $\texttt{H3}$ for subsections, and so on. Do not skip levels.
            \item Ensure heading text is descriptive and unique.
        \end{itemize}
    \item \textbf{Audit Tools}: MS Office Accessibility Checker, AccessODF, Lighthouse, Deque axe DevTools, WAVE, ANDI, IBM Equal Access Accessibility Checker. Manual review with any screen reader.
\end{itemize}

\subsection{Missing or Low-Quality Alternative Text (Alt Text) for Images and Objects}
\begin{itemize}
    \item \textbf{Problem}: Screen readers cannot "see" visual content. Without descriptive alt text for images, charts, graphs, and other non-text elements, users who are blind or have low vision miss crucial information. Poor alt text (e.g., file names, "image of," or redundant text) is equally unhelpful.
    \item \textbf{Remediation}:
        \begin{itemize}
            \item \textbf{Microsoft Office/LibreOffice/Google Workspace}: Right-click on the image/object $\rightarrow$ "Format Picture/Object" $\rightarrow$ "Alt Text" (or "Size & Properties" $\rightarrow$ Alt Text).
            \item Provide concise, descriptive alt text that conveys the purpose or content of the image.
            \item For purely decorative images, mark them as decorative (often by leaving alt text blank or setting it to `""` depending on the application/context, or by using specific "Mark as decorative" options if available).
            \item For complex images (e.g., charts, infographics), provide a brief alt text summary and then a longer, detailed description in the surrounding text or in an appendix.
        \end{itemize}
    \item \textbf{Audit Tools}: MS Office Accessibility Checker, AccessODF, Lighthouse, Deque axe DevTools, WAVE, ANDI, IBM Equal Access Accessibility Checker. Manual review with any screen reader.
\end{itemize}

\subsection{Inaccessible Tables}
\begin{itemize}
    \item \textbf{Problem}: Screen readers navigate tables cell by cell. Without proper header rows/columns, merged/split cells, or overly complex structures, the context of the data within the table is lost, making it impossible to understand the relationships between cells.
    \item \textbf{Remediation}:
        \begin{itemize}
            \item \textbf{Microsoft Office/LibreOffice/Google Workspace}:
            \item Use simple table structures whenever possible.
            \item Define header rows (and header columns if applicable) using the application's built-in table properties. (e.g., Table Tools $\rightarrow$ Design $\rightarrow$ Header Row/First Column in Word/PowerPoint; Table $\rightarrow$ Table Properties $\rightarrow$ Text Flow $\rightarrow$ Repeat heading in LibreOffice Writer; Data $\rightarrow$ Create a filter/filter views in Google Sheets for column headers).
            \item Avoid merged or split cells. If unavoidable, ensure the reading order and context are explicitly managed for screen readers.
            \item Avoid nesting tables within other tables.
            \item For complex data presentations, consider providing the data in an alternative, accessible format (e.g., a simple list or a link to an accessible spreadsheet).
        \end{itemize}
    \item \textbf{Audit Tools}: MS Office Accessibility Checker, AccessODF (implicitly by checking for simple structures), Lighthouse, Deque axe DevTools, WAVE, IBM Equal Access Accessibility Checker. Manual review with any screen reader is critical for tables.
\end{itemize}

\subsection{Non-Descriptive Links}
\begin{itemize}
    \item \textbf{Problem}: Generic link text like "click here," "read more," or the full URL provides insufficient context for screen reader users who often navigate by a list of links.
    \item \textbf{Remediation}:
        \begin{itemize}
            \item \textbf{Microsoft Office/LibreOffice/Google Workspace}: Ensure all hyperlinks use descriptive text that clearly indicates the link's purpose and destination.
            \item Example: Instead of "Click here for our privacy policy," use "Read our Privacy Policy."
        \end{itemize}
    \item \textbf{Audit Tools}: MS Office Accessibility Checker, Lighthouse, Deque axe DevTools, WAVE, ANDI, IBM Equal Access Accessibility Checker. Manual review with any screen reader (listening to the list of links).
\end{itemize}

\subsection{Improper Reading/Tab Order (Especially in Presentations)}
\begin{itemize}
    \item \textbf{Problem}: In applications like PowerPoint or Google Slides, objects are read by screen readers in the order they were added to the slide, which may not correspond to their visual or logical reading order. Similarly, interactive elements (forms, buttons) may have a non-logical keyboard tab order.
    \item \textbf{Remediation}:
        \begin{itemize}
            \item \textbf{Microsoft PowerPoint}: Use the "Selection Pane" (Home tab $\rightarrow$ Arrange $\rightarrow$ Selection Pane or Alt+F10). The objects are listed in reverse reading order (bottom of the list is read first). Reorder objects by dragging them up or down to ensure a logical reading order.
            \item \textbf{Google Slides}: While there isn't a direct "Reading Order Pane," the order of objects generally follows their stacking order. Ensure elements are laid out logically and that interactive components have a natural tab order.
            \item \textbf{General}: For all interactive elements (form fields, buttons), ensure they are accessible via keyboard (Tab key, Shift+Tab) and that their tab order is logical and intuitive.
        \end{itemize}
    \item \textbf{Audit Tools}: MS Office Accessibility Checker (PowerPoint's "Check reading order"), Accessibility Insights for Windows (Tab Stops), IBM Equal Access Accessibility Checker (Keyboard checker mode), ANDI (for web-based tab order). Manual keyboard testing with a screen reader is essential.
\end{itemize}

\subsection{Insufficient Color Contrast}
\begin{itemize}
    \item \textbf{Problem}: Text and background colors with insufficient contrast make content unreadable for users with low vision, color blindness, or in challenging lighting conditions. While primarily a visual issue, it impacts screen reader users who may also have low vision.
    \item \textbf{Remediation}:
        \begin{itemize}
            \item \textbf{Microsoft Office/LibreOffice/Google Workspace}: Ensure a minimum contrast ratio of 4.5:1 for normal text and 3:1 for large text (at least 18pt or 14pt bold) as per WCAG AA guidelines.
            \item Use a color contrast checker (see tools below) to verify ratios.
            \item Avoid conveying information solely by color (e.g., "red indicates required fields"). Use additional visual cues (e.g., asterisks, text labels).
        \end{itemize}
    \item \textbf{Audit Tools}: MS Office Accessibility Checker, AccessODF, Color Contrast Analyzer, Accessibility Insights for Windows (Color Contrast Analyzer), Lighthouse, WAVE, ARC Toolkit, ANDI, IBM Equal Access Accessibility Checker.
\end{itemize}

\subsection{Improperly Formatted Lists}
\begin{itemize}
    \item \textbf{Problem}: Using hyphens, asterisks, or numbers typed manually to simulate a list instead of using the application's built-in list formatting. Screen readers will not identify these as lists or announce the number of items, making it difficult for users to understand the structure.
    \item \textbf{Remediation}:
        \begin{itemize}
            \item \textbf{Microsoft Office/LibreOffice/Google Workspace}: Always use the dedicated bulleted or numbered list features available in the application's formatting toolbar.
        \end{itemize}
    \item \textbf{Audit Tools}: MS Office Accessibility Checker, Lighthouse, WAVE. Manual review with a screen reader.
\end{itemize}

\section{Conclusion}
\label{sec:office-conclusion}

Conducting a comprehensive accessibility audit for Microsoft Office, LibreOffice, and Google Suite tools requires a multi-faceted approach. It involves a foundational understanding of WCAG, Section 508, and ARIA, leveraging the built-in accessibility checkers provided by the office suites, and employing a range of commercial and open-source audit tools. Crucially, automated checks must be complemented by manual screen reader testing to ensure a truly accessible user experience. By diligently addressing the common failures identified in this report, organizations and individuals can significantly enhance the accessibility of their digital documents, fostering inclusivity and providing equitable access to information for all users.
