\hypertarget{intro}{}\chapter*{Introduction}\label{intro}
\renewcommand{\cftchapleader}{\cftdotfill{\cftdotsep}}
\extramarks{Vision Department Technology Needs}{Introduction}
\pagestyle{fancyplain}
\fancyfoot[C]{\thepage}
In the pursuit of an inclusive and equitable educational landscape, it is imperative to recognize the unique challenges faced by students with visual impairments. The Individuals with Disabilities Education Improvement Act (IDEIA) of 2004\footnote{\raggedright \href{http://sites.ed.gov/idea/statuteregulations/}{20 U.S.C. § 1400, et.} \url{http://sites.ed.gov/idea/statuteregulations/}} underscores the commitment to providing every student with a free and appropriate education, regardless of their abilities. For students with visual impairments, technology plays a pivotal role in dismantling barriers, fostering independence, and unlocking opportunities for academic success\footnote{\raggedright \textit{cf}., \href{http://ectacenter.org/topics/atech/laws.asp}{list of federal regulations pertaining to assistive technology} \hfill\break\url{http://ectacenter.org/topics/atech/laws.asp}}.

This document delves into the critical importance of addressing the technology needs of students with visual impairments within the framework of IDEIA, which mandates that students with disabilities, including those with visual impairments, must be given access to assistive technology to ensure they can participate fully in the curriculum. Screen magnification is one such assistive technology that can help students with visual impairments access their free public education\footnotemark[\value{footnote}]. The overarching goal is to shed light on the essential role that technology plays in not only accommodating these students but empowering them to thrive in educational environments. By understanding and meeting their specific technological requirements, we can bridge the accessibility gap, promote inclusivity, and ensure that visually impaired students receive the education they deserve\footnote{\raggedright \href{https://qiat.org/new/wp-content/uploads/2020/06/TEBO_VI_Resource_Guide.pdf}{TEBO VI Resource Guide. (2020). Quality Indicators for Assistive Technology in Education. Retrieved December 19, 2023} \url{https://qiat.org/new/wp-content/uploads/2020/06/TEBO_VI_Resource_Guide.pdf}}.

It is evident that technology is not merely an auxiliary tool but a catalyst for educational equality. The integration of appropriate technology is fundamental to providing a level playing field, enabling visually impaired students to engage with educational content, interact with peers, and pursue academic excellence with the same vigor as their sighted counterparts. Throughout this document, we will delve into the diverse spectrum of technological solutions available, ranging from adaptive devices to assistive software, and explore how these tools contribute to an enriched learning experience
