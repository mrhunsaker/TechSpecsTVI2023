\chapter{Text-to-Speech and DAISY}\label{ch7:chap:text-to-speech-daisy}
\glsreset{ocr}\glsreset{icr}\glsreset{tts}\glsreset{llm}\glsreset{uia}\glsreset{msaa}\glsreset{pdfua}\glsreset{api}\glsreset{cpu}
\raggedright

\section{~~Overview}\label{ch07:sec:overview}
\gidx{texttospeech}{Text-to-speech} (\gls{texttospeech})text-to-speech and the Digital Accessible Information System
(\gls{daisy}\index{DAISY}) ecosystem form a critical foundation for equitable \gidx{accessibility}{accessibility}
access to curricular and literacy materials for students who are blind or have low vision. Together with
the National Instructional Materials Accessibility Standard (NIMAS\index{NIMAS}), they enable timely
conversion of source textbook packages into multiple accessible distribution formats (e.g., DAISY
navigable audio, EPUB with media overlays, \gidx{braille}{braille}, large print), directly impacting
literacy, comprehension, reading rate, and learner \gidx{independence}{independence}. For Teachers of Students with
Visual Impairments (TVIs), understanding the production workflow, technical standards, and implementation
strategies allows proactive coordination so that accessible copies reach learners concurrently with print
distribution—minimizing lost instructional time and remediation workload.

This chapter builds conceptual fluency (how \gls{tts} pipelines, \gidx{navigation}{navigation} models, and media overlays work),
practical selection and evaluation strategies (\gidx{latency}{latency}, intelligibility, markup fidelity), compliance
mapping (NIMAS, DAISY, EPUB Accessibility, WCAG, Section 508), and forward-looking insights (neural
prosody, adaptive reading analytics, multilingual alignment). It is self-contained so you can apply the
content directly to program planning, Individualized Education Program (IEP) collaboration, procurement,
and student training.

\section{~~Learning Objectives}\label{ch07:sec:learning-objectives}
After completing this chapter you will be able to:
\begin{enumerate}
	\item Differentiate core structural, navigational, and synchronization features of DAISY, EPUB 3
	      accessibility, and NIMAS packages (\emph{analyze}).
	\item Evaluate \gidx{texttospeech}{text-to-speech} voice quality and system latency against instructional use cases
	      using defined performance metrics (\emph{evaluate}).
	\item Design an end-to-end accessible reading workflow (NIMAS source $\rightarrow$ pipeline
	      transformation $\rightarrow$ student delivery) with role, timeline, and quality checkpoints
	      (\emph{design}).
	\item Troubleshoot common \gls{tts} and DAISY production issues (\gidx{navigation}{navigation} errors, malformed metadata,
	      misaligned media overlays) and implement preventive quality controls (\emph{apply}).
\end{enumerate}

\section{~~Key Terms}\label{ch07:sec:key-terms}
\begin{description}
	\item[DAISY:] \gls{daisy}; structured digital talking book standard supporting rich \gidx{navigation}{navigation}.
	\item[NIMAS:] U.S. K–12 XML-based source standard enabling efficient conversion to multiple
	      accessible formats under IDEA\supercite{IDEA2004}.
	\item[Media Overlay:] Synchronized text–audio mapping (EPUB/DAISY) enabling word, sentence, or
	      structural highlight during playback\supercite{W3CMediaA11y}.
	\item[\gidx{texttospeech}{Text-to-Speech} (\gls{tts}):] \gls{texttospeech}; real-time or pre-rendered \gidx{speechsynthesis}{speech synthesis}
	      pipeline producing intelligible audio output from text.
	\item[Latency:] End-to-end elapsed time between user action (e.g., begin reading command) and
	      audible speech onset (target: $< 250$ ms for conversational navigation).
	\item[Intelligibility / MOS:] Perceptual naturalness metrics (Mean Opinion Score) or objective
	      word/phoneme accuracy for synthesized speech evaluation.
	\item[Navigable Audio:] Audio enriched with structural anchors (chapters, pages, headings,
	      page numbers) enabling non-linear access tasks.
	\item[EPUB Accessibility 1.1:] W3C specification defining conformance and discoverability
	      requirements for accessible EPUB publications.
\end{description}

\section{~~Historical and Policy Context}\label{ch07:sec:historical-policy}
The emergence of DAISY and NIMAS is rooted in legislative mandates requiring equal instructional
access: IDEA (ensuring Free Appropriate Public Education), the Americans with Disabilities Act (ADA),
and Section 504/508—collectively obligating accessible procurement, timely delivery, and format
equivalence\supercite{IDEA2004, Section508}. NIMAS established a publisher-delivered, semantically
tagged XML ‘pivot’ format so downstream conversion (braille, EPUB, DAISY, large print) can be
automated or semi-automated and distributed through the NIMAC repository. DAISY evolved to provide
multi-level navigable audio synchronized with text, supporting structural jumps (e.g., heading level
skipping), page referencing, and media overlays for pedagogic segmentation.

Subsequent standards (EPUB 3 Accessibility 1.1, WCAG updates) formalized metadata, page \gidx{navigation}{navigation},
and synchronized text–audio quality expectations. Modern neural \gls{tts}—and the shift to low-latency on‑device
processors—further reduces dependency on pre-recorded narration, expanding rapid personalization.

\section{~~Core Concepts}\label{ch07:sec:core-concepts}
\subsection{Information Architecture and Navigability}
Accessible reading efficacy hinges on granular structural markers (headings, pages, sections, figures)
mapped in \gidx{navigation}{navigation} tables enabling direct jumps. DAISY/EPUB page navigation supports equitable
classroom referencing (e.g., “turn to page 142”) without linear scanning.

\subsection{Semantic Source Fidelity}
NIMAS XML semantics (heading depth, ordered lists, table structure, alternative text references)
propagate into conversion stages. Loss of semantics (e.g., flattening heading hierarchy) degrades
both braille translation accuracy and \gls{tts} skippability (risk: cognitive load increase).

\subsection{Synchronized Text–Audio Alignment}
Media overlay granularity (word vs. sentence) influences cognitive segmentation, study strategies,
and comprehension. Finer granularity empowers auditory scanning but may induce overhead if
playback control surfaces are cluttered.

\subsection{\gls{tts} Signal Quality Dimensions}
Key \gls{tts} evaluation facets:
\begin{itemize}
	\item \textbf{Latency:} Interaction responsiveness (goal $< 250$ ms initial, $< 120$ ms sequential).
	\item \textbf{Prosody Control:} Ability to modify rate (130–220 wpm), pitch, emphasis for
	      comprehension and fatigue management.
	\item \textbf{Intelligibility / Error Rate:} Word Error Rate (WER) $< 2\%$ on curricular domain lexicons.
	\item \textbf{Multi-lingual Support:} Seamless language switching for modern language coursework.
	\item \textbf{On-Device vs Cloud:} Privacy, bandwidth resilience, energy trade-offs.
\end{itemize}

\subsection{Metadata and Discoverability}
Accurate accessibility metadata (accessMode, accessibilityFeature) supports library/distribution
search filters and procurement compliance (EPUB Accessibility 1.1 discoverability model).

\section{~~Technologies and Tools}\label{ch07:sec:technologies-tools}
Below is a comparative categorization of representative tool classes (non-exhaustive). Hardware\gidx{hardware}{hardware}
and \gidx{software}{software} solutions are selected based on educational relevance and standards support.

\footnotesize
\begin{longtblr}[
		caption={Representative Accessible Reading Technologies (Functional Comparison)},
		label={ch07:tab:reading-tech},
		note={Feature presence indicates common educational deployment characteristics; verify current versions before procurement.},
	]{colspec={X[l] X[l] X[l] X[l] X[l]}, rowhead=1, hlines}
	\toprule
	Category                                                          & Example Tools / Platforms                                                                  & Core Function                                        & Key Accessibility Features                                                                                 & Evaluation Considerations                                               \\
	\midrule
	Structured Audio Readers                                          & DAISY player apps (Voice Dream Reader\supercite{VoiceDreamReader}, Bookshare, BARD Mobile) & Navigable audio + text sync                          & Multi-level TOC, page nav, bookmarks, media overlay highlighting                                           & \gidx{navigation}{Navigation} depth, sync accuracy, battery consumption \\
	Desktop / Mobile Screen Readers\gidx{screenreader}{screen reader} & JAWS, NVDA, VoiceOver, TalkBack                                                            & Real-time UI + content speech                        & Structural \gidx{navigation}{navigation} (headings, links), \gidx{brailledisplay}{braille display} support & Latency on large documents, braille translation accuracy                \\
	Conversion Pipelines                                              & DAISY Pipeline / Publisher XML Transformers                                                & Source XML → multi-format (EPUB, braille, DAISY)     & Automated validation, page map preservation                                                                & Schema validation, throughput (files/hour)                              \\
	On-Device Neural \gls{tts} Engines                                & Platform voices (Apple, Microsoft), local neural voices                                    & Low-latency \gidx{speechsynthesis}{speech synthesis} & Natural prosody, offline resilience, rate/pitch control                                                    & Startup latency, MOS benchmarks, \gls{cpu}/NPU utilization              \\
	Cloud \gls{tts} APIs                                              & Commercial neural \gls{tts} services                                                       & High-fidelity multi-voice rendering                  & Large voice catalog, SSML control                                                                          & Network jitter impact, privacy risk (data egress)                       \\
	Annotation / Study Tools                                          & Accessible EPUB readers with markup                                                        & Highlight, note, export for study                    & Keyboard + braille annotation, structural export                                                           & Note export format fidelity, accessible markup semantics                \\
	\bottomrule
\end{longtblr}
\normalsize

\subsection{Retained Original Sections}
The following original sections (with preserved labels) now conceptually reside within this
Technologies and Tools domain.

\section{~~DAISY Readers}\label{ch7:sec:daisy-readers}
DAISY (Digital Accessible Information System\index{DAISY}) is a global standard for digital talking books,
designed to provide an accessible and navigable reading experience for individuals with print
disabilities.\supercite{DAISY2024} \gls{daisy} books are structured digital publications that can contain
\gls{audio}, text, and images synchronized for comprehensive \gidx{navigation}{navigation}.\supercite{MDPI2022}

\subsubsection{DAISY Readers and Digital Audio Players}
\begin{description}
	\item[Dedicated DAISY Players:] Standalone hardware featuring tactile controls, high-contrast displays,
	      and robust battery life.
	\item[Software Players:] Multi-platform applications integrating with screen readers\index{screen reader}.
	\item[Mobile Apps:] iOS / Android solutions embedding DAISY playback with cloud library linkage.
\end{description}

\section{~~\gidx{texttospeech}{Text-to-Speech}}\label{ch07:sec:tts}
Text-to-speech technology leverages \gidx{speechsynthesis}{speech synthesis} pipelines converting textual
content into speech.\supercite{VisionAid2025} Modern neural \gls{tts} delivers improved naturalness,
contextual prosody, and low-latency response, broadening usage for research reading, rapid skimming,
and multimodal study. Integration into screen readers, standalone devices, and reading platforms
enables pervasive access.

\section{~~Implementation Strategies}\label{ch07:sec:implementation-strategies}
Effective district or program adoption requires a structured workflow and continuous \index{assessment}.

\subsection{Workflow Blueprint}
\begin{enumerate}
	\item \textbf{Source Acquisition:} Confirm publisher provides certified NIMAS package; validate against schema.
	\item \textbf{Conversion Orchestration:} Use automated pipeline (e.g., DAISY Pipeline) to generate EPUB (media overlay),
	      braille (\texttt{.brf}), large print—log validation messages.
	\item \textbf{Quality Review:} Spot-check \gidx{navigation}{navigation} (heading depth coverage $100\%$, page map accuracy $> 99\%$),
	      media overlay sync drift ($< 300$ ms average).
	\item \textbf{Student Profiling:} Match modality (audio-first, braille-first, dual) to individualized goals (IEP).
	\item \textbf{Deployment:} Distribute via authenticated library platforms supporting offline caching.
	\item \textbf{Training:} Provide scaffolded instruction—basic navigation $\rightarrow$ annotation $\rightarrow$ study strategies.
	\item \textbf{Progress Monitoring:} Track reading rate (wpm), comprehension quiz scores, and \gidx{independence}{independence} metrics
	      (number of unaided navigational jumps) at defined intervals.
	\item \textbf{Iteration:} Adjust settings (voice rate, punctuation verbosity) based on data and learner feedback.
\end{enumerate}

\subsection{Assessment Metrics (Illustrative)}
\begin{itemize}
	\item Reading speed delta pre/post adoption (target improvement $\ge 15\%$ for grade-level passages).
	\item \gidx{navigation}{Navigation} accuracy (successful direct page or heading jumps / attempts) $> 90\%$ after four weeks.
	\item Latency adherence: median initial speech onset $< 250$ ms; sustained page-turn transition $< 150$ ms.
	\item Error incidence in structural announcements (misreported page, heading level) $< 1$ per 500 navigation events.
\end{itemize}

\section{~~Standards and Compliance}\label{ch07:sec:standards-compliance}
\begin{itemize}
	\item \textbf{NIMAS:} Ensures timely, standardized source exchange under IDEA; mandates \gidx{semantictagging}{semantic tagging}
	      (heading levels, lists, tables) enabling reliable downstream transformation.
	\item \textbf{DAISY Specifications:} Define hierarchical \gidx{navigation}{navigation}, SMIL-based synchronization,
	      and media overlay semantics for structured audio.
	\item \textbf{EPUB Accessibility 1.1:} Requires discoverability metadata (e.g., \texttt{accessMode},
	      \texttt{accessibilityFeature}) and encourages robust page navigation\supercite{W3CMediaA11y}.
	\item \textbf{WCAG / Section 508:} Aligns perception, operability, and robust principles with reading system UI.
	\item \textbf{Section 121 / Marrakesh Treaty (contextual):} Facilitates cross-border accessible format exchange (if applicable).
\end{itemize}

Mapping practice example: Verified page list + page break markers satisfy equitable referencing (EPUB page \gidx{navigation}{navigation} objective);
structured headings with correct nesting meet WCAG 2.x navigation success criteria; alternative text pipeline ensures
non-text content perceivability.

\section{~~Case Studies and Applied Examples}\label{ch07:sec:case-studies}
\subsection{Case Study 1: Middle School Science Text Delivery}
\textbf{Context:} 7th-grade mainstream science course adopting a new textbook.
\textbf{Intervention:} NIMAS source ingested; DAISY Pipeline generated synchronized EPUB + BRF; custom \gls{tts} voice with slower rate and
explicit punctuation for formula reading.
\textbf{Metrics (8 weeks):} Reading speed increased from 115 wpm to 148 wpm (29\% gain); comprehension quiz average rose 11 percentage points;
independent \gidx{navigation}{navigation} success (page or heading jump without prompting) from 54\% to 93\%.
\textbf{Outcome:} Improved synchronous classroom participation and reduced instructor redirection time.

\subsection{Case Study 2: Accelerated Literature Unit}
\textbf{Context:} High school AP Literature module with dense prose.
\textbf{Intervention:} Media overlay EPUB used in dual-mode (audio + highlight) for close reading; prosody emphasis configuration
modified to reduce monotonic phrase endings.
\textbf{Metrics:} Annotation counts per chapter (student-created) increased 2.3x; analytical essay citing textual evidence improved rubric
cohesion score by 15\%.
\textbf{Outcome:} Enhanced text engagement and analysis depth.

\section{~~Best Practices}\label{ch07:sec:best-practices}
\begin{itemize}
	\item Perform pre-semester NIMAS acquisition audit to eliminate reactive scrambling.
	\item Standardize \gls{tts} evaluation rubric (latency, MOS proxy via small user panel, prosody control granularity).
	\item Maintain a version-controlled transformation recipe (XSLT / pipeline configs) for reproducibility.
	\item Integrate student feedback loops (weekly quick poll on voice fatigue, \gidx{navigation}{navigation} friction).
	\item Enforce structured QA checklist: headings, page map, alt text coverage, SMIL sync drift.
	\item Provide tiered training: foundational navigation $\rightarrow$ search/annotation $\rightarrow$ study workflows.
	\item Track longitudinal metrics (wpm, comprehension) to justify procurement renewals.
	\item Ensure privacy-aligned configuration (on-device \gls{tts} where sensitive notes are spoken).
\end{itemize}

\section{~~Troubleshooting and Common Pitfalls}\label{ch07:sec:troubleshooting}
\footnotesize
\begin{longtblr}[
		caption={Troubleshooting Guide for \gls{tts} and DAISY Workflows},
		label={ch07:tab:troubleshooting},
		note={ReferenceKey points to existing citation keys; ensure bibliography entries exist.},
	]{colspec={X[l] X[l] X[l] X[l] X[l] X[l]}, rowhead=1, hlines}
	\toprule
	Issue                                              & RootCause                                       & ImpactOnLearner                                             & ResolutionSteps                                     & PreventivePractice                               & ReferenceKey  \\
	\midrule
	High initial \gls{tts} latency ($>500$ ms)         & On-demand model load / network call             & Breaks reading flow, reduced engagement                     & Pre-warm engine; cache voice; switch to local voice & Measure cold/warm latency pre-deployment         & W3CMediaA11y  \\
	Misaligned media overlays (sentence offset)        & Incorrect SMIL timing export                    & Comprehension disruption; mistrust of highlight             & Re-run timing generation; validate with diff tool   & Automated sync drift test (<300 ms avg)          & W3CMediaA11y  \\
	Missing page \gidx{navigation}{navigation} entries & Omitted page list generation                    & Inability to follow in-class page references                & Regenerate navigation doc; insert page-list         & Build pipeline step validating page count parity & W3CMediaA11y  \\
	Heading hierarchy flattened                        & Improper heading level mapping in XML transform & Loss of structural \gidx{navigation}{navigation} efficiency & Adjust XSLT mapping; re-validate semantics          & Schema test + heading depth coverage report      & Section508    \\
	Incorrect alt text association                     & ID/REF mismatch in NIMAS package                & Screen reader skips descriptions                            & Correct ID references; repackage                    & Link integrity automated check                   & Section508    \\
	\gls{tts} mispronounces domain terms               & Missing custom lexicon / SSML overrides         & Reduced comprehension of technical vocabulary               & Add pronunciation dictionary; test                  & Maintain shared lexicon repository               & VisionAid2025 \\
	Voice fatigue reported after long sessions         & Excessive speech rate / monotone prosody        & Declining retention over session                            & Tune rate (target 150–180 wpm); enable breaks       & Provide training on adjusting rate/pitch         & Kelly2011     \\
	EPUB metadata incomplete (no accessibilityFeature) & Export script omission                          & Reduced discoverability in repositories                     & Patch export; add metadata injection step           & Metadata validation gate in CI                   & W3CMediaA11y  \\
	\bottomrule
\end{longtblr}
\normalsize

\section{~~Emerging Trends and Future Directions}\label{ch07:sec:emerging-trends}
\begin{itemize}
	\item \textbf{Neural Expressive \gls{tts}:} Multi-speaker, emotion, and context-aware prosody models
	      enabling adaptive emphasis for instructional segments.
	\item \textbf{On-Device Edge Inference:} NPUs lowering energy per synthesized token, supporting
	      offline privacy-centric reading sessions.
	\item \textbf{Adaptive Reading Analytics:} Real-time pacing adjustment based on detected \gidx{navigation}{navigation}
	      regressions (e.g., frequent sentence replays).
	\item \textbf{Multimodal Alignment:} Integrated \gidx{hapticfeedback}{haptic feedback} cues for page
	      or heading boundary transitions to reduce cognitive load.
	\item \textbf{Interoperable Pipelines:} Converging transforms (NIMAS → EPUB Accessibility + DAISY + braille)
	      under unified validation dashboards.
\end{itemize}

\section{~~Ethical, Equity, and Privacy Considerations}\label{ch07:sec:ethics}
Equity considerations include ensuring simultaneous delivery (no lag vs print), parity of navigational
granularity (pages, figures, sidebars), and culturally inclusive voice selections (avoiding only a single
accent set). Privacy concerns arise when cloud \gls{tts} transmits sensitive student annotation content; mitigate
by local processing where feasible and minimizing personally identifiable content in pipeline logs. Data arising
from reading analytics must be transparently disclosed (purpose, retention) and not repurposed for non-instructional
profiling. Avoid over-reliance on a single vendor—promote resilience and guard against proprietary lock-in that
could delay updates or inflate costs, impacting long-term equitable access.

\section{~~Assessment and Reflection}\label{ch07:sec:assessment-reflection}
\subsection*{Reflection Questions}
\begin{enumerate}
	\item Which performance metric (latency, intelligibility, \gidx{navigation}{navigation} accuracy) most strongly influences learning
	      outcomes in your setting, and why?
	\item How would you redesign a workflow step if NIMAS source delivery is delayed two weeks before semester start?
	\item What safeguards can you implement to ensure privacy when annotating sensitive student materials with \gls{tts}?
\end{enumerate}

\subsection*{Applied Exercise (Mini Project)}
Design and document a “48-hour rapid conversion playbook”:
\begin{enumerate}
	\item Select a hypothetical new textbook adoption date.
	\item Outline steps, tools, and roles from NIMAS retrieval to student-ready DAISY + EPUB + braille.
	\item Include quantitative acceptance criteria (e.g., $> 98\%$ structural element preservation).
	\item Propose an automated validation script checklist (headings, page map, alt text coverage).
\end{enumerate}

\section{~~Summary}\label{ch07:sec:summary}
DAISY, NIMAS, and modern neural \gidx{texttospeech}{text-to-speech} collectively operationalize timely, multi-modal access
to curricular materials, enabling navigable, high-fidelity reading experiences that support literacy growth,
content comprehension, and learner \gidx{independence}{independence}. Effective educational deployment requires disciplined
conversion workflows, rigorous structural and timing validation, data-informed tuning (e.g., latency and
reading speed metrics), and continuous student-centered training. Proactive standards alignment (EPUB
Accessibility 1.1, WCAG, Section 508) and ethical safeguards (privacy, inclusive voice selection) ensure
sustainable, equitable service delivery. Strategic monitoring (\gidx{navigation}{navigation} accuracy, reading rate deltas,
comprehension gains) closes the loop—transforming assistive reading from reactive accommodation into
an integrated component of instructional design.

\section{~~References}\label{ch07:sec:references}

\endinput
