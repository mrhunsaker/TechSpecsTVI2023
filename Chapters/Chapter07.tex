\chapter{Beyond Boundaries: Text-to-Speech and DAISY as Catalysts for Literacy and Success in Visual Impairment Education}\label{chap:text-to-speech-daisy}

The National Instructional Materials Accessibility Standard (NIMAS) and Digital Accessible Information System (DAISY) are two important tools for the education of students with visual impairments. NIMAS is a technical standard used by publishers to prepare electronic files that are used to convert instructional materials into accessible formats. The purpose of NIMAS is to help increase the availability and timely delivery of instructional materials in accessible formats for qualifying students in elementary and secondary schools, with the National Instructional Materials Access Center (NIMAC) serving as a federally funded, online file repository of more than 83,000 K-12 NIMAS files~ \cite{NIMAC2025}. DAISY is a digital format for audio books that is designed to be more accessible to people with visual impairments, with files that can be used to produce formats like braille, EPUB, DAISY, and large print~ \cite{DAISY2024}.

Under IDEA, students must be receiving special education services under IDEA and meet the qualification criteria of the National Library Service for the Blind and Print Disabled (NLS) statute to be eligible for NIMAS-derived materials through the NIMAC~ \cite{CTEducation2025}.

NIMAS and DAISY are important because they help make educational materials more accessible to students with visual impairments. By providing instructional materials in accessible formats, students with visual impairments can participate more fully in the general education curriculum. This can help improve their academic performance and increase their chances of success in school.

Finally, NIMAS and DAISY can help students with visual impairments become more independent. By providing instructional materials in accessible formats, students can read books, take notes, and communicate with others more easily. This can help them lead more fulfilling lives and become more active members of their communities.

In the evolving landscape of education, the pursuit of literacy is a journey marked by innovation and inclusivity. For visually impaired students, the traditional pathways to literacy take on a distinctive form, guided by the transformative power of audiobook and DAISY readers. This chapter explores the indispensable role that these tools play in breaking down barriers to literacy, ensuring access to knowledge, and propelling visually impaired students towards academic success.

\section{DAISY Readers}\label{sec:daisy-readers}

Assistive technology is a crucial tool for students with visual impairments or blindness to receive a free and appropriate public education. DAISY (Digital Accessible Information System) is a technical standard for digital audiobooks, periodicals, and computerized text that is designed to be a complete audio substitute for print material and is specifically designed for use by people with print disabilities, including blindness, impaired vision, and dyslexia~ \cite{Wikipedia2024}.

Based on the MP3 and XML formats, the DAISY format has advanced features in addition to those of a traditional audiobook. Users can search, place bookmarks, precisely navigate line by line, and regulate the speaking speed without distortion~ \cite{Wikipedia2024}. DAISY also provides aurally accessible tables, references and additional information. As a result, DAISY allows visually impaired listeners to navigate something as complex as an encyclopedia or textbook, otherwise impossible using conventional audio recordings~ \cite{MDPI2022}.

DAISY books provide full accessibility for the reader and allow the reader to play the book on a variety of hardware and software players depending on the DAISY file type (e.g., CD players, mobile phones, computers, iPods \& iPads, tablets, etc.)\footnote{\href{https://www.cde.ca.gov/re/pn/sm/dtbdefinitions.asp}{California Department of Education. (2025). Digital Talking Books. Retrieved July 4, 2025}}.

\subsubsection{DAISY Readers and Digital Audio Players}

\tagpdfsetup{table/header-rows={1}}

\centering
\begin{longtblr}[
  caption = {DAISY readers and digital audio players: models, function, and company (Updated 2025)},
  label = {tab:chapter7:daisy-readers},
  note = {This table provides a comprehensive list of DAISY-compatible devices for reading digital books and audio content, including both portable and desktop options. It highlights device functions, manufacturers, and notes legacy models that may be discontinued, helping users identify suitable technology for accessible reading.}
]{
  colspec = {X[l] X[l] X[l]},
  rowhead = 1,
  hlines,
  stretch = 1.5
}
Model & Function & Company \\
Milestone 212 Ace Book Reader & DAISY Reader, Digital Audio Player & Bones \\
Orbit Player & DAISY Reader, Digital Audio Player & Orbit Research \\
PlexTalk PTN2 & DAISY Reader, CD Player & PlexTalk \\
PlexTalk Pocket & DAISY Reader, Digital Audio Player & PlexTalk \\
Reizen DAISY Digital Recorder & DAISY Reader, Digital Audio Player & Reizen \\
Victor Reader Stratus & DAISY Reader, Digital Audio Player (Desktop) & Humanware \\
Victor Reader Stream & Digital Audio Player & Humanware \\
Victor Reader Trek & GPS, Digital Audio Player & Humanware \\
Victor Reader Stratus 12 & Advanced DAISY Reader, Digital Audio Player & Humanware \\
\end{longtblr}

\section{Text-to-Speech}\label{text-to-speech}

The use of assistive technology, including Text-to-Speech, is required for all students with disabilities that show a need under the Individuals with Disabilities Education Act (IDEA), with eligibility criteria including those with visual impairments, perceptual disabilities, or those otherwise unable through physical disability to hold or manipulate a book or to focus or move the eyes to the extent that would be normally acceptable for reading\footnote{\href{https://ncademi.org/resources/publications/aem-guide/}{National Center on Accessible Digital Educational Materials. (2024). Addressing Accessible Educational Materials in the 2024 Assistive Technology Guidance. Retrieved July 4, 2025}}.

With the push of a single button, text to speech devices photograph your reading material and within seconds, begin reading it aloud. These easy-to-use devices are ideal for blind users of all ages and abilities. Some devices offer the option of smart reading features that enable you to simply ask for the text that interests you and some even offer facial and product recognition~ \cite{VisionAid2025}.

\subsubsection{Text-to-Speech Devices}

\tagpdfsetup{table/header-rows={1}}

\centering
\begin{longtblr}[
  caption = {Text-to-speech devices: model, function, and company (Updated 2025)},
  label = {tab:chapter7:text-to-speech-devices},
  note = {This table lists available text-to-speech devices for visually impaired users, ranging from handheld scanners to wearable solutions. It details device models, functions, and manufacturers, and provides guidance on pricing and availability for selecting appropriate assistive technology.}
]{
  colspec = {X[l] X[l] X[l]},
  rowhead = 1,
  hlines,
  stretch = 1.5
}
Model & Function & Company \\
C-Pen Reader & Pen Scanner, Text-to-Speech Reader & C-Pen \\
OrCam MyEye Pro & AI-powered Smart Glasses, Text-to-Speech, Object Recognition & OrCam \\
OrCam Read 3 & Hand-held Text-to-Speech Reader & OrCam \\
Scanmarker Air & Hand-held Text-to-Speech Scanner & Scanmarker \\
Enhanced Vision Smart Reader HD & Portable Text-to-Speech Reader & Enhanced Vision \\
Voice Dream Reader & Software-based Text-to-Speech (iOS/Android) & Voice Dream \\
KNFB Reader & Mobile App Text-to-Speech & KNFB \\
Speechify & AI-powered Text-to-Speech Software & Speechify \\
\end{longtblr}

\section{Emerging Technologies and Future Directions}\label{emerging-tech}

The landscape of assistive technology for visual impairment education continues to evolve rapidly. Assistive technology for people with disabilities has come a long way, with 2025 bringing new tools that are improving their lives~ \cite{Speechify2025}.

Advanced text-to-speech tools and voice cloning technology are bringing new levels of inclusivity to visually impaired people and individuals with speech disorders~ \cite{Respeecher2024}.

Modern text-to-speech technology has advanced significantly, offering more natural-sounding voices, better pronunciation of technical terms, and improved navigation features. These advancements make educational materials more accessible and engaging for students with visual impairments.

\section{Implementation and Best Practices}\label{implementation}

Educational institutions must ensure proper implementation of these technologies to maximize their effectiveness. The Department of Education provides guidance to states, state educational agencies (SEAs), local educational agencies, and other interested parties with information to facilitate implementation of the National Instructional Materials Accessibility Standard and coordination with the National Instructional Materials Access Center~ \cite{USDeptEducation2021}.

\subsubsection{Key Considerations for Implementation}

\begin{itemize}

    \item \textbf{Early identification and assessment of students who would benefit from these technologies}
    \item \textbf{Proper training for educators and support staff on the use of DAISY and text-to-speech devices}
    \item \textbf{Regular evaluation and updating of assistive technology resources}
    \item \textbf{Collaboration between special education professionals, general education teachers, and technology specialists}
    \item \textbf{Ensuring compatibility between different systems and formats}

\end{itemize}

\section{Conclusion}\label{conclusion}

The integration of NIMAS, DAISY, and text-to-speech technologies represents a significant advancement in educational accessibility for students with visual impairments. These tools not only provide access to educational materials but also promote independence, enhance learning outcomes, and prepare students for success in academic and professional environments.

The DAISY Consortium continues to work with over 150 partners all around the world to improve access to reading for people with print disabilities~ \cite{DAISY2024}, demonstrating the ongoing commitment to advancing these technologies.

\begin{thebibliography}{99}
\bibitem{NIMAC2025} NIMAC. (2025). National Instructional Materials Access Center. Retrieved July 4, 2025, from \url{https://www.nimac.us/}.
\bibitem{DAISY2024} DAISY Consortium. (2024). What is DAISY? Retrieved July 4, 2025, from \url{https://daisy.org/}.
\bibitem{CTEducation2025} Connecticut State Department of Education. (2025). Accessible Educational Materials. Retrieved July 4, 2025, from \url{https://portal.ct.gov/SDE/Publications/Assistive-Technology-Guidelines-Section-1-For-Ages-3-22/Accessible-Educational-Materials}.
\bibitem{Wikipedia2024} Wikipedia. (2024). Digital Accessible Information System. Retrieved July 4, 2025, from \url{https://en.wikipedia.org/wiki/Digital_Accessible_Information_System}.
\bibitem{MDPI2022} Encyclopedia MDPI. (2022). DAISY Digital Talking Book. Retrieved July 4, 2025, from \url{https://encyclopedia.pub/entry/33638}.
\bibitem{VisionAid2025} VisionAid. (2025). Text-to-speech. Retrieved July 4, 2025, from \url{https://www.visionaid.co.uk/text-to-speech}.
\bibitem{Speechify2025} Speechify. (2025). 10 assistive technology tools to help people with disabilities in 2025 and beyond. Retrieved July 4, 2025, from \url{https://speechify.com/blog/assistive-technology-tools-to-help-people/}.
\bibitem{Respeecher2024} Respeecher. (2024). How to Empower the Visually Impaired with Advanced Text-to-Speech Tools. Retrieved July 4, 2025, from \url{https://www.respeecher.com/blog/empower-the-visually-impaired-advanced-text-to-speech-tools}.
\bibitem{USDeptEducation2021} U.S. Department of Education. (2021). Questions and Answers on the National Instructional Materials Accessibility Standard (NIMAS). Retrieved July 4, 2025, from \url{https://sites.ed.gov/idea/idea-files/questions-and-answers-on-the-national-instructional-materials-accessibility-standard-nimas-aug-9-2021/}.
\end{thebibliography}

As technology continues to evolve, the future holds even greater promise for breaking down barriers to literacy and ensuring that all students, regardless of their visual abilities, have equal access to educational opportunities. The continued development and implementation of these assistive technologies will play a crucial role in creating truly inclusive educational environments.
