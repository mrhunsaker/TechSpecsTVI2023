\chapter{Text-to-Speech and DAISY}\label{ch7:chap:text-to-speech-daisy}
\raggedright
The National Instructional Materials\index{instructional materials} Accessibility\index{accessibility} Standard (NIMAS\index{NIMAS}) and Digital Accessible Information System (DAISY\index{DAISY}) are two important tools\index{sonification!tools} for the education of students with visual impairments\index{visual impairment}. NIMAS is a technical standard used by publishers to prepare electronic files that are used to convert instructional materials into accessible formats. The purpose of NIMAS is to help increase the availability and timely delivery of instructional materials in accessible formats for qualifying students in elementary and secondary schools, with the National Instructional Materials Access Center (NIMAC\index{NIMAC}) serving as a federally funded, online file repository of more than 83,000 K-12 NIMAS files~ \supercite{NIMAC2025}. The U.S. Department of Education provides comprehensive guidance on NIMAS implementation and compliance requirements \supercite{USDeptEducation2021}. DAISY is a digital format for audio books that is designed to be more accessible to people with visual impairments, with files that can be used to produce formats like braille\index{braille}, EPUB, DAISY, and large print~ \supercite{DAISY2024, Wikipedia2024, MDPI2022}.

Under IDEA, students must be receiving special education services under IDEA and meet the qualification criteria of the National Library Service for the Blind and Print Disabled (NLS) statute to be eligible for NIMAS\index{NIMAS}-derived materials through the NIMAC~ \supercite{CTEducation2025, IDEA2004}.

NIMAS and DAISY\index{DAISY} are important because they help make educational materials more accessible to students with visual impairments. By providing instructional materials\index{instructional materials} in accessible formats, students with visual impairments can participate more fully in the general\index{daily living aids!general} education curriculum. This can help improve their academic performance\index{troubleshooting!performance} and increase their chances of success in school.\supercite{Kelly2011, StudentOutcomesResearch}

Finally, NIMAS and DAISY can help students with visual impairments\index{visual impairment} become more independent. By providing instructional materials in accessible formats, students can read books, take notes, and communicate with others more easily. This can help them lead more fulfilling lives and become more active members of their communities.\supercite{Holbrook2006, Day2021}

In the evolving landscape of education, the pursuit of literacy is a journey marked by innovation and inclusivity. For visually impaired students, the traditional pathways to literacy take on a distinctive form, guided by the transformative power of audiobook\index{audiobook} and DAISY readers. This chapter explores the indispensable role that these tools\index{sonification!tools} play in breaking down barriers to literacy, ensuring access to knowledge, and propelling visually impaired students towards academic success.\supercite{Bookshare, BARDMobile, Audible}

\section{~~DAISY Readers}\label{ch7:sec:daisy-readers}

DAISY (Digital Accessible Information System\index{DAISY}) is a global standard for digital talking books, designed to provide an accessible and navigable reading experience for individuals with print disabilities. DAISY readers are the devices and software\index{software} that bring these books to life, offering a rich, structured, and user-friendly alternative to traditional audiobooks.\supercite{DAISY2024, Wikipedia2024}

\gls{daisy} books are more than just \gls{audio} recordings; they are structured digital publications that can contain \gls{audio}, text, and images, all synchronized to provide a comprehensive reading experience. This structure allows users to navigate the book by chapter, section, page, and even sentence, mirroring the way a sighted person would interact with a physical book.\supercite{MDPI2022}

\subsubsection{DAISY Readers and Digital Audio Players}

A wide range of devices and \gls{software} applications are available to play DAISY books, catering to different needs and preferences.

\begin{description}
	\item[Dedicated DAISY Players\index{DAISY!DAISY reader}:] These are standalone hardware\index{hardware} devices specifically designed for reading DAISY books. They often feature tactile buttons, simple interfaces, and long battery life, making them ideal for users who prefer a dedicated device.
	\item[Software DAISY Players:] These are applications that can be installed on computers, smartphones, and tablets\index{tablet}. They offer a high degree of flexibility and can be used with screen readers\index{screen reader} and other assistive technologies\index{assistive technology}.
	\item[Mobile Apps:] Many DAISY\index{DAISY} player apps are available for iOS and Android\index{operating system!Android} devices, providing a portable and convenient way to access DAISY books on the go.
\end{description}

The choice of a DAISY reader depends on the individual user's needs, technical skills, and lifestyle. For students, a software\index{software}-based or mobile solution is often the most practical, as it can be integrated with their existing devices and used for both academic and leisure reading.\supercite{VoiceDreamReader, Bookshare}

\section{~~Text-to-Speech}

Text-to-speech\index{text-to-speech} (TTS) technology\index{technology} is a form of speech synthesis that converts written text into spoken voice output. TTS has become an indispensable tool for visually impaired students, providing access to a vast range of digital content that would otherwise be inaccessible.\supercite{VisionAid2025, Speechify2025}

From reading emails and websites to accessing digital textbooks and research papers, TTS\index{text-to-speech} empowers students to engage with the digital world on an equal footing with their sighted peers. The technology has advanced significantly in recent years, with modern TTS engines offering a wide range of natural-sounding voices and languages.\supercite{Respeecher2024}

\subsubsection{Text-to-Speech Devices}

\gls{tts} functionality is now integrated into a wide variety of devices and applications, making it more accessible than ever before.

\begin{description}
	\item[Screen Readers\index{screen reader}:] Screen readers, such as JAWS\index{screen reader!JAWS}, NVDA, and VoiceOver, are the primary tools\index{sonification!tools} used by visually impaired individuals to interact with computers and mobile devices. They use TTS to read aloud the content of the screen, including text, icons, and menus.
	\item[Standalone Reading Machines:] These devices combine a scanner with OCR\index{OCR} and TTS technology\index{technology} to read printed documents aloud. They are ideal for accessing materials that are not available in a digital format.
	\item[Portable Reading Devices:] Many portable devices, such as the OrCam\index{video magnifier!OrCam} Read, are designed to provide on-the-go text-to-speech\index{text-to-speech} capabilities. These devices can read text from any surface, from a book to a signpost.
	\item[Software\index{software} Applications:] Numerous software applications and browser extensions are available that can add TTS\index{text-to-speech} functionality to any computer or mobile device. These tools are often used for reading long articles, proofreading documents, and studying.
\end{description}

The widespread availability of TTS technology has had a profound impact on the education of visually impaired students, fostering independence\index{independence}, improving literacy skills, and opening up a world of information.\supercite{Kelly2011, Holbrook2006}

\section{~~Emerging Technologies and Future Directions}

The fields of text-to-speech and DAISY\index{DAISY} are constantly evolving, with new technologies and innovations emerging all the time.\supercite{aimodels2024}

\begin{itemize}
	\item \textbf{AI\index{AI}-Powered Voices:} Artificial intelligence is being used to create increasingly natural and expressive TTS voices, making the listening experience more engaging and less fatiguing.
	\item \textbf{Integration with Smart Assistants:} TTS and DAISY functionality is being integrated into smart assistants like Amazon Alexa and Google\index{tablet!Google} Assistant, allowing users to access their books and other content with simple voice commands.
	\item \textbf{Haptic Feedback\index{haptic feedback}:} Researchers are exploring the use of haptic feedback\index{haptic feedback} to enhance the reading experience, providing tactile cues to indicate formatting and structure.
	\item \textbf{Immersive Reading Environments:} Virtual and augmented reality technologies have the potential to create immersive reading environments that can bring books to life in new and exciting ways.
\end{itemize}

As these technologies continue to develop, they will further enhance the accessibility\index{accessibility} and richness of the reading experience for visually impaired students.

\section{~~Implementation and Best Practices}

To ensure that text-to-speech\index{text-to-speech} and DAISY technologies are used effectively in educational settings, it is important to follow best practices for implementation.\supercite{Burgstahler2015}

\subsubsection{Key Considerations for Implementation}

\begin{itemize}
	\item \textbf{Assessment of Needs:} A thorough assessment of each student's individual needs is essential to determine the most appropriate tools\index{sonification!tools} and strategies.
	\item \textbf{Training and Support\index{troubleshooting!support}:} Both students and teachers require comprehensive training on how to use the technology\index{technology} effectively. Ongoing technical support is also crucial.
	\item \textbf{Access to Content:} Schools must ensure that students have access to a wide range of accessible instructional materials\index{instructional materials} in formats like DAISY\index{DAISY} and EPUB.
	\item \textbf{Integration into the Curriculum:} The use of these technologies should be seamlessly integrated into the curriculum, rather than being treated as a separate, add-on activity.
	\item \textbf{Collaboration:} Collaboration between teachers, parents, and specialists is key to creating a supportive learning environment that meets the needs of each student.
\end{itemize}

By following these best practices, schools can harness the full potential of \gls{texttospeech} and DAISY to create a more inclusive and equitable learning environment for all students.

\section{~~Conclusion}

Text-to-speech and DAISY are more than just assistive technologies\index{assistive technology}; they are powerful catalysts for literacy and success. By providing visually impaired students with access to the written word, these tools break down barriers to learning, foster independence\index{independence}, and empower students to reach their full potential.\supercite{StudentOutcomesResearch, wjaets2024}

As \gls{technology} continues to advance, the future of accessible reading is bright. The ongoing innovation in this field promises to create even more powerful and intuitive tools that will further enrich the lives of visually impaired students, ensuring that the world of knowledge is truly open to all.\supercite{maitraye2024, arxiv2503}
