\chapter{PDF Accessibility: Auditing, Tagging, and Remediation}
\label{chap:pdf-accessibility-report}

\section{Introduction}
\label{sec:pdf-report-introduction}
This chapter provides a comprehensive guide to PDF\index{PDF} accessibility, covering auditing, tagging, and remediation\index{accessibility!remediation}. It is intended for developers, content creators, and accessibility\index{accessibility} professionals who need to ensure that PDF documents are compliant with modern accessibility standards\index{accessibility!standards}.

\section{Understanding PDF Accessibility Standards}
\label{sec:pdf-accessibility-standards}
Ensuring PDF accessibility\index{PDF!PDF accessibility} requires adherence to established international standards. The two most important standards are the Web Content Accessibility Guidelines (WCAG\index{WCAG}) and PDF/UA\index{PDF!UA} (ISO 14289).

\subsection{Web Content Accessibility Guidelines (WCAG)}
\label{subsec:wcag-pdf}

\subsubsection{Overview}
\label{ssubsec:wcag-pdf-overview}
\gls{WCAG}, developed by the World Wide Web Consortium (W3C), is the most widely recognized international standard for \gls{webaccessibility}. While primarily aimed at web content, its principles are applicable to PDFs, which are often distributed via the web.

\subsection{PDF/UA (ISO 14289-1)}
\label{subsec:pdf-ua}
PDF/UA, or Universal Access, is the international standard specifically for accessible PDF technology\index{technology}. It provides a set of requirements for creating fully accessible PDF documents and applications that read them.

\subsubsection{Overview}
\label{ssubsec:pdf-ua-overview}
Key requirements of \gls{pdf} include:
\begin{itemize}
	\item \textbf{Tagged Content}: All meaningful content must be tagged in a logical structure.
	\item \textbf{Logical Reading Order}: The reading order must be specified and logical.
	\item \textbf{Alternate Text}: All figures and graphics must have alternative text\index{images and media!alternative text}.
	\item \textbf{Document Title and Language}: The document title and primary language must be specified.
	\item \textbf{Interactive Elements}: Form fields, links\index{accessibility!links}, and annotations must be accessible.
	\item \textbf{Security}: Security settings must not interfere with assistive technology\index{assistive technology}.
\end{itemize}

\section{Tools for PDF Accessibility Auditing}
\label{sec:pdf-auditing-tools}
A variety of tools\index{sonification!tools} are available for auditing the accessibility of PDF\index{PDF} documents. These range from comprehensive desktop \index{desktop!accessibility!tools!desktop} applications to convenient online \index{online!accessibility!tools!online} checkers.

\subsection{Adobe Tools}
\label{subsec:adobe-tools}
Adobe, the creator of PDF, provides powerful tools for accessibility auditing and remediation\index{accessibility!remediation}.

\subsubsection{Adobe Acrobat Pro (Windows, macOS)}
\label{ssubsec:adobe-acrobat-pro}
Adobe Acrobat Pro is the industry-standard tool for creating and editing PDFs. Its built-in accessibility checker can test against both WCAG\index{WCAG!WCA2.1} 2.1 and PDF/UA standards\index{accessibility!standards}.
\begin{itemize}
	\item \textbf{Features}: Full accessibility check\index{accessibility!accessibility testing}, reading order tool, tags panel, autotagging, and remediation\index{accessibility!remediation strategies} tools.
	\item \textbf{Pros}: Comprehensive, industry-standard, allows for both auditing and remediation.
	\item \textbf{Cons}: Requires a paid subscription.
\end{itemize}

\subsection{Non-Adobe Tools}
\label{subsec:non-adobe-tools}
Several third-party tools also offer robust PDF \gls{accessibility}!\gls{accessibility} \index{auditing!testing} features.

\subsubsection{Desktop Applications}
\label{ssubsec:desktop-apps-pdf}
\begin{itemize}
	\item \textbf{axesPDF (Windows)}: A powerful plug-in for Microsoft documents\index{PDF!source documents!Word} that helps create PDF/UA\index{PDF!PDF/UA}-compliant documents from the source.
	      \begin{itemize}
		      \item \textbf{Pros}: Excellent for creating accessible PDFs from Word.
		      \item \textbf{Cons}: Windows-only, commercial software\index{software}.
	      \end{itemize}
	\item \textbf{CommonLook PDF Validator (Windows)\supercite{AllyantValidator}}: A free tool for testing PDF documents against Section 508\index{accessibility!legal accessibility}, WCAG 2.0, and PDF/UA standards.
	      \begin{itemize}
		      \item \textbf{Pros}: Free, comprehensive validation.
		      \item \textbf{Cons}: Windows-only, does not offer remediation tools\index{sonification!tools}.
	      \end{itemize}
	\item \textbf{Foxit PDF Editor (Windows, macOS, Linux\index{operating system!Linux})}: An alternative to Acrobat\index{PDF!Adobe Acrobat} with its own set of accessibility\index{accessibility} tools.
	      \begin{itemize}
		      \item \textbf{Pros}: Cross-platform, often more affordable than Acrobat.
		      \item \textbf{Cons}: Accessibility features may not be as mature as Acrobat's.
	      \end{itemize}
\end{itemize}

\subsubsection{Online Tools (Cross-Platform)}
\label{ssubsec:online-tools-pdf}
\begin{itemize}
	\item \textbf{PDF Accessibility Checker (PAC) 2021}: Developed by the PDF/UA\index{PDF!UA} Foundation, PAC is a definitive tool for checking PDF/UA compliance.
	      \begin{itemize}
		      \item \textbf{Pros}: Authoritative PDF/UA checker, free.
		      \item \textbf{Cons}: Windows-only desktop application, not an online tool.
	      \end{itemize}
	\item \textbf{PAVE (PDF Accessibility Validation Engine)}: An online tool that can identify accessibility issues and perform some automatic remediation\index{accessibility!remediation strategies}.
	      \begin{itemize}
		      \item \textbf{Pros}: Free, web-based, provides some automatic fixes.
		      \item \textbf{Cons}: May not catch all issues, less comprehensive than desktop tools.
	      \end{itemize}
\end{itemize}

\section{Conducting a Full Accessibility Audit}
\label{sec:full-accessibility-audit}
A full accessibility audit\index{accessibility!accessibility testing} involves a combination of automated tools and manual checks to ensure compliance with accessibility standards\index{accessibility!standards}.

\subsection{General Audit Steps}
\label{subsec:general-audit-steps}
\begin{enumerate}
	\item \textbf{Automated Check}: Run the PDF\index{PDF} through an automated checker like Adobe Acrobat's or PAC 2021.
	\item \textbf{Manual Verification}: Manually check the items that automated tools flag for manual review.
	      \begin{itemize}
		      \item \textbf{Reading Order}: Verify that the reading order is logical.
		      \item \textbf{Color Contrast}: Use a color contrast\index{accessibility!Manual Testing} analyzer to check text and background colors.
		      \item \textbf{Alternate Text}: Ensure that alt text\index{images and media!alternative text} is meaningful and descriptive.
	      \end{itemize}
	\item \textbf{Assistive Technology Testing}: Test the document with a screen reader\index{screen reader} (e.g., NVDA, JAWS, VoiceOver\index{screen reader!VoiceOver}) to experience it as a user with a visual impairment\index{visual impairment} would.
	\item \textbf{Generate Report}: Document all findings, including failures\index{accessibility!failures} and areas needing improvement.
\end{enumerate}

\subsection{Step-by-Step Audit in Adobe Acrobat Pro (Windows/macOS)}
\label{subsec:audit-in-acrobat}
\begin{enumerate}
	\item Open the PDF in Adobe Acrobat Pro.
	\item Go to \textbf{Tools} > \textbf{Accessibility}.
	\item In the Accessibility panel, click \textbf{Full Check}.
	\item In the \textbf{Accessibility Checker Options} dialog, ensure all checks are selected and click \textbf{Start Checking}.
	\item The results will appear in the \textbf{Accessibility Checker} panel on the left.
	\item Review each item in the report. Right-click on any failed item for options to fix it, explain the issue, or skip the rule.
	\item Manually inspect the following:
	      \begin{itemize}
		      \item \textbf{Tags Panel}: Open the \textbf{Tags} panel to check the document's structure and reading order.
		      \item \textbf{Reading Order Tool}: Use the \textbf{Reading Order} tool to visually inspect and correct the reading order on each page.
		      \item \textbf{Content Panel}: Check for any content that is not tagged and should be, or content that should be marked as an artifact.
	      \end{itemize}
\end{enumerate}

\subsection{Auditing with Non-Adobe Tools}
\label{subsec:auditing-non-adobe}
\begin{itemize}
	\item \textbf{Using PAC 2021 (Windows\index{operating system!Windows})}:
	      \begin{enumerate}
		      \item Open your PDF\index{PDF} in PAC 2021.
		      \item The tool will automatically perform a check against PDF/UA\index{PDF!UA} criteria.
		      \item The results are displayed in a clear, color-coded report.
		      \item Use the \textbf{Screenreader\index{screen reader} Preview} to see how a screen reader\index{screen reader} would interpret the document structure\index{document structure}.
	      \end{enumerate}
	\item \textbf{Using PAVE (Online)}:
	      \begin{enumerate}
		      \item Go to the PAVE website and upload your PDF.
		      \item PAVE will analyze the document and present a report!accessibility\index{accessibility} of issues.
		      \item For some issues, PAVE will offer to automatically apply a fix.
		      \item Download the remediated PDF and perform further manual checks.
	      \end{enumerate}
\end{itemize}

\section{Creating Fully Tagged PDFs from Source Documents}
\label{sec:creating-tagged-pdfs}
The most effective way to create an accessible PDF is to start with an accessible source document\index{PDF!source documents}. Proper use of authoring tools can automate much of the tagging process.

\subsection{Generating Tagged PDFs with LaTeX}
\label{subsec:tagged-pdfs-latex}
Creating tagged PDFs from LaTeX\index{LaTeX} can be challenging, but with the right packages and techniques, it is achievable.

\subsubsection{Best Practices for Accessible LaTeX PDFs}
\label{ssubsec:accessible-latex-pdfs}
\begin{itemize}
	\item \textbf{Use the \texttt{axessibility} package}: This package is designed to improve the accessibility of PDFs generated from LaTeX. It helps with tagging and structure.
	\item \textbf{Use the \texttt{hyperref} package}: This is essential for creating clickable links\index{accessibility!links} and bookmarks, which are important for navigation.
	      \begin{verbatim}
        \usepackage[pdfa,
                    pdftitle={My Document Title},
                    pdfauthor={Author Name},
                    pdfsubject={Subject},
                    pdfkeywords={Keywords},
                    pdfproducer={LaTeX},
                    pdfcreator={pdfLaTeX},
                    bookmarks=true,
                    bookmarksopen=true,
                    bookmarksnumbered=true,
                    hypertexnames=false,
                    colorlinks=true,
                    linkcolor=blue,
                    citecolor=blue,
                    urlcolor=blue,
                    breaklinks=true]{hyperref}
    \end{verbatim}
	\item \textbf{Provide Alternate Text for Images}: Use the \texttt{\textbackslash includegraphics} command with the \texttt{alt} option from the \texttt{axessibility} package.
	      \begin{verbatim}
        \includegraphics[alt={Descriptive alt text}]{image.png}
    \end{verbatim}
	\item \textbf{Use Semantic Commands}: Use semantic commands like \texttt{\textbackslash section}, \texttt{\textbackslash emph}, and \texttt{\textbackslash item} rather than manual formatting commands. This helps create a logical document structure\index{document structure}.
\end{itemize}

\subsection{Best Practices for Source Documents (Microsoft Word, Adobe InDesign)}
\label{subsec:best-practices-source-docs}
Most modern office applications have built-in tools\index{sonification!tools} to improve accessibility\index{accessibility}.

\subsubsection{Microsoft Word}
\label{ssubsec:ms-word-pdf}
\begin{itemize}
	\item \textbf{Use Styles}: Use built-in heading styles (Heading 1, Heading 2, etc.) to create a logical document structure.
	\item \textbf{Add Alt Text}: Right-click on any image and select \textbf{Edit Alt Text} to add a description.
	\item \textbf{Use the Accessibility Checker}: Go to \textbf{File} > \textbf{Info} > \textbf{Check for Issues} > \textbf{Check Accessibility}.
	\item \textbf{Create PDF\index{PDF}}: When saving as a PDF, go to \textbf{Options} and ensure that \textbf{Document structure tags for accessibility} is checked.
\end{itemize}

\subsubsection{Adobe InDesign}
\label{ssubsec:adobe-indesign-pdf}
\begin{itemize}
	\item \textbf{Paragraph Styles}: Use paragraph styles to define headings\index{Markdown!headings}, body text, lists, etc. Map these styles to export tags (\textbf{Edit All Export Tags}).
	\item \textbf{Object Export Options}: Add alt text to images via \textbf{Object Export Options}.
	\item \textbf{Articles Panel}: Use the \textbf{Articles} panel to define the reading order of your document.
	\item \textbf{Exporting to PDF}: When exporting to PDF (Print or Interactive), ensure that \textbf{Create Tagged PDF} is checked. For interactive PDFs, select \textbf{Use Structure for Tab Order}.
\end{itemize}

\section{Manually Tagging PDF Content}
\label{sec:manual-tagging-pdf}
Even when starting from a well-structured source document\index{PDF!source documents}, some manual tagging in Acrobat is often necessary to achieve full accessibility.

\subsection{Manual Tagging in Adobe Acrobat Pro (Windows/macOS)}
\label{subsec:manual-tagging-acrobat}
The \textbf{Tags} panel in Acrobat is the primary tool for manual tagging\index{PDF!tagged PDF}.

\subsubsection{Manual Tagging Steps}
\label{ssubsec:manual-tagging-steps}
\begin{enumerate}
	\item \textbf{Open the Tags Panel}: Go to \textbf{View} > \textbf{Show/Hide} > \textbf{Navigation Panes} > \textbf{Tags}.
	\item \textbf{Review Existing Tags}: Examine the tag tree to see if it accurately represents the document's structure.
	\item \textbf{Create New Tags}: Right-click in the Tags panel and select \textbf{New Tag} to create a new tag. Choose the appropriate tag type (e.g., \texttt{<H1>}, \texttt{<P>}, \texttt{<Figure>}).
	\item \textbf{Find Untagged Content}: In the Tags panel options, select \textbf{Find}. In the Find Element dialog, choose \textbf{Unmarked Content} and click \textbf{Find}.
	\item \textbf{Tag Untagged Content}: For each piece of unmarked content, click \textbf{Tag Element} and select the appropriate tag type.
	\item \textbf{Create Artifacts}: Some content (e.g., decorative lines, page numbers) should be ignored by screen readers\index{screen reader}. Use the Reading Order tool to select this content and mark it as \textbf{Background/Artifact}.
	\item \textbf{Correct Reading Order}: In the Tags panel, you can drag and drop tags to change their order. The order in the Tags panel determines the order in which a screen reader\index{screen reader} reads the content.
	\item \textbf{Add Alternate Text}: Find the \texttt{<Figure>} tag for an image, right-click it, and select \textbf{Properties}. In the \textbf{Tag} tab, enter the alternate text.
	\item \textbf{Tag Tables}: Ensure tables are correctly tagged with \texttt{<Table>}, \texttt{<TR>}, \texttt{<TH>} (for header cells), and \texttt{<TD>} (for data cells). Use the Table Editor in the Reading Order tool to associate header cells with data cells.
\end{enumerate}

\subsection{Manual Tagging in Non-Adobe Software}
\label{subsec:manual-tagging-non-adobe}
Some non-Adobe tools also provide manual tagging capabilities.

\subsubsection{Manual Tagging in Other Tools}
\label{ssubsec:manual-tagging-other-tools}
\begin{itemize}
	\item \textbf{Foxit PDF\index{PDF} Editor}: Provides a tag tree and tools for adding and editing tags, similar to Acrobat\index{PDF!Adobe Acrobat}.
	\item \textbf{axesPDF (with axesPDF for Word)}: While primarily a Word!source documents\index{PDF!source documents} add-in, it provides extensive control over the process during PDF creation, reducing the need for manual post-editing.
\end{itemize}

\section{Remediating Accessibility Audit Failures}
\label{sec:remediating-audit-failures}
Remediation\index{accessibility!remediation} is the process of fixing the accessibility issues identified during an audit.

\subsection{Common Accessibility Audit Failures}
\label{subsec:common-audit-failures}
\begin{itemize}
	\item \textbf{Untagged PDF}: The document has no tags at all.
	\item \textbf{Missing Alt Text}: Images are missing alternative text\index{images and media!alternative text}.
	\item \textbf{Incorrect Reading Order}: The reading order is illogical.
	\item \textbf{No Document Language or Title}: The document properties are incomplete.
	\item \textbf{Improperly Tagged Tables}: Tables lack proper header and data cell definitions.
	\item \textbf{Low Color Contrast}: Text is difficult to read against its background.
	\item \textbf{Untagged Links\index{accessibility!links} and Form Fields}: Interactive elements are not accessible.
\end{itemize}

\subsection{Remediation in Adobe Acrobat Pro (Windows/macOS)}
\label{subsec:remediation-in-acrobat}
Acrobat provides a rich set of tools for remediation.

\subsubsection{General Document Properties}
\label{ssubsec:general-doc-properties}
\begin{itemize}
	\item \textbf{Title}: Go to \textbf{File} > \textbf{Properties}. In the \textbf{Description} tab, enter a title. In the \textbf{Initial View} tab, set \textbf{Show} to \textbf{Document Title}.
	\item \textbf{Language}: In \textbf{File} > \textbf{Properties} > \textbf{Advanced}, select the document language\index{PDF!document language}.
\end{itemize}

\subsubsection{Tagging and Reading Order Issues}
\label{ssubsec:tagging-reading-order-issues}
\begin{itemize}
	\item \textbf{Autotag Document}: If the document is completely untagged, use \textbf{Tools} > \textbf{Accessibility} > \textbf{Autotag Document} as a starting point. This will require significant manual cleanup.
	\item \textbf{Reading Order Tool}: Use this tool to correct reading order, tag elements, and create artifacts.
	\item \textbf{Tags Panel}: Use this panel for fine-grained control over the tag tree, as described in the manual tagging section.
\end{itemize}

\subsubsection{Forms Remediation}
\label{ssubsec:forms-remediation}
\begin{itemize}
	\item \textbf{Prepare Form Tool}: Use the \textbf{Prepare Form} tool to add and edit form fields.
	\item \textbf{Tag Form Fields}: Ensure each form field is tagged correctly in the tag tree.
	\item \textbf{Add Tooltips}: Right-click a form field, select \textbf{Properties}, and in the \textbf{General} tab, add a descriptive \textbf{Tooltip}. The tooltip serves as the label for the form field for screen reader\index{screen reader} users.
	\item \textbf{Set Tab Order}: In the \textbf{Page Thumbnails} panel, select all pages, right-click, and choose \textbf{Page Properties}. In the \textbf{Tab Order} tab, select \textbf{Use Document Structure\index{document structure}}.
\end{itemize}

\subsubsection{Table Remediation}
\label{ssubsec:table-remediation}
\begin{itemize}
	\item \textbf{Reading Order Tool}: Select the table and click the \textbf{Table} button. This will show the table structure.
	\item \textbf{Table Editor}: Right-click the table in the Reading Order tool and select \textbf{Table Editor}.
	\item \textbf{Assign Header Cells}: In the Table Editor, right-click on cells that should be headers and set their \textbf{Type} to \textbf{Header Cell}.
	\item \textbf{Set Scope}: For each header cell, set its \textbf{Scope} to \textbf{Row}, \textbf{Column}, or \textbf{Both}.
	\item \textbf{Associate Headers with Cells}: For complex tables, you may need to manually assign a unique \textbf{ID} to each header cell and then, for each data cell, list the corresponding header IDs in its \textbf{Associated Header Cell IDs} property.
\end{itemize}

\subsubsection{List Remediation}
\label{ssubsec:list-remediation}
Ensure lists\index{Markdown!lists} are tagged with a \texttt{<L>} tag containing \texttt{<LI>} (list item) tags. Each \texttt{<LI>} tag should contain an \texttt{<LBL>} (label, e.g., the bullet or number) and an \texttt{<LBody>} (the text of the list item).

\subsubsection{Low Color Contrast}
\label{ssubsec:low-color-contrast-remediation}
This must be fixed in the source document\index{PDF!source documents}. It is not possible to change the color of text or backgrounds in a tagged PDF\index{PDF} without potentially breaking the tag structure.
\begin{enumerate}
	\item Identify the low-contrast elements.
	\item Return to the source document!
	\item Adjust the colors to meet at least WCAG\index{WCAG} AA contrast ratios (4.5:1 for normal text, 3:1 for large text).
	\item Regenerate the PDF.
\end{enumerate}

\subsection{Remediation in Non-Adobe Software}
\label{subsec:remediation-non-adobe}
Remediation\index{accessibility!remediation} in non-Adobe tools\index{sonification!tools} tools follows similar principles but with different interfaces.

\subsubsection{Remediation in Other Tools}
\label{ssubsec:remediation-other-tools}
\begin{itemize}
	\item \textbf{Pave}: As mentioned, Pave can automatically fix some issues, such as adding a document title or tagging some elements. However, its capabilities are limited.
	\item \textbf{Foxit PDF Editor}: Offers a suite of accessibility\index{accessibility} tools, including a tag editor, reading order tool, and autotagging features that can be used for remediation\index{accessibility!remediation strategies}.
	\item \textbf{CommonLook PDF GlobalAccess (Windows\index{operating system!Windows})\supercite{AllyantCommonLook}}: A commercial plugin for Acrobat that provides advanced remediation features, often simplifying complex tasks like table remediation.
\end{itemize}

\section{Conclusion}
\label{sec:pdf-report-conclusion}
Achieving full PDF accessibility is a multi-step process that begins with accessible authoring practices and continues with thorough auditing and remediation. By using the tools and techniques outlined in this chapter, you can create PDF\index{PDF} documents that are usable by everyone, regardless of ability.
