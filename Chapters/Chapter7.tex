\chapter{Beyond Boundaries: Text-to-Speech and DAISY as Catalysts for Literacy and Success in Visual Impairment Education}\label{audio}

The National Instructional Materials Accessibility Standard (NIMAS) and Digital Accessible Information System (DAISY) are two important tools for the education of students with visual impairments. NIMAS is a technical standard used by publishers to prepare “electronic files” that are used to convert instructional materials into accessible formats. The purpose of NIMAS is to help increase the availability and timely delivery of instructional materials in accessible formats for qualifying students in elementary and secondary schools\footnote{\href{https://aem.cast.org/nimas-nimac/nimas-nimac}{AEM Center. (n.d.). NIMAS \& NIMAC. Retrieved December 19, 2023}}. DAISY is a digital format for audio books that is designed to be more accessible to people with visual impairments. DAISY books can be read using specialized software that allows users to navigate through the book using headings, bookmarks, and other features\footnote{\href{https://daisy.org/about_us/what-is-daisy/}{DAISY Consortium. (n.d.). What is DAISY? Retrieved December 19, 2023}}.

NIMAS and DAISY are important because they help make educational materials more accessible to students with visual impairments. By providing instructional materials in accessible formats, students with visual impairments can participate more fully in the general education curriculum. This can help improve their academic performance and increase their chances of success in school.

Finally, NIMAS and DAISY can help students with visual impairments become more independent. By providing instructional materials in accessible formats, students can read books, take notes, and communicate with others more easily. This can help them lead more fulfilling lives and become more active members of their communities.

In the evolving landscape of education, the pursuit of literacy is a journey marked by innovation and inclusivity. For visually impaired students, the traditional pathways to literacy take on a distinctive form, guided by the transformative power of audiobook and DAISY readers. This chapter explores the indispensable role that these tools play in breaking down barriers to literacy, ensuring access to knowledge, and propelling visually impaired students towards academic success.

\section{DAISY Readers}\label{text-to-speech-music-podcast}

Assistive technology is a crucial tool for students with visual impairments or blindness to receive a free and appropriate public education. One such technology is the DAISY format, which is designed to provide an accessible and navigable format for digital books and other publications. DAISY books can be read using specialized software that provides text-to-speech functionality, allowing students to listen to the content of the book in a digitized voice. This technology can be a game-changer for students who struggle with reading text in written form, as it allows them to access the same materials as their peers.

DAISY is a standard format for digital audio books, magazines, and computerized text. DAISY-encoded educational content is an essential tool for students with visual impairments to receive a free and appropriate public education. DAISY books can be read with specialized software that allows the user to navigate through the book using bookmarks, headings, and other navigational aids. This allows students with visual impairments to access the same educational materials as their sighted peers. DAISY books can also be read aloud using text-to-speech software, which can help improve literacy skills and comprehension. Additionally, DAISY books can include tactile graphics, which can help students with visual impairments better understand complex concepts. By providing students with visual impairments access to DAISY-encoded educational content, we can help ensure that they have the tools they need to succeed in their studies and beyond.

\tagpdfsetup{table/header-rows={1}}
\centering
\begin{longtblr}[
  caption = {DAISY readers and digital audio players: models, function, and company},
  label = {tab:chapter7:daisy-readers},
  note = {Comprehensive list of DAISY-compatible devices for reading digital books and audio content, including portable and desktop options}
]{
  colspec = {X[l] X[l] X[l]},
  rowhead = 1,
  hlines,
  stretch = 1.5,
}
Model & Function & Company \\
Milestone 212 Ace Book Reader & DAISY Reader, Digital Audio Player & Bones \\
PlexTalk PTN2 & DAISY Reader, CD Player & PlexTalk \\
PlexTalk Pocket & DAISY Reader, Digital Audio Player & PlexTalk \\
Reizen DAISY Digital Recorder & DAISY Reader, Digital Audio Player & Reizen \\
Victor Reader Stratus & DAISY Reader, Digital Audio Player (Not very portable) & Humanware \\
Victor Reader Stream & Digital Audio Player & Humanware \\
Victor Reader Trek & GPS, Digital Audio Player & Humanware \\
\end{longtblr}

\section{Text-to-Speech}\label{text-to-speech}

The use of assistive technology, including Text-to-Speech, is required for all students with disabilities that show a need under the Individuals with Disabilities Education Act (IDEIA)\footnote{\href{http://sites.ed.gov/idea/statuteregulations/}{20 U.S.C. § 1400, et.}}. Text-to-Speech technology is a powerful tool that can help students with visual impairments or blindness receive a free and appropriate public education.

\tagpdfsetup{table/header-rows={1}}
\centering
\begin{longtblr}[
  caption = {Text-to-speech devices: model, function, and company},
  label = {tab:chapter7:text-to-speech-devices},
  note = {Available text-to-speech devices ranging from handheld scanners to wearable solutions, with their respective functionalities}
]{
  colspec = {X[l] X[l] X[l]},
  rowhead = 1,
  hlines,
  stretch = 1.5,
}
Model & Function & Company \\
c-Pen2 & Pen Scanner, Text-to-Speech Reader & c-Pen \\
MyEye Pro & Glasses Mounted, Text to Speech & OrCam \\
LyriQ & Text to Speech & Zyrlo \\
Read 3 & Hand-held Text to Speech & OrCam \\
Scanmarker Air & Hand-held Text to Speech & Scanmarker \\
Smart Reader HD & Portable Text to Speech & Enhanced Vision \\
\end{longtblr}
