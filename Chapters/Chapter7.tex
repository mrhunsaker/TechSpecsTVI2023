\hypertarget{audio}{}\chapter[\raggedright Beyond Boundaries:\hfill\break Text-to-Speech and DAISY as Catalysts for Literacy and Success\hfill\break in Visual Impairment Education]{Beyond Boundaries: Text-to-Speech and DAISY as Catalysts for Literacy and Success in Visual Impairment Education}\label{audio}
\extramarks{Vision Department Technology Needs}{Chapter 7: Beyond Boundaries}
\noindent\makebox[\linewidth]{\rule{\linewidth}{0.4pt}}
{\let\clearpage\relax\localtableofcontents\let\clearpage\relax\locallistoftables}\newpage
The National Instructional Materials Accessibility Standard (NIMAS) and Digital Accessible Information System (DAISY) are two important tools for the education of students with visual impairments. NIMAS is a technical standard used by publishers to prepare “electronic files” that are used to convert instructional materials into accessible formats. The purpose of NIMAS is to help increase the availability and timely delivery of instructional materials in accessible formats for qualifying students in elementary and secondary schools\footnote{\raggedright \href{https://aem.cast.org/nimas-nimac/nimas-nimac}{AEM Center. (n.d.). NIMAS \& NIMAC. Retrieved December 19, 2023}}. DAISY is a digital format for audio books that is designed to be more accessible to people with visual impairments. DAISY books can be read using specialized software that allows users to navigate through the book using headings, bookmarks, and other features\footnote{\raggedright \href{https://daisy.org/about_us/what-is-daisy/}{DAISY Consortium. (n.d.). What is DAISY? Retrieved December 19, 2023}}.

NIMAS and DAISY are important because they help make educational materials more accessible to students with visual impairments. By providing instructional materials in accessible formats, students with visual impairments can participate more fully in the general education curriculum. This can help improve their academic performance and increase their chances of success in school.

Finally, NIMAS and DAISY can help students with visual impairments become more independent. By providing instructional materials in accessible formats, students can read books, take notes, and communicate with others more easily. This can help them lead more fulfilling lives and become more active members of their communities.

In the evolving landscape of education, the pursuit of literacy is a journey marked by innovation and inclusivity. For visually impaired students, the traditional pathways to literacy take on a distinctive form, guided by the transformative power of audiobook and DAISY (Digital Accessible Information System) readers. This chapter embarks on a profound exploration of the indispensable role that these tools play in breaking down barriers to literacy, ensuring access to a rich tapestry of knowledge, and propelling visually impaired students towards academic success.

The quest for literacy is deeply entwined with the ability to access and engage with textual content. Audiobook and DAISY readers emerge as champions in this narrative, providing a dynamic bridge that transcends the limitations posed by traditional print materials. As we delve into this chapter, we will unravel the nuanced functionalities of these tools, showcasing how they offer not just access to literature but a pathway to immersive and engaging learning experiences.

In subjects ranging from literature to science, where the written word is the gateway to understanding, Audiobook and DAISY readers become indispensable companions for visually impaired students. This chapter will explore how these technologies facilitate not only independent reading but also active participation in classroom discussions and assignments. By providing a platform that transforms written content into spoken words or navigable digital formats, these tools empower students to explore the vast realms of knowledge with autonomy and confidence.

The significance of Audiobook and DAISY readers extends beyond mere accessibility; they are key enablers of success. It is evident that these tools are not just aids for the visually impaired; they are essential components of an inclusive education landscape. Audiobook and DAISY readers ensure that every student, regardless of their visual abilities, can embark on a literary journey that is both enriching and empowering.

\pagebreak \hypertarget{text-to-speech-music-podcast}{}\section{DAISY Readers}\label{text-to-speech-music-podcast}

Assistive technology is a crucial tool for students with visual impairments or blindness to receive a free and appropriate public education. One such technology is the Digital Accessible Information System (DAISY) format, which is designed to provide an accessible and navigable format for digital books and other publications. DAISY books can be read using specialized software that provides text-to-speech functionality, allowing students to listen to the content of the book in a digitized voice. This technology can be a game-changer for students who struggle with reading text in written form, as it allows them to access the same materials as their peers.

DAISY (Digital Accessible Information System) is a standard format for digital audio books, magazines, and computerized text. DAISY-encoded educational content is an essential tool for students with visual impairments to receive a free and appropriate public education. DAISY books can be read with specialized software that allows the user to navigate through the book using bookmarks, headings, and other navigational aids. This allows students with visual impairments to access the same educational materials as their sighted peers. DAISY books can also be read aloud using text-to-speech software, which can help improve literacy skills and comprehension. Additionally, DAISY books can include tactile graphics, which can help students with visual impairments better understand complex concepts. By providing students with visual impairments access to DAISY-encoded educational content, we can help ensure that they have the tools they need to succeed in their studies and beyond. Table \ref{tab:table22} lists current available DAISY readers.

\pagebreak 
 
\begin{longtable}[]{@{}
	>{\raggedright\arraybackslash}m{.35\textwidth}
	>{\raggedright\arraybackslash}m{.15\textwidth}
	>{\raggedright\arraybackslash}m{.25\textwidth}
	>{\raggedright\arraybackslash}b{.2\textwidth}@{}
	}
	\toprule
	
	\textbf{Model}                & \textbf{Cost} & \textbf{Function}                                               & \textbf{Company} \\
	\midrule
	\endhead \hline                                                                                                                      \\
	\multicolumn{4}{r}{\textbf{Continued on Next Page}} \endfoot
	\endlastfoot
	Milestone 212 Ace Book Reader & \$380         & DAISY Reader\break Digital Audio Player                         & Bones            \\ \cdashline{1-4}
	PlexTalk PTN2                 & \$375         & DAISY Reader\break CD Player                                    & PlexTalk         \\ \cdashline{1-4}
	PlexTalk Pocket               & \$275         & DAISY Reader\break Digital Audio Player                         & PlexTalk         \\ \cdashline{1-4}
	Reizen DAISY Digital Recorder & \$219         & DAISY Reader\break Digital Audio Player                         & Reizen           \\ \cdashline{1-4}
	Victor Reader Stratus         & \$495         & DAISY Reader\break Digital Audio Player\break Not very portable & Humanware        \\ \cdashline{1-4}
	Victor Reader Stream          & \$550         & Digital Audio Player                                            & Humanware        \\ \cdashline{1-4}
	Victor Reader Trek            & \$975         & GPS\break Digital Audio Player                                  & Humanware        \\[1.0em]\hline
	\caption{ DAISY/Audiobook/Podcast Devices }\label{tab:table22}
\end{longtable}\clearpage

\pagebreak \hypertarget{text-to-speech}{}\section{Text-to-Speech}\label{text-to-speech}
The use of assistive technology, including Text-to-Speech, is required for all students with disabilities that show a need under the Individuals with Disabilities Education Act (IDEIA)\footnote{\raggedright \href{http://sites.ed.gov/idea/statuteregulations/}{20 U.S.C. § 1400, et.}}. It is important to note that the use of assistive technology helps prepare students for independent living, vocational pursuits, or higher education following graduation from high school. 

Text-to-Speech technology is a powerful tool that can help students with visual impairments or blindness receive a free and appropriate public education. These students face unique challenges in the educational environment, as they must be able to access text information across all curricular areas and participate fully in instruction that is often rich with visual content. Assistive technology is one way of supporting them in that process. Text-to-Speech software allows the computer to “read” digital text to the student in a digitized voice, which can be a game-changer for students who struggle with text in written form. Some programs will highlight words as they are read, allowing students to follow along. Refreshable braille displays can be connected to the digital text source, providing students with the option to read the text tactually Text-to-speech technology offers an alternative way for students of all ages and learning abilities to access a variety of texts with ease. Using it correctly, the text-to-speech benefits students who struggle with text in written form By providing students with visual impairments access to text-to-speech software, we can help ensure that they have the tools they need to succeed in their studies and beyond. Table \ref{tab:table23} lists current available text-to-speech devices.

\pagebreak 
 
\begin{longtable}[]{@{}
	>{\raggedright\arraybackslash}m{.25\textwidth}
	>{\raggedright\arraybackslash}m{.15\textwidth}
	>{\raggedright\arraybackslash}m{.25\textwidth}
	>{\raggedright\arraybackslash}b{.2\textwidth}@{}
	}
	\toprule
	
	\textbf{Model}  & \textbf{Cost} & \textbf{Function}                       & \textbf{Company} \\
	\midrule
	\endhead \hline                                                                                              \\
	\multicolumn{4}{r}{\textbf{Continued on Next Page}} \endfoot
	\endlastfoot
	c-Pen2          & \$399         & Pen Scanner\break Text-to-Speech Reader & c-Pen            \\ \cdashline{1-4}
	MyEye Pro       & \$4,250       & Glasses Mounted\break Text to Speech    & OrCam            \\ \cdashline{1-4}
	LyriQ           & \$1,960       & Text to Speech                          & Zyrlo            \\ \cdashline{1-4}
	Read 3          & \$2,490       & Hand-held Text to Speech                & OrCam            \\ \cdashline{1-4}
	Scanmarker Air  & \$150         & Hand-held Text to Speech                & Scanmarker       \\ \cdashline{1-4}
	Smart Reader HD & \$2,495       & Portable Text to Speech                 & Enhanced Vision  \\[1.0em]\hline
	\caption{ Text-to-Speech Devices}\label{tab:table23}
\end{longtable}\clearpage
