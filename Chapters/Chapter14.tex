\chapter{Computer Specifications and Embosser Compatibility for Tactile Graphics Software}

\section{Executive Summary}

This report provides an overview of software solutions for generating tactile graphics for individuals who are blind or visually impaired, covering Windows, MacOS, and Linux operating systems. It includes commercial and open-source applications, detailing their computer specifications, embosser compatibility, and limitations.

Commercial tactile graphics software is predominantly centered on Windows platforms, with MacOS users often requiring Windows emulators. Linux supports some open-source vector graphics editors but has fewer dedicated tactile graphics applications. Hardware requirements are generally moderate, with increased RAM enhancing performance for complex design tasks.

Embosser compatibility varies widely. Some programs support multiple embosser brands, while others are integrated with specific ecosystems. The adoption of Scalable Vector Graphics (SVG) as an intermediate file format facilitates interoperability between design tools and tactile output devices.

The report categorizes software into specialized tools for tactile graphic creation and general-purpose software adaptable for tactile output. The choice depends on user needs, technical proficiency, and the required tactile output type.

\section{Introduction to Tactile Graphics Technology}

\subsection{The Role of Tactile Graphics for the Visually Impaired}

Tactile graphics are specialized images designed to be interpreted through touch, providing critical non-textual information to individuals who are blind or have low vision. These graphics encompass a wide array of visual data, including pictures, diagrams, maps, and graphs. Their importance cannot be overstated, as they serve as essential tools for orientation, route planning, educational attainment, and fostering greater independence and inclusion in a visually-dominated world. By converting visual information into raised lines, textures, and patterns, tactile graphics enable visually impaired individuals to access and comprehend spatial relationships, abstract concepts, and complex data that would otherwise be inaccessible.\footnote{\url{https://www.perkins.org/resource/tactile-graphics-library/} \url{https://www.tactileview.com/} \url{https://bpb-us-e1.wpmucdn.net/wp.nyu.edu/dist/c/1540/files/2022/08/Design-Guidelines-for-Tactile-Maps.pdf} \url{https://touch-mapper.org/en/help} \url{https://aeldata.com/stem-accessibility/tactile-graphics/}}

\subsection{Overview of Software Categories for Tactile Production}

The ecosystem of software for tactile graphics production is diverse, comprising several distinct categories. Dedicated tactile design tools are purpose-built for creating and editing tactile content, often featuring specialized functions for braille labeling and multi-dot height embossing. Braille translation software frequently includes integrated graphics capabilities, allowing for the combination of braille text with tactile images within a single document. Specialized map generators focus on creating accessible tactile maps from geographical data. Lastly, general-purpose vector graphics editors, though not designed exclusively for tactile output, can be adapted to produce files suitable for embossing or other tactile printing methods. These software categories collectively facilitate the transformation of visual designs into physical tactile outputs via braille embossers, swell paper machines, or 3D printers.\footnote{\url{https://touch-mapper.org/en/help} \url{https://www.duxburysystems.com/dbt_details.asp} \url{https://irie-at.com/product/tactileview-design-software/} \url{https://bpb-us-e1.wpmucdn.net/wp.nyu.edu/dist/c/1540/files/2022/08/Workflow-For-In-House-Production-Of-Paper-Tactile-Maps.pdf} \url{https://getbraille.com/tactile-graphics/} \url{https://www.problind.org/en/create/} \url{https://www.duxburysystems.com/products.asp} \url{https://blindsvg.com/}}

\section{Commercial Tactile Graphics Software}

\subsection{TactileView Design Software}

\subsubsection{Core Functionality and Features}

TactileView is a robust and comprehensive design software specifically engineered for creating tactile graphics. It offers an intuitive editor with a wide array of drawing tools and image processing filters, allowing users to easily design and produce tactile pictures, diagrams, and maps.\footnote{\url{https://www.duxburysystems.com/dbt_details.asp} \url{https://blindsvg.com/}} A notable feature is its integrated math module, which can generate graphs directly from equations, and the powerful RouteTactile map maker, capable of producing tactile maps at various scales, from country-level overviews to detailed street layouts.\footnote{\url{https://www.duxburysystems.com/dbt_details.asp} \url{https://blindsvg.com/} \url{https://www.duxburysystems.com/dbt_details.asp} \url{https://www.duxburysystems.com/news.asp} \url{https://www.duxburysystems.com/dbt_details.asp}}

The software prioritizes accessibility, ensuring full compatibility with mouse, keyboard, and screen readers, thereby providing a seamless experience for all users.\footnote{\url{https://www.duxburysystems.com/dbt_details.asp} \url{https://blindsvg.com/} \url{https://www.duxburysystems.com/dbt_details.asp} \url{https://www.duxburysystems.com/dbt_details.asp}} TactileView supports the import of a broad range of image file formats, including.txt,.svg,.jpg,.png,.bmp,.tiff, and.gif, and saves designs in its proprietary.bpx format.\footnote{\url{https://www.duxburysystems.com/dbt_details.asp} \url{https://blindsvg.com/} \url{https://www.duxburysystems.com/news.asp} \url{https://www.duxburysystems.com/dbt_details.asp} \url{https://www.duxburysystems.com/dbt_details.asp}} An extensive online catalog provides access to thousands of ready-to-use designs, enhancing efficiency for creators.\footnote{\url{https://www.duxburysystems.com/dbt_details.asp} \url{https://blindsvg.com/} \url{https://www.duxburysystems.com/news.asp} \url{https://www.duxburysystems.com/dbt_details.asp}} While TactileView can be used offline, certain advanced functions, such as the map maker and access to the designs catalog, require an internet connection.\footnote{\url{https://www.duxburysystems.com/dbt_details.asp} \url{https://blindsvg.com/} \url{https://www.duxburysystems.com/news.asp} \url{https://www.duxburysystems.com/dbt_details.asp} \url{https://www.duxburysystems.com/dbt_details.asp}} A significant advantage is its ability to integrate with Duxbury Braille Translator (DBT), allowing for the creation of documents that combine both tactile graphics and braille text.\footnote{\url{https://www.duxburysystems.com/dbt_details.asp}}

\subsubsection{System Requirements (Windows, MacOS via Emulator)}

TactileView is primarily a Windows-native application. It supports a broad spectrum of Microsoft Windows operating systems, including older versions such as XP, Vista, 7, 8, 8.1, and newer iterations like Windows 10 and 11.\footnote{\url{https://www.duxburysystems.com/dbt_details.asp} \url{https://blindsvg.com/} \url{https://www.duxburysystems.com/news.asp} \url{https://www.duxburysystems.com/dbt_details.asp} \url{https://www.duxburysystems.com/dbt_details.asp}} For users operating on Mac OS X, a Windows emulator is explicitly required to run TactileView, which introduces an additional layer of software and potential performance considerations.\footnote{\url{https://www.duxburysystems.com/dbt_details.asp} \url{https://blindsvg.com/} \url{https://www.duxburysystems.com/news.asp} \url{https://www.duxburysystems.com/dbt_details.asp} \url{https://www.duxburysystems.com/dbt_details.asp}}

The provided information does not specify explicit minimum or recommended CPU, RAM, or storage requirements for TactileView.\footnote{\url{https://www.duxburysystems.com/dbt_details.asp} \url{https://blindsvg.com/} \url{https://www.duxburysystems.com/news.asp} \url{https://www.duxburysystems.com/dbt_details.asp} \url{https://www.duxburysystems.com/dbt_details.asp}} This suggests that the software is designed to operate efficiently on a wide range of standard, modern computer systems, and does not typically demand high-end hardware specifications beyond what is generally expected for the supported operating systems.

\subsubsection{Embosser and Thermal Device Compatibility}

TactileView demonstrates extensive compatibility with a wide array of braille embossers and thermal tactile graphics machines. It supports embossers from major manufacturers, including Index (V2, V3, V4, Basic-D V5, Everest-D V5), ViewPlus (all models, such as VP Columbia 2, VP Delta 2, VP Max, VP SpotDot, VP Premier, VP Elite), and Enabling Technologies (all models, including Romeo 60 and Juliet 120).\footnote{Source:  \url{https://www.duxburysystems.com/dbt_details.asp} \url{https://blindsvg.com/} \url{https://www.duxburysystems.com/dbt_details.asp} \url{https://www.duxburysystems.com/dbt_details.asp}} The software also supports printing on swell paper (microcapsule paper) and is compatible with thermal devices like PIAF (Pictures in a Flash) and Swell Form machines.\footnote{Source:  \url{https://www.duxburysystems.com/dbt_details.asp} \url{https://www.duxburysystems.com/dbt_details.asp} \url{https://www.duxburysystems.com/dbt_details.asp}} This broad compatibility ensures that users with various existing embossing hardware can leverage TactileView for their tactile graphic production needs.

\subsection{ElPicsPrint}

\subsubsection{Core Functionality and Features}

ElPicsPrint is a specialized software designed for the preparation and embossing of tactile images on braille embossers.\footnote{Source:  \url{https://irie-at.com/product/tactileview-design-software/}} It offers a comprehensive suite of tools for object processing, including scaling, rotating, flipping, grouping, and alignment.\footnote{Source:  \url{https://irie-at.com/product/tactileview-design-software/}} Users can modify contour properties such as stroke thickness, line type (solid, dotted), fill color, and tactile texture type, including dot height.\footnote{Source:  \url{https://irie-at.com/product/tactileview-design-software/}} The software also facilitates the creation of complex shapes through operations like adding, subtracting, or intersecting contours, and allows for layering objects with adjustable overlay order (Z-index).\footnote{Source:  \url{https://irie-at.com/product/tactileview-design-software/}} For full-color images, ElPicsPrint provides options to adjust upper and lower thresholds for image detection, which can significantly influence the embossed output.\footnote{Source:  \url{https://irie-at.com/product/tactileview-design-software/}}

A key strength of ElPicsPrint is its robust text features. It enables the creation and editing of text fields, with automatic conversion to braille using the Liblouis library and selected translation tables.\footnote{Source:  \url{https://irie-at.com/product/tactileview-design-software/}} Users can also input braille directly via six-key entry and utilize braille label recognition to differentiate braille text from the image.\footnote{Source:  \url{https://irie-at.com/product/tactileview-design-software/}} For printing, ElPicsPrint supports various methods, including its own dot algorithm for braille embossing, direct printing to microcapsule paper, and utilizing standard operating system printer drivers.\footnote{Source:  \url{https://irie-at.com/product/tactileview-design-software/}} It supports common paper sizes like A4, A3, Letter, Legal, and Ledger, and allows for margin adjustments and image rotation for portrait orientation.\footnote{Source:  \url{https://irie-at.com/product/tactileview-design-software/}} The software can import.jpg,.png, and.bmp image files, and supports opening and saving Scalable Vector Graphics (SVG) files (contours only).\footnote{Source:  \url{https://irie-at.com/product/tactileview-design-software/}} It also uses its own.elpe (editable SVG) and.elpp (ready-to-emboss dot information) file formats.\footnote{Source:  \url{https://www.duxburysystems.com/dbt_details.asp}}

\subsubsection{System Requirements (Windows)}

ElPicsPrint is designed to run on Windows operating systems. While the research material explicitly states its compatibility with various embossers and its ability to process SVG files, it \emph{does not provide specific minimum or recommended CPU, RAM, or storage requirements} for the software itself.\footnote{Source:  \url{https://irie-at.com/product/tactileview-design-software/} \url{https://www.duxburysystems.com/dbt_details.asp}} This absence of detailed hardware specifications suggests that ElPicsPrint is likely optimized to operate effectively on standard Windows computers without demanding high-performance components. The focus appears to be on its compatibility with a wide range of output devices rather than stringent computational demands.

\subsubsection{Embosser and Thermal Device Compatibility}

ElPicsPrint offers broad compatibility with a range of braille embossers and thermal tactile graphics devices. It is specifically designed to work with embossers manufactured by Index Braille, ViewPlus, and Enabling Technologies.\footnote{\url{https://irie-at.com/product/tactileview-design-software/}}

Supported Embosser Models:
\begin{itemize}
    \item \emph{Index Braille:} Everest-D, BrailleBox, Basic-D, and FanFold-D (version 4 or higher).\footnote{\url{https://irie-at.com/product/tactileview-design-software/} \url{https://www.duxburysystems.com/dbt_details.asp}}
    \item \emph{ViewPlus:} VP Delta, VP Columbia, VP Roque, VP EmBraille, VP Max, VP SpotDot, VP Elite, VP Premier.\footnote{\url{https://irie-at.com/product/tactileview-design-software/} \url{https://www.duxburysystems.com/dbt_details.asp}}
    \item \emph{Enabling Technologies:} Romeo and Juliet.\footnote{\url{https://irie-at.com/product/tactileview-design-software/} \url{https://www.duxburysystems.com/dbt_details.asp}}
\end{itemize}

In addition to embossers, ElPicsPrint supports devices with thermal technology for printing on microcapsule paper, including Swell Form and PIAF machines.\footnote{Source:  \url{https://irie-at.com/product/tactileview-design-software/}} This extensive compatibility allows users to select from a wide range of hardware for producing tactile graphics.

\subsection{Duxbury Braille Translator (DBT) \& QuickTac}

\subsubsection{Core Functionality and Features (Braille Translation with Graphics Integration)}

Duxbury Braille Translator (DBT) is an industry-leading software primarily known for its robust print-to-braille translation capabilities, supporting over 180 languages and various braille codes, including UEB (Unified English Braille).\footnote{Source:  \url{https://www.duxburysystems.com/dbt_details.asp} \url{https://www.perkins.org/resource/how-create-3d-printable-maps-using-touch-mapper/} \url{https://viewplus.com/downloads/cutsheets/TSS8_PS_EN_8.5x11_2023.pdf} \url{https://irie-at.com/product/brailletrac-120/} \url{https://elitagroup.com/prod/elpicsprint/} \url{https://viewplus.com/product/tiger-software-suite9/}} It excels at formatting braille pages, automating the conversion process, and allowing direct editing in both print and braille views.\footnote{Source:  \url{https://www.perkins.org/resource/how-create-3d-printable-maps-using-touch-mapper/}} DBT can import a wide range of modern file formats, including Microsoft Word documents (2003-365), Open Office, and Excel files.\footnote{Source:  \url{https://www.perkins.org/resource/how-create-3d-printable-maps-using-touch-mapper/} \url{https://irie-at.com/product/brailletrac-120/} \url{https://elitagroup.com/prod/elpicsprint/} \url{https://viewplus.com/product/tiger-software-suite9/}} It also supports mathematics and science text translation into braille.\footnote{Source:  \url{https://www.perkins.org/resource/how-create-3d-printable-maps-using-touch-mapper/} \url{https://elitagroup.com/prod/elpicsprint/} \url{https://viewplus.com/product/tiger-software-suite9/}}

While DBT's core strength is braille translation, it possesses the crucial ability to include tactile graphics files for mixed text-and-graphic documents.\footnote{Source:  \url{https://touch-mapper.org/en/help} \url{https://www.perkins.org/resource/how-create-3d-printable-maps-using-touch-mapper/} \url{https://elitagroup.com/prod/elpicsprint/} \url{https://elitagroup.com/prod/elpicsprint/}} This integration is largely facilitated by \emph{QuickTac}, a companion freeware program from Duxbury Systems specifically designed for creating embosser graphics.\footnote{Source:  \url{https://touch-mapper.org/en/help} \url{https://elitagroup.com/prod/elpicsprint/} \url{https://elitagroup.com/prod/elpicsprint/} \url{https://viewplus.com/downloads/cutsheets/TSS8_PS_EN_8.5x11_2023.pdf} \url{https://www.duxburysystems.com/dbt_details.asp} \url{https://softorage.com/software/inkscape/}} QuickTac functions as a paint program that builds a grid of dots, which can then be directly embossed or saved into a file (specifically the.SIG format) for import into DBT.\footnote{Source:  \url{https://touch-mapper.org/en/help} \url{https://www.duxburysystems.com/dbt_details.asp}} This workflow addresses the challenge of tactile graphics, where a continuous line is often difficult to achieve with embossers.\footnote{Source:  \url{https://viewplus.com/downloads/cutsheets/TSS8_PS_EN_8.5x11_2023.pdf}} The integration of QuickTac allows DBT users to produce documents containing both braille text and tactile illustrations, enhancing accessibility for complex materials.\footnote{Source:  \url{https://touch-mapper.org/en/help} \url{https://www.duxburysystems.com/dbt_details.asp} \url{https://www.perkins.org/resource/how-create-3d-printable-maps-using-touch-mapper/} \url{https://elitagroup.com/prod/elpicsprint/} \url{https://elitagroup.com/prod/elpicsprint/}}

\subsubsection{System Requirements (Windows, MacOS)}

\emph{Duxbury Braille Translator (DBT):}
\begin{itemize}
    \item \emph{Operating Systems:} DBT for Windows (DBT Win 14.1 SR1) requires Windows 8, 10, or 11.\footnote{\url{https://www.perkins.org/resource/how-create-3d-printable-maps-using-touch-mapper/} \url{https://irie-at.com/product/brailletrac-120/} \url{https://elitagroup.com/prod/elpicsprint/}} Older versions also supported Windows 7.\footnote{\url{https://elitagroup.com/prod/elpicsprint/}} For Mac users, DBT for Mac 14.1 recommends Mac OS X El Capitan (10.11) or higher, including Sierra (10.12) and High Sierra (10.13).\footnote{\url{https://irie-at.com/product/brailletrac-120/} \url{https://elitagroup.com/prod/elpicsprint/} \url{https://elitagroup.com/prod/elpicsprint/}}
    \item \emph{Hardware:} The provided documentation for DBT does not specify explicit minimum or recommended CPU, RAM, or storage requirements.\footnote{\url{https://www.perkins.org/resource/how-create-3d-printable-maps-using-touch-mapper/} \url{https://viewplus.com/downloads/cutsheets/TSS8_PS_EN_8.5x11_2023.pdf} \url{https://irie-at.com/product/brailletrac-120/} \url{https://elitagroup.com/prod/elpicsprint/} \url{https://viewplus.com/product/tiger-software-suite9/} \url{https://www.duxburysystems.com/dbt_details.asp}} This suggests that DBT is optimized to run efficiently on standard computer configurations that meet the specified operating system requirements.
\end{itemize}

\emph{QuickTac:}
\begin{itemize}
    \item \emph{Operating Systems:} QuickTac for Windows runs on Windows 7 or above, with updates released in July 2020 to maintain compatibility.\footnote{\url{https://www.duxburysystems.com/products.asp}} For Mac users, QuickTac for Mac runs on Mac OS X Yosemite or above, also updated in July 2020.\footnote{\url{https://www.duxburysystems.com/products.asp}}
    \item \emph{Hardware:} Similar to DBT, explicit CPU, RAM, or storage requirements for QuickTac are not detailed in the provided information.\footnote{\url{https://elitagroup.com/prod/elpicsprint/} \url{https://elitagroup.com/prod/elpicsprint/} \url{https://www.duxburysystems.com/products.asp} \url{https://elitagroup.com/manuals/elpicsprint/}} QuickTac is described as a "paint program building a grid of dots" \cite{}, implying it is not resource-intensive and can run on general system requirements for the specified operating systems.
\end{itemize}

\subsubsection{Embosser Compatibility}

\emph{Duxbury Braille Translator (DBT):}
DBT is designed to support "all commercial embossers," ranging from very old to quite recent models.\footnote{\url{https://www.perkins.org/resource/how-create-3d-printable-maps-using-touch-mapper/} \url{https://irie-at.com/product/brailletrac-120/} \url{https://elitagroup.com/prod/elpicsprint/} \url{https://viewplus.com/product/tiger-software-suite9/} \url{https://elitagroup.com/prod/elpicsprint/}} This broad compatibility is a significant advantage, allowing users to integrate DBT into diverse existing embossing setups. Recent updates have improved support for specific models, including ViewPlus Columbia, ViewPlus Delta (supporting booklet printing on 11x17 paper), Braillo (including Series 2 models like 300, 450, 600, and variable line spacing), Irie embossers (correcting front/back page alignment on BrailleTrac 120), TactPlus EasyTactix, ViewPlus SpotDot, APH PageBlaster, and APH PixBlaster.\footnote{\url{https://softorage.com/software/inkscape/}} DBT also supports Z-fold format on Index FanFold embossers.\footnote{\url{https://softorage.com/software/inkscape/}}

\emph{QuickTac:}
QuickTac produces tactile graphics files that can be imported into DBT.\footnote{\url{https://touch-mapper.org/en/help} \url{https://elitagroup.com/prod/elpicsprint/} \url{https://elitagroup.com/prod/elpicsprint/} \url{https://viewplus.com/downloads/cutsheets/TSS8_PS_EN_8.5x11_2023.pdf} \url{https://www.duxburysystems.com/dbt_details.asp} \url{https://softorage.com/software/inkscape/}} When setting up an embosser in QuickTac, only models capable of producing graphics are displayed.\footnote{\url{https://www.duxburysystems.com/dbt_details.asp}} QuickTac-generated graphics (in.SIG format) can be produced on ViewPlus embossers when embedded in a DBT document.\footnote{\url{https://softorage.com/software/inkscape/}} This indicates that QuickTac's compatibility is primarily through its integration with DBT, leveraging DBT's extensive embosser support.

\subsection{Tiger Software Suite (TSS)}

\subsubsection{Core Functionality and Features (ViewPlus Embosser Ecosystem)}

The Tiger Software Suite (TSS) is a comprehensive software package developed by ViewPlus, designed to maximize the functionality of ViewPlus embossers.\footnote{Source:  \url{https://emeraldcoastvisionaids.com/shop/products/tactileview-design-software/} \url{https://elitagroup.com/manuals/elpicsprint/} \url{https://www.duxburysystems.com/products.asp} \url{https://www.duxburysystems.org/documentation/NimPro/quick.htm}} It provides essential tools for braille translation and tactile graphics creation, making it a complete solution for braille production needs.\footnote{Source:  \url{https://emeraldcoastvisionaids.com/shop/products/tactileview-design-software/} \url{https://www.duxburysystems.com/products.asp}}

TSS includes several key components:
\begin{itemize}
    \item \emph{VP Formatter:} This add-in integrates directly with Microsoft Word and Excel, allowing users to translate text to braille and convert images to tactile graphics within familiar Windows applications.\footnote{Source:  \url{https://emeraldcoastvisionaids.com/shop/products/tactileview-design-software/} \url{https://elitagroup.com/manuals/elpicsprint/} \url{https://www.duxburysystems.org/documentation/NimPro/quick.htm}} It supports interline printing, displaying print text alongside braille, which is beneficial for proofreading and for users new to braille.\footnote{Source:  \url{https://emeraldcoastvisionaids.com/shop/products/tactileview-design-software/}} The VP Formatter offers expert-level output with easy-to-use controls, enabling single-click embossing and customizable settings.\footnote{Source:  \url{https://emeraldcoastvisionaids.com/shop/products/tactileview-design-software/} \url{https://www.duxburysystems.org/documentation/NimPro/quick.htm}}
    \item \emph{Tiger Designer:} This is a standalone tactile graphics software that allows users to create or edit tactile graphics, leveraging ViewPlus embossers' capability for 8 different dot heights to produce high-resolution, world-class tactile graphics.\footnote{Source:  \url{https://emeraldcoastvisionaids.com/shop/products/tactileview-design-software/} \url{https://elitagroup.com/manuals/elpicsprint/} \url{https://www.duxburysystems.com/products.asp} \url{https://www.duxburysystems.org/documentation/NimPro/quick.htm}} It can quickly convert existing images or generate unique tactile graphics using intuitive tools.\footnote{Source:  \url{https://emeraldcoastvisionaids.com/shop/products/tactileview-design-software/} \url{https://www.duxburysystems.org/documentation/NimPro/quick.htm}} Tiger Designer also supports opening or importing PDFs, adjusting size, snap, and rotation with braille text recognition.\footnote{Source:  \url{https://elitagroup.com/manuals/elpicsprint/}} It offers a pattern editor with default patterns and the ability to assign tactile patterns to specific colors, applying color mapping automatically when printing to compatible ViewPlus embossers.\footnote{Source:  \url{https://elitagroup.com/manuals/elpicsprint/}}
    \item \emph{VP Translator:} This tool can convert text from various Windows applications (e.g., PowerPoint, text files, websites) into braille based on user-preferred settings and send the document to a ViewPlus embosser.\footnote{Source:  \url{https://emeraldcoastvisionaids.com/shop/products/tactileview-design-software/} \url{https://www.duxburysystems.org/documentation/NimPro/quick.htm}}
\end{itemize}

TSS supports customizable UEB Grade 2 contraction tables, updates to Liblouis, and includes Braille36 fonts for better OS interoperability across Windows, Mac, and Linux.\footnote{Source:  \url{https://elitagroup.com/manuals/elpicsprint/}} It also allows experienced graphic designers to use familiar applications like Adobe Illustrator and CorelDraw and print directly to ViewPlus embossers.\footnote{Source:  \url{https://elitagroup.com/manuals/elpicsprint/} \url{https://www.duxburysystems.com/products.asp} \url{https://touch-mapper.org/}} The software enables ink printing to any mainstream color printer.\footnote{Source:  \url{https://elitagroup.com/manuals/elpicsprint/}}

\subsubsection{System Requirements (Windows)}

The provided information for Tiger Software Suite primarily focuses on its features and embosser compatibility rather than detailed system requirements. While it is stated that TSS runs on Windows, specific minimum or recommended CPU, RAM, or storage requirements are not explicitly provided.\footnote{Source:  \url{https://emeraldcoastvisionaids.com/shop/products/tactileview-design-software/} \url{https://elitagroup.com/manuals/elpicsprint/} \url{https://www.duxburysystems.com/products.asp} \url{https://www.duxburysystems.org/documentation/NimPro/quick.htm}} However, the compatibility section for ViewPlus embossers, which are powered by TSS, mentions Windows 7, 8/8.1, and 10 as compatible operating systems.\footnote{Source:  \url{https://www.duxburysystems.com/products.asp}} This suggests that TSS is designed to function effectively on standard Windows systems that meet general modern computing needs.

\subsubsection{Embosser Compatibility}

Tiger Software Suite is specifically developed to power and fully utilize \emph{all ViewPlus embossers}, from portable models like EmBraille to high-volume/high-speed units such as the Elite.\footnote{Source:  \url{https://emeraldcoastvisionaids.com/shop/products/tactileview-design-software/} \url{https://elitagroup.com/manuals/elpicsprint/} \url{https://www.duxburysystems.com/products.asp} \url{https://www.duxburysystems.org/documentation/NimPro/quick.htm}} This tight integration ensures optimal performance and access to advanced features like 8-dot height tactile graphics.\footnote{Source:  \url{https://emeraldcoastvisionaids.com/shop/products/tactileview-design-software/} \url{https://www.duxburysystems.com/products.asp}}

While primarily designed for ViewPlus embossers, TSS is also compatible with Duxbury and other mainstream braille software, indicating a degree of interoperability within the broader braille production ecosystem.\footnote{Source:  \url{https://emeraldcoastvisionaids.com/shop/products/tactileview-design-software/} \url{https://www.duxburysystems.com/products.asp} \url{https://touch-mapper.org/} \url{https://steemit.com/utopianio/@jingis07/reviewing-inkscape-professional-vector-graphics-editor-1554816723337}} The software leverages standardized Windows printer drivers, allowing for basic printing to ViewPlus embossers even without specialized software, but TSS enhances this with multi-language braille translation and document formatting.\footnote{Source:  \url{https://elitagroup.com/manuals/elpicsprint/}} The ability to import and work with Scalable Vector Graphics (SVG) files within Tiger Designer further extends its utility, as SVG is a widely recognized format for vector imagery.\footnote{Source:  \url{https://elitagroup.com/manuals/elpicsprint/} \url{https://irie-at.com/product/brailletrac-120/}} This allows users of other vector graphics applications to prepare designs for ViewPlus embossers.

\section{Open-Source Tactile Graphics Software}

\subsection{Inkscape}

\subsubsection{Core Functionality and Features (Vector Graphics for Tactile Output)}

Inkscape is a professional-quality, free, and open-source vector graphics software that runs natively on Windows, Mac OS X, and GNU/Linux.\footnote{\url{https://getbraille.com/tactile-graphics/} \url{https://blindhelp.net/software/DBT}} It is widely used by designers and hobbyists for creating a diverse range of graphics, including illustrations, icons, logos, diagrams, maps, and web graphics.\footnote{\url{https://blindhelp.net/software/DBT}} Inkscape's native format is SVG (Scalable Vector Graphics), an open standard from the W3C.\footnote{\url{https://blindhelp.net/software/DBT}}

While not exclusively designed for tactile graphics, Inkscape is highly adaptable for this purpose due to its vector-based nature. Tactile graphics can be created quickly and easily using its tools.\footnote{Source:  \url{https://getbraille.com/tactile-graphics/}} Key functions relevant to tactile output include flexible drawing tools, shape tools, interactive transformations (move, scale, rotate, skew), layer organization, grouping, and precise control over fill and stroke settings.\footnote{Source:  \url{https://www.duxburysystems.com/faq2.asp?faq=13}} It supports multi-line text and text placement along paths, which is crucial for adding braille labels.\footnote{Source:  \url{https://www.duxburysystems.com/faq2.asp?faq=13}} The software's ability to handle SVG files is particularly advantageous, as SVG is a common format for generating embossed graphics, raised-line prints with Swell-Form printing, and even extrusion maps for 3D printers.\footnote{Source:  \url{https://www.problind.org/en/create/}} Users can create custom illustrations, schematics, blueprints, charts, graphs, and maps for tactile production.\footnote{Source:  \url{https://www.problind.org/en/create/}}

Specific considerations for tactile graphics in Inkscape include limiting drawings to a few colors (e.g., black, grey, oldLace) to produce strong lines and good textures with multi-dot height embossers, and using a minimum stroke width of 2 units for embossed graphics (1 unit for Swell-Form/3D printing).\footnote{Source:  \url{https://www.problind.org/en/create/}} The software allows for the creation of templates, which can then be uploaded to databases like "Share" for community access.\footnote{Source:  \url{https://getbraille.com/tactile-graphics/}}

\subsubsection{System Requirements (Windows, MacOS, Linux)}

Inkscape is known for being relatively light on system resources and is compatible across major operating systems.\footnote{\url{https://blindhelp.net/software/DBT} \url{https://blindhelp.net/software/DBT}}

\begin{itemize}
    \item Operating Systems: Inkscape runs on Windows, Mac OS X, and GNU/Linux.\footnote{\url{https://blindhelp.net/software/DBT}} It is also available on FreeBSD and web-based platforms.\footnote{\url{https://blindhelp.net/software/DBT}}
    \item Minimum Hardware Requirements:
    \begin{itemize}
        \item CPU: 1 GHz processor.\footnote{\url{https://blindhelp.net/software/DBT}}
        \item RAM: 256 MB.\footnote{\url{https://blindhelp.net/software/DBT}}
        \item Storage: The executable file size for installation is approximately 80 MB, and it occupies about 375 MB after installation, indicating a small footprint.\footnote{\url{https://blindhelp.net/software/DBT}}
    \end{itemize}
    \item Recommended Hardware for Complex Graphics:
    \begin{itemize}
        \item For smoother performance, especially when working with large or high-resolution images, a 2.40 GHz processor with 8 GB of RAM is commonly used.\footnote{\url{https://blindhelp.net/software/DBT}}
        \item For handling multiple tasks without slowing down, particularly with complex graphics work, 16 GB of RAM is recommended.\footnote{\url{https://blindhelp.net/software/DBT}}
        \item Any graphics card capable of displaying the work is generally sufficient.\footnote{\url{https://blindhelp.net/software/DBT}}
    \end{itemize}
    \item Limitations: While Inkscape is lightweight, it may experience performance issues, including hangs, lags, or crashes, when dealing with very large or complicated projects.\footnote{\url{https://blindhelp.net/software/DBT}} Compatibility issues can arise when transferring files between Inkscape and other graphics software, potentially leading to a loss of certain features.\footnote{\url{https://blindhelp.net/software/DBT}} It also lacks built-in cloud-based services for file synchronization and collaboration.\footnote{\url{https://blindhelp.net/software/DBT}}
\end{itemize}

\subsubsection{Embosser Workflow and Compatibility Considerations}

Inkscape does not offer direct, built-in embosser support or specific drivers. Instead, its strength lies in its ability to produce SVG files, which serve as an intermediate format for tactile output.\footnote{\url{https://www.problind.org/en/create/} \url{https://www.duxburysystems.org/documentation/NimPro/quick.htm}} The workflow for creating tactile graphics with Inkscape and sending them to an embosser or swell form machine typically involves several steps:

\begin{enumerate}
    \item Design in Inkscape: Create the tactile graphic design, ensuring adherence to tactile design principles (e.g., minimum object distance, limited textures, appropriate line thickness).\footnote{\url{https://www.problind.org/en/create/} \url{https://www.duxburysystems.com/dbt_details.asp}} Braille labels can be added directly, often using specific fonts like Courier New 27 pt.\footnote{\url{https://getbraille.com/tactile-graphics/}}
    \item Export/Save as SVG: Save the design as an SVG file.\footnote{\url{https://www.problind.org/en/create/}} This vector format is crucial as it allows for scaling without pixelation, maintaining quality for tactile output.\footnote{\url{https://blindhelp.net/software/DBT}}
    \item Preparation for Embossing:
    \begin{itemize}
        \item For braille embossers, the SVG file can often be printed through standard operating system tools, where the printer driver generates the tactile image.\footnote{\url{https://irie-at.com/product/tactileview-design-software/}} Some embossers, particularly ViewPlus Tiger embossers, are designed to work with SVG files and can interpret color brightness to create different dot heights.\footnote{\url{https://www.problind.org/en/create/} \url{https://www.duxburysystems.com/products.asp} \url{https://www.duxburysystems.com/products.asp} \url{https://irie-at.com/product/brailletrac-120/}}
        \item For swell paper machines (e.g., PIAF, Swell Form), the SVG design is typically printed onto swell paper using a laser printer, then passed through the thermal machine to raise the inked areas.\footnote{\url{https://www.duxburysystems.com/products.asp} \url{https://www.duxburysystems.org/documentation/NimPro/quick.htm} \url{https://www.duxburysystems.org/name=Products;action-uri=/OldRoot/products.html}}
        \item The workflow may involve duplicating layers for color and black-and-white versions for combined tactile and visual prints.\footnote{\url{https://www.duxburysystems.org/documentation/NimPro/quick.htm}}
    \end{itemize}
    \item External Translation/Processing: In some cases, the SVG file might be imported into specialized tactile graphics software (like ElPicsPrint or Tiger Designer) for final processing, braille translation (if not done in Inkscape), and embossing.\footnote{Source:  \url{https://www.duxburysystems.com/dbt_details.asp} \url{https://elitagroup.com/manuals/elpicsprint/}} This approach leverages Inkscape's design capabilities while relying on dedicated software for optimal embosser control.
    \item Browser-based Printing: For certain embossers, printing SVGs directly from web browsers like Google Chrome or Firefox can be effective.\footnote{Source:  \url{https://www.problind.org/en/create/}}
\end{enumerate}

The flexibility of SVG as an open standard makes Inkscape a valuable tool in tactile graphics production, enabling a workflow that can be adapted to various embossing technologies, even without direct software-level embosser integration.\footnote{\url{https://www.problind.org/en/create/}}

\subsection{Touch Mapper}

\subsubsection{Core Functionality and Features (Specialized Tactile Map Generation)}

Touch Mapper is a specialized web-based tool designed for the easy creation of custom outdoor tactile maps for individuals who are blind or partially sighted.\footnote{\url{https://aeldata.com/stem-accessibility/tactile-graphics/} \url{https://bpb-us-e1.wpmucdn.net/wp.nyu.edu/dist/c/1540/files/2022/08/Workflow-For-In-House-Production-Of-Paper-Tactile-Maps.pdf} \url{https://ability2access.com/products-2/tiger-software-suite-tss/}} Its primary function is to help users orient themselves and plan routes by providing tactile representations of geographical areas.\footnote{\url{https://aeldata.com/stem-accessibility/tactile-graphics/} \url{https://bpb-us-e1.wpmucdn.net/wp.nyu.edu/dist/c/1540/files/2022/08/Workflow-For-In-House-Production-Of-Paper-Tactile-Maps.pdf}}

The process is straightforward: users enter an address, click "Search," then "Create tactile map," and finally choose to print or order the map.\footnote{Source:  \url{https://bpb-us-e1.wpmucdn.net/wp.nyu.edu/dist/c/1540/files/2022/08/Workflow-For-In-House-Production-Of-Paper-Tactile-Maps.pdf}} Touch Mapper utilizes OpenStreetMap data as its source, which, while generally excellent, may occasionally have missing features.\footnote{Source:  \url{https://aeldata.com/stem-accessibility/tactile-graphics/}} The maps generated are optimized for clarity and the practical needs of visually impaired individuals, rather than being true-to-life representations.\footnote{Source:  \url{https://aeldata.com/stem-accessibility/tactile-graphics/}} They can include roads (pedestrian roads are raised more), buildings, railways, and water bodies (with wavy surfaces, narrow streams as lines).\footnote{Source:  \url{https://aeldata.com/stem-accessibility/tactile-graphics/}} A raised dot on one corner typically marks the Northeast corner for orientation.\footnote{Source:  \url{https://aeldata.com/stem-accessibility/tactile-graphics/} \url{https://ability2access.com/products-2/tiger-software-suite-tss/}}

A key limitation is that Touch Mapper-generated maps do not include any text or labels directly on the map, necessitating a separate written or brailled description to accompany the map for full utility.\footnote{Source:  \url{https://aeldata.com/stem-accessibility/tactile-graphics/} \url{https://ability2access.com/products-2/tiger-software-suite-tss/}} The platform supports creating maps in multiple parts for larger areas, which can be laid side-by-side.\footnote{Source:  \url{https://aeldata.com/stem-accessibility/tactile-graphics/}} Commercial use of maps created with Touch Mapper is permitted, provided the source (www.touch-mapper.org) is clearly credited.\footnote{Source:  \url{https://aeldata.com/stem-accessibility/tactile-graphics/}}

\subsubsection{System Requirements (Web-Based)}

As a web-based application, Touch Mapper's system requirements are primarily dependent on the user's web browser and internet connectivity.

\begin{itemize}
    \item Operating Systems: Compatible with any operating system capable of running a modern web browser (Windows, MacOS, Linux, mobile OS).\footnote{Source:  \url{https://aeldata.com/stem-accessibility/tactile-graphics/} \url{https://bpb-us-e1.wpmucdn.net/wp.nyu.edu/dist/c/1540/files/2022/08/Workflow-For-In-House-Production-Of-Paper-Tactile-Maps.pdf} \url{https://ability2access.com/products-2/tiger-software-suite-tss/}}
    \item Hardware: Minimal hardware requirements for the computer itself, as the processing is done server-side. A stable internet connection is necessary to access the service and generate maps.\footnote{Source:  \url{https://aeldata.com/stem-accessibility/tactile-graphics/} \url{https://bpb-us-e1.wpmucdn.net/wp.nyu.edu/dist/c/1540/files/2022/08/Workflow-For-In-House-Production-Of-Paper-Tactile-Maps.pdf} \url{https://ability2access.com/products-2/tiger-software-suite-tss/}}
    \item Local Printing: If printing maps locally, the computer must meet the requirements of the specific 3D printer, braille embosser, or swell paper printer being used.
\end{itemize}

\subsubsection{Embosser, Swell Paper, and 3D Printer Compatibility}

Touch Mapper supports various output methods for tactile maps:

\begin{itemize}
    \item 3D Printers: Users can download an STL file of the map at no charge and print it themselves using almost any 3D printer, including affordable hobbyist devices.\footnote{Source:  \url{https://aeldata.com/stem-accessibility/tactile-graphics/} \url{https://bpb-us-e1.wpmucdn.net/wp.nyu.edu/dist/c/1540/files/2022/08/Workflow-For-In-House-Production-Of-Paper-Tactile-Maps.pdf} \url{https://ability2access.com/products-2/tiger-software-suite-tss/}} Recommendations for 3D printing include layer thickness of 0.25 mm or 0.3 mm, a base thickness of 0.6 mm (two layers), two top layers, and one horizontal shell.\footnote{Source:  \url{https://aeldata.com/stem-accessibility/tactile-graphics/}} Printing time for a 17 cm map is around 3-4 hours.\footnote{Source:  \url{https://aeldata.com/stem-accessibility/tactile-graphics/}}
    \item Embossers: Tactile embossers are supported for printing the maps.\footnote{Source:  \url{https://bpb-us-e1.wpmucdn.net/wp.nyu.edu/dist/c/1540/files/2022/08/Workflow-For-In-House-Production-Of-Paper-Tactile-Maps.pdf}} The color brightness in the digital design can influence the resolution and dot height on a braille embosser.\footnote{Source:  \url{https://www.duxburysystems.com/products.asp}}
    \item Swell Paper Printers (Thermal Devices): Swell paper printers are also supported.\footnote{Source:  \url{https://bpb-us-e1.wpmucdn.net/wp.nyu.edu/dist/c/1540/files/2022/08/Workflow-For-In-House-Production-Of-Paper-Tactile-Maps.pdf}} This method involves printing the map design on swell paper using a laser printer and then passing it through a Swell Form Machine to raise the inked areas.\footnote{Source:  \url{https://www.duxburysystems.com/products.asp}} This method is noted for being easy to learn and reproduce, offering a greater level of graphical detail than braille embossers, though the paper cost is higher and braille text may be less clear.\footnote{Source:  \url{https://www.duxburysystems.com/products.asp}}
\end{itemize}

Touch Mapper's flexibility in output formats makes it a versatile tool for creating tactile maps, catering to different production capabilities and user preferences.

\subsection{BrailleBlaster}

\subsubsection{Core Functionality and Features (Braille Production with Graphics Potential)}

BrailleBlaster is an open-source braille transcription and production software primarily focused on creating braille documents. It is designed to assist in producing braille for various materials, including textbooks. While its core strength lies in braille translation and formatting, it has the potential for integrating or handling graphics within braille documents. The software aims to be a comprehensive solution for braille production, supporting multiple languages and offering features for complex document layouts.

\subsubsection{System Requirements (Windows, MacOS, Linux)}

BrailleBlaster is designed to run on common desktop operating systems, with specific hardware recommendations for optimal performance.

\begin{itemize}
    \item Operating Systems:
    \begin{itemize}
        \item Windows: Windows 10 or newer.\footnote{Source:  \url{https://sterlingadaptives.com/products/viewplus-premier-braille-embosser-tiger-software-suite-included}} Older versions supported Windows 7 or higher.\footnote{Source:  \url{https://sterlingadaptives.com/products/viewplus-premier-braille-embosser-tiger-software-suite-included}}
        \item Mac OS: Mac OS X or newer.\footnote{Source:  \url{https://sterlingadaptives.com/products/viewplus-premier-braille-embosser-tiger-software-suite-included}}
        \item Linux: Ubuntu Linux 20.04 or later. For universal zip archive distribution, Java 15 or higher is required.\footnote{Source:  \url{https://sterlingadaptives.com/products/viewplus-premier-braille-embosser-tiger-software-suite-included}} Older versions supported Linux with Java 8 installed.\footnote{Source:  \url{https://sterlingadaptives.com/products/viewplus-premier-braille-embosser-tiger-software-suite-included}}
    \end{itemize}
    \item Hardware Requirements:
    \begin{itemize}
        \item Processor: 64-bit computer with a 64-bit operating system installed.\footnote{Source:  \url{https://sterlingadaptives.com/products/viewplus-premier-braille-embosser-tiger-software-suite-included}} No specific CPU speed or core count is mentioned, implying general modern processor compatibility.
        \item RAM: Minimum of 4 GB RAM, with 6 GB recommended for better performance.\footnote{Source:  \url{https://sterlingadaptives.com/products/viewplus-premier-braille-embosser-tiger-software-suite-included}}
        \item Storage: Not explicitly stated, but typically modest for document processing software.
        \item Graphics: Works best with a high-resolution monitor, though it is not a strict requirement.\footnote{Source:  \url{https://sterlingadaptives.com/products/viewplus-premier-braille-embosser-tiger-software-suite-included}}
    \end{itemize}
\end{itemize}

\subsubsection{Embosser Compatibility}

The provided information indicates that BrailleBlaster is a tool for braille production, which inherently implies compatibility with braille embossers. However, the available snippets \emph{do not explicitly detail a list of specific embosser brands or models} that BrailleBlaster directly supports for tactile graphics output. Its primary function as a braille editor suggests it would likely utilize standard printer drivers for embossing, similar to how other braille translation software interacts with hardware. The focus of BrailleBlaster is on translating and formatting text into braille, with the potential for graphics integration being a secondary or indirect capability.

\section{Comparative Analysis: Choosing the Right Solution}\label{sec:comparative-analysis}

The selection of appropriate software for generating tactile graphics involves a careful evaluation of operating system compatibility, hardware demands, embosser integration, feature sets, and the fundamental choice between commercial and open-source models.

\subsection{Operating System Support Across Platforms}

The landscape of tactile graphics software exhibits a clear bias towards Windows, which hosts the majority of commercial, feature-rich applications. TactileView and ElPicsPrint are Windows-native, requiring a Windows emulator for MacOS users.\footnote{Source:  \url{https://www.duxburysystems.com/dbt_details.asp} \url{https://blindsvg.com/} \url{https://www.duxburysystems.com/news.asp} \url{https://www.duxburysystems.com/dbt_details.asp} \url{https://www.duxburysystems.com/dbt_details.asp}} This presents a barrier for MacOS users who prefer not to run virtual environments. Duxbury Braille Translator (DBT) offers native versions for both Windows (Windows 8, 10, 11) and MacOS (OS X El Capitan or higher) \cite{[22, 21, }, making it a more versatile choice for cross-platform environments. Tiger Software Suite (TSS) is primarily a Windows solution, deeply integrated with Microsoft Office applications.\footnote{Source:  \url{https://emeraldcoastvisionaids.com/shop/products/tactileview-design-software/} \url{https://elitagroup.com/manuals/elpicsprint/}}

For Linux users, the options are more limited in the commercial space. Open-source solutions like Inkscape provide robust vector graphics editing capabilities natively on Linux, Windows, and MacOS.\footnote{Source:  \url{https://blindhelp.net/software/DBT}} BrailleBlaster also supports all three major operating systems (Windows 10+, Mac OS X+, Ubuntu Linux 20.04+).\footnote{Source:  \url{https://sterlingadaptives.com/products/viewplus-premier-braille-embosser-tiger-software-suite-included}} Web-based tools like Touch Mapper offer the highest degree of OS independence, as they can be accessed from any device with a modern web browser and internet connection.\footnote{Source:  \url{https://aeldata.com/stem-accessibility/tactile-graphics/} \url{https://bpb-us-e1.wpmucdn.net/wp.nyu.edu/dist/c/1540/files/2022/08/Workflow-For-In-House-Production-Of-Paper-Tactile-Maps.pdf} \url{https://ability2access.com/products-2/tiger-software-suite-tss/}} This broadens accessibility for users across different computing environments, including those running Linux or mobile operating systems.

\subsection{Hardware Demands: CPU, RAM, Storage, Graphics}

Detailed hardware specifications are not consistently provided across all software, particularly for commercial products like TactileView and ElPicsPrint, which typically state only operating system compatibility. This often indicates that these applications are not exceptionally resource-intensive and can run effectively on standard modern computers.

\begin{itemize}
    \item \emph{TactileView \& ElPicsPrint:} Specific CPU, RAM, or storage requirements are not listed for these commercial applications.\footnote{Source:  \url{https://www.duxburysystems.com/dbt_details.asp} \url{https://irie-at.com/product/tactileview-design-software/} \url{https://blindsvg.com/} \url{https://www.duxburysystems.com/news.asp} \url{https://www.duxburysystems.com/dbt_details.asp} \url{https://www.duxburysystems.com/dbt_details.asp} \url{https://www.duxburysystems.com/dbt_details.asp}} Their performance is likely tied to the underlying operating system's general requirements.
    \item \emph{Duxbury Braille Translator (DBT) \& QuickTac:} Similar to the above, explicit hardware requirements are largely absent.\footnote{Source:  \url{https://www.perkins.org/resource/how-create-3d-printable-maps-using-touch-mapper.pdf} \url{https://viewplus.com/downloads/cutsheets/TSS8_PS_EN_8.5x11_2023.pdf} \url{https://irie-at.com/product/brailletrac-120/} \url{https://elitagroup.com/prod/elpicsprint/} \url{https://viewplus.com/product/tiger-software-suite9/} \url{https://elitagroup.com/prod/elpicsprint/} \url{https://www.duxburysystems.com/products.asp} \url{https://elitagroup.com/manuals/elpicsprint/}} QuickTac, being a simple paint program, is inherently light.\footnote{Source:  \url{https://www.duxburysystems.com/dbt_details.asp}}
    \item \emph{Inkscape:} As an open-source vector graphics editor, Inkscape has modest minimum requirements (1 GHz CPU, 256 MB RAM, ~375 MB storage).\footnote{Source:  \url{https://blindhelp.net/software/DBT}} However, for handling large or high-resolution images and complex projects, 8 GB of RAM is commonly used, and 16 GB is recommended for smoother performance and multitasking.\footnote{Source:  \url{https://blindhelp.net/software/DBT} \url{https://blindhelp.net/software/DBT}} It is generally considered light on system resources, with CPU usage remaining low unless working on demanding files.\footnote{Source:  \url{https://blindhelp.net/software/DBT}}
    \item \emph{BrailleBlaster:} This open-source braille production software requires a 64-bit computer with a 64-bit operating system, a minimum of 4 GB RAM (6 GB recommended), and works best with a high-resolution monitor.\footnote{Source:  \url{https://sterlingadaptives.com/products/viewplus-premier-braille-embosser-tiger-software-suite-included}}
    \item \emph{Touch Mapper:} Being web-based, its hardware demands on the client machine are minimal, relying primarily on a stable internet connection and a modern web browser.\footnote{Source:  \url{https://aeldata.com/stem-accessibility/tactile-graphics/} \url{https://bpb-us-e1.wpmucdn.net/wp.nyu.edu/dist/c/1540/files/2022/08/Workflow-For-In-House-Production-Of-Paper-Tactile-Maps.pdf} \url{https://ability2access.com/products-2/tiger-software-suite-tss/}} The hardware requirements for actual printing (3D printer, embosser) are separate.
\end{itemize}

In summary, while most dedicated tactile graphics software does not impose stringent hardware demands, general-purpose vector editors like Inkscape can benefit significantly from more substantial RAM, especially when dealing with intricate designs that are common in tactile graphics.

\subsection{Embosser Compatibility Matrix and Limitations}

Embosser compatibility is a critical factor, and software solutions vary widely in their support.

\begin{longtblr}[
  caption = {Tactile Graphics Software Overview \& OS Compatibility},
  label = {tab:os_compatibility}
]{
  colspec = {|l|l|l|l|l|},
  rowhead = 1,
  hlines,
  stretch = 1.5
}
Software Name & Type & Primary Function & Supported Operating Systems & Notes on Emulators/Specific Versions \\
TactileView Design Software & Commercial & Tactile graphics editor, map maker, math graphs & Windows XP, Vista, 7, 8, 8.1, 10, 11\footnote{Source:  \url{https://www.duxburysystems.com/dbt_details.asp} \url{https://blindsvg.com/} \url{https://www.duxburysystems.com/news.asp} \url{https://www.duxburysystems.com/dbt_details.asp} \url{https://www.duxburysystems.com/dbt_details.asp}} & MacOS X requires Windows emulator.\footnote{Source:  \url{https://www.duxburysystems.com/dbt_details.asp} \url{https://blindsvg.com/} \url{https://www.duxburysystems.com/news.asp} \url{https://www.duxburysystems.com/dbt_details.asp} \url{https://www.duxburysystems.com/dbt_details.asp}} \\
ElPicsPrint & Commercial & Tactile graphics embossing preparation & Windows\footnote{Source:  \url{https://irie-at.com/product/tactileview-design-software/}} & No MacOS/Linux support specified. \\
Duxbury Braille Translator (DBT) & Commercial & Braille translation, tactile graphics integration \&Windows 8, 10, 11; MacOS X El Capitan+\footnote{Source:  \url{https://www.perkins.org/resource/how-create-3d-printable-maps-using-touch-mapper/} \url{https://irie-at.com/product/brailletrac-120/} \url{https://elitagroup.com/prod/elpicsprint/}} & QuickTac for graphics.\footnote{Source:  \url{https://touch-mapper.org/en/help} \url{https://elitagroup.com/prod/elpicsprint/} \url{https://elitagroup.com/prod/elpicsprint/}} \\
QuickTac & Freeware (Companion to DBT) & Tactile graphics composition (dot-based) & Windows 7+; MacOS X Yosemite+\footnote{Source:  \url{https://www.duxburysystems.com/products.asp}} & Primarily integrates with DBT.\footnote{Source:  \url{https://touch-mapper.org/en/help} \url{https://elitagroup.com/prod/elpicsprint/} \url{https://elitagroup.com/prod/elpicsprint/}} \\
Tiger Software Suite (TSS) & Commercial & Braille translation, tactile graphics design for ViewPlus embossers & Windows 7, 8/8.1, 10\footnote{Source:  \url{https://www.duxburysystems.com/products.asp}} & Primarily for ViewPlus hardware.\footnote{Source:  \url{https://emeraldcoastvisionaids.com/shop/products/tactileview-design-software/} \url{https://elitagroup.com/manuals/elpicsprint/} \url{https://www.duxburysystems.com/products.asp}} \\
Inkscape & Open Source & Vector graphics editor for tactile output & Windows, MacOS X, GNU/Linux, FreeBSD\footnote{Source:  \url{https://blindhelp.net/software/DBT}} & Requires external embosser drivers/workflows.\footnote{Source:  \url{https://www.problind.org/en/create/} \url{https://www.duxburysystems.org/documentation/NimPro/quick.htm}} \\
Touch Mapper & Open Source (Web-based) & Specialized tactile map generation & Web-based (OS independent)\footnote{Source:  \url{https://aeldata.com/stem-accessibility/tactile-graphics/} \url{https://bpb-us-e1.wpmucdn.net/wp.nyu.edu/dist/c/1540/files/2022/08/Workflow-For-In-House-Production-Of-Paper-Tactile-Maps.pdf} \url{https://ability2access.com/products-2/tiger-software-suite-tss/}} & Output for 3D printers, embossers, swell paper.\footnote{Source:  \url{https://bpb-us-e1.wpmucdn.net/wp.nyu.edu/dist/c/1540/files/2022/08/Workflow-For-In-House-Production-Of-Paper-Tactile-Maps.pdf} \url{https://ability2access.com/products-2/tiger-software-suite-tss/}} \\
BrailleBlaster & Open Source & Braille production, graphics potential & Windows 10+; MacOS X+; Ubuntu Linux 20.04+\footnote{Source:  \url{https://sterlingadaptives.com/products/viewplus-premier-braille-embosser-tiger-software-suite-included}} & Embosser compatibility implied for braille. \\
\end{longtblr}

\begin{longtblr}[
  caption = {Detailed System Requirements by Software},
  label = {tab:system_requirements}
]{
  colspec = {|l|l|l|l|l|l|l|l|},
  rowhead = 1,
  hlines,
  stretch = 1.5
}
Software Name & OS & Minimum CPU & Recommended CPU & Minimum RAM & Recommended RAM & Storage Footprint & Graphics Card Notes \\
\hline
TactileView Design Software & Windows & Not specified\footnote{Source:  \url{https://www.duxburysystems.com/dbt_details.asp} \url{https://blindsvg.com/} \url{https://www.duxburysystems.com/news.asp} \url{https://www.duxburysystems.com/dbt_details.asp} \url{https://www.duxburysystems.com/dbt_details.asp}} & Not specified\footnotemark & Not specified\footnotemark & Not specified\footnotemark & Not specified\footnotemark & Not specified\footnotemark \\
 & MacOS & Not specified\footnotemark & Not specified\footnotemark & Not specified\footnotemark & Not specified\footnotemark & Not specified\footnotemark & Not specified (via emulator)\footnotemark \\
ElPicsPrint & Windows & Not specified\footnote{Source:  \url{https://irie-at.com/product/tactileview-design-software/} \url{https://www.duxburysystems.com/dbt_details.asp}} & Not specified\footnotemark & Not specified\footnotemark & Not specified\footnotemark & Not specified\footnotemark & Not specified\footnotemark \\
Duxbury Braille Translator (DBT) & Windows & Not specified\footnote{Source:  \url{https://www.perkins.org/resource/how-create-3d-printable-maps-using-touch-mapper/} \url{https://viewplus.com/downloads/cutsheets/TSS8_PS_EN_8.5x11_2023.pdf} \url{https://irie-at.com/product/brailletrac-120/} \url{https://elitagroup.com/prod/elpicsprint/} \url{https://viewplus.com/product/tiger-software-suite9/} \url{https://www.duxburysystems.com/dbt_details.asp}} & Not specified\footnotemark & Not specified\footnotemark & Not specified\footnotemark & Not specified\footnotemark & Not specified\footnotemark \\
 & MacOS & Not specified\footnotemark & Not specified\footnotemark & Not specified\footnotemark & Not specified\footnotemark & Not specified\footnotemark & Not specified\footnotemark \\
QuickTac & Windows & Not specified\footnote{Source:  \url{https://elitagroup.com/prod/elpicsprint/} \url{https://elitagroup.com/prod/elpicsprint/} \url{https://www.duxburysystems.com/products.asp} \url{https://elitagroup.com/manuals/elpicsprint/}} & Not specified\footnotemark & Not specified\footnotemark & Not specified\footnotemark & Not specified\footnotemark & Not specified\footnotemark \\
 & MacOS & Not specified\footnotemark & Not specified\footnotemark & Not specified\footnotemark & Not specified\footnotemark & Not specified\footnotemark & Not specified\footnotemark \\
Tiger Software Suite (TSS) & Windows & Not specified\footnote{Source:  \url{https://emeraldcoastvisionaids.com/shop/products/tactileview-design-software/} \url{https://elitagroup.com/manuals/elpicsprint/} \url{https://www.duxburysystems.com/products.asp} \url{https://www.duxburysystems.org/documentation/NimPro/quick.htm}} & Not specified\footnotemark & Not specified\footnotemark & Not specified\footnotemark & Not specified\footnotemark & Not specified\footnotemark \\
Inkscape & Windows, MacOS, Linux & 1 GHz\footnote{Source:  \url{https://blindhelp.net/software/DBT}} & 2.40 GHz\footnotemark & 256 MB\footnotemark & 8 GB (16 GB for complex projects)\footnote{Source:  \url{https://blindhelp.net/software/DBT} \url{https://blindhelp.net/software/DBT}} & ~375 MB installed\footnotemark & Any capable of display; can struggle with large images\footnotemark \\
Touch Mapper & Web-based & N/A (Browser-based)\footnote{Source:  \url{https://aeldata.com/stem-accessibility/tactile-graphics/} \url{https://bpb-us-e1.wpmucdn.net/wp.nyu.edu/dist/c/1540/files/2022/08/Workflow-For-In-House-Production-Of-Paper-Tactile-Maps.pdf} \url{https://ability2access.com/products-2/tiger-software-suite-tss/}} & N/A (Browser-based)\footnotemark & N/A (Browser-based)\footnotemark & N/A (Browser-based)\footnotemark & N/A (Browser-based)\footnotemark & N/A (Browser-based)\footnotemark \\
BrailleBlaster & Windows, MacOS, Linux & 64-bit CPU\footnote{Source:  \url{https://sterlingadaptives.com/products/viewplus-premier-braille-embosser-tiger-software-suite-included}} & Not specified & 4 GB\footnotemark & 6 GB\footnotemark & Not specified & Works best with high-resolution monitor\footnotemark \\
\end{longtblr}
