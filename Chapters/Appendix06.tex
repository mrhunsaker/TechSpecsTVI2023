\chapter{Comprehensive Review of Communication Strategies for Individuals with Visual Impairments, Blindness, and Deafblindness}\label{chap:comm-strategies}

\textbf{Accessibility Note:} This chapter provides a comprehensive and structured review of communication strategies, AAC programs, and resources for individuals with visual impairments, blindness, and deafblindness. The content and structure have been enhanced for clarity, navigation, and accessibility, including context for all directories, lists, and resources.


\section{Executive Summary}\label{sec:exec-summary}

This report provides a comprehensive review of instructional programs and theoretical frameworks surrounding Augmentative and Alternative Communication (AAC) for individuals with visual impairments, blindness, and deafblindness. It explores high-tech and low-tech AAC solutions, including 3D printed tactile communication modalities and accessible switches. The review also includes details on various programs, their costs, and direct access links, alongside resources for creating and utilizing self-made, low-cost communication tools.

\section{Introduction}\label{sec:intro}

Communication is a fundamental human right, and for individuals with visual impairments, blindness, or deafblindness, Augmentative and Alternative Communication (AAC) systems are crucial for fostering independence and interaction. This report delves into the diverse strategies, theories, and instructional programs that support AAC usage within this population. We'll explore a spectrum of solutions, from foundational communication principles to the innovative application of 3D printing for tactile symbols and the use of accessible switches for communication. Our aim is to provide a comprehensive resource for educators, therapists, families, and individuals seeking to enhance communication for those with sensory impairments.

---

\section{Defining Augmentative and Alternative Communication (AAC)}\label{sec:defining-aac}

AAC encompasses a broad range of methods used to supplement or replace spoken or written communication for individuals with communication disorders  \cite{ASHA}. These methods can be unaided, relying on the individual’s body (e.g., gestures, manual signs), or aided, requiring external tools or devices. For individuals with visual impairments, blindness, and/or deafblindness, aided AAC often involves tactile, auditory, or multi-modal approaches. The goal of AAC is to enable individuals to communicate effectively, participate in daily life, and express their thoughts, needs, and desires  \cite{ASHA}.

\subsection{General Principles of AAC Assessment and Intervention}\label{subsec:aac-principles}

Effective AAC implementation begins with a thorough assessment of an individual's strengths, needs, and preferred communication modalities. This often involves observing communication attempts, evaluating sensory capabilities, and considering motor skills for accessing communication tools, including the use of \textbf{switch-based communication}  \cite{ASHA}. Intervention strategies are highly individualized, focusing on teaching the individual how to use their AAC system, as well as educating communication partners on how to interpret and respond to AAC messages  \cite{ASHA}. The ultimate aim is to facilitate meaningful communication in various environments.

---

\section{Foundational AAC Strategies and Theoretical Frameworks in Training}\label{sec:aac-strategies}

Many programs emphasize foundational communication strategies that are particularly vital for individuals with sensory impairments. These include establishing cause-and-effect, fostering intentional communication, and developing symbolic understanding through multi-sensory experiences.

\subsection{Touch for Connection and Communication}\label{subsec:touch-connection}

Tactile interaction is paramount for individuals who are deafblind. Training often focuses on developing communication through touch, including body language, natural gestures, and haptic cues. The Helen Keller National Center (HKNC) offers dedicated training on this, emphasizing how touch can convey information about people, objects, and the environment  \cite{HKNC_Haptics}. This foundational skill is critical for building a bridge to more complex symbolic communication.

\subsection{Progressing from Non-symbolic to Symbolic Communication}\label{subsec:progress-symbolic}

Instructional programs guide learners from pre-symbolic communication (e.g., crying, body movements, changes in muscle tone) to symbolic communication using objects, pictures, or tactile symbols. This progression is often non-linear and tailored to the individual's cognitive and sensory development. The University of South Dakota's DeafBlind Communication Strategies course covers this journey, focusing on developing abstract thought and language concepts  \cite{USD}.

\subsection{Object Communication Systems}\label{subsec:object-comm}

Object communication systems are a highly effective low-tech AAC strategy, particularly for individuals with significant visual impairments or deafblindness. These systems use real objects or miniature objects to represent concepts, activities, or choices. For example, a spoon might represent "eat" or "lunch." Training emphasizes associating the object with its meaning through consistent pairing and naturalistic experiences. The AAC Academy's "AAC \& CVI Series" specifically addresses object communication systems and their use with children in early communication phases  \cite{AACA_CVI}. The Texas School for the Blind and Visually Impaired (TSBVI) also champions principles of Active Learning, which inherently supports the development of object-based communication through active exploration of environments and materials  \cite{TSBVI_ActiveLearning}.

\subsection{The Universal Core Vocabulary and its Application in AAC}

The Universal Core Vocabulary is a small set of words (often 36 or 60) that makes up a large percentage of what we say and hear every day. It's designed to be portable and consistent across different communication systems. For individuals with visual impairments or deafblindness, the core vocabulary can be represented through tactile symbols or 3D printed objects. Project Core provides resources for teaching this vocabulary, emphasizing its application in natural communication contexts  \cite{ProjectCore}. Organizations like Volksswitch and Makers Making Change have significantly expanded on this concept by offering accessible and expandable sets of tactile symbols, directly addressing the need for broader vocabulary while maintaining the core principles  \cite{Volksswitch_Bliss}.

---

\section{Directory of Comprehensive AAC Training Programs}

This section provides an overview of various instructional programs that teach the theory and strategies behind AAC usage for individuals with visual impairments, blindness, and/or deafblindness.


\textbf{Context:} The following directory lists major AAC training programs and organizations. Each entry includes a brief description and reference for further exploration. For accessibility, each program is introduced with a semantic description environment.

\begin{description}
\item[National Center on Deafblindness (NCDB):] See below.
\item[University of South Dakota (USD) - DeafBlind Program:] See below.
\item[Helen Keller National Center (HKNC):] See below.
\item[National Research \& Training Center on Blindness \& Low Vision (NRTC) at Mississippi State University:] See below.
\item[Perkins School for the Blind:] See below.
\item[Texas School for the Blind and Visually Impaired (TSBVI):] See below.
\item[Association for Education and Rehabilitation of the Blind and Visually Impaired (AERBVI):] See below.
\item[Closing The Gap, Inc.:] See below.
\item[NextSense (Australia):] See below.
\item[The AAC Academy:] See below.
\item[Tactile Communications, LLC:] See below.
\end{description}

\subsection{National Center on Deafblindness (NCDB)}
The NCDB serves as a central hub for professional development related to deafblindness. While they don't directly offer all courses, they list numerous opportunities from various partners.
\begin{itemize}
    \item \textbf{Open Hands, Open Access (OHOA) Modules}: These are self-guided online modules providing foundational knowledge in deafblindness.
    \begin{itemize}
        \item \textbf{Content}: Modules cover communication principles, emergent communication, touch for connection, and progressing from non-symbolic to symbolic communication. They are foundational to understanding AAC in deafblindness.
        \item \textbf{Access}: Free.
        \item \textbf{Link}: \url{https://www.nationaldb.org/products/online-learning/ohoa-modules/}
    \end{itemize}
\end{itemize}

\subsection{University of South Dakota (USD) - DeafBlind Program}
USD offers a professional development series focused on deafblindness and high-intensity support needs.
\begin{itemize}
    \item \textbf{Professional Development: Concentration in DeafBlindness and High-Intensity Support Needs}: This series includes four courses designed for educators and related service providers.
    \begin{itemize}
        \item \textbf{Course 2: DeafBlind Communication Strategies}: This course focuses heavily on communication strategies, including touch for connection and communication, and supporting the progression from non-symbolic to symbolic and complex language.
        \item \textbf{Access}: Paid, approximately \$120 per 3-credit course (based on general USD graduate credit fees).
        \item \textbf{Link}: \url{https://www.usd.edu/Academics/Colleges-and-Schools/sanford-school-of-medicine/Research-and-Outreach-Centers/Center-for-Disabilities/Programs-and-Services/DeafBlind-Program}
    \end{itemize}
\end{itemize}

\subsection{Helen Keller National Center (HKNC)}
HKNC offers specialized courses directly related to communication with individuals who are deafblind.
\begin{itemize}
    \item \textbf{Communicating with Individuals Who Are DeafBlind}:
    \begin{itemize}
        \item \textbf{Content}: Explores various communication methods used by individuals who are deafblind, including tactile approaches like print-on-palm.
        \item \textbf{Access}: Paid, \$29.99.
        \item \textbf{Link}: \url{https://www.helenkeller.org/courses/communicating-with-individuals-who-are-deafblind/}
    \end{itemize}
    \item \textbf{Haptics: Using Touch to Convey Visual and Environmental Information to People who are Deaf-Blind}:
    \begin{itemize}
        \item \textbf{Content}: Focuses on the systematic use of touch to provide environmental and contextual information.
        \item \textbf{Access}: Paid, \$179.99.
        \item \textbf{Link}: \url{https://www.helenkeller.org/hknc/class/haptics-using-touch-convey-visual-and-environmental-information-people-who-are-deaf-blind}
    \end{itemize}
\end{itemize}

\subsection{National Research \& Training Center on Blindness \& Low Vision (NRTC) at Mississippi State University}
\begin{itemize}
    \item \textbf{Introduction to Working with Individuals Who are Deafblind}:
    \begin{itemize}
        \item \textbf{Content}: Provides an overview of deafblindness, including communication strategies and assistive technology.
        \item \textbf{Access}: Free.
        \item \textbf{Link}: \url{https://nrtc.catalog.instructure.com/courses/intro-deaf-blind}
    \end{itemize}
\end{itemize}

\subsection{Perkins School for the Blind}
Perkins offers a variety of professional development opportunities through their eLearning platform.
\begin{itemize}
    \item \textbf{Perkins eLearning}: Offers webinars and self-paced courses, many of which touch upon communication for individuals with visual impairments and deafblindness. Specific courses on tactile learning and multi-modal approaches are available, though direct courses on 3D printed AAC creation are not explicitly listed.
    \begin{itemize}
        \item \textbf{Access}: Varies; some webinars are free, self-paced courses may be paid (e.g., "Active Learning Is More Than a Piece of Equipment! Tutorial" is \$55  \cite{NCDB_Perkins}).
        \item \textbf{Link}: \url{https://www.perkinselearning.org/}
    \end{itemize}
    \item \textbf{Paths to Literacy}: A collaborative website between Perkins and TSBVI, offering extensive resources, including articles and strategies for literacy development in deafblind and multiply disabled learners. This includes discussions around tactile learning and materials  \cite{PathsToLiteracy}.
    \begin{itemize}
        \item \textbf{Access}: Free.
        \item \textbf{Link}: \url{https://www.pathstoliteracy.org/}
    \end{itemize}
\end{itemize}

\subsection{Texas School for the Blind and Visually Impaired (TSBVI)}
TSBVI offers valuable free online courses focusing on education for students with visual impairments and deafblindness.
\begin{itemize}
    \item \textbf{Introduction to the Intervener Team Model Training}:
    \begin{itemize}
        \item \textbf{Content}: Covers the role of interveners in supporting communication and access for deafblind students.
        \item \textbf{Access}: Free.
        \item \textbf{Link}: \url{https://www.tsbvi.edu/statewide-resources/professional-development/online-learning}
    \end{itemize}
    \item \textbf{Principles of Active Learning}: Based on Dr. Lilli Nielsen's approach, this course is highly relevant for fostering tactile exploration and developing pre-linguistic and object-based communication skills.
    \begin{itemize}
        \item \textbf{Access}: Free.
        \item \textbf{Link}: \url{https://www.tsbvi.edu/statewide-resources/professional-development/online-learning}
    \end{itemize}
\end{itemize}

\subsection{Association for Education and Rehabilitation of the Blind and Visually Impaired (AERBVI)}
AERBVI provides professional development opportunities for professionals working with individuals with visual impairment.
\begin{itemize}
    \item \textbf{eLearning Center}: Offers various courses and webinars relevant to the field, though specific courses dedicated solely to tactile AAC or 3D printing were not explicitly found in a broad search of their offerings.
    \begin{itemize}
        \item \textbf{Access}: Varies; membership often provides discounts. Specific course costs need to be confirmed via their platform.
        \item \textbf{Link}: \url{https://www.aerbvi.org/elearning-center}
    \end{itemize}
\end{itemize}

\subsection{Closing The Gap, Inc.}
Closing The Gap provides professional development focused on assistive technology, including AAC, through their conferences and extensive online library.
\begin{itemize}
    \item \textbf{Member On Demand Library}: Contains numerous webinars.
    \begin{itemize}
        \item \textbf{Bringing Literacy to Life: Using AAC \& Sensory Elements to Build Connections}: This webinar explores integrating AAC, literacy, and sensory components, which is relevant for tactile communication strategies.
        \item \textbf{Access}: Paid, \$65.00 for the on-demand webinar, or included with a Solutions Membership  \cite{CTG_Literacy}.
        \item \textbf{Link}: \url{https://www.closingthegap.com/product/bringing-literacy-to-life-using-aac-sensory-elements-to-build-connections-on-demand/}
    \end{itemize}
\end{itemize}

\subsection{NextSense (Australia)}
NextSense offers professional education focused on hearing and vision loss.
\begin{itemize}
    \item \textbf{Professional Education Courses}: While not exclusively focused on AAC, their courses address communication strategies for individuals with deafblindness, including discussions of object symbols and tactile signing. They also explicitly mention using 3D printed materials in their accessibility services, indicating a practical engagement with these modalities  \cite{NextSense_Services}.
    \begin{itemize}
        \item \textbf{Access}: Varies; specific course costs need to be confirmed.
        \item \textbf{Link}: \url{https://www.nextsense.org.au/professional-education/}
    \end{itemize}
\end{itemize}

\subsection{The AAC Academy}
\begin{itemize}
    \item \textbf{AAC \& CVI Series: Beyond Red \& Yellow}: A series of sessions focusing on AAC for children with Cortical Visual Impairment (CVI).
    \begin{itemize}
        \item \textbf{Session 2}: Explicitly covers "Object communication systems and how they are used with children in early Phase II."
        \item \textbf{Access}: Paid, \$100/year for membership access to courses  \cite{AACA_CVI}.
        \item \textbf{Link}: \url{https://www.theaacacademy.org/course/aac-for-children-with-cvi}
    \end{itemize}
\end{itemize}

\subsection{Tactile Communications, LLC}
\begin{itemize}
    \item \textbf{Training in Protactile Language and Adaptive Strategies}: Offers individualized training directly from DeafBlind instructors.
    \begin{itemize}
        \item \textbf{Content}: Highly specialized training in Protactile Language and other tactile communication systems.
        \item \textbf{Access}: Paid; rates vary based on services provided  \cite{TactileComms}.
        \item \textbf{Link}: \url{https://www.tactilecommunications.org/Services/}
    \end{itemize}
\end{itemize}

---

\section{Instructional Programs for Teaching Switch-Based Communication Skills}
Teaching individuals with visual impairments, blindness, or deafblindness to use switches for communication requires specific strategies that often emphasize auditory and tactile feedback, as well as clear cause-and-effect relationships. While dedicated comprehensive courses are less common, several organizations and resources offer valuable instructional guidance.

\subsection{Foundational Principles and Progression}
Many instructional approaches emphasize a progression of switch skills, starting with basic cause-and-effect and moving towards more complex communication.
\begin{itemize}
    \item \textbf{Cause and Effect Activities}: Essential for introducing switches, allowing learners to understand that pressing the switch produces a predictable outcome (e.g., turning on a toy, playing a sound)  \cite{TeachingVI_Switches}.
    \item \textbf{Switch Progression Road Map}: Developed by Ian Bean and highlighted by resources like My Dynamic Therapy and Spectronics, this road map outlines a systematic progression from initial exploration to single-switch timing, two-switch scanning, and eventually more complex switch access methods  \cite{MyDynamicTherapy}. Resources like the SENICT Resource Portal provide materials to support this progression.
\item \textbf{Auditory Scanning}: For individuals with visual impairments, auditory scanning is a crucial access method. The system speaks choices aloud, and the user activates the switch when they hear their desired option. Programs focusing on CVI and motor planning for high-tech devices implicitly cover these concepts  \cite{Perkins_CVI}.
\end{itemize}

\subsection{Webinars and Instructional Guides}
Several organizations offer webinars and online guides that provide practical strategies for teaching switch use.
\begin{itemize}
    \item \textbf{Closing The Gap Webinars}:
    \begin{itemize}
        \item \textbf{Making Switch Activities Fun!}: This webinar offers practical ideas for engaging individuals in switch-based activities, which is key for developing initial skills and motivation  \cite{CTG_Fun}.
        \item \textbf{Switching to Self-Regulation: Using Assistive Technology to Create a Sensory-Based Calming Space for Students with Complex Communication \& Access Needs}: This session explores using switches to control sensory environments for self-regulation, demonstrating a functional application of switch skills beyond direct communication  \cite{CTG_SelfRegulation}.
        \item \textbf{Access}: Paid, often around \$65 per on-demand webinar, or included with a Solutions Membership.
        \item \textbf{Link}: Check Closing The Gap's "Member On Demand Library" for current offerings \url{https://www.closingthegap.com/webinars/archived-webinars/}
    \end{itemize}
    \item \textbf{Online Guides and Resources}: Websites like Teaching Visually Impaired and Number Analytics provide extensive articles and guides on setting up and teaching switch access, including customizing settings on various devices and integrating switches into daily routines  \cite{TeachingVI_Switches, NumberAnalytics_Switches}.
\end{itemize}

---

\section{Purchase and Access Options for 3D Printable and Tactile Graphic Communication Modalities}
Low-tech AAC, especially tactile-based systems, provides robust and accessible communication solutions. The advent of 3D printing has significantly expanded the possibilities for creating customized tactile symbols.

\subsection{Project Core 3D Universal Core Vocabulary}
Project Core offers a standardized set of core vocabulary words that can be 3D printed, providing consistent tactile symbols for communication.
\begin{itemize}
    \item \textbf{Free Downloadable STL Files}: Individuals and organizations can download the STL (STereoLithography) files for all 36 Universal Core Vocabulary symbols and print them using a 3D printer. This is a free and highly customizable option.
    \begin{itemize}
        \item \textbf{Access}: Free to download.
        \item \textbf{Link}: \url{https://www.project-core.com/communication-systems/}
    \end{itemize}
    \item \textbf{Pre-Printed 3D Symbols}: For those without access to a 3D printer, some vendors offer physical, pre-printed Project Core symbols for purchase.
    \begin{itemize}
        \item \textbf{Example Vendor}: 64ouncebraille.com offers individual Project Core symbols (e.g., 'go' for \$11.00) and complete sets (e.g., Project Core Kit for \$349.00).
        \item \textbf{Access}: Paid.
        \item \textbf{Link}: \url{https://64ouncebraille.com/products/project-core-3d-universal-core-vocabulary}
    \end{itemize}
    \item \textbf{Teaching Guide}: Project Core also provides a guide on their website, "How to teach students to use the 3D symbols," offering practical strategies for implementation  \cite{ProjectCore_Teaching}.
\end{itemize}

\subsection{Bliss Tactile Symbols and Expansions}
Blissymbols are a graphic language where symbols represent concepts rather than words. Tactile versions make them accessible for individuals with visual impairments.
\begin{itemize}
    \item \textbf{Volksswitch - Extended Core Words for Students with Intellectual Disabilities}: Volksswitch has developed over 240 ready-to-3D-print Bliss Tactile Symbols, significantly expanding on core vocabulary. This set includes all 36 Project Core Universal Core Words and extends to a much larger vocabulary. They provide detailed documentation and teaching plans.
    \begin{itemize}
        \item \textbf{Access}: Free downloadable STL files.
        \item \textbf{Link}: \url{https://volksswitch.org/index.php/bliss-tactile-symbols/}
    \end{itemize}
    \item \textbf{Makers Making Change}: This organization also hosts and supports the creation of 3D printable Bliss Tactile Symbols, making these resources widely available through their open-source library  \cite{MakersMakingChange}.
    \begin{itemize}
        \item \textbf{Access}: Free downloadable STL files.
        \item \textbf{Link}: Explore their AT Solutions Library, often under "Communication" or "Deafblind" categories, at \url{https://www.makersmakingchange.com/}
    \end{itemize}
    \item \textbf{Bliss Tactile Symbol Carrier}: Augmentative Resources offers a physical carrier designed to organize and present Bliss Tactile Symbols (symbols not included).
    \begin{itemize}
        \item \textbf{Access}: Paid, \$45.00.
        \item \textbf{Link}: \url{https://www.augresources.com/Bliss-Tactile-Symbol-Carrier-p/033138.htm}
    \end{itemize}
\end{itemize}

\subsection{Tactile Graphics Creation Tools and Services}
Beyond 3D printed symbols, other methods exist for creating tactile graphics.
\begin{itemize}
    \item \textbf{Tactile Graphics Machines}: Companies like American Thermoform sell machines (e.g., Swell Form machines) and supplies (swell paper, braille paper) that allow users to create raised-line tactile graphics from printed images. This is a solution for creating customized tactile communication boards or symbols.
    \begin{itemize}
        \item \textbf{Access}: Paid (equipment and supplies).
        \item \textbf{Link}: \url{https://americanthermoform.com/product-category/tactile-graphics/}
    \end{itemize}
    \item \textbf{NextSense (Australia)}: Offers services for producing 3D printed materials, 2D tactile materials, and large print graphics as part of their accessibility and inclusion services  \cite{NextSense_Services}.
    \begin{itemize}
        \item \textbf{Access}: Service-based, costs vary.
        \item \textbf{Link}: \url{https://www.nextsense.org.au/services/vision/accessibility-and-inclusion}
    \end{itemize}
\end{itemize}

---

\section{Resources for Low-Cost, Self-Made Accessible Switches}
Accessible switches are vital for individuals with limited motor control to access communication devices. Several initiatives provide designs and instructions for creating affordable, self-made switches.

\subsection{DIY Switch Designs}
\begin{itemize}
    \item \textbf{Exploratorium - Homemade Switches}: Provides simple, creative ideas for making switches using common, inexpensive materials like cardboard, aluminum foil, and craft supplies. Examples include "donut switches," "clothespin switches," and "feather switches." These are excellent for understanding the basic principles and creating highly customized, low-cost solutions  \cite{Exploratorium_Switches}.
    \begin{itemize}
        \item \textbf{Access}: Free instructions.
        \item \textbf{Link}: \url{https://www.exploratorium.edu/tinkering/homemade-switches}
    \end{itemize}
    \item \textbf{Battery Interrupters}: A common low-cost adaptation is to use battery interrupters. These are thin discs with conductive material that can be inserted into the battery compartment of a battery-operated toy or device, allowing a standard accessible switch to control the device's power. Many AT providers or DIY guides explain how to make or use them  \cite{ATKansans_Interrupters}.
    \begin{itemize}
        \item \textbf{Access}: Low-cost materials or pre-made adapters.
    \end{itemize}
\end{itemize}

\subsection{3D Printable Switch Designs}
\begin{itemize}
    \item \textbf{Switched Adapted Toys - 3D Printed Switches}: Offers free, open-source downloadable STL files for 3D printing various accessible switch buttons. These designs are often optimized for ease of activation and durability.
    \begin{itemize}
        \item \textbf{Content}: Includes designs for Mini, Standard, and Mega switch buttons with interchangeable tops, allowing for customization based on user needs.
        \item \textbf{Access}: Free downloadable STL files.
        \item \textbf{Link}: \url{https://switchedadaptedtoys.com/collections/3d-print-files}
    \end{itemize}
\end{itemize}

\subsection{Makers Making Change}
Makers Making Change is a strong advocate for open-source assistive technology, including accessible switches.
\begin{itemize}
    \item \textbf{Open-Source AT Solutions Library}: They provide an extensive library of designs, including various accessible switches, complete with build instructions, material lists, and downloadable 3D print files. Users can either build the devices themselves or request them from volunteer makers.
    \begin{itemize}
        \item \textbf{Access}: Free designs and instructions, or request from a maker (cost varies based on materials and maker fees).
        \item \textbf{Link}: Explore their AT Solutions Library, often under "Switches" or "Input Devices," at \url{https://www.makersmakingchange.com/at-solutions-library/}
    \end{itemize}
\end{itemize}

---

\section{Conclusion}
The landscape of AAC for individuals with visual impairments, blindness, and deafblindness is rich with innovative strategies and supportive resources. From foundational principles of tactile communication and object-based systems to the expanding possibilities of 3D printed tactile symbols and accessible, self-made switches, the options for enhancing communication are more diverse than ever. Instructional programs, whether comprehensive university courses, specialized webinars, or free online guides, equip professionals and families with the knowledge to implement these vital tools effectively. By leveraging these resources, we can continue to empower individuals with sensory impairments to connect with their world and communicate their unique voices.

---

\begin{thebibliography}{99}

\bibitem{AACA_CVI} The AAC Academy. (n.d.). \textit{AAC \& CVI Series: Beyond Red \& Yellow}. Retrieved from \url{https://www.theaacacademy.org/course/aac-for-children-with-cvi}

\bibitem{ASHA} American Speech-Language-Hearing Association. (n.d.). \textit{Augmentative and Alternative Communication (AAC)}. Retrieved from \url{https://www.asha.org/practice-portal/professional-issues/augmentative-and-alternative-communication/}

\bibitem{ATKansans_Interrupters} Assistive Technology for Kansans. (n.d.). \textit{Low Tech AT Solutions}. Retrieved from \url{https://www.atk.ku.edu/low-tech-at-solutions}

\bibitem{AugResources_Carrier} Augmentative Resources. (n.d.). \textit{Bliss Tactile Symbol Carrier}. Retrieved from \url{https://www.augresources.com/Bliss-Tactile-Symbol-Carrier-p/033138.htm}

\bibitem{CTG_Fun} Closing The Gap. (n.d.). \textit{Making Switch Activities Fun!}. Retrieved from \url{https://www.closingthegap.com/shop/?add-to-cart=64326&product_order=desc&product_orderby=popularity&product_view=grid&product_count=36}

\bibitem{CTG_Literacy} Closing The Gap. (n.d.). \textit{Bringing Literacy to Life: Using AAC \& Sensory Elements to Build Connections On-Demand}. Retrieved from \url{https://www.closingthegap.com/product/bringing-literacy-to-life-using-aac-sensory-elements-to-build-connections-on-demand/}

\bibitem{CTG_SelfRegulation} Closing The Gap. (n.d.). \textit{Switching to Self-Regulation: Using Assistive Technology to Create a Sensory-Based Calming Space for Students with Complex Communication \& Access Needs}. Retrieved from \url{https://www.closingthegap.com/webinars/archived-webinars/}

\bibitem{Exploratorium_Switches} Exploratorium. (n.d.). \textit{Homemade Switches}. Retrieved from \url{https://www.exploratorium.edu/tinkering/homemade-switches}

\bibitem{HKNC_Haptics} Helen Keller National Center. (n.d.). \textit{Haptics: Using Touch to Convey Visual and Environmental Information to People who are Deaf-Blind}. Retrieved from \url{https://www.helenkeller.org/hknc/class/haptics-using-touch-convey-visual-and-environmental-information-people-who-are-deaf-blind}

\bibitem{HKNC_Communicate} Helen Keller National Center. (n.d.). \textit{Communicating with Individuals Who Are DeafBlind}. Retrieved from \url{https://www.helenkeller.org/courses/communicating-with-individuals-who-are-deafblind/}

\bibitem{MakersMakingChange} Makers Making Change. (n.d.). \textit{AT Solutions Library}. Retrieved from \url{https://www.makersmakingchange.com/at-solutions-library/}

\bibitem{MyDynamicTherapy} My Dynamic Therapy. (n.d.). \textit{Getting Started with Auditory Scanning and Switch Progression}. Retrieved from \url{https://www.mydynamictherapy.com/blog/getting-started-with-auditory-scanning-and-switch-progression}

\bibitem{NCDB_OHOA} National Center on Deafblindness. (n.d.). \textit{Open Hands, Open Access (OHOA) Modules}. Retrieved from \url{https://www.nationaldb.org/products/online-learning/ohoa-modules/}

\bibitem{NCDB_Perkins} National Center on Deafblindness. (n.d.). \textit{Perkins eLearning - Active Learning Is More Than a Piece of Equipment! Tutorial}. Retrieved from \url{https://www.nationaldb.org/products/online-learning/perkins-elearning-active-learning-tutorial/}

\bibitem{NextSense_Services} NextSense. (n.d.). \textit{Accessibility and Inclusion Services}. Retrieved from \url{https://www.nextsense.org.au/services/vision/accessibility-and-inclusion}

\bibitem{NumberAnalytics_Switches} Number Analytics. (n.d.). \textit{Switch Access in AAC: A Complete Guide}. Retrieved from \url{https://numberanalytics.com/blog/switch-access-in-aac-a-complete-guide/}

\bibitem{PathsToLiteracy} Paths to Literacy. (n.d.). \textit{Home}. Retrieved from \url{https://www.pathstoliteracy.org/}

\bibitem{Perkins_CVI} Perkins eLearning. (n.d.). \textit{CVI Certificate Program}. Retrieved from \url{https://www.perkinselearning.org/earn-ceus/cvi-certificate-program}

\bibitem{ProjectCore} Project Core. (n.d.). \textit{Communication Systems}. Retrieved from \url{https://www.project-core.com/communication-systems/}

\bibitem{ProjectCore_Teaching} Project Core. (n.d.). \textit{How to teach students to use the 3D symbols}. Retrieved from \url{https://www.project-core.com/communication-systems/3d-symbols/}

\bibitem{Spectronics_Switches} Spectronics. (n.d.). \textit{Making Sense of Switches}. Retrieved from \url{https://www.spectronics.com.au/article/making-sense-of-switches}

\bibitem{SwitchedAdaptedToys} Switched Adapted Toys. (n.d.). \textit{3D Print Files}. Retrieved from \url{https://switchedadaptedtoys.com/collections/3d-print-files}

\bibitem{TactileComms} Tactile Communications, LLC. (n.d.). \textit{Services}. Retrieved from \url{https://www.tactilecommunications.org/Services/}

\bibitem{TeachingVI_Switches} Teaching Visually Impaired. (n.d.). \textit{Switch Access}. Retrieved from \url{https://www.teachingvisuallyimpaired.com/switch-access.html}

\bibitem{TSBVI_ActiveLearning} Texas School for the Blind and Visually Impaired. (n.d.). \textit{Principles of Active Learning (Online Course)}. Retrieved from \url{https://www.tsbvi.edu/statewide-resources/professional-development/online-learning}

\bibitem{USD} University of South Dakota. (n.d.). \textit{DeafBlind Program}. Retrieved from \url{https://www.usd.edu/Academics/Colleges-and-Schools/sanford-school-of-medicine/Research-and-Outreach-Centers/Center-for-Disabilities/Programs-and-Services/DeafBlind-Program}

\bibitem{Volksswitch_Bliss} Volksswitch. (n.d.). \textit{Bliss Tactile Symbols}. Retrieved from \url{https://volksswitch.org/index.php/bliss-tactile-symbols/}

\bibitem{64ouncebraille} 64ouncebraille.com. (n.d.). \textit{Project Core 3D Universal Core Vocabulary}. Retrieved from \url{https://64ouncebraille.com/products/project-core-3d-universal-core-vocabulary}

\end{thebibliography}
