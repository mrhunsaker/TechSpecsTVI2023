\chapter{Communication Accommodations for Individuals with Visual Impairments, Blindness, and Deafblindness}\label{app6:appx6}

\begin{raggedright}
	\textbf{Accessibility\index{accessibility} Note:} This appendix provides a comprehensive review of communication strategies for individuals with visual impairments\index{visual impairment}, blindness, and deafblindness. The content has been structured for clarity, navigation, and accessibility\index{accessibility}, with semantic markup and descriptive context.
\end{raggedright}

\section{Executive Summary}\label{app6:exec-summary}
This appendix provides a comprehensive review of communication strategies for individuals with visual impairments\index{visual impairment}, blindness, and deafblindness. It covers a wide range of topics, from foundational principles of Augmentative and Alternative Communication (AAC) to specific training programs, instructional materials, and resources for creating accessible communication tools. The appendix is designed to be a practical guide for educators, therapists, families, and individuals seeking to enhance communication skills and independence\index{independence}.

\section{Introduction}\label{app6:intro}
Effective \gls{communication} is a fundamental human right and is essential for learning, social interaction, and quality of life. For individuals with visual impairments, blindness, and deafblindness, \gls{communication} can be challenging due to sensory loss. However, with the right strategies, tools, and support, these individuals can become effective communicators. This appendix explores the diverse landscape of \gls{communication} strategies, with a focus on AAC, to support this population.

\section{Defining Augmentative and Alternative Communication (AAC)}\label{app6:define-aac}
Augmentative and Alternative Communication (AAC) refers to a wide range of strategies and tools that supplement or replace spoken language. For individuals with visual impairments, AAC can include tactile symbols, braille, switch access, and specialized software\index{software}. The goal of AAC is to provide a means of communication that is effective, efficient, and tailored to the individual's unique needs and abilities.

\subsection{General Principles of AAC Assessment and Intervention}\label{app6:aac-principles}
A successful AAC intervention begins with a thorough assessment of the individual's sensory abilities, motor skills, cognitive level, and communication needs. The assessment should be a collaborative process involving the individual, family, educators, and therapists. Intervention should be person-centered, focusing on teaching functional communication skills in natural environments. The chosen AAC system should be flexible and adaptable to grow with the individual over time.

\section{Foundational AAC Strategies and Theoretical Frameworks in Training}\label{app6:aac-frameworks}
Several foundational strategies and theoretical frameworks guide the teaching of AAC to individuals with visual impairments\index{visual impairment}. These approaches emphasize the importance of starting with concrete, meaningful interactions and progressing toward more abstract forms of communication.

\subsection{Touch for Connection and Communication}\label{app6:touch-comm}
For individuals with deafblindness, touch is the primary channel for receiving information and building relationships. Communication partners must learn to use touch in a way that is respectful, predictable, and meaningful. This includes using \gls{tactile} cues, hand-under-hand guidance, and \gls{tactile} sign language.

\subsection{Progressing from Non-symbolic to Symbolic Communication}\label{app6:symbolic-comm}
Many individuals begin with non-symbolic communication, such as gestures, facial expressions, and body language. The goal is to help them transition to symbolic communication, where a symbol (e.g., an object, a picture, a word) represents a concept. This transition requires systematic instruction and consistent use of symbols in meaningful contexts.

\subsection{Object Communication Systems}\label{app6:object-comm}
Object communication systems use tangible objects to represent activities, people, or places. For example, a spoon might represent "time to eat." This is a concrete way to introduce symbolic communication and is often a bridge to more abstract systems like tactile symbols or braille\index{braille}.

\subsection{The Universal Core Vocabulary and its Application in AAC}\label{app6:core-vocab}
Core vocabulary consists of a small set of high-frequency words (e.g., "go," "more," "want," "not") that make up the majority of what we say. Teaching core vocabulary is a highly effective strategy in AAC because it allows individuals to communicate a wide range of messages across different contexts. The Universal Core Vocabulary has been adapted for use with individuals with visual impairments\index{visual impairment}, including the creation of 3D printable tactile symbols.

\section{Directory of Comprehensive AAC Training Programs}\label{app6:aac-training}
A wide range of organizations offer training and professional development in AAC for individuals with visual impairments\index{visual impairment}. These programs provide valuable knowledge and skills for educators, therapists, and families.

\begin{itemize}
	\item \textbf{National Center on Deafblindness (NCDB):} Offers a wealth of resources, webinars, and technical assistance on communication strategies for individuals with deafblindness.
	\item \textbf{University of South Dakota (USD) - DeafBlind Program:} Provides specialized training and coursework for professionals working with individuals who are deafblind.
	\item \textbf{Helen Keller National Center (HKNC):} A leading national resource for individuals with deafblindness, offering comprehensive training, research, and support.
	\item \textbf{National Research \& Training Center on Blindness \& Low Vision (NRTC) at Mississippi State University:} Conducts research and provides training on a wide range of topics related to blindness and low vision, including communication.
	\item \textbf{Perkins School for the Blind:} Offers a variety of professional development opportunities, including webinars, workshops, and online courses on topics such as CVI\index{CVI}, deafblindness, and AAC.
	\item \textbf{Texas School for the Blind and Visually Impaired (TSBVI):} Provides extensive online resources, publications, and training on educational strategies for students with visual impairments.
	\item \textbf{Association for Education and Rehabilitation of the Blind and Visually Impaired (AERBVI):} A professional organization that offers conferences, webinars, and publications on all aspects of visual impairment, including communication.
	\item \textbf{Closing The Gap, Inc.:} Hosts an annual conference and provides resources on assistive technology\index{assistive technology}, including AAC.
	\item \textbf{NextSense (Australia):} An organization that provides services and conducts research for individuals with hearing and vision loss, including resources on communication.
	\item \textbf{The AAC Academy:} Offers online courses and training on AAC assessment and intervention.
	\item \textbf{Tactile Communications, LLC:} Provides training and resources on tactile communication strategies for individuals with deafblindness.
\end{itemize}

\subsection{National Center on Deafblindness (NCDB)}\label{app6:ncdb}
The NCDB is a national technical assistance center funded by the U.S. Department of Education. It provides a wide range of resources, including publications, webinars, and online courses, on topics related to deafblindness. The NCDB's website is a valuable source of information on communication strategies, assessment, and intervention for individuals with deafblindness.

\subsection{University of South Dakota (USD) - DeafBlind Program}\label{app6:usd-db}
The University of South Dakota offers a graduate program in deafblindness, preparing professionals to work with this unique population. The program covers topics such as communication, assessment, curriculum, and collaboration. Graduates are equipped with the knowledge and skills to provide high-quality educational services to individuals with deafblindness.

\subsection{Helen Keller National Center (HKNC)}\label{app6:hknc}
The Helen Keller National Center is a comprehensive national program that provides training and resources for individuals who are deafblind. HKNC offers a variety of programs, including vocational training, independent living skills\index{independent living skills}, and communication instruction. The center is a leader in the field of deafblindness and a valuable resource for individuals, families, and professionals.

\subsection{National Research \& Training Center on Blindness \& Low Vision (NRTC) at Mississippi State University}\label{app6:nrtc}
The NRTC conducts research and provides training on a wide range of topics related to blindness and low vision. The center's work focuses on improving employment outcomes, independent living, and quality of life for individuals with visual impairments\index{visual impairment}. The NRTC offers online courses, publications, and other resources that are relevant to communication and assistive technology\index{assistive technology}.

\subsection{Perkins School for the Blind}\label{app6:perkins}
Perkins is a world-renowned leader in the education of students with visual impairments and deafblindness. The school offers a variety of professional development opportunities, including online courses, webinars, and workshops. Perkins' eLearning platform, Perkins eLearning, provides a wealth of resources on topics such as CVI\index{CVI}, deafblindness, and communication.

\subsection{Texas School for the Blind and Visually Impaired (TSBVI)}\label{app6:tsbvi}
TSBVI is a state-funded school that provides a wide range of services to students with visual impairments in Texas. The school's website is an extensive resource for educators, families, and professionals, offering publications, webinars, and instructional materials on a variety of topics, including communication and assistive technology\index{assistive technology}.

\subsection{Association for Education and Rehabilitation of the Blind and Visually Impaired (AERBVI)}\label{app6:aerbvi}
AERBVI is a professional organization for individuals who work in the field of blindness and \gls{visualimpairment}. The organization offers conferences, webinars, and publications that cover a wide range of topics, including communication, \gls{assistivetechnology}, and education. AERBVI is a valuable resource for networking and professional development.

\subsection{Closing The Gap, Inc.}\label{app6:ctg}
Closing The Gap is an organization that focuses on assistive \gls{technology} for individuals with disabilities. The organization hosts an annual conference that brings together experts in the field to share the latest research and best practices. Closing The Gap also provides a variety of resources, including a magazine and online articles, on assistive \gls{technology} topics.

\subsection{NextSense (Australia)}\label{app6:nextsense}
NextSense is an Australian organization that provides services and conducts research for individuals with hearing and vision loss. The organization offers a variety of programs, including early intervention, school support, and cochlear implants. NextSense is a leader in the field of sensory loss and a valuable resource for individuals and families.

\subsection{The AAC Academy}\label{app6:aac-academy}
The AAC Academy is an online platform that offers courses and training on AAC assessment and intervention. The academy's courses are designed for professionals who work with individuals who use AAC, including speech-language pathologists, educators, and therapists. The AAC Academy is a valuable resource for building skills in this specialized area.

\subsection{Tactile Communications, LLC}\label{app6:tactile-comm}
Tactile Communications provides training and resources on tactile communication strategies for individuals with deafblindness. The company offers workshops and consultations on topics such as tactile sign language, protactile, and the use of tactile symbols. Tactile Communications is a specialized resource for those working with individuals who rely on touch for communication.

\section{Instructional Programs for Teaching Switch-Based Communication Skills}\label{app6:switch-comm}
For individuals with significant physical disabilities, switches can provide a reliable way to access communication devices. Teaching switch skills requires a systematic approach that builds from basic cause-and-effect to more complex scanning and selection techniques.

\subsection{Foundational Principles and Progression}\label{app6:switch-principles}
\begin{itemize}
	\item \textbf{Cause and Effect:} Start with activities where activating a switch produces a direct, motivating result (e.g., turning on a toy or music).
	\item \textbf{Building Skills:} Gradually introduce concepts like waiting for the right time to activate the switch and using the switch for different functions.
	\item \textbf{Scanning:} Introduce scanning techniques (e.g., row-column scanning) where the individual activates the switch to select an item from a group.
	\item \textbf{Timing and Accuracy:} Work on improving the speed and accuracy of switch activation.
\end{itemize}

\subsection{Webinars and Instructional Guides}\label{app6:switch-guides}
\begin{itemize}
	\item \textbf{AbleNet:} Offers a variety of webinars and resources on teaching switch skills, from foundational concepts to advanced applications.
	\item \textbf{Linda Burkhart:} A leading expert in switch access, Linda Burkhart's website offers a wealth of articles, handouts, and presentations on teaching switch skills.
	\item \textbf{Jane Farrall:} An independent consultant and trainer who provides resources and workshops on literacy, AAC, and assistive technology\index{assistive technology}, including switch access.
	\item \textbf{Ian Bean:} Offers a range of free, accessible switch activities and resources through his website, SENict.
\end{itemize}

\section{Purchase and Access Options for 3D Printable and Tactile Graphic Communication Modalities}\label{app6:3d-comm}
The advent of 3D printing\index{3D printing} has opened up new possibilities for creating customized, low-cost tactile communication tools. Several projects and organizations are making 3D printable models and tactile graphics\index{tactile graphics} widely available.

\subsection{Project Core 3D Universal Core Vocabulary}\label{app6:project-core}
Project Core, based at the University of North Carolina at Chapel Hill, has developed a set of 3D printable tactile symbols for the Universal Core Vocabulary. These symbols are designed to be easily recognizable by touch and are available for free download. This initiative makes it possible for schools and families to create their own sets of core vocabulary symbols.\supercite{64ouncebraille}

\subsection{Bliss Tactile Symbols and Expansions}\label{app6:bliss-symbols}
Blissymbolics is a graphical communication system that has been adapted for tactile use.
\begin{itemize}
	\item \textbf{Bliss Tactile Symbols:} A set of standardized tactile symbols based on the Blissymbolics system.
	\item \textbf{APH\index{braille embosser!APH} TactileBliss Symbols:} The American Printing House for the Blind offers a set of durable, thermoformed tactile Blissymbols.
	\item \textbf{3D Printable Models:} Community-driven efforts have led to the creation of 3D printable versions of Blissymbols, making them more accessible.
\end{itemize}

\subsection{Tactile Graphics Creation Tools and Services}\label{app6:tactile-graphics}
\begin{itemize}
	\item \textbf{APH Tactile Graphic Image Library (TGIL):} A free online resource from APH that contains thousands of downloadable tactile graphics on a wide range of subjects.
	\item \textbf{Touch Graphics, Inc.:} A company that specializes in the design and production of high-quality tactile graphics\index{tactile graphics} and maps.
	\item \textbf{Swell Paper\index{tactile graphics!swell paper}:} A type of microcapsule paper\index{tactile graphics!capsule paper} that allows users to create raised-line graphics by printing with a standard inkjet or laser printer and then heating the paper in a specialized oven.
	\item \textbf{3D Pens:} A low-cost tool that can be used to create simple, on-the-fly
\end{itemize}
