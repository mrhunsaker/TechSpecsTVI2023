\chapter{Analysis of Windows Screenreader Technologies}

\section{~~Introduction to Screenreader Technology and its Ecosystem}

\subsection{Defining the Fundamental Role of Screenreaders in Digital Accessibility}
Screenreaders are specialized software programs designed to provide digital accessibility for individuals who are blind or visually impaired. Their core function is to interpret the text and graphical information displayed on a computer screen and convert it into an alternative format, primarily synthesized speech or braille output via a refreshable braille display.\supercite{kingsbury2025} This transformation is crucial as it allows users to perceive and interact with digital content that would otherwise be inaccessible. In essence, a screenreader acts as the indispensable interface, bridging the gap between the computer's operating system, its diverse applications, and the user.\supercite{kingsbury2025}
The operational mechanism of screenreaders fundamentally substitutes vision with auditory or tactile feedback. Instead of using a mouse for navigation and command execution, users rely on specific keystroke combinations. These keystrokes can be either unique commands developed specifically for the screenreader program itself or, more commonly, standard Windows-based keystrokes that are universally recognized by the operating system and its applications.\supercite{kingsbury2025} The effectiveness of this interface layer directly determines a user's ability to engage with any digital content. If the screenreader, as the primary intermediary, encounters inconsistencies or limitations at the operating system or application level, the user's access can be significantly impeded, even if the screenreader itself is robust. This foundational role underscores why a singular approach to screenreader usage is often insufficient, necessitating proficiency across multiple screenreader tools and a deep understanding of the underlying operating system and application commands.

\subsection{Overview of Leading Windows Screenreaders: JAWS, NVDA, and Windows Narrator}
This report focuses on the three most prominent screenreader programs utilized within the Windows operating system environment: Job Access with Speech (JAWS), Non-Visual Desktop Access (NVDA), and Windows Narrator.\supercite{kingsbury2025} These tools form the bedrock of digital interaction for many blind and visually impaired individuals globally.
The scope of this analysis, mirroring the comprehensive nature of its source material, aims to provide a robust understanding of how these screenreaders integrate with and enable access to a wide array of commonly used PC applications. This includes the Windows operating system itself, the four primary applications within the Microsoft Office Suite—Word, Outlook, Excel, and PowerPoint—and the three most popular web browsers: Google Chrome, Microsoft Edge, and Mozilla Firefox.\supercite{kingsbury2025} Beyond these core applications, the report also delves into cloud-based file sharing programs such as Dropbox, OneDrive, and Google Drive, the widely used Adobe Acrobat Reader for accessing PDF files, and popular remote teleconferencing platforms like Zoom and Microsoft Teams.\supercite{kingsbury2025} Furthermore, it covers audio and video editing tools like Audacity, Windows MovieMaker, and Machete Lite, and provides insights into the most popular Google Workspace applications, including Docs, Sheets, Slides, Gmail, Calendar, and Forms.\supercite{kingsbury2025}
The technical foundation for this analysis is rooted in specific software versions to ensure accuracy and relevance. The information presented is based on observations from Windows 10 23H2 (released in Fall 2023), the Fall 2023 release of Windows 11 (23H2), JAWS 2024, NVDA 2023.3, and the version of Narrator bundled with Windows 11 23H2. For Microsoft Office applications, Office 365 is the primary reference, with noted differences where they exist for purchased versions dating back to 2016.\supercite{kingsbury2025} The dynamic nature of technology, particularly in assistive technologies, means that software undergoes continuous updates. This inherent fluidity necessitates that educational resources in this field remain adaptable and current. The source material explicitly acknowledges this, stating that "technology is always changing, so some of the text may become outdated as time passes. The hope is to continue updating the book every year, making it a living document rather than a one-off that loses its relevance in just a short time – a built-in problem with technology books".\supercite{kingsbury2025} This perspective highlights that screenreader proficiency is not a static acquisition but demands ongoing learning and adaptation from users. For instructors, it mandates constant curriculum revision, and for developers, it underscores the need for consistent, well-documented updates and careful consideration of backward compatibility.

\subsection{Foundational Interaction Paradigms: Modifier Keys and Universal Keystrokes}
A fundamental aspect of screenreader interaction revolves around the use of modifier keys. Each of the three primary screenreader programs—JAWS, NVDA, and Narrator—allows for the use of up to three such keys. These modifier keys, when pressed in combination with other keys, trigger specific functions of the screenreader program. Common modifier keys include the physical Insert key (also known as the Extended Insert key), the Zero key on the numeric keypad, and the Caps Lock key.\supercite{kingsbury2025} While all desktop keyboards typically feature these three keys, some laptop models may lack one or both Insert keys.\supercite{kingsbury2025}
Default settings for modifier keys vary by screenreader. Narrator, for instance, has all three modifier keys (Insert, Num Pad Zero, and Caps Lock) set by default. In contrast, JAWS and NVDA default to using the Insert and Num Pad Zero keys, requiring users to manually configure the Caps Lock key as a modifier if desired.\supercite{kingsbury2025} When the Caps Lock key is designated as a modifier, its traditional function of toggling capital letters on and off is accessed by tapping it twice quickly. Users of Fusion, a combined ZoomText and JAWS product, need to tap Caps Lock three times for this purpose.\supercite{kingsbury2025} In practical usage, many users find it advantageous to employ the Caps Lock key with their left hand, especially when the accompanying combination keys are located on the right side of the keyboard.\supercite{kingsbury2025} It is important to note that user guides for these screenreaders may also refer to modifier keys as the "JAWS key," "NVDA key," or "Narrator key".\supercite{kingsbury2025} For consistency and brevity within this report, the term "Insert key" will generally be used when referencing the use of modifier keys.\supercite{kingsbury2025}
A critical aspect of screenreader mastery, and a central theme of the source material, is the reliance on universal Windows keystrokes. The document explicitly states that "the vast majority of keystrokes covered in this book are actually Windows keystrokes rather than keystrokes created by screenreader program developers".\supercite{kingsbury2025} This observation underscores a crucial pedagogical and practical reality: while screenreaders provide the auditory interface, true proficiency in navigating a PC environment hinges more on mastering the fundamental commands of the underlying operating system and its applications. This means that the learning curve for transitioning between different screenreaders, once a user has become comfortable with one, is relatively gentle because the core Windows and Microsoft Office keystrokes remain consistent across all of them.\supercite{kingsbury2025} This consistency also proves invaluable in troubleshooting, as it allows users to more effectively discern whether an issue is specific to a particular screenreader or a broader problem related to how the operating system or an application interacts with assistive technologies. This understanding significantly streamlines problem-solving and informs effective training methodologies.

\section{~~Comprehensive Comparative Analysis of Windows Screenreaders}

\subsection{Economic and Licensing Models: A Detailed Cost-Benefit Perspective}
The economic and licensing models of JAWS, NVDA, and Windows Narrator present a clear spectrum of cost and accessibility, significantly influencing their adoption and usage patterns.
\subsubsection{JAWS: Proprietary Licensing and Cost}
JAWS operates on a proprietary licensing model, meaning it is a paid software product.\supercite{kingsbury2025} Its licensing structure permits use on up to three computers, provided the user is the primary operator of those machines. Freedom Scientific, the developer of JAWS, offers both Home and Pro licenses. While the features across these license types are identical, Home licenses are intended for personal use, whereas Pro licenses are designated for institutional settings.\supercite{kingsbury2025}
The outright purchase cost for the current version of JAWS is approximately \$1,000 in the United States. Many users initially acquire JAWS through state blindness agencies, which may cover the initial purchase. Subsequent upgrades, however, typically become the user's responsibility.\supercite{kingsbury2025} Annual upgrades are released in late October or early November. The most economical method for maintaining an updated JAWS license is through a two-year Software Maintenance Agreement (SMA). For instance, purchasing an SMA for JAWS 2024 and 2025 before January 1, 2023, would have cost \$150. Delaying this purchase beyond the deadline would increase the cost, for example, to \$180.\supercite{kingsbury2025} An alternative is an annual license, which, as of early 2023, cost \$95. This option is particularly relevant for users whose full price license is not covered by an agency or employer, or for those who are several versions behind on their SMA renewals.\supercite{kingsbury2025} Vispero, the creator of JAWS, also offers periodic discounts, such as a 20% discount for all US subscribers since 2020, often coinciding with national conventions like those of the American Council of the Blind and the National Federation of the Blind.\supercite{kingsbury2025}
\subsubsection{NVDA: Open-Source Model and Donation-Based Support}
In stark contrast to JAWS, NVDA is a free, open-source software.\supercite{kingsbury2025} While it is freely downloadable and installable, users are encouraged to make a donation if they have the financial means to support its ongoing development.\supercite{kingsbury2025} Although NVDA utilizes voices built into the Windows operating system by default, users who desire the same high-quality Eloquence and Vocalizer Expressive voices available with JAWS can purchase them for a modest price, around \$69 as of March 2024 from third-party vendors like AT Guys.\supercite{kingsbury2025}
\subsubsection{Windows Narrator: Integrated System Utility and Value Proposition}
Windows Narrator is Microsoft's native screenreader, integrated directly into the Windows operating system. This means it is available "out of the box" on any Windows 10 or 11 PC, making it a cost-free solution for users.\supercite{kingsbury2025}
The differing economic models for these screenreaders reveal a significant dynamic in screenreader adoption: a discernible economic divide. The substantial cost associated with JAWS, despite its advanced features and dedicated support, creates a barrier to access for many individuals and organizations globally. This financial consideration directly contributes to observed usage patterns, where JAWS tends to be the primary screenreader in more affluent nations like the United States, while NVDA, being free, gains prominence in less affluent regions.\supercite{kingsbury2025} This disparity highlights that digital accessibility is not solely a technical challenge but is profoundly influenced by economic factors. The availability of free alternatives like NVDA is therefore crucial for promoting equitable access to digital information worldwide. This also has implications for training programs, which may need to prioritize instruction in free tools to reach a broader audience, or advocate for funding for commercial solutions when their advanced features are essential for specific professional or educational contexts.

\subsection{Distinctive Strengths and Advanced Feature Sets}
Each of the leading Windows screenreaders possesses unique strengths and feature sets that cater to different user needs and preferences.
\subsubsection{JAWS: Unparalleled Customization, Robust Support Ecosystem, and Specialized Accessibility Tools}
JAWS is renowned for its high degree of customization capabilities, allowing users to tailor their computing experience extensively.\supercite{kingsbury2025} It is supported by an outstanding ecosystem of help and training resources, available in various formats including text, YouTube videos, and podcasts, complemented by excellent technical support accessible via phone or email.\supercite{kingsbury2025} JAWS also offers access to high-quality Vocalizer Expressive voices in over forty languages without additional cost, enhancing the auditory experience for a global user base.\supercite{kingsbury2025} A significant advantage for professional users is the ability for specialists to write custom scripts, which can profoundly improve the accessibility of specific applications in workplace environments, a factor that can be critical for job retention or acquisition.\supercite{kingsbury2025}
JAWS also features several unique and powerful tools:
\begin{itemize}
	\item \textbf{PictureSmart Generative AI:} Introduced in March 2024, this feature provides detailed descriptions of graphics, including charts and images in Excel and PowerPoint presentations, significantly enhancing access to visual content.\supercite{kingsbury2025}
	\item \textbf{Convenient OCR:} This utility enables the reading of otherwise inaccessible PDF files, particularly those created from scanned images, by converting them into readable text.\supercite{kingsbury2025}
	\item \textbf{Text Analyzer and Speech and Sound Schemes:} These are advanced tools designed to aid in proofreading documents, offering sophisticated methods for identifying and reviewing text attributes and errors.\supercite{kingsbury2025}
	\item \textbf{Speech History:} A highly valued tool that allows users to access and copy up to the last 500 announcements spoken by the JAWS synthesizer to the Clipboard. This is invaluable for capturing ephemeral information such as error messages, Zoom chat text, or content from dialog boxes.\supercite{kingsbury2025}
	\item \textbf{Sound Splitter:} Introduced with JAWS 2022, this feature allows users with stereo headphones or speakers to route JAWS speech to one ear while routing audio from other applications (e.g., remote meeting participants, YouTube videos) to the other. This is particularly useful in remote meetings, enabling screen sharing without the audience hearing the screenreader's speech.\supercite{kingsbury2025}
	\item \textbf{Notification History:} A feature introduced in JAWS 2022 that simplifies the customization of many notifications, offering options to mute, play a sound instead, or shorten the wording of alerts, either generally or for specific applications.\supercite{kingsbury2025}
	\item \textbf{FS Clipboard:} This feature, specific to Freedom Scientific products, retains multiple copied text items, which are appended using the keystroke Insert Windows C. Unlike the standard clipboard, it allows for the accumulation of text selections.\supercite{kingsbury2025}
\end{itemize}
\subsubsection{NVDA: Agility, Community-Driven Innovation (Add-ons), and Targeted Functional Advantages}
NVDA's primary strength lies in its cost-effectiveness, being a free software solution that makes it widely accessible globally.\supercite{kingsbury2025} Despite being free, it is fully capable of performing all necessary functions for beginner and intermediate screenreader users, and many advanced tasks as well.\supercite{kingsbury2025} In certain scenarios, NVDA can even perform functions more reliably or in less complicated ways than JAWS, such as its detection of cell text visibility issues in Excel.\supercite{kingsbury2025} Its functionality is further extended through a vibrant ecosystem of numerous small, open-source add-ons available from the NVDA Community Add-ons website, allowing users to customize and expand its capabilities.\supercite{kingsbury2025}
\subsubsection{Windows Narrator: Seamless Integration, High Responsiveness, and Evolving Natural-Sounding Voice Quality in Windows 11}
Narrator's most significant advantage is its seamless integration directly into the Windows operating system. This "out of the box" availability means users can immediately employ Narrator to initiate the setup process on a new computer.\supercite{kingsbury2025} It is characterized by its high responsiveness, opening almost instantaneously with the Control Windows Enter keystroke, exhibiting minimal lag compared to JAWS and NVDA.\supercite{kingsbury2025} With Windows 11, Narrator has also significantly improved its voice quality, offering several excellent natural-sounding voices (including Jenny, Aria, and Guy, with more added in 23H2) that are quick to install and maintain clarity and responsiveness across different speaking rates.\supercite{kingsbury2025}
The distinct strengths of these screenreaders position them along a spectrum of assistive technology solutions: JAWS as the premium, feature-rich, and extensively supported option; NVDA as the pragmatic, community-driven, and highly capable free alternative; and Narrator as the integrated, responsive baseline utility. This strategic positioning means that user choices are influenced not only by specific features but also by budgetary constraints, the need for dedicated technical support, and particular workflow requirements. JAWS, with its comprehensive features and robust support, is often favored in professional environments where specialized scripting and advanced tools are critical. NVDA's open-source model and community-led innovation foster broader adoption, empowering users who may not have access to paid software. Narrator, while less feature-rich, serves as a vital entry point for new users and a reliable backup or troubleshooting tool, demonstrating Microsoft's commitment to fundamental accessibility within its operating system.
To provide a concise overview of these differentiators, the following table presents a comparative feature matrix.

\begin{longtblr}[
	caption = {Comparative Feature Matrix of JAWS, NVDA, and Narrator},
	label = {tab:feature_matrix}
	]{
	colspec={X[0.8,l]X[l]X[l]X[l]},
	rowhead = 1
	}
	\toprule
	\textbf{Feature Category}  & \textbf{JAWS}                                                                                                                               & \textbf{NVDA}                                                                 & \textbf{Narrator}                                             \\
	\midrule
	Cost/Licensing Model       & Proprietary (Paid)                                                                                                                          & Free (Donation-based)                                                         & Free (OS Integrated)                                          \\
	\midrule
	Key Strengths              & Unparalleled Customization, Robust Support, Scripting Capabilities                                                                          & Agility, Community-Driven Innovation, Cost-Effectiveness                      & Seamless Integration, High Responsiveness, Out-of-the-Box Use \\
	\midrule
	Unique Features (Examples) & PictureSmart AI, Convenient OCR, Speech History, Sound Splitter, Notification History, FS Clipboard, Text Analyzer, Speech \& Sound Schemes & Add-ons, Targeted Functional Advantages (e.g., Excel cell text visibility)    & Evolving Natural Voices (Win 11), Immediate Startup           \\
	\midrule
	Support \& Training        & Excellent (Phone, Email, Extensive Resources)                                                                                               & Community-based (No dedicated desk), Less abundant resources                  & Built-in Help, Online Guide (Microsoft)                       \\
	\midrule
	Customization              & Highly Customizable                                                                                                                         & Moderate (via Settings \& Add-ons)                                            & Limited (via Settings)                                        \\
	\midrule
	Voice Quality/Options      & High-quality, Multi-language (40+) Vocalizer Expressive voices included                                                                     & Good (Windows voices), Premium Eloquence/Vocalizer Expressive optional (\$69) & Excellent Natural Voices (Win 11), Standard Voices (Win 10)   \\
	\midrule
	Responsiveness/Lag         & Lags on opening, Higher resource usage, Potential hiccups                                                                                   & Lags on opening, Moderate resource usage                                      & Immediate startup, Lower resource usage                       \\
	\midrule
	Primary Use Case           & Professional, Advanced Tasks, Complex Workflows                                                                                             & General Use, Budget-Conscious Users, Everyday Tasks                           & Initial System Setup, Fallback/Troubleshooting, Basic Use     \\
	\bottomrule
\end{longtblr}

\subsection{Identified Limitations and Practical Disadvantages}
While each screenreader offers distinct advantages, they also come with their own set of limitations and practical disadvantages that users must consider.
\subsubsection{JAWS: Resource Demands and Potential Performance Hiccups}
The primary drawback of JAWS is its proprietary cost, which can be a significant barrier to access for many individuals and organizations.\supercite{kingsbury2025} Furthermore, due to its powerful and comprehensive feature set, JAWS tends to consume more computer resources compared to its counterparts. This higher resource demand can sometimes lead to performance issues, such as the screenreader "going silent" or experiencing other "hiccups," even on reasonably fast computers. This can occur multiple times daily, disrupting workflow.\supercite{kingsbury2025} Additionally, JAWS typically experiences noticeable lags when opening, particularly when compared to the near-instantaneous startup of Windows Narrator.\supercite{kingsbury2025}
\subsubsection{NVDA: Support Infrastructure Gaps, Documentation Limitations, and Feature Development Pace Compared to JAWS}
Despite its significant advantage of being free, NVDA has several limitations. It lacks a dedicated technical support desk, meaning users must rely on community forums and online resources for assistance.\supercite{kingsbury2025} While its documentation is comprehensive, it is generally less abundant and centralized than the extensive resources available for JAWS.\supercite{kingsbury2025} The pace of new feature introduction in NVDA, often through community-developed add-ons, is generally slower and the features themselves may be less significant or frequent compared to JAWS. The documentation for these add-ons can also be limited, and some add-ons may not be consistently updated, potentially leading to compatibility issues or abandonment.\supercite{kingsbury2025} Similar to JAWS, NVDA also experiences some lag when opening, particularly in comparison to Narrator.\supercite{kingsbury2025}
\subsubsection{Windows Narrator: Feature Parity Challenges and "Full Service" Capabilities Compared to Commercial Alternatives}
Windows Narrator, while seamlessly integrated and highly responsive, is generally not characterized as a "full service" screenreader. Its capabilities are more limited compared to the extensive feature sets of JAWS and NVDA.\supercite{kingsbury2025} It remains uncertain whether Microsoft intends to develop Narrator to directly compete with the broader range of features offered by JAWS and NVDA.\supercite{kingsbury2025}
Specific feature gaps in Narrator, as highlighted in the source material, include:
\begin{itemize}
	\item The absence of an equivalent to JAWS's Insert F1 for screen-sensitive help, which provides contextual information on elements like margin dimensions, headers, and footers in Word documents.\supercite{kingsbury2025}
	\item The current inability to read Word table headers, which significantly hampers data interpretation in complex documents.\supercite{kingsbury2025}
	\item A lack of quick navigation methods to footnote superscripts in the main text, making it harder to locate references efficiently.\supercite{kingsbury2025}
	\item No comparable tool to JAWS's Skim Reading utility for quickly finding format changes or highlighted text within documents.\supercite{kingsbury2025}
	\item Inability to identify Excel cell dimension information or detect cell text visibility problems, which can lead to issues with data display for sighted collaborators.\supercite{kingsbury2025}
	\item Unreliable identification of Excel border formatting, making it difficult for users to confirm visual presentation.\supercite{kingsbury2025}
	\item No auditory announcements when entering or exiting multi-column text passages in Word, potentially confusing navigation.\supercite{kingsbury2025}
	\item The absence of keystrokes for listing tables in Word or worksheets in Excel, which impedes efficient navigation in multi-table or multi-sheet documents.\supercite{kingsbury2025}
	\item Poor or non-existent spellchecking functionality in PowerPoint presentations, necessitating workarounds.\supercite{kingsbury2025}
	\item A critical limitation where Google Slides presentation mode is inaccessible for delivery, rendering the application largely unusable for live presentations by screenreader users.\supercite{kingsbury2025}
\end{itemize}
The identified limitations across these screenreaders underscore a fundamental trade-off between cost, features, and usability. JAWS's comprehensive power comes with a financial investment and a higher demand on system resources. NVDA's open-source nature provides broad access but entails reliance on community support and a slower pace of core feature development. Narrator's deep integration into the operating system offers immediate availability and responsiveness but at the expense of advanced functionalities. These trade-offs mean that a screenreader's "lack" of a feature is not merely an omission but has practical implications for different user needs. For instance, Narrator's limitations prevent it from being a viable primary screenreader for complex professional tasks, yet its instant availability makes it an invaluable tool for initial system setup or as a reliable fallback when other screenreaders encounter issues. This dynamic emphasizes the necessity of multi-screenreader proficiency, as users often combine tools to meet their diverse and evolving digital access requirements.

\subsection{Strategic Rationale for Multi-Screenreader Proficiency}
Becoming proficient in the use of more than one screenreader program offers significant advantages, transforming a user's approach to digital accessibility from reliance on a single tool to a versatile, adaptive strategy.
Firstly, the learning curve for additional screenreaders, once a user is comfortable with one, is surprisingly gentle. This is primarily because the most crucial screenreader keystrokes are often identical across different programs. Furthermore, the core skills required for PC usage—interacting with Windows, Microsoft Office applications, and web browsers—rely heavily on universal Windows keystrokes and processes that remain consistent regardless of the screenreader being used.\supercite{kingsbury2025} This consistency means that much of the foundational knowledge is transferable, making the acquisition of new screenreader skills relatively straightforward.
Secondly, employing multiple screenreaders is a powerful strategy for troubleshooting accessibility issues. When a user encounters a problem, for example, within Microsoft Word, and the issue persists irrespective of which screenreader is active (JAWS, NVDA, or Narrator), it strongly suggests that the problem lies with the application itself rather than the screenreader. This diagnostic capability greatly aids in articulating the issue more precisely when seeking technical support, increasing the likelihood of a swift resolution.\supercite{kingsbury2025}
Thirdly, web accessibility is not uniform across all browsers and screenreaders. Websites are often developed and tested with specific combinations in mind, leading to varying degrees of accessibility. When confronted with a challenging or "finicky" web page, switching to an alternative browser or screenreader can often help overcome navigation or content access roadblocks.\supercite{kingsbury2025} This adaptability is key to maintaining consistent access to online information.
Finally, for specialized tasks, the combined strengths of multiple screenreaders can be indispensable. Professionals involved in web accessibility testing, for instance, frequently find that using both JAWS and NVDA in tandem allows them to identify a broader spectrum of accessibility issues than would be possible with a single tool.\supercite{kingsbury2025} This "toolbox" approach to digital navigation extends beyond mere options; it cultivates a robust and adaptive set of strategies. The emphasis on troubleshooting, navigating web inconsistencies, and performing specialized tasks highlights a proactive and flexible user mindset. This adaptability is critical for overcoming the inherent inconsistencies in digital accessibility, where not all applications or websites are equally optimized for every screenreader. It shifts the paradigm from identifying the single "best" screenreader to developing a versatile skill set that maximizes digital independence.

\section{~~Current Trends and User Adoption Patterns in the Screenreader Community}

\subsection{Analysis of Recent WebAIM Survey Data (February 2024)}
Understanding the landscape of screenreader usage is crucial for developers, educators, and users alike. The WebAIM organization, an accessibility training and consulting service based in Utah, conducts a bi-annual Screenreader User Survey that provides valuable insights into these patterns.\supercite{kingsbury2025} The results of their Tenth survey, completed in February 2024, offer a comprehensive snapshot of current adoption.
The survey gathered responses from 1,539 individuals globally. A significant portion of respondents hailed from North America (47.2\%), followed by Europe (30.7\%), with the remaining 22.1\% from other parts of the world.\supercite{kingsbury2025} This geographic distribution should be considered when interpreting the data, as it may introduce some regional biases, potentially favoring screenreaders more prevalent in North America.\supercite{kingsbury2025}
The survey inquired about users' primary screenreader programs:
\begin{itemize}
	\item JAWS was reported as the primary screenreader by 40.5\% of respondents.\supercite{kingsbury2025}
	\item NVDA closely followed, with 37.7\% of users identifying it as their primary screenreader.\supercite{kingsbury2025}
	\item Windows Narrator had a significantly smaller share, reported as primary by only 0.7\% of users.\supercite{kingsbury2025}
	\item Other screenreaders cited by respondents included VoiceOver on the Mac, Zoomtext/Fusion, System Access, and ChromeVox.\supercite{kingsbury2025}
\end{itemize}
Historical trends reveal a notable shift in primary screenreader usage over time. NVDA's adoption as a primary tool has experienced consistent growth, surging from a mere 3\% in 2009 to 38\% in the latest survey. Concurrently, JAWS's share as the primary screenreader has declined from 68\% in 2009 to 40\% in 2024.\supercite{kingsbury2025}
Beyond primary usage, the survey also highlights a prevalent trend of multi-screenreader proficiency among users. When asked which screenreaders they commonly use, the figures are considerably higher:
\begin{itemize}
	\item NVDA is commonly used by 65.6\% of respondents.\supercite{kingsbury2025}
	\item JAWS is commonly used by 60.5\% of respondents.\supercite{kingsbury2025}
	\item Narrator sees a substantial increase in common usage, jumping to 37.3\% from its very low primary usage figure.\supercite{kingsbury2025}
\end{itemize}
This data strongly suggests that most experienced users do not rely on a single screenreader but rather leverage multiple tools. The survey also points to a socio-economic and geographic influence on primary screenreader choice: JAWS tends to be more frequently adopted as the primary tool in more affluent countries like the United States, while NVDA's usage predominates in less affluent regions.\supercite{kingsbury2025}
The WebAIM survey data points to a significant democratization of screenreader access and an evolving sophistication among users. NVDA's rapid ascent in primary usage, nearly equaling JAWS, is not merely a statistical shift; it reflects a broader trend towards more accessible and open-source solutions. The high percentage of users who commonly employ multiple screenreaders (both NVDA and JAWS exceeding 60\%, and Narrator over 37\%) indicates a user base that strategically utilizes different tools for varying contexts, moving beyond a sole dependency on any single screenreader. This trend suggests that accessibility solutions are becoming more diverse and user-centric. The decrease in JAWS's primary market share, despite its continued high common usage, implies that while its advanced features remain valued, its cost may be a barrier to being the exclusive solution for many. Narrator's significant common usage, despite its minimal primary adoption, reinforces its role as a critical fallback or initial system setup tool. This evolving landscape necessitates that assistive technology instructors and developers prioritize interoperability, offer comprehensive training across various tools, and deeply understand the diverse needs of a multi-tool user base.
To visualize these trends, the following table presents the key findings from the WebAIM survey.

\begin{longtblr}[
	caption = {WebAIM Screenreader User Survey Results (February 2024)},
	label = {tab:webaim_survey}
	]{
	colspec={X[l]X[l]X[l]X[l]},
	rowhead = 1
	}
	\toprule
	\textbf{Metric}                      & \textbf{JAWS}                   & \textbf{NVDA}                  & \textbf{Narrator}       \\
	\midrule
	Primary Usage (\%)                   & 40.5\%                          & 37.7\%                         & 0.7\%                   \\
	\midrule
	Common Usage (\%)                    & 60.5\%                          & 65.6\%                         & 37.3\%                  \\
	\midrule
	Trend (Primary Usage, 2009 vs. 2024) & From 68\% (2009) to 40\% (2024) & From 3\% (2009) to 38\% (2024) & N/A (Low initial usage) \\
	\bottomrule
\end{longtblr}

\subsection{Implications of Evolving Usage Trends for Accessibility Development and User Training}
The evolving usage patterns of screenreaders carry significant implications for both accessibility development and user training methodologies. The choice of a primary screenreader is often influenced by external factors, including the willingness of state agencies and employers to fund proprietary software like JAWS, the availability of specialized training programs, and the user's geographic location.\supercite{kingsbury2025} However, ultimately, the most effective screenreader choice should align with an individual's specific needs, the demands of their school or workplace environment, their personal preferences, and their financial circumstances.\supercite{kingsbury2025}
From an instructional standpoint, the increasing diversity of screenreader options is a positive development. The author, who has utilized JAWS since 2004, expresses appreciation for this expanded landscape, noting that it is "no longer the 'only game in town'".\supercite{kingsbury2025} This perspective highlights a broader shift towards personalized assistive technology solutions. The interplay of policy, pedagogy, and personalization in assistive technology adoption is evident in these trends. The data linking usage patterns to factors like funding and training availability indicates that screenreader adoption is not merely a technical decision but is heavily shaped by systemic and educational infrastructures. This understanding means that for policy-makers and funding bodies, supporting a diverse range of assistive technology options, including open-source tools, is crucial for ensuring equitable access. For educators, it mandates training programs that emphasize adaptability and the strategic use of multiple screenreaders, rather than focusing on a single product. For developers, it underscores the need for robust interoperability and consistent accessibility standards across platforms, acknowledging that users will likely combine tools to meet their complex and varied digital needs.

\section{~~Advanced Accessibility Practices and Nuanced Insights Across Key Applications}

\subsection{Optimizing the Windows Operating System Environment for Screenreader Users}
The Windows operating system forms the foundational layer for all screenreader interactions. Optimizing its environment is paramount for ensuring a smooth, efficient, and accessible computing experience. This involves configuring critical initial settings, mastering file and folder management, and leveraging advanced Windows features.

\subsubsection{Critical Initial Configuration Settings for Enhanced Usability}
Upon acquiring a new computer, several default settings in the Windows environment can significantly impact screenreader usability. Modifying these initial configurations is crucial for enhancing efficiency and preventing common frustrations. The following table summarizes these crucial initial Windows environment settings for screenreader optimization.

\begin{longtblr}[
	caption = {Recommended Initial Windows Environment Settings for Screenreader Optimization},
	label = {tab:windows_settings}
	]{
	colspec={X[l]X[l]X[l]X[l]},
	rowhead = 1
	}
	\toprule
	\textbf{Setting}               & \textbf{Recommended Action}          & \textbf{Benefit for Screenreader Users}                                                            & \textbf{Key Consideration/Caveat}                                     \\
	\midrule
	Deletion Confirmations         & Enable                               & Prevents accidental data loss by providing auditory confirmation.                                  & Default prioritizes speed over safety for AT users.                   \\
	\midrule
	File Extensions                & Unhide                               & Easier to distinguish files from folders and identify file types.                                  & Enhances auditory information density.                                \\
	\midrule
	Protected View (Office)        & Disable Permanently                  & Improves file behavior and responsiveness; reduces usability friction.                             & Introduces security trade-off; rely on user vigilance.                \\
	\midrule
	FN Key (Laptop)                & Set to Classic Function              & Enables standard F-key commands (e.g., Alt F4, F7, Insert F12).                                    & May require inaccessible BIOS access for some models.                 \\
	\midrule
	Touchpad (Laptop)              & Disable                              & Prevents accidental cursor movements and window switches.                                          & Essential to know re-enable method for sighted assistance.            \\
	\midrule
	Wi-Fi Network                  & Configure                            & Establishes essential internet connectivity.                                                       & Standard network connection process.                                  \\
	\midrule
	Control Panel View             & Change to Icons View                 & Ensures consistent and accessible navigation within Control Panel.                                 & Improves discoverability of settings.                                 \\
	\midrule
	Startup Apps                   & Review \& Disable Unnecessary        & Improves boot-up time, reduces annoying notifications, prevents performance impediments.           & Proactive performance management is critical for AT stability.        \\
	\midrule
	File Path Display in Title Bar & Turn Off                             & Reduces auditory verbosity, improving cognitive load.                                              & Full path still accessible via other commands (Alt D, JAWS Insert A). \\
	\midrule
	Default Programs               & Configure Preferred                  & Sets desired applications for various file types/tasks.                                            & Windows 11 changes are more complex than Windows 10.                  \\
	\midrule
	Windows 10 Startup Sound       & Enable                               & Provides auditory assurance of proper boot-up.                                                     & Personal preference; Windows 11 has default sound.                    \\
	\midrule
	File Explorer Default Open     & Reassign to "This PC" (or cloud app) & Provides a more useful and accessible starting point for file navigation than Windows 11's "Home." & Tailors the digital workspace for non-visual efficiency.              \\
	\midrule
	Control S for Classic Save As  & Enable                               & Streamlines the initial saving process by reverting to a familiar dialog.                          & Avoids a potentially confusing alternative dialog.                    \\
	\bottomrule
\end{longtblr}

\subsubsection{Efficient File and Folder Management Strategies: Structure, Navigation, and Organization}
Effective file and folder management is a cornerstone of efficient computer usage for screenreader users. A clear understanding of the computer's folder structure, coupled with proficient navigation and organization techniques, significantly reduces the likelihood of losing files, enhances confidence in downloading content, simplifies the management of external devices, streamlines cloud application usage, and facilitates reliable data backups.\supercite{kingsbury2025}
The hard drive's (C: Drive) folder structure forms a hierarchical system. The highest level is "This PC," which serves as the gateway to the entire computer's folder structure. This can be accessed via Windows E (if reassigned) or by typing "This PC" in the Windows search bar.\supercite{kingsbury2025} Below "This PC" are "Devices and drives" (including the C: drive itself), then the C: drive containing "Program Files" and the "Users" folder. Within the "Users" folder, individual user folders reside, and within a user's personal folder, key directories like "Documents," "Downloads," "Favorites," "OneDrive," "Music," "Pictures," and "Video" are found.\supercite{kingsbury2025} The "Documents" folder is typically the primary location for user-created files, while "Downloads" is the default for internet content.\supercite{kingsbury2025} Navigation within this hierarchy involves pressing Enter to descend into a folder and Backspace, Alt Left arrow, or Alt Up arrow to ascend. The Alt Up arrow is generally more reliable for moving up the hierarchy regardless of the entry point.\supercite{kingsbury2025}
Creating new folders is straightforward: press Control Shift N, type the desired name, and press Enter.\supercite{kingsbury2025} Folders can be renamed using the F2 key, followed by editing the text and pressing Enter.\supercite{kingsbury2025} Selecting files and folders can be done individually by arrowing, or by pressing Spacebar on the first item if it's not already selected. Contiguous selections are made by holding Shift and arrowing, while non-contiguous selections involve holding Control and pressing Spacebar on desired items while skipping others.\supercite{kingsbury2025} All items in a folder can be selected with Control A, and selections from the cursor to the top or bottom of a list are achieved with Control Shift Home/End.\supercite{kingsbury2025} Once selected, items can be copied (Control C), cut (Control X), or pasted (Control V).\supercite{kingsbury2025} Deleting files or folders involves selecting them and pressing the Delete key; if deletion confirmations are enabled, a prompt will appear.\supercite{kingsbury2025} The Recycle Bin, accessed via the Desktop, allows for permanent deletion or restoration of items, though files deleted from external devices are permanently lost.\supercite{kingsbury2025}
Proactive desktop management is essential for efficiency. Maintaining a limit of 40-50 desktop icons, restricting web page shortcuts to frequently visited sites, and creating shortcuts for often-accessed folders (e.g., Documents, Downloads, cloud apps) and short-term project files are recommended practices.\supercite{kingsbury2025} Desktop icons can be reorganized by right-clicking (Applications key) on the Desktop, accessing the View submenu to change size (large, medium, small), and the Sort by submenu to arrange by name, size, item type, or date modified.\supercite{kingsbury2025}
File Explorer offers various views. The "Details View," common for documents and audio files, displays filename, last modified date, file type, and size.\supercite{kingsbury2025} Users can navigate right to hear these additional details. NVDA and JAWS can also reveal folder size via Properties (Applications key -\textgreater{} Properties), though Narrator cannot.\supercite{kingsbury2025} This view can be customized to show specific information.\supercite{kingsbury2025} The "Large Icons View," typical for pictures and videos, is visually oriented and can be confusing for screenreader users due to its grid layout, requiring right/left arrowing to detect all files.\supercite{kingsbury2025} Switching views in Windows 10 involves Alt, right-arrow to View tab, tabbing to "Layout Toolbar change your view" drop-down, and selecting "Details".\supercite{kingsbury2025} In Windows 11, it's Alt, right-arrow to View button, Spacebar, then down-arrow to "Details".\supercite{kingsbury2025} These changes can be set as defaults.\supercite{kingsbury2025} Files within views can be sorted by various criteria (name, date, type, size) and in ascending or descending order.\supercite{kingsbury2025} Folder tree views organize folders hierarchically, allowing navigation through levels using arrow keys to expand/collapse, but cannot open individual files.\supercite{kingsbury2025}
Opening applications and individual files can be done in several ways. Applications can be launched by pressing the Windows key, typing the application name, and pressing Enter.\supercite{kingsbury2025} Existing files are opened by navigating to them in a folder and pressing Enter, which launches them in their default program.\supercite{kingsbury2025} Desktop shortcuts provide quick access for files, folders, and applications. For folders and files, navigate to the item, open the Applications key menu, select "Send To" -\textgreater{} "Desktop create shortcut".\supercite{kingsbury2025} For applications, use the Windows key to search for the app, open its file location via the Applications key, then use "Send To" to create a desktop shortcut.\supercite{kingsbury2025} If a shortcut becomes "unavailable," returning to the Desktop and trying again often resolves the issue.\supercite{kingsbury2025} Custom shortcut keys (hotkeys) can also be assigned to programs (Control Alt + letter/number) for quick access from anywhere, though care must be taken to avoid conflicts with existing key combinations.\supercite{kingsbury2025}

\subsubsection{Leveraging Windows 11 Quick Settings and the Windows Clipboard for Streamlined Workflows}
Windows 11 introduces "Quick Settings," a convenient feature accessed by pressing Windows A. This consolidated panel provides rapid access to frequently adjusted system parameters such as Wi-Fi connectivity, Bluetooth, Airplane mode, Accessibility options (Magnifier, Color filters, Narrator, Mono audio, Live captions, Sticky keys), screen brightness, and volume.\supercite{kingsbury2025} This consolidation of system controls is a positive development in accessible design, as it streamlines common tasks by reducing the number of steps and disparate locations a user needs to navigate, thereby enhancing efficiency for screenreader users. Users can also manage Wi-Fi connections, adjust brightness and volume sliders, and check battery status from this panel.\supercite{kingsbury2025}
Managing notifications is another critical aspect of optimizing the Windows environment. General strategies include tailoring notifications within individual applications (e.g., Outlook) and disabling unnecessary startup apps (e.g., Microsoft Teams), which can generate excessive and disruptive alerts.\supercite{kingsbury2025} In Windows 10, notifications are managed through the Actions Center (Windows A), where users can manage notifications for individual apps, clear recent alerts, or clear all notifications. "Focus Assist" can also be configured to silence notifications during specific times.\supercite{kingsbury2025} Windows 11 utilizes a "Notifications Center" (Windows N), which displays recent notifications and allows for app-specific settings or clearing all alerts.\supercite{kingsbury2025} However, adding an app to the Notifications Center in Windows 11 can be cumbersome, requiring navigation through multiple settings menus.\supercite{kingsbury2025} This inconsistency and potential for notification overload highlight the cognitive burden placed on users and the need for more consistent, easily configurable, and non-intrusive notification systems in software design.
The Windows Clipboard is a powerful feature that enhances multitasking efficiency by allowing users to retain multiple copied items, unlike the traditional clipboard which only holds the last copied item.\supercite{kingsbury2025} This functionality must first be enabled by navigating to Windows Settings (Windows I), searching for "clipboard," selecting "Clipboard settings," and then tabbing to and enabling the "Save multiple items to the Clipboard" toggle.\supercite{kingsbury2025} Once enabled, users can copy several items (text only) and then paste them by pressing Windows V in a new document. This opens a chronological list of copied items, from which individual items can be pasted one at a time.\supercite{kingsbury2025} The clipboard content persists across multiple applications and can be cleared by shutting down the computer or by manually clearing it from the Windows Clipboard panel.\supercite{kingsbury2025} A limitation is that it does not retain formatting, and Excel formulas paste as values only.\supercite{kingsbury2025} This extended clipboard functionality significantly boosts productivity by streamlining workflows that involve information aggregation, allowing users to collect multiple pieces of data before pasting them.

\subsection{Mastering Microsoft Office Suite Accessibility for Enhanced Productivity}
The Microsoft Office Suite—comprising Word, Excel, and PowerPoint—is central to many professional and academic workflows. Mastering the accessibility features within these applications is crucial for screenreader users to achieve enhanced productivity.

\subsubsection{Word: Strategic Application of Styles and Headings for Document Structure and Navigation}
Microsoft Word is a powerful word processing application, and effective use of its features, particularly ribbon menus, formatting tools, and structural elements like headings and styles, is vital for screenreader users.
\paragraph{Ribbon Menus Overview:}
Word's interface is structured around ribbon menus, introduced in Office 2007. These consist of an Upper ribbon with tabs (e.g., File, Home, Insert, Design, Layout, References, Mailings, Review, View, Help, accessed via Alt + a letter shortcut like Alt F for File) and a Lower ribbon with grouped icons and commands, accessed by tabbing from the Upper ribbon.\supercite{kingsbury2025} Context-specific tabs, such as "Table Tools Design" and "Layout," appear only when relevant objects (like tables) are selected.\supercite{kingsbury2025} For correct screenreader behavior, ribbons should always be expanded (Control F1); a "File tab" as the initial landing point often indicates a collapsed state.\supercite{kingsbury2025} While ribbons visually organize many commands, they can be inefficient for screenreader users due to extensive tabbing. Shortcut keys (e.g., Control D for Font) or the Applications key (for context-sensitive menus) are often more efficient.\supercite{kingsbury2025} Office 2016 and later versions include a "Command Search" feature (Alt Q), allowing users to search for commands, access help, or add commands to the Quick Access Toolbar. This feature's functionality may vary across Outlook views.\supercite{kingsbury2025}
\paragraph{Formatting Basics:}
Format Checking (Insert F): A valuable keystroke for instantly identifying formatting. JAWS and NVDA reveal both font (attributes, name, color) and paragraph (line spacing, indentation, alignment) formatting, while Narrator only provides font information. Pressing Insert F twice allows for line-by-line reading of formatting details.\supercite{kingsbury2025}
Font Dialog (Control D or Applications key -\textgreater{} Font): This dialog allows changes to font name, style (Regular, Italic, Bold, Bold Italic), and size.\supercite{kingsbury2025} Changing font color is a somewhat quirky process, involving pressing Spacebar on the "Color" button, navigating a color palette, pressing Enter, and then tabbing to verify the change.\supercite{kingsbury2025} Basic attributes like bold, italic, and underline can be toggled on/off quickly using Control B, Control I, and Control U, respectively.\supercite{kingsbury2025}
Paragraph Dialog (Alt H P G or Applications key -\textgreater{} Paragraph): Key formatting options here include alignment (Left, Center, Right, Justify), left and right indentation (e.g., 0.5 inches), special indentation (First line or Hanging), and crucial "Before" and "After" paragraph spacing.\supercite{kingsbury2025} The default "After" spacing is 8 points; setting this to 0 is important if users manually add blank lines (by pressing Enter twice) to avoid excessive spacing between paragraphs.\supercite{kingsbury2025} Line spacing (Single or Double) is also configured here.\supercite{kingsbury2025}
Setting Margins (Alt P M or JAWS Insert F1): Default margins are one inch on all sides.\supercite{kingsbury2025} Pre-determined options like "Normal," "Narrow," or "Moderate" are available. Custom margins can be set by entering specific values for Top, Bottom, Left, and Right margins in the "Custom Margins" dialog.\supercite{kingsbury2025} JAWS's Insert F1 provides quick auditory feedback on current margin dimensions.\supercite{kingsbury2025}
Changing Default Font, Paragraph, and Margin Settings: To apply formatting changes as defaults for future documents, users must make the changes in the respective dialogs, then select "Set as default," and crucially, choose "All documents based on the Normal template".\supercite{kingsbury2025}
\paragraph{Bulleting and Numbering Lists:}
These features are used to organize items, with bulleting for unordered lists and numbering for ordered lists (e.g., recipes).
Bulleting: Text can be bulleted by selecting it and pressing Alt H U (Bullets split button), then Enter for the default solid dot, or Alt Down arrow for other options. Control Shift L provides a quick way to apply the default solid circle bullet, though it is not a toggle for removal.\supercite{kingsbury2025}
Numbering: Similar to bulleting, select text and press Alt H N (Numbering split button), then Enter for default numbering (1., 2., 3.) or Alt Down arrow for other styles (letters, Roman numerals).\supercite{kingsbury2025}
Adding/Removing Items: Pressing Enter at the end of a list item adds a new bullet or number; a second Enter ends the list.\supercite{kingsbury2025}
Removing: To remove bullets or numbering, select the list and use Alt H U/N, then navigate to "None" and press Enter or Spacebar.\supercite{kingsbury2025}
\paragraph{Headers and Footers (Including Page Numbering):}
Headers appear between the top edge of the page and the top margin, while footers are between the bottom margin and the page's physical bottom edge. By default, they repeat on every page.\supercite{kingsbury2025}
Page Numbers: Can be inserted in either headers or footers using Alt Shift P when in the respective pane.\supercite{kingsbury2025}
Checking (JAWS): JAWS's Insert F1 can indicate the presence of headers/footers and their text.\supercite{kingsbury2025} NVDA and Narrator lack this direct tool.\supercite{kingsbury2025}
Insertion: Access the Header (Alt N H) or Footer (Alt N O) submenu, select "Blank," and type text. Use Tab to align text (left, center, right).\supercite{kingsbury2025}
Removing/Editing: Headers/footers can be removed using Alt N H R or Alt N O R. Editing is often easier by removing and redoing.\supercite{kingsbury2025} Font consistency with the main document body is important and can be adjusted by selecting all text in the header/footer pane (Control A) and using the Font dialog (Control D).\supercite{kingsbury2025}
Different First Page: To have a unique header/footer on the first page, check the "Different first page" checkbox in the Header/Footer tab (accessed via Alt then Control Right arrow to Options group when in the header pane).\supercite{kingsbury2025}
Different in Sections: For varied headers/footers within a document, section breaks are required (discussed in Chapter 3.9.2). Within the header pane of the new section, the "Link to previous section" button must be unchecked to delink it from the prior section's header.\supercite{kingsbury2025}
\paragraph{Headings and Styles:}
Styles are powerful tools for ensuring consistent document formatting and aiding navigation, particularly through the use of headings. Headings organize and format text, establishing a hierarchical structure (Level 1, Level 2, etc.).\supercite{kingsbury2025}
Navigation:
\begin{itemize}
	\item JAWS/Narrator: Insert F6 provides a list of headings for quick navigation.\supercite{kingsbury2025}
	\item JAWS Quick Keys: Insert Z toggles Quick Keys mode, allowing navigation by heading (H/Shift H) or by heading level (1, 2, 3 on number row).\supercite{kingsbury2025}
	\item NVDA Browse Mode: Insert Spacebar toggles Browse mode, enabling similar navigation by heading (H/Shift H) and heading level (1, 2, 3).\supercite{kingsbury2025}
	\item NVDA Elements List: Insert F7 opens the Elements list, where headings can be navigated, though the level numbering can be confusing for screenreader users.\supercite{kingsbury2025}
\end{itemize}
Adding Headings: Headings are easily applied by placing the cursor on the desired line and pressing Control Alt 1, 2, or 3 for Heading 1, 2, or 3 respectively.\supercite{kingsbury2025}
Removing Heading Format: Control Shift N removes heading formatting, converting the text to "Normal" style.\supercite{kingsbury2025}
What is a Style?: A style encapsulates all font and paragraph formatting for a text segment. Heading styles visually differentiate content, creating a hierarchy (e.g., larger point size, bolding, different spacing for higher-level headings).\supercite{kingsbury2025} Key formatting considerations include font name, attributes, point size, color, alignment, indentation, and "Before/After" paragraph spacing.\supercite{kingsbury2025}
Changing Styles for a Single Document: To modify a style for a specific document, select text formatted with that style, apply the desired font and paragraph changes, then open the Styles submenu (Alt H L), use the Applications key on the relevant "Heading X" or "Normal" button, and select "update Heading X to match selection".\supercite{kingsbury2025}
Changing Default Styles for Future Documents: This critical modification must be performed in a new, blank document. After applying the desired font and paragraph changes to a heading or "Normal" text, open the Styles submenu (Alt H L), use the Applications key on the style button, select "Modify," then choose "New documents based on this template" before confirming.\supercite{kingsbury2025}
Defining Shortcut Keys for Lower Level Headings: While Word provides shortcuts for levels 1-3, users can define custom shortcuts for lower levels (e.g., Level 4+). This involves navigating to the desired heading level (Alt Shift Right arrow), opening the Styles submenu (Alt H L), modifying the style, and assigning a custom shortcut key (e.g., Control Alt 4).\supercite{kingsbury2025}
Selecting Similar Text to Mark As Headings: For documents with visually formatted but unmarked headings, users can select one instance of the text, then use Home tab -\textgreater{} Editing group -\textgreater{} Select submenu -\textgreater{} "Text with similar formatting" to select all similarly formatted text. This allows for quick application of the correct heading style.\supercite{kingsbury2025}
Shifting Heading Levels: To globally adjust heading levels (e.g., turning all Level 3 headings into Level 4), it is crucial to start with the lowest level headings and work upwards. Select the target heading level, use the Styles submenu to "Select all (no data)" for that style, then apply the new heading level. For shifting levels upwards, start with the highest level and work downwards.\supercite{kingsbury2025}
\paragraph{Table Accessibility: Ensuring Readability with Column/Row Headers and Gridline Management}
Tables are used to present information clearly, though for numeric calculations, Excel is often preferred.\supercite{kingsbury2025} Tables should generally be limited to 5-6 narrow columns to fit on a page.\supercite{kingsbury2025} JAWS typically provides a superior experience with tables compared to NVDA and Narrator, offering clearer audio feedback and more formatting information.\supercite{kingsbury2025}
Navigation within Tables: Users can navigate cells using Control Alt + four arrow keys, or Tab/Shift Tab.\supercite{kingsbury2025} JAWS offers a "Layered Mode" (Insert Spacebar, T) for additional navigation flexibility (e.g., moving to first/last cell in row/column, top-left/bottom-right cell).\supercite{kingsbury2025} JAWS also has specific keystrokes to read the current column (Alt Windows Period) or row (Alt Windows Comma), and its Insert F1 provides screen-sensitive help for table formatting.\supercite{kingsbury2025}
Navigation Between Tables: JAWS Quick Keys (Insert Z, T/Shift T) or its List of Tables (Control Insert T) allow quick movement between tables.\supercite{kingsbury2025} NVDA and Narrator use Browse/Scan Mode (Insert Spacebar, T/Shift T) but lack a dedicated list of tables.\supercite{kingsbury2025}
Reading Column and Row Headers: JAWS and NVDA should announce cell data with their corresponding column and row headers. JAWS typically speaks headers first with a lower pitch, while NVDA speaks them second.\supercite{kingsbury2025} JAWS's Quick Settings can correct improper header reading.\supercite{kingsbury2025} Narrator currently does not read Word table headers.\supercite{kingsbury2025}
Creating Tables: The table body is created via the Insert tab, Tables submenu (Alt N T), by selecting the desired grid size.\supercite{kingsbury2025} Default tables have thin gridlines, which can be verified with JAWS Insert F1.\supercite{kingsbury2025} Gridlines can be removed or customized (e.g., setting thick borders) via the Table Tools Design tab's "Borders and shading" option (Alt J T O).\supercite{kingsbury2025} Rows and columns can be added or deleted using the Applications key's "Insert" or "Delete" submenus.\supercite{kingsbury2025}
Title or Caption: A title should be added above the table using the "Insert Captions" command in the References tab (Alt S P), which facilitates consistent formatting and numbering.\supercite{kingsbury2025} Table numbering can be updated after reordering or inserting tables by selecting the entire document and pressing F9.\supercite{kingsbury2025}
Notes on Sources: Additional notes or sources can be added by inserting a new row at the bottom of the table, merging its cells, and then customizing the gridlines to hide them, making them appear as external notes while remaining part of the table structure.\supercite{kingsbury2025}
Table Styles: Word offers a gallery of over 100 visual table styles, though these are difficult for screenreader users to preview.\supercite{kingsbury2025} Custom table styles can be created (Alt H L S) to ensure consistent formatting across multiple tables, defining aspects like fonts and borders.\supercite{kingsbury2025}
Importing Data from Excel: Data from Excel spreadsheets can be easily imported into Word tables. Select the data range in Excel, copy it (Control C), then paste it into Word using Control V (for no gridlines) or "Paste options" -\textgreater{} "Use destination styles" (for gridlines and consistent fonts).\supercite{kingsbury2025}
\paragraph{Footnotes and Endnotes:}
Footnotes and endnotes provide supplementary information without cluttering the main text; footnotes appear at the bottom of the page, while endnotes are at the end of a document or chapter.\supercite{kingsbury2025}
Creating and Editing Footnotes: To create a footnote, place the cursor precisely at the insertion point in the main text, then press Alt S F (References tab, Insert footnote), and type the text in the Footnote pane. Exit the pane by arrowing up or down.\supercite{kingsbury2025} Footnotes can be edited by navigating to the Footnote pane or by using Alt S H (Show notes button) to access a list of all notes.\supercite{kingsbury2025}
Navigating to Footnotes in the Text Body and Deleting Them: The "Go to next footnote" (Alt S O N) and "Go to previous footnote" (Alt S O P) commands navigate to the superscript in the main text. Footnotes are deleted by deleting their corresponding superscript in the main body, not by deleting the text in the Footnote pane.\supercite{kingsbury2025} JAWS offers quicker navigation with Quick Keys (Insert Z, O/Shift O) or a Virtual Viewer (Windows Semi-Colon, then "Footnotes").\supercite{kingsbury2025} NVDA and Narrator primarily rely on ribbon shortcuts for navigation.\supercite{kingsbury2025}
Footnote Styles: Footnote text format should be consistent with the main document, often with a slightly smaller font size. The default is Calibri 10pt.\supercite{kingsbury2025} To change the footnote style, place the cursor in the Footnote pane, use Applications key -\textgreater{} Styles, select "Footnote text," "Modify," and adjust font/paragraph settings.\supercite{kingsbury2025} A custom shortcut key (e.g., Control Alt F) can be assigned to the footnote text style.\supercite{kingsbury2025}
Endnotes: Created similarly to footnotes via Alt S E (References tab), endnotes typically appear at the end of a document section or chapter, with default lowercase Roman numeral superscripts.\supercite{kingsbury2025} JAWS can navigate endnotes with Quick Keys (Insert Z, D) or the Virtual Viewer (Windows Semi-Colon, then "Endnotes").\supercite{kingsbury2025} Footnotes and endnotes can be converted between each other using the "Footnote and endnote dialog" in the References tab.\supercite{kingsbury2025}
Customizing How You Hear Footnotes with JAWS: JAWS allows granular control over auditory feedback for footnotes. In JAWS Quick Settings (Insert V), users can type "footnote" and cycle through options like "Off," "On," "On with text" (reads footnote text, lower pitch for Eloquence voices), or "On plus count" (reads text and indicates number of superscripts on a line).\supercite{kingsbury2025} This fine-grained control over auditory verbosity allows users to balance informational needs with reading flow, reducing cognitive load. NVDA and Narrator have fewer customization options for footnote audio.\supercite{kingsbury2025}
\paragraph{Citations and Bibliographies:}
Word's built-in "Citations and Bibliography" feature streamlines the creation and formatting of academic and professional documentation.\supercite{kingsbury2025}
Style Type: Users first select a citation style (e.g., APA, Chicago, MLA) from the "Styles" combo box (Alt S L) in the References tab.\supercite{kingsbury2025}
Creating Citations: Citations are created by placing the cursor at the desired insertion point, then using Alt S C (Insert Citation submenu) and selecting "Add new source." Users fill in relevant information (author, title, year, publisher, etc.) for various source types (book, journal, website).\supercite{kingsbury2025}
Inserting Existing Citations: Previously created citations can be inserted from the current document's list or from a master list of citations accessible via the "Manage sources" dialog (Alt S M).\supercite{kingsbury2025}
Creating a Bibliography: A bibliography can be generated at the end of the document (typically on a new page via Control Enter) by selecting the appropriate option (References, Works Cited, or Bibliography) from the "Bibliography submenu" (Alt S B) in the References tab.\supercite{kingsbury2025}
Editing and Updating Citations and Bibliographies: Individual citations can be edited via the "Manage sources" dialog (Alt S M).\supercite{kingsbury2025} Bibliographies can be updated (e.g., for page number changes) by placing the cursor in the bibliography and using the Applications key -\textgreater{} "Update field".\supercite{kingsbury2025} Deleting a citation requires manually deleting its superscript in the main text and then recreating the bibliography to remove it from the list.\supercite{kingsbury2025}
Bibliography Styles: Bibliographies often require specific font and paragraph formatting (e.g., left alignment, hanging indentation, 0pt "Above paragraph spacing" for reference entries) to align with academic style guides.\supercite{kingsbury2025} A custom bibliography style can be created and saved for consistent application.\supercite{kingsbury2025}
Exporting Your Master List of Citations to Another Computer: The master list of citations is stored locally as a "Sources.xml" file. This file can be copied (Control C) from the "Manage sources" dialog -\textgreater{} "Browse" and then pasted to a cloud application folder or a thumb drive for use on other computers, ensuring consistency across devices.\supercite{kingsbury2025}
\paragraph{Advanced Text Manipulation: Leveraging Find/Replace, Skim Reading, Bookmarks, Highlighting, and Extended Selection}
Beyond basic text editing, Word offers advanced tools for efficient text manipulation, navigation, and review.
Find (Control F) and Find and Replace (Control H): The "Find" command (Control F) allows users to search for specific text, announcing the number of results and jumping to occurrences. Control Page down/up navigate between occurrences.\supercite{kingsbury2025} The "Find and Replace" command (Control H) is a significant time-saver for replacing numerous instances of text or special characters (e.g., correcting "Jaws" to "JAWS," removing extra hard returns \textasciicircum{}p\textasciicircum{}p to \textasciicircum{}p, or deleting text by leaving the replace field empty).\supercite{kingsbury2025} Special characters like section breaks (\textasciicircum{}B), graphics (\textasciicircum{}G), and tabs (\textasciicircum{}t) can also be found.\supercite{kingsbury2025}
Skim Reading Text: To quickly grasp the general content of a document without reading every word, screenreader users can employ skim reading strategies. This involves jumping from paragraph to paragraph (Control Down arrow for all screenreaders).\supercite{kingsbury2025} JAWS and Narrator also offer paragraph navigation in their respective Quick Keys/Scan Modes (Insert Z / Insert Spacebar, then P).\supercite{kingsbury2025} JAWS provides a dedicated "Skim Reading Utility" (Insert F2, then S) that can read the first line or sentence of every paragraph continuously, with options to interrupt and move to the next/previous unit.\supercite{kingsbury2025}
Go To Command (Control G or F5): This command allows users to quickly navigate to a specific page number or other document elements like bookmarks.\supercite{kingsbury2025}
Bookmarks: Bookmarks enable quick navigation to important passages. They are added by placing the cursor at the desired location, then pressing Alt N K or Control Shift F5, typing a unique name (no spaces), and adding it.\supercite{kingsbury2025} Bookmarks can be navigated using the "Go To" command or, with JAWS, via Quick Keys (Insert Z, B/Shift B).\supercite{kingsbury2025} Bookmarks can also be deleted from the Bookmarks dialog.\supercite{kingsbury2025}
Highlighting Text and Accessing It With the JAWS Skim Reading Tool: Text can be highlighted (e.g., with a yellow background) using Alt H I (Text highlight split button).\supercite{kingsbury2025} The JAWS Skim Reading tool can then be configured to quickly access these highlighted passages by creating a rule based on background color. This tool can also generate a summary document containing all highlighted text segments.\supercite{kingsbury2025} Similarly, it can detect and navigate text with foreground color changes (e.g., red text for review).\supercite{kingsbury2025} Removing color formatting can be a finicky process involving the Find and Replace command, requiring precise navigation within the dialog.\supercite{kingsbury2025}
Extended Text Selection: For selecting large blocks of text efficiently, JAWS offers Control Windows K to mark a temporary placemarker and then Insert Spacebar, M to select to the current cursor position.\supercite{kingsbury2025} NVDA uses Insert F9 to mark the start and Insert F10 to mark the end of a selection.\supercite{kingsbury2025} A universal Windows method involves pressing F8 to turn on extended selection mode, then navigating to the end of the desired text.\supercite{kingsbury2025}
\paragraph{Practical Applications of "Paste Text Only" and Word Count Features:}
Paste Text Only: When copying text from other applications (e.g., websites), using "Paste text only" (Alt H V T or Applications key -\textgreater{} Paste options -\textgreater{} Keep text only) prevents importing undesired formatting, ensuring the pasted text adopts the destination document's default style.\supercite{kingsbury2025}
Word Count: Word count for an entire document (Alt R W or Control Shift G) or a selected segment can be obtained from the Review tab or via the JAWS Status bar (Insert Page up).\supercite{kingsbury2025}
Customizing Status Bar: The Status bar at the bottom of the screen can be customized to display useful information like page number, vertical position, and word count. This is done by pressing F6 to reach the Status bar, then using the Applications key to check desired options.\supercite{kingsbury2025}
Inserting Special Characters: Foreign accents, currency symbols, and other special characters can be inserted using specific keystroke combinations (e.g., Control apostrophe + E for French acute E), an online table of international character shortcuts, or JAWS's special characters list (Insert 4).\supercite{kingsbury2025} For less common characters, the four-digit ASCII or Unicode number can be typed using the numeric keypad while holding Alt.\supercite{kingsbury2025}

\subsubsection{Excel: Data Interpretation and Management for Non-Visual Users}
Microsoft Excel is a powerful spreadsheet application, and its accessibility for non-visual users hinges on specific features and navigation techniques that convert visual data into an interpretable auditory format.
\paragraph{Basic Spreadsheet Elements:}
A spreadsheet is composed of cells, which are the intersections of rows and columns.\supercite{kingsbury2025} Data tables typically include row headers (text down the left column), column headers (text across the top row), and the table body where numeric or textual values appear.\supercite{kingsbury2025}
\paragraph{The "Define Name" Command: A Cornerstone for Spreadsheet Accessibility and Data Interpretation}
The "Define Name" command in Excel is a critical accessibility feature that significantly improves data interpretation for screenreader users. Sighted users can quickly glance at a cell (e.g., E8) and immediately understand its context (e.g., "rent for April"). For a screenreader user, this would typically require time-consuming navigation (up-arrowing to the column header, left-arrowing to the row header).\supercite{kingsbury2025} The "Define Name" command bridges this visual-auditory gap by automatically announcing the corresponding column and row headers as the user navigates cell by cell. With JAWS, headers are spoken first, often with a lower pitch, followed by the cell content. NVDA also provides this functionality, but speaks the headers second. As of the source material's publication, Narrator did not support this feature.\supercite{kingsbury2025}
To activate this feature, the user must place focus on the intersection cell of the desired row and column headers (e.g., cell A2 if row headers are in column A and column headers are in row 2). Then, open the Applications key menu, navigate up to "Define Name," press Enter, type the word "title" (without quotes), and press Enter.\supercite{kingsbury2025} This transforms a potentially confusing grid of data into a structured, audibly navigable table, drastically improving efficiency and reducing cognitive load for non-visual users. Variants of this command allow for hearing only column headers (by placing focus on the leftmost cell of the column headers and defining the name "columntitle") or only row headers (by placing focus on the cell just above the topmost row header and defining the name "rowtitle").\supercite{kingsbury2025}
\paragraph{Formulas, Data Sorting, and Multi-Worksheet Navigation Best Practices}
Formulas: Excel's power lies in its ability to perform calculations using formulas. Formulas begin with an equals sign (=), followed by cell references (e.g., =A1+B1).\supercite{kingsbury2025} The "AutoSum" feature (Alt H U) quickly sums a selected range of cells.\supercite{kingsbury2025} Formulas can be copied from one cell and pasted to others (Control C, Control V), with Excel automatically adjusting cell references (relative references) unless absolute references are specified using F4 to toggle dollar signs (\$)).\supercite{kingsbury2025} Formula auditing tools, like "Trace Precedents" and "Trace Dependents" (in the Formulas tab), help understand formula relationships, and error checking is available via Alt M E.\supercite{kingsbury2025}
Data Sorting: Data can be sorted in ascending (Alt A S A) or descending (Alt A S D) order based on the first column.\supercite{kingsbury2025} For more complex sorting, "Custom Sort" (Alt A S S) allows sorting by multiple columns, with options to add or delete sorting levels.\supercite{kingsbury2025}
Multi-Worksheet Navigation: Excel files can contain multiple worksheets. Users can navigate between them using Control Page down (next) and Control Page up (previous).\supercite{kingsbury2025} JAWS offers a "List of Worksheets" (Control Shift S) for efficient navigation and reordering, while NVDA provides a similar list via Insert F7 -\textgreater{} "Sheets" radio button. Narrator, however, does not have this capability.\supercite{kingsbury2025} Worksheets can be managed (inserted, deleted, renamed, moved, copied) using JAWS's context menu (Shift Insert S) or via ribbon shortcuts (e.g., Shift F11 for insert, Alt H O R for rename) for NVDA and Narrator users.\supercite{kingsbury2025} The "Paste link" command allows copying a formula from one worksheet and pasting it into another, maintaining a dynamic link so that changes in the source formula are reflected in the linked cell.\supercite{kingsbury2025}

\section{~~Conclusion}
The analysis of Windows screenreader technologies—JAWS, NVDA, and Narrator—reveals a dynamic and evolving landscape shaped by technological advancements, economic considerations, and user needs. Screenreaders serve as the fundamental interface between users with visual impairments and the digital world, making their design, functionality, and usability paramount.
The reliance on universal Windows keystrokes for the majority of screenreader interactions means that a user's proficiency with the underlying operating system and applications is more critical than memorizing screenreader-specific commands. This foundational understanding significantly flattens the learning curve when transitioning between different screenreaders and is invaluable for troubleshooting.
The comparative analysis highlights a clear spectrum of solutions. JAWS, as a proprietary, feature-rich tool with extensive support, caters to professional and complex use cases, offering unique capabilities like PictureSmart AI and advanced scripting. However, its cost and higher resource demands present barriers. NVDA, as a free, open-source, and community-driven alternative, has democratized access, offering robust functionality for a wide range of users, though it lacks dedicated technical support and its feature development pace may be slower. Narrator, integrated directly into Windows, provides immediate, responsive access and is improving its natural-sounding voices, serving as a vital baseline and troubleshooting tool despite its more limited feature set.
Recent WebAIM survey data confirms a significant shift in primary screenreader usage, with NVDA rapidly gaining ground on JAWS, indicating a broader trend towards more accessible and diverse solutions. The high percentage of users who commonly employ multiple screenreaders underscores a sophisticated user base that strategically leverages different tools to overcome accessibility inconsistencies across various applications and websites. This "toolbox" approach is essential for navigating the complexities of the digital environment.
Optimizing the Windows operating system environment is crucial for screenreader users. This involves configuring initial settings such as enabling deletion confirmations, unhiding file extensions, managing startup apps, and tailoring verbosity, all of which directly impact usability, safety, and efficiency. Furthermore, mastering file and folder management, leveraging Windows 11's Quick Settings, and utilizing the Windows Clipboard are key to streamlined workflows.
Within Microsoft Office applications, strategic application of features like Word's styles and headings is vital for document structure and navigation, while Excel's "Define Name" command is a cornerstone for non-visual data interpretation. The ability to customize auditory feedback, manage notifications, and efficiently manipulate text are all critical for enhancing productivity.
In conclusion, the landscape of Windows screenreaders is characterized by diversity, continuous evolution, and the imperative for users to adopt adaptable strategies. The choice of screenreader is a personalized decision influenced by a confluence of technical requirements, financial considerations, and the specific demands of a user's digital ecosystem. For developers, this necessitates a focus on interoperability and consistent accessibility standards. For educators, it mandates training that emphasizes adaptability and the strategic use of multiple screenreaders. For users, it means cultivating a versatile skill set to maximize digital independence.
