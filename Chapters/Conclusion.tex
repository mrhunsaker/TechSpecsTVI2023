\hypertarget{conclusion}{}\chapter[\raggedright Conclusion\hfill\break ]{Conclusion}\label{conclusion}
\noindent\makebox[\linewidth]{\rule{\linewidth}{0.4pt}}
\extramarks{Vision Department Technology Needs}{Conclusion}
In conclusion, the Individuals with Disabilities Education Act (IDEIA) mandates that students with disabilities, including those with visual impairments, must be given access to assistive technology to ensure they can participate fully in the curriculum\footnote{\raggedright \href{http://sites.ed.gov/idea/statuteregulations/}{20 U.S.C. § 1400, et.}}. Screen magnification is one such assistive technology that can help students with visual impairments access their free public education. The overarching goal of this document has been to shed light on the essential role that technology plays in not only accommodating these students but empowering them to thrive in educational environments. By providing students with visual impairments access to the technology they need, we can help ensure that they have the tools they need to succeed in their studies and beyond.

Assistive technology is a critical component of ensuring that students with visual impairments receive a free and appropriate public education. The technology needs of these students must be addressed within the framework of IDEIA, which mandates that students with disabilities must be given access to assistive technology to ensure they can participate fully in the curriculum. Screen magnification is one such assistive technology that can help students with visual impairments access their free public education. By providing students with visual impairments access to the technology they need, we can help ensure that they have the tools they need to succeed in their studies and beyond.

In addition to helping students with visual impairments access information and participate in classroom activities, assistive technology can also help these students become more independent. By providing students with the tools they need to access information and communicate with others, assistive technology can help them become more self-sufficient and less reliant on others 1. This can help improve their self-esteem and confidence, which can have a positive impact on their academic performance and overall well-being.

Finally, it is important to note that the use of assistive technology is not a one-size-fits-all solution. The technology needs of students with visual impairments can vary widely depending on their individual needs and abilities. Therefore, it is important to work with students, families, and educators to identify the most appropriate assistive technology solutions for each student. By doing so, we can help ensure that students with visual impairments have the tools they need to succeed in school and beyond.

In conclusion, the use of assistive technology is critical for students with visual impairments to receive a free and appropriate public education. The technology needs of these students must be addressed within the framework of IDEIA\footnotemark[\value{footnote}], which mandates that students with disabilities must be given access to assistive technology to ensure they can participate fully in the curriculum. The use of assistive technology is essential for students with visual impairments to access the same educational materials as their sighted peers. Assistive technology can help students with visual impairments access text information across all curricular areas and participate fully in instruction that is often rich with visual content. The use of assistive technology also helps prepare students for independent living, vocational pursuits, or higher education following graduation from high school. By providing students with visual impairments access to the technology they need, we can help ensure that they have the tools they need to succeed in their studies and beyond.
\addtocontents{toc}{\protect\clearpage}