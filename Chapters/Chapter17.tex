\chapter{Creating and Verifying PDF Accessibility with Adobe Acrobat}
\glsreset{ocr}\glsreset{icr}\glsreset{tts}\glsreset{llm}\glsreset{uia}\glsreset{msaa}\glsreset{pdfua}\glsreset{api}\glsreset{cpu}
% ------------------------------------------------------------------
% Added standardized pedagogical scaffold sections (do not remove).
% Existing content (Overview, Check Accessibility, Fix Issues,
% Accessibility Issues, etc.) should remain below and be adjusted
% manually if duplication occurs.
% ------------------------------------------------------------------

\section{~~Overview}\label{ch17:sec:overview-expanded}
This chapter (expanded) provides a structured, end-to-end framework for creating, auditing, and remediating accessible PDFs using Adobe Acrobat and complementary tools. The scaffold introduces learning objectives, key terminology, standards mapping, implementation strategies, a troubleshooting matrix, equity/ethics considerations, and assessment materials—augmenting the original procedural focus.

\section{~~Learning Objectives}\label{ch17:sec:learning-objectives}
After completing this chapter, the reader will be able to:
\begin{enumerate}
	\item Distinguish core PDF accessibility requirements (tagging, logical order, alt text, forms, metadata, security).
	\item Execute a phased PDF accessibility workflow (source authoring → tagging → auditing → remediation → validation).
	\item Map common PDF issues to WCAG 2.x and \gls{pdfua} requirements and articulate remediation steps.
	\item Diagnose and resolve recurring remediation pitfalls using a structured troubleshooting matrix.
\end{enumerate}

\section{~~Key Terms}\label{ch17:sec:key-terms}
\begin{description}
	\item[Tagged PDF] A PDF containing a logical structure (tag) tree exposing \gidx{readingorder}{reading order} and semantics to assistive technology.
	\item[Logical Reading Order] The sequence in which content is presented to assistive technologies—independent of visual layout.
	\item[Artifact] Non-informational content (decorative lines, page numbers) intentionally excluded from the tag tree.
	\item[Alternative Text] Textual replacement describing the purpose or content of a non-text element.
	\item[Role Map] Mapping between custom and standard tag names ensuring interoperability with assistive technology.
	\item[Form Field Tooltip / Accessible Name] Programmatic label announced by a screen reader for interactive fields.
	\item[\gls{pdfua}] ISO 14289 standard specifying requirements for universally accessible PDF documents and viewers.
\end{description}

\section{~~Historical and Policy Context}\label{ch17:sec:historical-policy}
PDF accessibility practices evolved from ad hoc post-production tagging toward source-first authoring and standards-driven validation as WCAG, Section 508 refresh, and \gls{pdfua} matured. Increasing digital textbook distribution, legal enforcement, and the emergence of enterprise document automation elevated expectations for consistent, testable accessibility outcomes.

\section{~~Core Concepts}\label{ch17:sec:core-concepts}
\begin{itemize}
	\item \textbf{Source Integrity:} Accessible tagging success depends heavily on semantic structure in the source (styles, headings, alt text placeholders).
	\item \textbf{Separation of Content vs. Presentation:} Visual order rarely matches logical \gidx{readingorder}{reading order} without deliberate tagging.
	\item \textbf{Deterministic Remediation:} Each issue category (e.g., untagged figure) maps to reproducible corrective actions.
	\item \textbf{Iterative Verification:} Automated checks + manual inspection + assistive technology testing form a completeness triad.
\end{itemize}

\section{~~Technologies and Tools}\label{ch17:sec:technologies-tools}
\begin{itemize}
	\item \textbf{Adobe Acrobat Pro:} Primary environment (Autotag, \gidx{readingorder}{Reading Order}, Tags panel, Preflight).
	\item \textbf{Validation Utilities:} PAC 2021, CommonLook Validator for \gls{pdfua} conformance depth.
	\item \textbf{Authoring Environments:} Word / InDesign / LaTeX with accessibility-conscious export settings.
	\item \textbf{Scripting / Batch:} Preflight profiles or command-line tools (where applicable) to enforce metadata, tagging presence.
\end{itemize}

\section{~~Implementation Strategies}\label{ch17:sec:implementation-strategies}
\subsection*{1. Source-First Authoring}
Mandate heading styles, meaningful alt text drafts, table header definition, list semantics, and language metadata before export.
\subsection*{2. Controlled Export}
Enable “Document structure tags for accessibility,” include bookmarks, and preserve \gidx{readingorder}{reading order} (e.g., “Use document structure for tab order”).
\subsection*{3. Initial Tag Audit}
Run Acrobat Full Check; review Tags tree; correct role misclassification, nested order anomalies, artifact omissions.
\subsection*{4. Remediation Pass}
Address structural issues (headings, lists, tables), figures alt text, form field tooltips, and reading order using the Reading Order and Tags panels.
\subsection*{5. Validation and Assistive Technology Testing}
Perform screen reader (NVDA/JAWS/VoiceOver) traversal verifying semantic clarity, order, table header association, form \gidx{navigation}{navigation}, and link purpose.
\subsection*{6. Governance}
Log remediation cycle time, defect density (issues/page), and rework events for continuous improvement.

\section{~~Standards and Compliance}\label{ch17:sec:standards-compliance}
\begin{itemize}
	\item \textbf{WCAG 2.x} (e.g., 1.1.1 Non-text Content, 1.3.1 Info and Relationships, 1.3.2 Meaningful Sequence, 1.4.x Contrast, 2.4.2 Page Titled, 3.1.1 Language of Page).
	\item \textbf{\gls{pdfua} (ISO 14289):} Tag tree completeness, correct role mapping, alt text for figures, language metadata, form field semantics, artifact classification.
	\item \textbf{Section 508 / EN 301 549:} Leverages WCAG alignment; incorporate into procurement and publication checklists.
	\item \textbf{Best Practice:} Maintain a conformance log (tool versions, test scope, exceptions).
\end{itemize}

\section{~~Best Practices}\label{ch17:sec:best-practices}
\begin{itemize}
	\item Enforce source templates embedding heading and table styles.
	\item Keep alt text concise (purpose-focused) and avoid redundancy with captions.
	\item Limit manual post-export restructuring through early style discipline.
	\item Verify \gidx{readingorder}{reading order} per page—do not rely solely on autotagging.
	\item Use Preflight profiles for batch metadata / tagging validation.
	\item Track common defect categories to target author training.
\end{itemize}

\section{~~Troubleshooting and Common Pitfalls}\label{ch17:sec:troubleshooting}
\footnotesize
\begin{longtblr}[
		caption = {Common PDF Accessibility Issues and Resolutions},
		label = {ch17:tab:troubleshooting},
		note = {Schema: Issue, RootCause, ImpactOnLearner, ResolutionSteps, PreventivePractice, ReferenceKey.}
	]{
		colspec = {X[l] X[l] X[l] X[l] X[l] X[l]},
		rowhead = 1,
		row{1} = {font=\bfseries},
		hlines
	}
	Issue                                & RootCause                                                           & ImpactOnLearner                                & ResolutionSteps                                                      & PreventivePractice                                 & ReferenceKey      \\
	Untagged content blocks              & Export omitted structural tags or manual insertion outside tag tree & Screen reader skips content                    & Autotag as baseline; manually retag missing nodes via Tags panel     & Enforce source styles; export with tagging enabled & AdobeAccessGuide  \\
	Figures lacking alt text             & Author oversight or bulk insertion                                  & Loss of contextual information                 & Add alt in Figure tag properties; ensure brevity                     & Alt text checklist in authoring stage              & AdobeAccessGuide  \\
	Incorrect heading nesting            & Manual styling or promotion/demotion errors                         & Disorientation; inefficient \gidx{navigation}{navigation}         & Reassign tag levels (H1→H2 sequence) in Tags panel                   & Template enforcing style cascade                   & AdobePDFUA        \\
	Table header misassociation          & Improper TH/TD tagging or missing scope attributes                  & Misinterpreted tabular relationships           & Rebuild table structure; set header scope; verify with screen reader & Use accessible table patterns in source            & PubComWCAGvsPDFUA \\
	\gidx{readingorder}{Reading order} mismatch               & Visual layout overrides logical flow                                & Fragmented comprehension                       & Adjust order numbers or tag sequence with Reading Order tool         & Source linearization before export                 & AdobeAccessGuide  \\
	Artifacts announced as content       & Decorative elements not marked Artifact                             & Noise; cognitive load                          & Mark as Artifact (\gidx{readingorder}{Reading Order} / Content panel)                     & Authoring guidance to segregate decorative objects & AdobePDFUA        \\
	Improper form field labeling         & Missing Tooltips / alt associations                                 & Ambiguous form control purpose                 & Add Tooltips; verify tag role / name                                 & Form template library with required labels         & AdobeAccessGuide  \\
	Security settings block AT           & Incorrect DRM / restrictions                                        & AT cannot access text                          & Remove content copying restrictions; re-save                         & Security policy review pre-publication             & AdobePDFUA        \\
	Language metadata missing            & Language not set in properties                                      & Incorrect pronunciation / hyphenation          & Set document language; tag inline language shifts                    & Export script enforcing metadata                   & AdobePDFUA        \\
	Autotag structural misclassification & Complex layout confusion                                            & Incorrect semantics (e.g., list vs. paragraph) & Manually retag misclassified nodes                                   & Validate sample pages before bulk processing       & AdobeAccessGuide  \\
\end{longtblr}
\normalsize

\section{~~Emerging Trends and Future Directions}\label{ch17:sec:emerging-trends}
\begin{itemize}
	\item AI-assisted autotag refinement (semantic role disambiguation).
	\item Automated \gidx{readingorder}{reading order} inference via layout vision models.
	\item Source-to-multi-format pipelines (PDF + EPUB + HTML) with synchronized semantic layers.
	\item Real-time accessibility validation in authoring environments.
\end{itemize}

\section{~~Ethical, Equity, and Privacy Considerations}\label{ch17:sec:ethics-equity}
\begin{itemize}
	\item \textbf{Equity:} Delayed remediation widens the access gap; track \gidx{latency}{latency} from publication to accessible release.
	\item \textbf{Accuracy vs. Automation:} Over-trust in AI autotagging risks silent structural errors—mandate human review.
	\item \textbf{Privacy:} When using cloud remediation, ensure no sensitive or personal data is exposed.
	\item \textbf{Transparency:} Provide conformance statements with scope, methods, and residual known issues.
\end{itemize}

\section{~~Assessment and Reflection}\label{ch17:sec:assessment-reflection}
\textbf{Reflection Questions}
\begin{enumerate}
	\item Which three defect categories (from the troubleshooting matrix) most frequently appear in your workflow and why?
	\item How would you re-engineer the source authoring template to eliminate one recurring remediation step?
	\item What metrics (e.g., defect density per page, remediation cycle time) will you instrument to measure continuous improvement?
\end{enumerate}
\textbf{Applied Exercise} Select a 10–15 page PDF with moderate layout complexity. Produce: (a) baseline issue log (SC mapped), (b) prioritized remediation plan, (c) post-remediation validation summary (tools + AT tests), (d) recommendations to reduce future defects at source.

\section{~~Summary}\label{ch17:sec:summary}
High-quality accessible PDFs emerge from source-first semantic rigor, disciplined tagging and \gidx{readingorder}{reading order} validation, standards mapping (WCAG + \gls{pdfua}), iterative remediation, and governance metrics. Automation accelerates but does not replace expert human verification; a structured troubleshooting matrix and equity-focused KPIs drive sustainable improvement.

% --- Original chapter content continues below (existing sections preserved) ---
\label{cha:creating-and-verifying-pdf-accessibility-with-adobe-acrobat}

\section{~~Overview}
\label{sec:overview-16}

This chapter provides a guide to using Adobe Acrobat Pro\index{PDF!Adobe Acrobat} to check and fix \gidx{accessibility}{accessibility} issues in PDF documents~\supercite{AdobeAccessGuide, AdobeHelpX}. Ensuring PDF accessibility\index{PDF!PDF accessibility} is crucial for compliance with standards like WCAG\index{WCAG}~\supercite{WCAG21, WCAG22} and for providing equal access to information for people with disabilities.

\textbf{By using Acrobat's built-in accessibility checker, identify common accessibility problems, and offer solutions to resolve them. By following these guidelines, you can create PDFs that are navigable and readable by assistive technologies, such as screen readers.}

The topics covered in this chapter include:

\begin{itemize}
	\item Checking \gls{pdf} \gls{accessibility}
	\item Fixing accessibility issues
	\item Common \gidx{accessibility}{accessibility} issues and solutions
\end{itemize}

\section{~~Check Accessibility of PDFs (Acrobat Pro)}
\label{sec:check-accessibility-of-pdfs-acrobat-pro}

Adobe Acrobat\index{PDF!Adobe Acrobat} Pro includes a built-in accessibility checker\index{accessibility!accessibility testing} that can help identify potential issues in your PDF\index{PDF} documents~\supercite{AdobeAccessGuide}. This tool tests the file against a set of \gidx{accessibility}{accessibility} criteria and generates a report that lists\index{Markdown!lists} any problems it finds.

\subsection{Steps to Check for Accessibility}\label{sub:steps-to-check-for-accessibility}\supercite{AdobeHelpX}

\begin{enumerate}
	\item Open the PDF in Adobe Acrobat Pro.
	\item Go to \textbf{Tools} and select \textbf{Accessibility}.
	\item In the Accessibility tool pane, click on \textbf{Full Check} or \textbf{Accessibility Check}.
	\item In the Accessibility Checker Options dialog box, select the checking options you want to use. It is recommended to leave the default settings.
	\item Click \textbf{Start Checking}.
	\item The results are displayed in the Accessibility Checker panel on the left. The report shows one of the following statuses for each rule check:
	      \begin{itemize}
		      \item \textbf{Passed:} The item is accessible.
		      \item \textbf{Needs Manual Check:} The feature could not be checked automatically. Verify the item manually.
		      \item \textbf{Failed:} The item did not pass the accessibility check.
	      \end{itemize}
\end{enumerate}

\section{~~Fix Accessibility Issues (Acrobat Pro)}
\label{sec:fix-accessibility-issues-acrobat-pro}

After running the accessibility checker, you can use the Accessibility Checker report provides links to specific problems in the document. To fix an issue, right-click on the item in the Accessibility\index{accessibility} Checker panel and select one of the following options:

\begin{itemize}
	\item \textbf{Fix:} Acrobat either fixes the item automatically or displays a dialog box prompting you to fix the item manually.
	\item \textbf{Explain:} Opens the online help documentation with details about the \gidx{accessibility}{accessibility} issue.
\end{itemize}

\section{~~Accessibility Issues}
\label{sec:accessibility-issues}

The following sections describe common accessibility issues that can be found in PDF\index{PDF} documents and how to address them using Adobe Acrobat\index{PDF!Adobe Acrobat} Pro.

\subsection{Document}
\label{sub:document}

\subsubsection{Prevent security settings from interfering with screen readers}
\label{ssub:prevent-security-settings-from-interfering-with-screen-readers}

Security settings can restrict the use of a PDF file, but they should not prevent a \gidx{screenreader}{screen reader} from accessing the text.

\paragraph{To check the security settings:}
\label{par:to-check-the-security-settings}

\begin{enumerate}
	\itemsep-0.5em
	\item Go to \textbf{File} > \textbf{Properties}.
	\item Select the \textbf{Security} tab.
	\item Ensure that the \textbf{Security Method} is set to \textbf{No Security}, or if security is required, that \textbf{Content Copying for Accessibility} is set to \textbf{Allowed}.
\end{enumerate}

\paragraph{To fix the security settings:}
\label{par:to-fix-the-security-settings}

If the security settings are too restrictive, you will need the password to change them. If you do not have the password, you will need to obtain a version of the PDF\index{PDF} without these restrictions.

\begin{itemize}
	\item \textbf{To fix this rule automatically:}
	      \begin{enumerate}
		      \item In the Accessibility Checker panel, right-click the \textbf{Accessibility Permission} flag.
		      \item Choose \textbf{Fix} from the context menu.
		      \item In the \textbf{Document Properties} dialog box, select the \textbf{Security} tab.
		      \item For \textbf{Security Method}, select \textbf{No Security}.
		      \item If a password is required, select \textbf{Password Security}. Under \textbf{Permissions}, ensure that \textbf{Enable text access for screen reader devices for the visually impaired} is selected.
		      \item Click \textbf{OK}.
	      \end{enumerate}
	\item \textbf{To check this rule manually:}
	      \begin{enumerate}
		      \item Go to \textbf{File > Properties}.
		      \item Select the \textbf{Security} tab.
		      \item Verify that the \textbf{Security Method} is set to \textbf{No Security}, or if it is, that \textbf{Enable text access for screen reader devices for the visually impaired} is checked.
	      \end{enumerate}
\end{itemize}

\subsubsection{Image-only PDF}
\label{ssubsec:pdf-image-only}

\textbf{Issue:} The PDF\index{PDF} contains no text that can be read by a \gidx{screenreader}{screen reader}, likely because it was created from a scanned document.

\begin{itemize}
	\item \textbf{To fix this rule manually:}
	      \begin{enumerate}
		      \item Use the \textbf{Scan \& \gls{ocr}\index{\gls{ocr}}} tool in Acrobat.
		      \item Select \textbf{Recognize Text > In This File}.
		      \item Acrobat will perform Optical Character Recognition\index{\gls{ocr}} (\gls{ocr}) to convert the image of text into actual, readable text.
		      \item After \gls{ocr} is complete, you will need to review the document for recognition errors and then proceed with tagging the document.
	      \end{enumerate}
\end{itemize}

\subsubsection{Tagged PDF}
\label{ssubsec:pdf-tagged}

\textbf{Issue:} The document is not tagged, meaning it lacks the structural information necessary for screen reader \gidx{navigation}{navigation}.

\begin{itemize}
	\item \textbf{To fix this rule automatically:}
	      \begin{enumerate}
		      \item In the Accessibility Checker panel, right-click the \textbf{Tagged PDF} flag.
		      \item Choose \textbf{Fix} from the context menu.
		      \item Acrobat automatically adds tags to the PDF\index{PDF}.
	      \end{enumerate}
	\item \textbf{To fix this rule manually:}
	      \begin{enumerate}
		      \item In the \textbf{Accessibility} tools pane, select \textbf{Autotag Document}.
		      \item This process adds tags to the document. You must then manually review the tags to ensure they are correct and logical.
	      \end{enumerate}
\end{itemize}

\subsubsection{Logical \gidx{readingorder}{reading order}}
\label{ssubsec:pdf-reading-order}

\textbf{Issue:} The \gidx{readingorder}{reading order}\index{PDF!reading order} of the document does not match the logical flow of the content. This must be checked manually.

\begin{itemize}
	\item Use the \textbf{\gidx{readingorder}{Reading Order}} tool in the Accessibility\index{accessibility} panel to view and correct the order of elements on each page. You can drag and drop items in the Order panel to re-sequence them.
\end{itemize}

\subsubsection{Document language}
\label{ssubsec:pdf-language}

\textbf{Issue:} The primary language of the document is not specified, which prevents screen readers\index{screen reader} from using the correct pronunciation rules.

\begin{itemize}
	\item \textbf{To fix this rule automatically:}
	      \begin{enumerate}
		      \item In the Accessibility Checker\index{accessibility!accessibility testing} panel, right-click the \textbf{Primary Language} flag.
		      \item Choose \textbf{Fix} from the context menu.
		      \item In the \textbf{Set Reading Language} dialog box, select the appropriate language from the dropdown menu.
		      \item Click \textbf{OK}.
	      \end{enumerate}
	\item \textbf{To fix this rule manually:}
	      \begin{enumerate}
		      \item Go to \textbf{File > Properties}.
		      \item Select the \textbf{Advanced} tab.
		      \item In the \textbf{Reading Options} section, choose the correct language from the \textbf{Language} dropdown.
	      \end{enumerate}
\end{itemize}

\subsubsection{Title}
\label{ssubsec:pdf-title}

\textbf{Issue:} The document title is not specified in the document properties.

\begin{itemize}
	\item \textbf{To fix this rule automatically:}
	      \begin{enumerate}
		      \item In the Accessibility Checker panel, right-click the \textbf{Title} flag.
		      \item Choose \textbf{Fix} from the context menu.
		      \item The \textbf{Description} dialog box will appear.
		      \item Uncheck \textbf{Leave As Is} and enter a descriptive title for the document.
		      \item Click \textbf{OK}.
	      \end{enumerate}
	\item \textbf{To fix this rule manually:}
	      \begin{enumerate}
		      \item Go to \textbf{File > Properties}.
		      \item Select the \textbf{Description} tab.
		      \item Enter a descriptive title in the \textbf{Title} field.
	      \end{enumerate}
\end{itemize}

\subsubsection{Bookmarks}
\label{ssubsec:pdf-bookmarks}

\textbf{Issue:} Long documents (21 pages or more) should have bookmarks that correspond to the document structure\index{document structure} for easy \gidx{navigation}{navigation}.

\begin{itemize}
	\item \textbf{To fix this rule manually:}
	      \begin{enumerate}
		      \item Open the \textbf{Bookmarks} panel on the left.
		      \item Use the \textbf{New Bookmark} icon to add bookmarks\index{document structure}.
		      \item It is best practice to create bookmarks from the document's heading structure. You can do this automatically from the Bookmarks panel options menu by selecting \textbf{New Bookmarks from Structure} and choosing the heading levels you want to include.
	      \end{enumerate}
\end{itemize}

\subsubsection{Color contrast}
\label{ssubsec:pdf-color-contrast}

\textbf{Issue:} The document contains text or content that does not have sufficient contrast against its background, making it difficult for users with low vision or color blindness to read. This must be checked manually.

\begin{itemize}
	\item \textbf{To fix this issue:}
	      \begin{enumerate}
		      \item Make sure that the document's content adheres to the guidelines outlined in WCAG section 1.4.3.
		      \item Or, include a recommendation that the PDF viewer use high-contrast colors:
		      \item Select the hamburger menu (Windows) or the Acrobat menu (macOS) > Preferences.
		      \item In the dialog that opens, from the left panel, select Accessibility.
		      \item Select Replace Document Colors and then select Use High-Contrast Colors.
		      \item From the High-contrast color combination, choose the color combination that you want and then select OK.
	      \end{enumerate}
\end{itemize}

\subsection{Page Content}
\label{subsec:pdf-page-content}

\subsubsection{Tagged content}
\label{ssubsec:pdf-tagged-content}

\textbf{Issue:} Some content on the page is not tagged.

\begin{itemize}
	\item \textbf{To fix this rule manually:}
	      \begin{enumerate}
		      \item In the Accessibility Checker\index{accessibility!accessibility testing} panel, right-click the \textbf{Tagged content} flag.
		      \item Choose \textbf{Show in Tags Panel}.
		      \item In the Tags panel, use the \textbf{Find} tool to search for untagged content.
		      \item Manually tag the content using the appropriate tag from the New Tag dialog box.
	      \end{enumerate}
\end{itemize}

\subsubsection{Tagged annotations\index{PDF!tagged PDF}}
\label{ssubsec:pdf-tagged-annotations}

\textbf{Issue:} Annotations like comments, links, or text highlights are not included in the tags tree.

\begin{itemize}
	\item \textbf{To fix this rule manually:}
	      \begin{enumerate}
		      \item In the Accessibility Checker panel, right-click the \textbf{Tagged annotations} flag.
		      \item Choose \textbf{Show in Tags Panel}.
		      \item In the Tags panel, use the \textbf{Find} tool and select \textbf{Unmarked Annotations} from the dropdown.
		      \item Click \textbf{Find}, and then click the \textbf{Tag Element} button for each found annotation.
	      \end{enumerate}
\end{itemize}

\subsubsection{Tab order}
\label{ssubsec:pdf-tab-order}

\textbf{Issue:} The tab order\index{PDF!tab order} for interactive elements is not specified or is not logical.

\begin{itemize}
	\item \textbf{To fix this rule automatically:}
	      \begin{enumerate}
		      \item In the Accessibility Checker\index{accessibility!accessibility testing} panel, right-click the \textbf{Tab Order} flag.
		      \item Choose \textbf{Fix} from the context menu. Acrobat\index{PDF!Adobe Acrobat} will set the tab order to follow the document structure\index{document structure}.
	      \end{enumerate}
	\item \textbf{To fix this rule manually:}
	      \begin{enumerate}
		      \item Go to the \textbf{Page Thumbnails} panel.
		      \item Select all page thumbnails (Ctrl+A or Cmd+A).
		      \item Right-click and select \textbf{Page Properties}.
		      \item In the \textbf{Tab Order} tab, select \textbf{Use Document Structure}.
		      \item Click \textbf{OK}.
	      \end{enumerate}
\end{itemize}

\subsubsection{Character encoding}
\label{ssubsec:pdf-character-encoding}

\textbf{Issue:} The document uses non-standard character encoding, which can cause text to be displayed incorrectly.

\begin{itemize}
	\item \textbf{To fix this rule manually:}
	      \begin{enumerate}
		      \item This issue often occurs when fonts\index{fonts} are not embedded in the PDF.
		      \item Go to \textbf{File > Properties}.
		      \item Select the \textbf{Fonts\index{fonts}} tab.
		      \item Ensure that all fonts used in the document are listed as \textbf{(Embedded Subset)} or \textbf{(Embedded)}.
		      \item If \gls{fonts} are not embedded, you must regenerate the PDF from the source document with the option to embed all \gls{fonts} selected.
	      \end{enumerate}
\end{itemize}

\subsubsection{Tagged multimedia}
\label{ssubsec:pdf-tagged-multimedia}

\textbf{Issue:} Multimedia elements (audio/video) are not tagged.

\begin{itemize}
	\item \textbf{To fix this rule manually:}
	      \begin{enumerate}
		      \item Use the \textbf{Tags} panel to find the untagged multimedia object.
		      \item Create a new tag for the object (e.g., `<Figure>`).
		      \item Ensure the multimedia object has alternative text\index{images and media!alternative text} or a caption describing its content.
		      \item For video, ensure captions are available. For \gls{audio}, provide a transcript.
	      \end{enumerate}
\end{itemize}

\subsubsection{Screen flicker}
\label{ssubsec:pdf-screen-flicker}

\textbf{Issue:} The document contains content that flashes between 2 and 55 times per second, which can trigger seizures.

\begin{itemize}
	\item \textbf{To fix this rule manually:}
	      \begin{enumerate}
		      \item This must be checked manually by visually inspecting the document.
		      \item If flashing content is found (often in animated GIFs or embedded videos), it must be removed or modified to stop flashing.
	      \end{enumerate}
\end{itemize}

\subsubsection{Scripts}
\label{ssubsec:pdf-scripts}

\textbf{Issue:} Scripts\index{PDF!scripts} (like JavaScript) in the document may be inaccessible.

\begin{itemize}
	\item \textbf{To fix this rule manually:}
	      \begin{enumerate}
		      \item This requires manual inspection of any JavaScript in the document.
		      \item Ensure that any functionality provided by a script is also available through other means (e.g., standard form fields) or that the script itself is written to be accessible to assistive technologies\index{assistive technology}.
	      \end{enumerate}
\end{itemize}

\subsubsection{Timed responses}
\label{ssubsec:pdf-timed-responses}

\textbf{Issue:} The document requires the user to respond within a specific time limit.

\begin{itemize}
	\item \textbf{To fix this rule manually:}
	      \begin{enumerate}
		      \item This must be checked manually.
		      \item If a timed response is required (e.g., by a script), the script must be modified to allow the user to extend the time limit or disable it entirely.
	      \end{enumerate}
\end{itemize}

\subsubsection{Accessible links}
\label{ssubsec:pdf-accessible-links}

\textbf{Issue:} Link text is not descriptive or is a bare URL.

\begin{itemize}
	\item \textbf{To fix this rule manually:}
	      \begin{enumerate}
		      \item Use the \textbf{Edit PDF} tool to change the visible link text to be descriptive.
		      \item For example, instead of "Click Here," use "Read the Q3 Financial Report."
		      \item This is best fixed in the source document before creating the PDF\index{PDF}.
	      \end{enumerate}
\end{itemize}

\subsection{Forms}
\label{subsec:pdf-forms}

\subsubsection{Tagged form fields}
\label{ssubsec:pdf-tagged-form-fields}

\textbf{Issue:} Interactive form fields are not tagged.

\begin{itemize}
	\item \textbf{To fix this rule automatically:}
	      \begin{enumerate}
		      \item In the Accessibility Checker\index{accessibility!accessibility testing} panel, right-click the \textbf{Tagged form fields} flag.
		      \item Choose \textbf{Fix} from the context menu. Acrobat\index{PDF!Adobe Acrobat} will tag the form fields.
	      \end{enumerate}
\end{itemize}

\subsubsection{Field descriptions}
\label{ssubsec:pdf-field-descriptions}

\textbf{Issue:} Form fields are missing tooltips (descriptive text\index{images and media!alternative text} that screen readers\index{screen reader} announce).

\begin{itemize}
	\item \textbf{To fix this rule automatically:}
	      \begin{enumerate}
		      \item In the Accessibility Checker panel, right-click the \textbf{Field descriptions\index{PDF!field descriptions}} flag.
		      \item Choose \textbf{Fix} from the context menu.
		      \item Acrobat will open a dialog box allowing you to add tooltips to all form fields.
	      \end{enumerate}
	\item \textbf{To fix this rule manually:}
	      \begin{enumerate}
		      \item Use the \textbf{Prepare Form} tool.
		      \item Right-click on a form field and select \textbf{Properties}.
		      \item In the \textbf{General} tab, enter a descriptive name and a helpful tooltip. The tooltip is what screen readers\index{screen reader} will announce.
	      \end{enumerate}
\end{itemize}

\subsection{Alternate Text}
\label{subsec:pdf-alternate-text}

\subsubsection{Figures alternate text}
\label{ssubsec:pdf-figures-alt-text}

\textbf{Issue:} Images and other figures are missing alternative text.

\begin{itemize}
	\item \textbf{To fix this rule automatically:}
	      \begin{enumerate}
		      \item In the Accessibility Checker\index{accessibility!accessibility testing} panel, right-click the \textbf{Figures alternate text} flag.
		      \item Choose \textbf{Fix} from the context menu.
		      \item Acrobat\index{PDF!Adobe Acrobat} will display each figure in the document and provide a dialog box for you to enter alt text\index{images and media!alternative text}.
		      \item If an image is purely decorative, check the \textbf{Decorative figure} box.
	      \end{enumerate}
\end{itemize}

\subsubsection{Nested alternate text}
\label{ssubsec:pdf-nested-alt-text}

\textbf{Issue:} An element has alternate text that is also applied to content within that element.

\begin{itemize}
	\item \textbf{To fix this rule manually:}
	      \begin{enumerate}
		      \item Open the \textbf{Tags} panel.
		      \item Locate the parent element that has nested content.
		      \item Right-click the parent tag and select \textbf{Properties}.
		      \item In the \textbf{Tag} tab, remove the alternate text from the parent element if the nested elements already have their own \gls{alttext}. The goal is to avoid redundant descriptions.
	      \end{enumerate}
\end{itemize}

\subsubsection{Associated with content}
\label{ssubsec:pdf-alt-text-associated}

\textbf{Issue:} Alternate text is for an element that is not part of the page content (an artifact).

\begin{itemize}
	\item This usually indicates a tagging error. The element should be marked as an artifact so that it is ignored by \gls{screenreader}. Use the \gidx{readingorder}{Reading Order} tool to mark the element as "Background/Artifact."
\end{itemize}

\subsubsection{Hides Annotation}
\label{ssubsec:pdf-hides-annotation}

\textbf{Issue:} An annotation (like a comment or highlight) is placed over content, hiding it from screen readers.

\begin{itemize}
	\item \textbf{To fix this rule manually:}
	      \begin{enumerate}
		      \item Locate the annotation in the document.
		      \item Move the annotation so that it does not obscure any underlying text or content.
		      \item Alternatively, if the annotation is not essential, you can delete it.
	      \end{enumerate}
\end{itemize}

\subsubsection{Other Elements Alternate Text}
\label{ssubsec:pdf-other-elements-alt-text}

\textbf{Issue:} Content other than figures (like complex tables or multimedia) may require alternate text.

\begin{itemize}
	\item \textbf{To fix this rule manually:}
	      \begin{enumerate}
		      \item Use the \textbf{Tags} panel to locate the element.
		      \item Right-click the tag and select \textbf{Properties}.
		      \item Add a concise, descriptive alternate text in the \textbf{Alternate Text} field.
	      \end{enumerate}
\end{itemize}

\subsection{References}
\label{subsec:pdf-references}

\begin{itemize}
	\item \textbf{Tagged content:} All content in the document must be tagged and included in the structure tree.
	\item \textbf{Tagged annotations:} All annotations, such as comments and links, must be tagged.
	\item \textbf{Tab order\index{PDF!tab order}:} The tab order for interactive elements like links and form fields must be logical.
	\item \textbf{Character encoding\index{PDF!character encoding}:} The document must use a standard character encoding to ensure text displays correctly.
	\item \textbf{Tagged multimedia\index{PDF!tagged PDF}:} All multimedia content must be tagged.
	\item \textbf{Screen flicker\index{accessibility!screen flicker}:} The document should not contain any content that flashes or flickers at a rate that could cause seizures.
	\item \textbf{Scripts:} Any scripts\index{PDF!scripts} in the document must be accessible.
	\item \textbf{Timed responses\index{accessibility!timed responses}:} The document should not require timed responses from the user.
	\item \textbf{Accessible links\index{accessibility!accessible links}:} Link text must be descriptive and make sense out of context.
\end{itemize}

