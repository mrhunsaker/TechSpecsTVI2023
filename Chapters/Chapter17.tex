\chapter{Creating and Verifying PDF Accessibility with Adobe Acrobat}
\label{cha:creating-and-verifying-pdf-accessibility-with-adobe-acrobat}

\section{~~Overview}
\label{sec:overview-16}

This chapter provides a guide to using Adobe Acrobat Pro\index{PDF!Adobe Acrobat} to check and fix accessibility\index{accessibility} issues in PDF documents~\supercite{AdobeAccessGuide, AdobeHelpX}. Ensuring PDF accessibility\index{PDF!PDF accessibility} is crucial for compliance with standards like WCAG\index{WCAG}~\supercite{WCAG21, WCAG22} and for providing equal access to information for people with disabilities.

\textbf{By using Acrobat's built-in accessibility checker, identify common accessibility problems, and offer solutions to resolve them. By following these guidelines, you can create PDFs that are navigable and readable by assistive technologies, such as screen readers.}

The topics covered in this chapter include:

\begin{itemize}
	\item Checking \gls{pdf} \gls{accessibility}
	\item Fixing accessibility issues
	\item Common accessibility\index{accessibility} issues and solutions
\end{itemize}

\section{~~Check Accessibility of PDFs (Acrobat Pro)}
\label{sec:check-accessibility-of-pdfs-acrobat-pro}

Adobe Acrobat\index{PDF!Adobe Acrobat} Pro includes a built-in accessibility checker\index{accessibility!accessibility testing} that can help identify potential issues in your PDF\index{PDF} documents~\supercite{AdobeAccessGuide}. This tool tests the file against a set of accessibility\index{accessibility} criteria and generates a report that lists\index{Markdown!lists} any problems it finds.

\subsection{Steps to Check for Accessibility}\label{sub:steps-to-check-for-accessibility}\supercite{AdobeHelpX}

\begin{enumerate}
	\item Open the PDF in Adobe Acrobat Pro.
	\item Go to \textbf{Tools} and select \textbf{Accessibility}.
	\item In the Accessibility tool pane, click on \textbf{Full Check} or \textbf{Accessibility Check}.
	\item In the Accessibility Checker Options dialog box, select the checking options you want to use. It is recommended to leave the default settings.
	\item Click \textbf{Start Checking}.
	\item The results are displayed in the Accessibility Checker panel on the left. The report shows one of the following statuses for each rule check:
	      \begin{itemize}
		      \item \textbf{Passed:} The item is accessible.
		      \item \textbf{Needs Manual Check:} The feature could not be checked automatically. Verify the item manually.
		      \item \textbf{Failed:} The item did not pass the accessibility check.
	      \end{itemize}
\end{enumerate}

\section{~~Fix Accessibility Issues (Acrobat Pro)}
\label{sec:fix-accessibility-issues-acrobat-pro}

After running the accessibility checker, you can use the Accessibility Checker report provides links to specific problems in the document. To fix an issue, right-click on the item in the Accessibility\index{accessibility} Checker panel and select one of the following options:

\begin{itemize}
	\item \textbf{Fix:} Acrobat either fixes the item automatically or displays a dialog box prompting you to fix the item manually.
	\item \textbf{Explain:} Opens the online help documentation with details about the accessibility\index{accessibility} issue.
\end{itemize}

\section{~~Accessibility Issues}
\label{sec:accessibility-issues}

The following sections describe common accessibility issues that can be found in PDF\index{PDF} documents and how to address them using Adobe Acrobat\index{PDF!Adobe Acrobat} Pro.

\subsection{Document}
\label{sub:document}

\subsubsection{Prevent security settings from interfering with screen readers}
\label{ssub:prevent-security-settings-from-interfering-with-screen-readers}

Security settings can restrict the use of a PDF file, but they should not prevent a screen reader\index{screen reader} from accessing the text.

\paragraph{To check the security settings:}
\label{par:to-check-the-security-settings}

\begin{enumerate}
	\itemsep-0.5em
	\item Go to \textbf{File} > \textbf{Properties}.
	\item Select the \textbf{Security} tab.
	\item Ensure that the \textbf{Security Method} is set to \textbf{No Security}, or if security is required, that \textbf{Content Copying for Accessibility} is set to \textbf{Allowed}.
\end{enumerate}

\paragraph{To fix the security settings:}
\label{par:to-fix-the-security-settings}

If the security settings are too restrictive, you will need the password to change them. If you do not have the password, you will need to obtain a version of the PDF\index{PDF} without these restrictions.

\begin{itemize}
	\item \textbf{To fix this rule automatically:}
	      \begin{enumerate}
		      \item In the Accessibility Checker panel, right-click the \textbf{Accessibility Permission} flag.
		      \item Choose \textbf{Fix} from the context menu.
		      \item In the \textbf{Document Properties} dialog box, select the \textbf{Security} tab.
		      \item For \textbf{Security Method}, select \textbf{No Security}.
		      \item If a password is required, select \textbf{Password Security}. Under \textbf{Permissions}, ensure that \textbf{Enable text access for screen reader devices for the visually impaired} is selected.
		      \item Click \textbf{OK}.
	      \end{enumerate}
	\item \textbf{To check this rule manually:}
	      \begin{enumerate}
		      \item Go to \textbf{File > Properties}.
		      \item Select the \textbf{Security} tab.
		      \item Verify that the \textbf{Security Method} is set to \textbf{No Security}, or if it is, that \textbf{Enable text access for screen reader devices for the visually impaired} is checked.
	      \end{enumerate}
\end{itemize}

\subsubsection{Image-only PDF}
\label{ssubsec:pdf-image-only}

\textbf{Issue:} The PDF\index{PDF} contains no text that can be read by a screen reader\index{screen reader}, likely because it was created from a scanned document.

\begin{itemize}
	\item \textbf{To fix this rule manually:}
	      \begin{enumerate}
		      \item Use the \textbf{Scan \& OCR\index{OCR}} tool in Acrobat.
		      \item Select \textbf{Recognize Text > In This File}.
		      \item Acrobat will perform Optical Character Recognition\index{OCR} (OCR) to convert the image of text into actual, readable text.
		      \item After \gls{OCR} is complete, you will need to review the document for recognition errors and then proceed with tagging the document.
	      \end{enumerate}
\end{itemize}

\subsubsection{Tagged PDF}
\label{ssubsec:pdf-tagged}

\textbf{Issue:} The document is not tagged, meaning it lacks the structural information necessary for screen reader navigation.

\begin{itemize}
	\item \textbf{To fix this rule automatically:}
	      \begin{enumerate}
		      \item In the Accessibility Checker panel, right-click the \textbf{Tagged PDF} flag.
		      \item Choose \textbf{Fix} from the context menu.
		      \item Acrobat automatically adds tags to the PDF\index{PDF}.
	      \end{enumerate}
	\item \textbf{To fix this rule manually:}
	      \begin{enumerate}
		      \item In the \textbf{Accessibility} tools pane, select \textbf{Autotag Document}.
		      \item This process adds tags to the document. You must then manually review the tags to ensure they are correct and logical.
	      \end{enumerate}
\end{itemize}

\subsubsection{Logical reading order}
\label{ssubsec:pdf-reading-order}

\textbf{Issue:} The reading order\index{PDF!reading order} of the document does not match the logical flow of the content. This must be checked manually.

\begin{itemize}
	\item Use the \textbf{Reading Order} tool in the Accessibility\index{accessibility} panel to view and correct the order of elements on each page. You can drag and drop items in the Order panel to re-sequence them.
\end{itemize}

\subsubsection{Document language}
\label{ssubsec:pdf-language}

\textbf{Issue:} The primary language of the document is not specified, which prevents screen readers\index{screen reader} from using the correct pronunciation rules.

\begin{itemize}
	\item \textbf{To fix this rule automatically:}
	      \begin{enumerate}
		      \item In the Accessibility Checker\index{accessibility!accessibility testing} panel, right-click the \textbf{Primary Language} flag.
		      \item Choose \textbf{Fix} from the context menu.
		      \item In the \textbf{Set Reading Language} dialog box, select the appropriate language from the dropdown menu.
		      \item Click \textbf{OK}.
	      \end{enumerate}
	\item \textbf{To fix this rule manually:}
	      \begin{enumerate}
		      \item Go to \textbf{File > Properties}.
		      \item Select the \textbf{Advanced} tab.
		      \item In the \textbf{Reading Options} section, choose the correct language from the \textbf{Language} dropdown.
	      \end{enumerate}
\end{itemize}

\subsubsection{Title}
\label{ssubsec:pdf-title}

\textbf{Issue:} The document title is not specified in the document properties.

\begin{itemize}
	\item \textbf{To fix this rule automatically:}
	      \begin{enumerate}
		      \item In the Accessibility Checker panel, right-click the \textbf{Title} flag.
		      \item Choose \textbf{Fix} from the context menu.
		      \item The \textbf{Description} dialog box will appear.
		      \item Uncheck \textbf{Leave As Is} and enter a descriptive title for the document.
		      \item Click \textbf{OK}.
	      \end{enumerate}
	\item \textbf{To fix this rule manually:}
	      \begin{enumerate}
		      \item Go to \textbf{File > Properties}.
		      \item Select the \textbf{Description} tab.
		      \item Enter a descriptive title in the \textbf{Title} field.
	      \end{enumerate}
\end{itemize}

\subsubsection{Bookmarks}
\label{ssubsec:pdf-bookmarks}

\textbf{Issue:} Long documents (21 pages or more) should have bookmarks that correspond to the document structure\index{document structure} for easy navigation.

\begin{itemize}
	\item \textbf{To fix this rule manually:}
	      \begin{enumerate}
		      \item Open the \textbf{Bookmarks} panel on the left.
		      \item Use the \textbf{New Bookmark} icon to add bookmarks\index{document structure}.
		      \item It is best practice to create bookmarks from the document's heading structure. You can do this automatically from the Bookmarks panel options menu by selecting \textbf{New Bookmarks from Structure} and choosing the heading levels you want to include.
	      \end{enumerate}
\end{itemize}

\subsubsection{Color contrast}
\label{ssubsec:pdf-color-contrast}

\textbf{Issue:} The document contains text or content that does not have sufficient contrast against its background, making it difficult for users with low vision or color blindness to read. This must be checked manually.

\begin{itemize}
	\item \textbf{To fix this issue:}
	      \begin{enumerate}
		      \item Make sure that the document's content adheres to the guidelines outlined in WCAG section 1.4.3.
		      \item Or, include a recommendation that the PDF viewer use high-contrast colors:
		      \item Select the hamburger menu (Windows) or the Acrobat menu (macOS) > Preferences.
		      \item In the dialog that opens, from the left panel, select Accessibility.
		      \item Select Replace Document Colors and then select Use High-Contrast Colors.
		      \item From the High-contrast color combination, choose the color combination that you want and then select OK.
	      \end{enumerate}
\end{itemize}

\subsection{Page Content}
\label{subsec:pdf-page-content}

\subsubsection{Tagged content}
\label{ssubsec:pdf-tagged-content}

\textbf{Issue:} Some content on the page is not tagged.

\begin{itemize}
	\item \textbf{To fix this rule manually:}
	      \begin{enumerate}
		      \item In the Accessibility Checker\index{accessibility!accessibility testing} panel, right-click the \textbf{Tagged content} flag.
		      \item Choose \textbf{Show in Tags Panel}.
		      \item In the Tags panel, use the \textbf{Find} tool to search for untagged content.
		      \item Manually tag the content using the appropriate tag from the New Tag dialog box.
	      \end{enumerate}
\end{itemize}

\subsubsection{Tagged annotations\index{PDF!tagged PDF}}
\label{ssubsec:pdf-tagged-annotations}

\textbf{Issue:} Annotations like comments, links, or text highlights are not included in the tags tree.

\begin{itemize}
	\item \textbf{To fix this rule manually:}
	      \begin{enumerate}
		      \item In the Accessibility Checker panel, right-click the \textbf{Tagged annotations} flag.
		      \item Choose \textbf{Show in Tags Panel}.
		      \item In the Tags panel, use the \textbf{Find} tool and select \textbf{Unmarked Annotations} from the dropdown.
		      \item Click \textbf{Find}, and then click the \textbf{Tag Element} button for each found annotation.
	      \end{enumerate}
\end{itemize}

\subsubsection{Tab order}
\label{ssubsec:pdf-tab-order}

\textbf{Issue:} The tab order\index{PDF!tab order} for interactive elements is not specified or is not logical.

\begin{itemize}
	\item \textbf{To fix this rule automatically:}
	      \begin{enumerate}
		      \item In the Accessibility Checker\index{accessibility!accessibility testing} panel, right-click the \textbf{Tab Order} flag.
		      \item Choose \textbf{Fix} from the context menu. Acrobat\index{PDF!Adobe Acrobat} will set the tab order to follow the document structure\index{document structure}.
	      \end{enumerate}
	\item \textbf{To fix this rule manually:}
	      \begin{enumerate}
		      \item Go to the \textbf{Page Thumbnails} panel.
		      \item Select all page thumbnails (Ctrl+A or Cmd+A).
		      \item Right-click and select \textbf{Page Properties}.
		      \item In the \textbf{Tab Order} tab, select \textbf{Use Document Structure}.
		      \item Click \textbf{OK}.
	      \end{enumerate}
\end{itemize}

\subsubsection{Character encoding}
\label{ssubsec:pdf-character-encoding}

\textbf{Issue:} The document uses non-standard character encoding, which can cause text to be displayed incorrectly.

\begin{itemize}
	\item \textbf{To fix this rule manually:}
	      \begin{enumerate}
		      \item This issue often occurs when fonts\index{fonts} are not embedded in the PDF.
		      \item Go to \textbf{File > Properties}.
		      \item Select the \textbf{Fonts\index{fonts}} tab.
		      \item Ensure that all fonts used in the document are listed as \textbf{(Embedded Subset)} or \textbf{(Embedded)}.
		      \item If \gls{fonts} are not embedded, you must regenerate the PDF from the source document with the option to embed all \gls{fonts} selected.
	      \end{enumerate}
\end{itemize}

\subsubsection{Tagged multimedia}
\label{ssubsec:pdf-tagged-multimedia}

\textbf{Issue:} Multimedia elements (audio/video) are not tagged.

\begin{itemize}
	\item \textbf{To fix this rule manually:}
	      \begin{enumerate}
		      \item Use the \textbf{Tags} panel to find the untagged multimedia object.
		      \item Create a new tag for the object (e.g., `<Figure>`).
		      \item Ensure the multimedia object has alternative text\index{images and media!alternative text} or a caption describing its content.
		      \item For video, ensure captions are available. For \gls{audio}, provide a transcript.
	      \end{enumerate}
\end{itemize}

\subsubsection{Screen flicker}
\label{ssubsec:pdf-screen-flicker}

\textbf{Issue:} The document contains content that flashes between 2 and 55 times per second, which can trigger seizures.

\begin{itemize}
	\item \textbf{To fix this rule manually:}
	      \begin{enumerate}
		      \item This must be checked manually by visually inspecting the document.
		      \item If flashing content is found (often in animated GIFs or embedded videos), it must be removed or modified to stop flashing.
	      \end{enumerate}
\end{itemize}

\subsubsection{Scripts}
\label{ssubsec:pdf-scripts}

\textbf{Issue:} Scripts\index{PDF!scripts} (like JavaScript) in the document may be inaccessible.

\begin{itemize}
	\item \textbf{To fix this rule manually:}
	      \begin{enumerate}
		      \item This requires manual inspection of any JavaScript in the document.
		      \item Ensure that any functionality provided by a script is also available through other means (e.g., standard form fields) or that the script itself is written to be accessible to assistive technologies\index{assistive technology}.
	      \end{enumerate}
\end{itemize}

\subsubsection{Timed responses}
\label{ssubsec:pdf-timed-responses}

\textbf{Issue:} The document requires the user to respond within a specific time limit.

\begin{itemize}
	\item \textbf{To fix this rule manually:}
	      \begin{enumerate}
		      \item This must be checked manually.
		      \item If a timed response is required (e.g., by a script), the script must be modified to allow the user to extend the time limit or disable it entirely.
	      \end{enumerate}
\end{itemize}

\subsubsection{Accessible links}
\label{ssubsec:pdf-accessible-links}

\textbf{Issue:} Link text is not descriptive or is a bare URL.

\begin{itemize}
	\item \textbf{To fix this rule manually:}
	      \begin{enumerate}
		      \item Use the \textbf{Edit PDF} tool to change the visible link text to be descriptive.
		      \item For example, instead of "Click Here," use "Read the Q3 Financial Report."
		      \item This is best fixed in the source document before creating the PDF\index{PDF}.
	      \end{enumerate}
\end{itemize}

\subsection{Forms}
\label{subsec:pdf-forms}

\subsubsection{Tagged form fields}
\label{ssubsec:pdf-tagged-form-fields}

\textbf{Issue:} Interactive form fields are not tagged.

\begin{itemize}
	\item \textbf{To fix this rule automatically:}
	      \begin{enumerate}
		      \item In the Accessibility Checker\index{accessibility!accessibility testing} panel, right-click the \textbf{Tagged form fields} flag.
		      \item Choose \textbf{Fix} from the context menu. Acrobat\index{PDF!Adobe Acrobat} will tag the form fields.
	      \end{enumerate}
\end{itemize}

\subsubsection{Field descriptions}
\label{ssubsec:pdf-field-descriptions}

\textbf{Issue:} Form fields are missing tooltips (descriptive text\index{images and media!alternative text} that screen readers\index{screen reader} announce).

\begin{itemize}
	\item \textbf{To fix this rule automatically:}
	      \begin{enumerate}
		      \item In the Accessibility Checker panel, right-click the \textbf{Field descriptions\index{PDF!field descriptions}} flag.
		      \item Choose \textbf{Fix} from the context menu.
		      \item Acrobat will open a dialog box allowing you to add tooltips to all form fields.
	      \end{enumerate}
	\item \textbf{To fix this rule manually:}
	      \begin{enumerate}
		      \item Use the \textbf{Prepare Form} tool.
		      \item Right-click on a form field and select \textbf{Properties}.
		      \item In the \textbf{General} tab, enter a descriptive name and a helpful tooltip. The tooltip is what screen readers\index{screen reader} will announce.
	      \end{enumerate}
\end{itemize}

\subsection{Alternate Text}
\label{subsec:pdf-alternate-text}

\subsubsection{Figures alternate text}
\label{ssubsec:pdf-figures-alt-text}

\textbf{Issue:} Images and other figures are missing alternative text.

\begin{itemize}
	\item \textbf{To fix this rule automatically:}
	      \begin{enumerate}
		      \item In the Accessibility Checker\index{accessibility!accessibility testing} panel, right-click the \textbf{Figures alternate text} flag.
		      \item Choose \textbf{Fix} from the context menu.
		      \item Acrobat\index{PDF!Adobe Acrobat} will display each figure in the document and provide a dialog box for you to enter alt text\index{images and media!alternative text}.
		      \item If an image is purely decorative, check the \textbf{Decorative figure} box.
	      \end{enumerate}
\end{itemize}

\subsubsection{Nested alternate text}
\label{ssubsec:pdf-nested-alt-text}

\textbf{Issue:} An element has alternate text that is also applied to content within that element.

\begin{itemize}
	\item \textbf{To fix this rule manually:}
	      \begin{enumerate}
		      \item Open the \textbf{Tags} panel.
		      \item Locate the parent element that has nested content.
		      \item Right-click the parent tag and select \textbf{Properties}.
		      \item In the \textbf{Tag} tab, remove the alternate text from the parent element if the nested elements already have their own \gls{alttext}. The goal is to avoid redundant descriptions.
	      \end{enumerate}
\end{itemize}

\subsubsection{Associated with content}
\label{ssubsec:pdf-alt-text-associated}

\textbf{Issue:} Alternate text is for an element that is not part of the page content (an artifact).

\begin{itemize}
	\item This usually indicates a tagging error. The element should be marked as an artifact so that it is ignored by \gls{screenreader}. Use the Reading Order tool to mark the element as "Background/Artifact."
\end{itemize}

\subsubsection{Hides Annotation}
\label{ssubsec:pdf-hides-annotation}

\textbf{Issue:} An annotation (like a comment or highlight) is placed over content, hiding it from screen readers.

\begin{itemize}
	\item \textbf{To fix this rule manually:}
	      \begin{enumerate}
		      \item Locate the annotation in the document.
		      \item Move the annotation so that it does not obscure any underlying text or content.
		      \item Alternatively, if the annotation is not essential, you can delete it.
	      \end{enumerate}
\end{itemize}

\subsubsection{Other Elements Alternate Text}
\label{ssubsec:pdf-other-elements-alt-text}

\textbf{Issue:} Content other than figures (like complex tables or multimedia) may require alternate text.

\begin{itemize}
	\item \textbf{To fix this rule manually:}
	      \begin{enumerate}
		      \item Use the \textbf{Tags} panel to locate the element.
		      \item Right-click the tag and select \textbf{Properties}.
		      \item Add a concise, descriptive alternate text in the \textbf{Alternate Text} field.
	      \end{enumerate}
\end{itemize}

\subsection{References}
\label{subsec:pdf-references}

\begin{itemize}
	\item \textbf{Tagged content:} All content in the document must be tagged and included in the structure tree.
	\item \textbf{Tagged annotations:} All annotations, such as comments and links, must be tagged.
	\item \textbf{Tab order\index{PDF!tab order}:} The tab order for interactive elements like links and form fields must be logical.
	\item \textbf{Character encoding\index{PDF!character encoding}:} The document must use a standard character encoding to ensure text displays correctly.
	\item \textbf{Tagged multimedia\index{PDF!tagged PDF}:} All multimedia content must be tagged.
	\item \textbf{Screen flicker\index{accessibility!screen flicker}:} The document should not contain any content that flashes or flickers at a rate that could cause seizures.
	\item \textbf{Scripts:} Any scripts\index{PDF!scripts} in the document must be accessible.
	\item \textbf{Timed responses\index{accessibility!timed responses}:} The document should not require timed responses from the user.
	\item \textbf{Accessible links\index{accessibility!accessible links}:} Link text must be descriptive and make sense out of context.
\end{itemize}
