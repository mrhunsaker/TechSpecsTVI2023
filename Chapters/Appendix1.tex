\hypertarget{troubleshooting}{}\chapter[\hfill\break\raggedright Troubleshooting Screenreader \& Magnifier Performance]{Troubleshooting Screenreader \& Magnifier Performance}\label{troubleshooting}
\extramarks{Vision Department Technology Needs}{Appendix A: Troubleshooting Screenreader \& Magnifier Performance}
\noindent\makebox[\linewidth]{\rule{\linewidth}{0.4pt}}
{\let\clearpage\relax\localtableofcontents}\newpage
\hypertarget{cache}{}\section{Clearing System Cache}\label{cache}
One thing that it often recommended to users of screenreaders is that they maintain a habit of clearing the browser and system cache(s) in order to optimize performance of their laptop. Clearing the computer and browser cache is a common practice to free up space on the hard drive and improve the performance of the computer. However, this practice does not speed up the response of a computer if it has a Solid State Drive (SSD) rather than a spinning hard drive. This is because SSDs work differently than spinning hard drives. When data is written to an SSD, it is written to a block of memory called a page. When the page is full, the data is moved to another block of memory called a block. The block must be erased before new data can be written to it. This process is called garbage collection and it happens automatically in the background. Clearing the cache does not speed up the garbage collection process.

In addition, SSDs have a limited number of write cycles. Every time data is written to an SSD, it uses up one of these write cycles. Clearing the cache causes more data to be written to the SSD, which can reduce the lifespan of the drive. This is because when the cache is cleared, the computer must download the data again, which requires writing the data to the SSD. This can cause unnecessary wear and tear on the drive and reduce its lifespan.

Finally, clearing the cache can actually slow down the response of a computer with an SSD. This is because the cache stores frequently accessed data, such as images and scripts, so that they can be loaded quickly. When the cache is cleared, the computer must download this data again, which can slow down the response time. In contrast, spinning hard drives are slower than SSDs and can benefit from clearing the cache. This is because spinning hard drives have to physically move a read/write head to access data, which can take longer than reading data from an SSD.

\pagebreak \hypertarget{response}{}\section{Slow Responsiveness}\label{response}
When a screen reader like JAWS or NVDA is not responding to input or is taking a long time to report changes on the screen, there are several things you can try to resolve the issue. First, try restarting the screen reader and the computer. This can help clear any temporary issues that may be causing the problem. If this does not work, try updating the screen reader to the latest version. Screen readers are updated regularly to fix bugs and improve performance. Updating to the latest version may help resolve the issue.

If the problem persists, try adjusting the settings of the screen reader. Some screen readers have settings that can be adjusted to improve performance. For example, you can adjust the verbosity level to reduce the amount of information that is read out loud. You can also adjust the speed of the screen reader to make it faster or slower. Experimenting with these settings may help improve the performance of the screen reader.

Finally, if none of these steps work, you may need to contact the manufacturer of the screen reader for further assistance. They may be able to provide additional troubleshooting steps or help you diagnose the problem. It’s important to remember that screen readers are complex pieces of software and may require specialized knowledge to troubleshoot. By following these steps, you can help ensure that your screen reader is working properly and providing you with the accessibility you need.

\pagebreak \hypertarget{report}{}\section{Official Support Contact}\label{report}
\begin{itemize}[leftmargin=*]
	\item JAWS/Fusion: You can submit a technical support request, call 727-803-8600 weekdays between 8:30 AM and 7:00 PM ET, or send an email to \href{mailto:support@freedomscientific.com}{Freedom Scientific Support}.
	\item Dolphin Products: You can contact Dolphin’s technical support team by emailing \href{mailto:support@yourdolphin.com}{Dolphin Support}.
	\item NVDA: You can submit a bug report or request support by emailing \href{mailto:info@nvaccess.org}{NVDA Support Desk}.
	\item Windows: You can contact Microsoft’s technical support team by visiting the following link: \href{http://support.microsoft.com/en-us/contactus/}{Microsoft Support}.
\end{itemize}
\hypertarget{listserv}{}\section{Community Support via ListServ}\label{listserv}
Sometimes asking a listserv that talks about screen readers may give faster responses than contacting official customer support. This is because listservs are online communities where people with similar interests can share information and help each other out. Members of these communities are often experts in their field and can provide quick and accurate answers to questions. In contrast, customer support teams may have to follow a set of procedures and protocols before they can provide assistance. This can take time and may not always result in a satisfactory resolution. Additionally, customer support teams may not be available 24/7, whereas listservs are often active around the clock. However, it’s important to remember that listservs are not official sources of information and the advice given may not always be accurate or up-to-date. It’s always a good idea to verify information before acting on it.

Here are links to relevant listserv for visual impairment accessibility needs.
\begin{itemize}[leftmargin=*]
	\item JAWS / Fusion
	      \begin{itemize}[leftmargin=2em]
	      	\item \href{http://www.groups.io/g/jfw/}{The JAWS for Windows Support List}
	      	\item \href{http://groups.io/g/jfw-users/}{JFW Users List}
	      	\item \href{http://groups.io/g/jawsdiscussion/}{Jaws Discussion}
	      	\item \href{http://groups.io/g/jawslite/}{Jaws Lite}  
	      	\item \href{http://groups.io/g/jawsscripting/}{JAWS Scripting}  
	      \end{itemize}
	\item NVDA
	      \begin{itemize}[leftmargin=2em]
	      	\item \href{http://nvda.groups.io/g/nvda/ }{NVDA Group}  
	      	\item \href{http://nvda-addons.groups.io/g/nvda-addons}{NVDA Addons Group}  
	      	\item \href{http://nvda.groups.io/g/chat/ }{Chat Subgroup of the NVDA Group}  
	      	\item \href{http://groups.io/g/nvda-devel/messages}{NVDA Development}  
	      	\item \href{http://groups.io/g/nvdadiscussion/messages}{NVDA Discussion}  
	      	\item \href{http://groups.io/g/NVDAhelp/messages}{NVDA Help}  
	      \end{itemize}
	\item Windows / General Accessibility
	      \begin{itemize}[leftmargin=2em]
	      	\item \href{http://winaccess.groups.io/g/winaccess}{Windows Access with Screen Readers}  
	      \end{itemize}
	\item General Technology (Screenreaders Discussed Frequently)
	      \begin{itemize}[leftmargin=2em]
	      	\item \href{http://groups.io/g/blindtechdiscuss/messages}{Blind tech Discuss}  
	      	\item \href{http://groups.io/g/tech-for-blind}{Tech For Blind}  
	      	\item \href{http://groups.io/g/blindadtech}{BlindADTech}  
	      	\item \href{http://groups.io/g/blind-techies/messages}{Blind Techies}  
	      \end{itemize}
\end{itemize}