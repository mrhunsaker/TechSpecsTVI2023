\chapter{Bridging Literacy: The Crucial Role of Refreshable Braille Displays in Empowering Visually Impaired Students}\label{braille-first-devices}

In the intricate tapestry of education, the pursuit of literacy is a fundamental thread, weaving through the academic journey of every student. For visually impaired learners, the path to literacy takes on a unique character, one in which the tactile elegance of braille becomes a vital conduit to knowledge. Within this narrative, refreshable braille displays emerge as indispensable companions, unlocking the doors to literacy, fostering engagement, and propelling students toward academic success. This chapter explores how refreshable braille displays are not merely tools but keystones in the quest for literacy and educational achievement among visually impaired students.

Refreshable braille displays integrate the tactile richness of braille with the dynamic capabilities of digital communication. These devices are pivotal in ensuring that visually impaired students not only read but actively participate in the discourse of knowledge acquisition.

Refreshable braille displays serve as conduits for accessing textual content, enabling the exploration of literature, textbooks, and diverse educational materials in a format that aligns with the tactile language of braille. They also empower students to actively contribute to the discourse, facilitating note-taking, writing, and engaging in classroom discussions with the same spontaneity and fluency as their sighted peers.

By providing visually impaired students with the means to interact with written information independently and dynamically, these devices foster a sense of agency and pave the way for academic success.

\section{Braille Notetakers and Laptops}\label{braille-notetakers-and-braille-laptop-computers}

Braille notetakers such as the BrailleSense6 and BrailleNote Touch Plus are essential tools for students with visual impairments to access their schoolwork and receive a free and accessible public education. These devices are small and portable, allowing students to take notes in class using either braille or standard (QWERTY) keyboard, or both. They can also be used to read books, write class assignments, find directions, record lectures, and listen to podcasts. The notes written on these devices can be transferred to a computer for storage or printed in either braille or print formats. Many note-taking devices have word processors, appointment calendars, calculators or clocks, and can do almost everything a computer can do. Some note-taking devices have a speech program with braille input. Many newer models are Bluetooth accessible which allows them to be used with iPads, iPhones and other Bluetooth devices as well as Wi-Fi access. Braille notetakers are useful not only for note taking in class, but also for composing and printing essays, writing notes, sending e-mails, or browsing the Internet. These devices can give students who are blind or have low vision support in all academic areas as well as in expanded core curriculum. By providing students with visual impairments access to braille notetakers, we can help ensure that they have the tools they need to succeed in their studies and beyond.

\tagpdfsetup{table/header-rows={1}}
\centering
\begin{longtblr}[
  caption = {Braille notetakers and laptops: device and operating system},
  label = {tab:chapter3:braille-notetakers-laptops}
]{
  colspec = {X[l] X[l]},
  rowhead = 1,
  hlines,
  stretch = 1.5,
}
Device Name & Operating System \\
BrailleNote Touch+ & Android 8 \\
BrailleSense 6 & Android 10 \\
BTSpeak Pro & Linux \\
Canute Console & Rasperian 12 \\
ElBraille 40 & Windows 10 \\
InsideONE+ & Windows 11 \\
Nattiq Note & Windows 11 \\
Notey the Notetaker & Windows 11 \\
Orbit Optima & Windows 11 \\
Seika Studio & Windows 10 \\
b.note & Windows 10 \\
b.book & Windows 10 \\
\end{longtblr}

\section{Braille Notetaker/Laptop Recommendations}\label{braille-notetakers-and-braille-laptop-computers-recommend}
The BrailleNote Touch Plus runs on Android 8.1 Oreo, while the BrailleSense 6 runs on Android 10\footnote{December 2023 HIMS Released an Update for the BrailleSense 6 to upgrade it to Android 12, however it remains an update rather than pre-installed on the device \href{https://hims-inc.com/wp-content/uploads/2023/11/Release-Note-for-BrailleSense-6-V2.0.docx}{Release Notes}}. Both operating systems are outdated, with Android 14 being the current version of the Android operating system as of \today\footnote{Android 15 will be released in Beta in February 2024 and as a stable version in October 2024}.

Using outdated operating systems can pose a security risk, as they no longer receive security updates. This makes it easier for harmful viruses, spyware, and other malicious software to gain access to your device. Hackers often target outdated operating systems because of their vulnerability, allowing them to breach your device and gain personal information. Preventing malicious access to hardware is one major reason why drivers and applications are made back-compatible only to versions of the operating system still receiving security updates.

It is important to keep your operating system up-to-date to ensure that you have access to the latest features and improvements. This can help improve the performance of your device and ensure that it is compatible with the latest software and hardware. Updating your operating system is a simple and effective way to keep your device running smoothly and securely.

However, updating an operating system is not always possible, as it depends on the device's hardware and software compatibility. It is also important to note that updating to the latest operating system may not always be the best option, as it may cause compatibility issues with older software and hardware.

\emph{Table \ref{tab:chapter3:braille-notetaker-laptop-recommendations}} gives the recommendations for currently available braille notetakers. An important note is that I favor Windows-based systems as the current most popular devices that run on the Android OS platform are both out-of-date with regards to their operating system as can be seen in \emph{Table \ref{tab:chapter3:braille-notetakers-laptops}} above.

\tagpdfsetup{table/header-rows={1}}
\centering
\begin{longtblr}[
  caption = {Braille notetaker and laptop recommendations with key specifications},
  label = {tab:chapter3:braille-notetaker-laptop-recommendations}
]{
  colspec = {X[l] X[l] X[l] X[l] X[l]},
  rowhead = 1,
  hlines,
  stretch = 1.5,
}
Display & Battery & Keyboard & Manufacturer & OS \\
Orbit Optima & TBD & QWERTY & Orbit Research & Windows 11 \\
Seika Studio & TBD & QWERTY & Nippon Telesoft & Windows 10 \\
b.book & 15h & Perkins & Eurobraille & Windows 10 \\
\end{longtblr}

\section{Refreshable Braille Displays}\label{refreshable-braille-displays}

Refreshable braille displays are essential tools for students with visual impairments to access digital content. The number of braille cells in a display is an important factor to consider when selecting a device. Displays with 32-40 cells are generally better than those with 14-20 cells for several reasons. Firstly, they provide more space for displaying text, which can help reduce the need for scrolling and improve reading speed. Secondly, they allow for more complex formatting, such as tables and graphs, which can be important for STEM subjects. Thirdly, they provide more context for the user, which can help improve comprehension and reduce errors. Fourthly, they are more versatile and can be used for a wider range of tasks, such as taking notes, writing essays, and browsing the internet. Finally, they are more future-proof, as they are more likely to be compatible with new technologies and software updates. While 14-20 cell displays may be more affordable, investing in a 32-40 cell display can provide significant benefits for students with visual impairments in the long run.

\subsection{14-20 cell Refreshable Braille Displays}\label{few-cell-refreshable-braille-displays}
There are some situations where 14-20 cell displays may be more appropriate. For example, if the student only needs to read short messages or simple documents, a smaller display may be sufficient. Additionally, smaller displays are more portable and can be easier to carry around. They may also be more affordable, which can be important for students on a tight budget. Finally, smaller displays may be more appropriate for younger students who are just learning braille and may not need as much space for displaying text. While 14-20 cell displays may not be as versatile as larger displays, they can still provide significant benefits for students with visual impairments in certain situations.

\tagpdfsetup{table/header-rows={1}}
\centering
\begin{longtblr}[
  caption = {14-20 cell refreshable braille displays: device and battery life},
  label = {tab:chapter3:braille-14-20cell},
  note = {Compact refreshable braille displays with 14-20 cells, comparing models by battery duration for portable use}
]{
  colspec = {X[l] X[l]},
  rowhead = 1,
  hlines,
  stretch = 1.5,
}
Device Name & Battery Life \\
Actilino & 16 hours \\
Basic Braille 20 & 16 hours \\
Brailliant BI20x & 14 hours \\
Chameleon 20 & 14 hours \\
Focus 14 Blue & 18 hours \\
Orbit Reader 20+ & 20 hours \\
Orbit Speak & 20 hours \\
BTSpeak & 15 hours \\
Seika 24 & 20 hours \\
Seika Mini Plus & 20 hours \\
VarioUltra 20 & 12 hours \\
b.note 20 & 15 hours \\
\end{longtblr}

\subsection{32-40 cell Refreshable Braille Displays}
Displays with 32-40 cells provide more space for displaying text, allow for more complex formatting, and are more versatile for a wider range of tasks. While 14-20 cell displays may be more affordable, investing in a 32-40 cell display can provide significant benefits for students with visual impairments in the long run.

\tagpdfsetup{table/header-rows={1}}
\centering
\begin{longtblr}[
  caption = {32-40 cell refreshable braille displays: features and manufacturers},
  label = {tab:chapter3:braille-32-40cell},
  note = {Full-size refreshable braille displays with 32-40 cells, comparing models by battery life, keyboard type, and manufacturer}
]{
  colspec = {X[l] X[l] X[l] X[l]},
  rowhead = 1,
  hlines,
  stretch = 1.5,
}
Display & Battery & Keyboard & Manufacturer \\
Activator & 40 & Perkins & Help Tech \\
Active Braille & 20 & Perkins & Help Tech \\
Active Star & 40 & Perkins & Help Tech \\
Alva 640 Comfort & 10 & Perkins & Optelec \\
Alva 640 USB & n/a & none & Optelec \\
Alva BC 640 & 10 & none & Alva \\
Basic Braille Plus & 12 & Perkins & Help Tech \\
Brailliant BI40x & 14 & Perkins & Humanware \\
Focus 40 Blue & 18 & Perkins & Vispero \\
Mantis Q40 & 14 & QWERTY & APH \\
Orbit Reader 40 & 20 & Perkins & Orbit Research \\
QBraille XL & 16 & Perkins & HIMS \\
Seika V5 & 20 & none & Nippon Telesoft \\
Vario 340 & 20 & none & VisioBraille \\
Vario 440 & 20 & none & VisioBraille \\
Vario Ultra 40 & 12 & Perkins & VisioBraille \\
b.note 40 & 15 & Perkins & Eurobraille \\
\end{longtblr}

\section{Multiple Line Braille Displays/Tablets}\label{multiple-line-refreshable-braille-displaystablets}
Multiple line braille displays are better than single line refreshable braille displays for students with visual impairments for several reasons. Firstly, they provide more space for displaying text, which can help reduce the need for scrolling and improve reading speed. Secondly, they allow for more complex formatting, such as tables and graphs, which can be important for STEM subjects. Thirdly, they provide more context for the user, which can help improve comprehension and reduce errors. Fourthly, they are more versatile and can be used for a wider range of tasks, such as taking notes, writing essays, and browsing the internet. Finally, they are more future-proof, as they are more likely to be compatible with new technologies and software updates. While single line refreshable braille displays may be more affordable, investing in a multiple line display can provide significant benefits for students with visual impairments in the long run.

\tagpdfsetup{table/header-rows={1}}
\centering
\begin{longtblr}[
  caption = {Multiple line braille displays and tablets: features and manufacturers},
  label = {tab:chapter3:multi-line-braille},
  note = {Advanced multi-line braille displays and tablets, showing comprehensive features and specifications for enhanced reading experience}
]{
  colspec = {X[l] X[l] X[l] X[l] X[l]},
  rowhead = 1,
  hlines,
  stretch = 1.5,
}
Display & Battery & Braille Lines & Keyboard & Manufacturer \\
APH Monarch & 11 hr & 10 row x 32 cell + 32 cell line & Perkins & Humanware, APH \\
Blitab & TBD & 14 row x 23 cell & Touch Interface & Blitab \\
BraillePad & TBD & 50 row x 40 cells & none & 4Blind \\
Cadence & TBD & 6 row x 8 cells, stack to 24 x 16 & Perkins & Tactile Engineering \\
Canute 360 & Req AC & 9 row x 40 cell & none & Bristol Braille \\
DotPad & 11 hr & 10 row x 32 cell + 20 cell line & Touch interface & Dot Inc. \\
Graphiti & 20-22 & 60 row x 40 cell & Perkins & Orbit Research \\
Graphiti Plus & 20-22 & 60 row x 40 cell + 40 cell line & Perkins & Orbit Research \\
Orbit Slate 340 & 20-22 & 5 row x 20 cell & Perkins & Orbit Research \\
Orbit Slate 520 & 20-22 & 5 row x 20 cell & Perkins & Orbit Research \\
TACTIS 100 & Req AC & 4 row x 25 cell & none & Tactisplay \\
TACTIS Table & Req AC & 25 row x 40 cell & none & Tactisplay \\
TACTIS Walk & Req AC & 10 row x 25 cell & none & Tactisplay \\
Tactile Pro & TBD & TBD & Perkins & PCT \\
Tactonom Pro & Req AC & 89 row x 119 cell & N/A & Tactonom \\
\end{longtblr}

\section{Braille Education Devices}\label{learning-tools}
In many cases, students do not learn braille as efficiently as their sighted peers learn print. One potential explanation is that there is limited time that a student has access to a teacher trained in braille. One solution is to provide devices that can be used to reinforce or train a student in braille skills without the need for a braille-fluent adult present. This is analogous to the Lexia, Prodigy, or other academic learning systems that allow for self-paced learning. In the last 5 years, a number of teaching tools have been developed, primarily by groups in India and South Korea to address these needs.

Specialized tools like Taptilo and Polly/Annie are crucial for teaching Braille to students with visual impairment. These tools provide a more interactive and engaging learning experience for students, which can help them learn Braille more effectively. Taptilo is a Braille learning device that uses a modular design to teach Braille in a fun and interactive way. It has a variety of features such as audio feedback, games, and quizzes that can help students learn Braille more effectively\footnote{\raggedright \href{https://www.taptilo.com/ }{Taptilo. (n.d.). Taptilo. Retrieved December 19, 2023} \href{https://www.taptilo.com/ }{https://www.taptilo.com/ }}. Polly and Annie are two Braille teaching tools that use a combination of hardware and software to teach Braille to students. They use a variety of interactive games and activities to help students learn Braille more effectively\footnote{\raggedright \href{https://www.thinkerbelllabs.com/}{Thinkerbell Labs. (n.d.). Polly. Retrieved December 19, 2023} \href{https://www.thinkerbelllabs.com/}{https://www.thinkerbelllabs.com/}}.

In addition to providing a more engaging learning experience, specialized tools like Taptilo and Polly/Annie can also help students learn Braille more quickly. These tools are designed to be intuitive and easy to use, which can help students learn Braille more quickly than traditional methods. Additionally, these tools can provide students with immediate feedback on their progress, which can help them identify areas where they need to improve.

Finally, specialized tools like Taptilo and Polly/Annie can help students with visual impairment become more independent. By learning Braille more effectively and quickly, students can become more independent in their daily lives. They can read books, take notes, and communicate with others more easily, which can help them lead more fulfilling lives.

\tagpdfsetup{table/header-rows={1}}
\centering
\begin{longtblr}[
  caption = {Braille education devices and their manufacturers},
  label = {tab:chapter3:braille-education-devices}
]{
  colspec = {X[l] X[l]},
  rowhead = 1,
  hlines,
  stretch = 1.5,
}
Equipment & Manufacturer \\
Braille Doodle & Touchpad Pro Foundation \\
Braille Teach & Braille Teach \\
BrailleBlox & BrailleBot \\
BrailleBuzz & APH \\
BrailleCoach & Logan Tech \\
Feelif Creator & Feelif Technology \\
Feelif Pro & Feelif Technology \\
Mountbatten Braille Tutor & Harpo \\
Polly\footnote{\raggedright Called ``Annie" outside the Unites States} & APH / Thinkerbell Labs \\
Read Read & EdVar Tech \\
SMART Brailler & Perkins \\
Taptilo & HIMS / OHFA Tech \\
\end{longtblr}
