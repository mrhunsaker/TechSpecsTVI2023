\hypertarget{braille-first-devices}{}\chapter[\raggedright Bridging Literacy:\hfill\break The Crucial Role of Refreshable Braille Displays in Empowering\hfill\break Visually Impaired Students]{Bridging Literacy: The Crucial Role of Refreshable Braille Displays in Empowering Visually Impaired Students}\label{braille-first-devices}
\noindent\makebox[\linewidth]{\rule{\linewidth}{0.4pt}}
{\let\clearpage\relax\localtableofcontents\let\clearpage\relax\locallistoftables}\newpage
\extramarks{Vision Department Technology Needs}{Chapter 3: Bridging Literacy}
In the intricate tapestry of education, the pursuit of literacy is a fundamental thread, weaving through the academic journey of every student. For visually impaired learners, the path to literacy takes on a unique character, one in which the tactile elegance of braille becomes a vital conduit to knowledge. Within this narrative, refreshable braille displays emerge as indispensable companions, unlocking the doors to literacy, fostering engagement, and propelling students toward academic success. This chapter embarks on a compelling exploration of how refreshable braille displays are not merely tools but keystones in the quest for literacy and educational achievement among visually impaired students.

At the heart of this exploration lies the transformative power of refreshable braille displays—a technological marvel that seamlessly integrates the tactile richness of braille with the dynamic capabilities of digital communication. This chapter delves into the ergonomic design and sophisticated functionalities of these devices, spotlighting their pivotal role in ensuring that visually impaired students not only read but actively participate in the discourse of knowledge acquisition.

Refreshable braille displays play a dual role in the educational narrative of visually impaired students. Firstly, they serve as conduits for accessing textual content, enabling the exploration of literature, textbooks, and diverse educational materials in a format that aligns with the tactile language of braille. Secondly, and perhaps more profoundly, these devices empower students to actively contribute to the discourse, facilitating note-taking, writing, and engaging in classroom discussions with the same spontaneity and fluency as their sighted peers.

By providing visually impaired students with the means to interact with written information independently and dynamically, these devices emerge not just as tools but as instruments of empowerment, fostering a sense of agency and paving the way for academic success in the rich landscape of education.

\pagebreak
\hypertarget{braille-notetakers-and-braille-laptop-computers}{}\section{Braille Notetakers and Laptops}\label{braille-notetakers-and-braille-laptop-computers}

Braille notetakers such as the BrailleSense6 and BrailleNote Touch Plus are essential tools for students with visual impairments to access their schoolwork and receive a free and accessible public education. These devices are small and portable, allowing students to take notes in class using either braille or standard (QWERTY) keyboard, or both. They can also be used to read books, write class assignments, find directions, record lectures, and listen to podcasts. The notes written on these devices can be transferred to a computer for storage or printed in either braille or print formats. Many note-taking devices have word processors, appointment calendars, calculators or clocks, and can do almost everything a computer can do. Some note-taking devices have a speech program with braille input. Many newer models are Bluetooth accessible which allows them to be used with iPads, iPhones and other Bluetooth devices as well as Wi-Fi access Braille notetakers are useful not only for note taking in class, but also for composing and printing essays, writing notes, sending e-mails, or browsing the Internet These devices can give students who are blind or have low vision support in all academic areas as well as in expanded core curriculum. By providing students with visual impairments access to braille notetakers, we can help ensure that they have the tools they need to succeed in their studies and beyond. \textit{Table \ref{tab:table11}} gives the specs for currently available braille notetakers.

\pagebreak 
 
\begin{longtable}[]{@{}
 >{\raggedright\arraybackslash}m{.22\textwidth}
 >{\raggedright\arraybackslash}m{.1\textwidth}
 >{\raggedright\arraybackslash}m{.1\textwidth}
 >{\raggedright\arraybackslash}m{.1\textwidth}
 >{\raggedright\arraybackslash}m{.15\textwidth}
 >{\raggedright\arraybackslash}b{.2\textwidth}@{}
 }
 \toprule
 
 \textbf{Display} & \textbf{Cost} & \textbf{Battery} & \textbf{Keyboard} & \textbf{Manufacturer} & \textbf{OS} \\
 \midrule
 \endhead \hline \\
 \multicolumn{6}{r}{\textbf{Continued on Next Page}} \endfoot
 \endlastfoot
 BrailleNote Touch+\footnote{\raggedright For both the BrailleNote Touch+ and BrailleSense 6, there is an emerging issue with outdated operating systems, WiFi connectivity inconsistencies, and incompatibility with Google applications.\hfill\break\textbullet\hspace{2.5mm} \href{http://perkins.org/braillenote-touch-outdated-os/}{Link to article from Perkins.org regarding the BrailleNote Touch Plus} \hfill\break\textbullet\hspace{2.5mm} \href{http://endoflife.date/android}{Continually Updated List for End of Life for all flavors of AndroidOS}} & \$5,795 & 12h & Perkins & Humanware & Android 8\footnote{\raggedright Android 8 `Oreo' Security Support Ended (\emph{i.e.}, End of Life) 2017-12-05} \\ \cdashline{1-6}
 BrailleSense 6 & \$5,795 & 12h & Perkins & HIMS & Android 10\footnote{\raggedright Android 10 `Queen Cake' Security Support Ended (\emph{i.e.}, End of Life) 2023-03-06}\fnsep\footnote{\raggedright The Braille Sense has a firmware update v2.0 released 2023-11-28 which updates the operating system to Android 12, but this is currently buggy and causing system overheating. (\emph{note:} Android 12 will only receive updates until \textasciitilde October of 2024)} \\ \cdashline{1-6}
 BTSpeak Pro\footnote{\raggedright This is a Linux Computer without a monitor that runs on ORCA, but it has settings to function like a braille notetaker}\fnsep\footnote{This device has only auditory output} & \$1,195 & 15h & Perkins & Blazie Tech & Linux \\[1em]
 Canute Console & \$6,890\footnote{\raggedright \$3,995 for the Canute Console+\$2,895 for the Canute Display} & 15h & QWERTY & Bristol Braille & Rasperian 12\footnote{\raggedright Debian 12 Linux for Raspberry Pi}\fnsep\footnote{\raggedright This is a console/terminal driven operating system that requires knowledge of Linux (or desire to learn) and a certain level of comfort using bash commands as a primary method of controlling the system} \\ \cdashline{1-6}ElBraille 40\footnote{\raggedright This is a Windows Computer without a monitor that runs on JAWS, but it has settings to function like a braille notetaker}\fnsep\footnote{\raggedright This is not included above as a laptop option since it has only 4GB of RAM} & \$6,000\footnote{\raggedright This price is for the ElBraille unit itself as well as a Focus 40 that docks into the unit as a display} & TBD & QWERTY & Elita & Windows 10\footnote{\raggedright Windows 11 is not yet officially supported, but users are updating to Windows 11 without issue} \\ \cdashline{1-6}InsideONE+ & \$5,499 & 6h & Perkins & InsideVision & Windows 11 \\ \cdashline{1-6}Nattiq Note & \$5,200 & 12h & QWERTY & Nattiq & Windows 11 \\ \cdashline{1-6}Notey the Notetaker & \textasciitilde\$750+\footnote{\raggedright Self build \href{http://notey-project.com/2023/03/07/notey-user-manual-v1-0-2/}{Specs for Notey the NoteTaker}} & & QWERTY Perkins & Miscs & Windows 11 \\ \cdashline{1-6}Orbit Optima\footnote{\raggedright This is a Windows Computer without a monitor that runs on any screenreader, but it has settings to function like a braille notetaker} & \$6,000\break \$9,000 & TBD & QWERTY & Orbit Research & Windows 11 \\ \cdashline{1-6}
 Seika Studio\footnote{\raggedright This is a Windows Computer without a monitor, but it has settings to function like a braille notetaker} & \$6,500 & TBD & QWERTY & Nippon Telesoft & Windows 10 \\ \cdashline{1-6}
 b.note & \$4,360 & 15h & Perkins & Eurobraille & Windows 10 \\ \cdashline{1-6}
 b.book & \$5,765 & 15h & Perkins & Eurobraille & Windows 10\footnote{\raggedright Windows 11 version coming Q1 2024} \\[1.0em]\hline
 \caption{ Braille NoteTakers and Laptops }\label{tab:table11}
\end{longtable}\clearpage
\pagebreak

\hypertarget{braille-notetakers-and-braille-laptop-computers-recommend}{}\section{Braille Notetaker/Laptop Recommendations}\label{braille-notetakers-and-braille-laptop-computers-recommend}
The BrailleNote Touch Plus runs on Android 8.1 Oreo, while the BrailleSense 6 runs on Android 10\footnote{\raggedright December 2023 HIMS Released an Update for the BrailleSense 6 to upgrade it to Android 12, \href{https://hims-inc.com/wp-content/uploads/2023/11/Release-Note-for-BrailleSense-6-V2.0.docx}{Release Notes}}. Both operating systems are outdated, with Android 14 being the current version of the Android operating system as of \today\footnote{\raggedright Android 15 will be released in Beta in February 2024 and as a stable version in October 2024}.

Using outdated operating systems can pose a security risk, as they no longer receive security updates. This makes it easier for harmful viruses, spyware, and other malicious software to gain access to your device. Hackers often target outdated operating systems because of their vulnerability, allowing them to breach your device and gain personal information. Preventing malicious access to hardware is one major reason why drivers and applications are made back-compatible only to versions of the operating system still receiving security updates. 

It is important to keep your operating system up-to-date to ensure that you have access to the latest features and improvements. This can help improve the performance of your device and ensure that it is compatible with the latest software and hardware. Updating your operating system is a simple and effective way to keep your device running smoothly and securely.

However, updating an operating system is not always possible, as it depends on the device’s hardware and software compatibility. It is also important to note that updating to the latest operating system may not always be the best option, as it may cause compatibility issues with older software and hardware.

\textit{Table \ref{tab:table111}} gives the recommendations for currently available braille notetakers. An important note is that I favor Windows-based system as the current most popular devices are both out-of-date with regards to their operating system as can be seed in \textit{Table \ref{tab:table11}} above.

\pagebreak 
 
\begin{longtable}[]{@{}
 >{\raggedright\arraybackslash}m{.22\textwidth}
 >{\raggedright\arraybackslash}m{.1\textwidth}
 >{\raggedright\arraybackslash}m{.1\textwidth}
 >{\raggedright\arraybackslash}m{.1\textwidth}
 >{\raggedright\arraybackslash}m{.2\textwidth}
 >{\raggedright\arraybackslash}b{.15\textwidth}@{}
 }
 \toprule
 
 \textbf{Display} & \textbf{Cost} & \textbf{Battery} & \textbf{Keyboard} & \textbf{Manufacturer} & \textbf{OS} \\
 \midrule
 \endhead \hline \\
 \multicolumn{6}{r}{\textbf{Continued on Next Page}} \endfoot
 \endlastfoot
 \rowcolor{red!10} Orbit Optima & \$6,000\break \$9,000 & TBD & QWERTY & Orbit Research & Windows 11 \\ \cdashline{1-6}
 Seika Studio & \$6,500 & TBD & QWERTY & Nippon Telesoft & Windows 10 \\ \cdashline{1-6}
 \rowcolor{red!10} b.book & \$5,765 & 15h & Perkins & Eurobraille & Windows 10 \\[1.0em]\hline
 \caption[Braille Notetaker and Laptop Recommendations]{Braille Notetaker and Laptop Recommendations. Overall recommendation highlighted in light red. }\label{tab:table111}
\end{longtable}\clearpage
\pagebreak
\hypertarget{refreshable-braille-displays}{}\section{Refreshable Braille
 Displays}\label{refreshable-braille-displays}

Refreshable braille displays are essential tools for students with visual impairments to access digital content. The number of braille cells in a display is an important factor to consider when selecting a device. Displays with 32-40 cells are generally better than those with 14-20 cells for several reasons. Firstly, they provide more space for displaying text, which can help reduce the need for scrolling and improve reading speed. Secondly, they allow for more complex formatting, such as tables and graphs, which can be important for STEM subjects. Thirdly, they provide more context for the user, which can help improve comprehension and reduce errors. Fourthly, they are more versatile and can be used for a wider range of tasks, such as taking notes, writing essays, and browsing the internet. Finally, they are more future-proof, as they are more likely to be compatible with new technologies and software updates. While 14-20 cell displays may be more affordable, investing in a 32-40 cell display can provide significant benefits for students with visual impairments in the long run.

\pagebreak
\hypertarget{few-cell-refreshable-braille-displays}{}\subsection{14-20 cell Refreshable Braille
 Displays}\label{few-cell-refreshable-braille-displays}
There are some situations where 14-20 cell displays may be more appropriate. For example, if the student only needs to read short messages or simple documents, a smaller display may be sufficient. Additionally, smaller displays are more portable and can be easier to carry around. They may also be more affordable, which can be important for students on a tight budget. Finally, smaller displays may be more appropriate for younger students who are just learning braille and may not need as much space for displaying text. While 14-20 cell displays may not be as versatile as larger displays, they can still provide significant benefits for students with visual impairments in certain situations. \textit{Table \ref{tab:table12}} lists current available display options.
\pagebreak \begin{flushleft} \pagebreak 
 
\begin{longtable}[]{@{}
 >{\raggedright\arraybackslash}m{.2\textwidth}
 >{\raggedright\arraybackslash}m{.15\textwidth}
 >{\raggedright\arraybackslash}m{.15\textwidth}
 >{\raggedright\arraybackslash}m{.15\textwidth}
 >{\raggedright\arraybackslash}b{.2\textwidth}@{}
 }
 \toprule
 
 \textbf{Display} & \textbf{Cost} & \textbf{Battery} & \textbf{Keyboard} & \textbf{Manufacturer} \\
 \midrule
 \endhead \hline \\
 \multicolumn{5}{r}{\textbf{Continued on next page}}
 \endfoot \endlastfoot
 Actilino & \$2,795 & 16 & Perkins & Help Tech \\ \cdashline{1-5}
 Basic Braille 20 & \$2,295 & 16 & none & Help Tech \\ \cdashline{1-5}
 Brailliant BI20x & \$2,199 & 14 & Perkins & Humanware \\ \cdashline{1-5}
 Chameleon 20 & \$1,715 & 14 & Perkins & APH \\ \cdashline{1-5}
 Focus 14 Blue & \$1,545 & 18 & Perkins & Vispero \\ \cdashline{1-5}
 Orbit Reader 20+ & \$799 & 20 & Perkins & Orbit Research \\ \cdashline{1-5}
 Orbit Speak\footnote{\raggedright This device has no braille output, but uses braille input and returns auditory output} & TBD & 20 & Perkins & Orbit Research \\ \cdashline{1-5}
 BTSpeak\footnotemark[\value{footnote}] & \$1,195 & 15 & Perkins & Blazie Tech \\[1em]
 Seika 24 & \$2,395 & 20 & none & Nippon Telesoft \\ \cdashline{1-5}
 Seika Mini Plus & \$2,795 & 20 & none & Nippon Telesoft \\ \cdashline{1-5}
 VarioUltra 20 & \$4,340 & 12 & Perkins & VisioBraille \\ \cdashline{1-5}
 b.note 20 & \$2,695 & 15 & Perkins & Eurobraille \\[1.0em] \hline
 \caption[ 14-20 cell Single Line Refreshable Braille Displays]{14-20 cell Single Line Refreshable Braille Displays}\label{tab:table12}
\end{longtable}\clearpage \end{flushleft}

\pagebreak
\hypertarget{cell-refreshable-braille-displays}{}\subsection{32-40 cell Refreshable Braille
 Displays}\label{cell-refreshable-braille-displays}
Refreshable braille displays are essential tools for students with visual impairments to access digital content. The number of braille cells in a display is an important factor to consider when selecting a device. Displays with 32-40 cells are generally better than those with 14-20 cells for several reasons. Firstly, they provide more space for displaying text, which can help reduce the need for scrolling and improve reading speed. Secondly, they allow for more complex formatting, such as tables and graphs, which can be important for STEM subjects. Thirdly, they provide more context for the user, which can help improve comprehension and reduce errors. Fourthly, they are more versatile and can be used for a wider range of tasks, such as taking notes, writing essays, and browsing the internet. Finally, they are more future-proof, as they are more likely to be compatible with new technologies and software updates. While 14-20 cell displays may be more affordable, investing in a 32-40 cell display can provide significant benefits for students with visual impairments in the long run. \textit{Table \ref{tab:table13}} lists current available display options.

\pagebreak 
 
\begin{longtable}[]{@{}
 >{\raggedright\arraybackslash}m{.2\textwidth}
 >{\raggedright\arraybackslash}m{.15\textwidth}
 >{\raggedright\arraybackslash}m{.15\textwidth}
 >{\raggedright\arraybackslash}m{.15\textwidth}
 >{\raggedright\arraybackslash}b{.2\textwidth}@{}
 }
 \toprule
 
 \textbf{Display} & \textbf{Cost} & \textbf{Battery} & \textbf{Keyboard} & \textbf{Manufacturer} \\
 \midrule
 \endhead \hline \\
 \multicolumn{5}{r}{\textbf{Continued on Next Page}} \endfoot
 \endlastfoot
 Activator & \$6,495 & 40 & Perkins & Help Tech \\ \cdashline{1-5}
 Active Braille & \$6,495 & 20 & Perkins & Help Tech \\ \cdashline{1-5}
 Active Star & \$6,795 & 40 & Perkins & Help Tech \\ \cdashline{1-5}
 Alva 640 Comfort & \$3,046 & 10 & Perkins & Optelec \\ \cdashline{1-5}
 Alva 640 USB & \$3837 & n/a & none & Optelec \\ \cdashline{1-5}
 Alva BC 640 & \$2,087 & 10 & none & Alva \\ \cdashline{1-5}
 Basic Braille Plus & \$3,295 & 12 & Perkins & Help Tech \\ \cdashline{1-5}
 Brailliant BI40x & \$3,195 & 14 & Perkins & Humanware \\ \cdashline{1-5}
 Focus 40 Blue & \$3,145 & 18 & Perkins & Vispero \\ \cdashline{1-5}
 Mantis Q40 & \$2,495 & 14 & QWERTY & APH \\ \cdashline{1-5}
 Orbit Reader 40 & \$1,399 & 20 & Perkins & Orbit Research \\ \cdashline{1-5}
 QBraille XL & \$3,195 & 16 & Perkins & HIMS \\ \cdashline{1-5}
 Seika V5 & \$2,495 & 20 & none & Nippon Telesoft \\ \cdashline{1-5}
 Vario 340 & \$5,138 & 20 & none & VisioBraille \\ \cdashline{1-5}
 Vario 440 & \$4,550 & 20 & none & VisioBraille \\ \cdashline{1-5}
 Vario Ultra 40 & \$7,643 & 12 & Perkins & VisioBraille \\ \cdashline{1-5}
 b.note 40 & \$3,565 & 15 & Perkins & Eurobraille \\[1.0em] \hline
 \caption{ 32-40 cell Single Line Refreshable Braille Displays }\label{tab:table13}
\end{longtable}\clearpage

\pagebreak
\hypertarget{multiple-line-refreshable-braille-displaystablets}{}\section{Multiple Line Braille Displays/Tablets}\label{multiple-line-refreshable-braille-displaystablets}
Multiple line braille displays are better than single line refreshable braille displays for students with visual impairments for several reasons. Firstly, they provide more space for displaying text, which can help reduce the need for scrolling and improve reading speed. Secondly, they allow for more complex formatting, such as tables and graphs, which can be important for STEM subjects. Thirdly, they provide more context for the user, which can help improve comprehension and reduce errors. Fourthly, they are more versatile and can be used for a wider range of tasks, such as taking notes, writing essays, and browsing the internet. Finally, they are more future-proof, as they are more likely to be compatible with new technologies and software updates. While single line refreshable braille displays may be more affordable, investing in a multiple line display can provide significant benefits for students with visual impairments in the long run. \textit{Table \ref{tab:table14}} lists current available display options.


\pagebreak 
 
\begin{longtable}[]{@{}
 >{\raggedright\arraybackslash}m{.2\textwidth}
 >{\raggedright\arraybackslash}m{.1\textwidth}
 >{\raggedright\arraybackslash}m{.1\textwidth}
 >{\raggedright\arraybackslash}b{.2\textwidth}
 >{\raggedright\arraybackslash}m{.1\textwidth}
 >{\raggedright\arraybackslash}b{.2\textwidth}@{}
 }
 \toprule
 
 \textbf{Display} & \textbf{Cost} & \textbf{Battery} & \textbf{Braille Lines} & \textbf{Keyboard} & \textbf{Manufacturer} \\
 \midrule
 \endhead \hline \\
 \multicolumn{6}{r}{\textbf{Continued on Next Page}} \endfoot
 \endlastfoot
 APH Monarch & \textasciitilde\$15,000 & 11 hr & 10 row x 32 cell \break+32 cell line & Perkins & Humanware\break APH \\ \cdashline{1-6}
 Blitab & \$500 & TBD & 14 row x 23 cell & Touch Interface & Blitab \\ \cdashline{1-5}
 BraillePad & \$4,390 & TBD & 50 row x 40 cells & none & 4Blind \\ \cdashline{1-6}
 Cadence & TBD & TBD & 6 row x 8 cells\break stack to 24 x 16 & Perkins & Tactile Engineering \\ \cdashline{1-6}
 Canute 360 & \$2,895 & Req AC & 9 row x 40 cell & none & Bristol Braille \\ \cdashline{1-6}
 DotPad & \textasciitilde\$15,000 & 11 hr & 10 row x 32 cell \break+ 20 cell line & Touch interface & Dot Inc. \\ \cdashline{1-6}
 Graphiti & \textasciitilde\$15,000 & 20-22 & 60 row x 40 cell & Perkins & Orbit Research \\ \cdashline{1-6}
 Graphiti Plus & \textasciitilde\$15,000 & 20-22 & 60 row x 40 cell \break+ 40 cell line & Perkins & Orbit Research \\ \cdashline{1-6}
 Orbit Slate 340 & \$3,995 & 20-22 & 5 row x 20 cell & Perkins & Orbit Research \\ \cdashline{1-6}
 Orbit Slate 520 & \$3,495 & 20-22 & 5 row x 20 cell & Perkins & Orbit Research \\ \cdashline{1-6}
 TACTIS 100 & \textasciitilde\$5,000 & Req AC & 4 row x 25 cell & none & Tactisplay \\ \cdashline{1-6}
 TACTIS Table & \textasciitilde\$15,,000 & Req AC & 25 row x 40 cell & none & Tactisplay \\ \cdashline{1-6}
 TACTIS Walk & \textasciitilde\$7,000 & Req AC & 10 row x 25 cell & none & Tactisplay \\ \cdashline{1-6}
 Tactile Pro & TBD & TBD & TBD & Perkins & PCT \\ \cdashline{1-6}
 Tactonom Pro & \textasciitilde\$15,000 & Req AC & 89 row x 119 cell & N/A & Tactonom \\[1.0em]\hline
 \caption{ Multiple Line Refreshable Braille Devices }\label{tab:table14}
\end{longtable}\clearpage
\pagebreak
\hypertarget{learning-tools}{}\section{Braille Education Devices}\label{learning-tools}
In many cases, students do not learn braille as efficiently as their sighted peers learn print. One potential explanation is that there is limited time that a student has access to a teacher trained in braille. One solution is to provide devices that can be used to reinforce or train a student in braille skills without the need for a braille-fluent adult present. This is analogous to the Lexia, Prodigy, or other academic learning systems that allow for self-paced learning. In the last 5 years, a number of teaching tools have been developed, primarily by groups in India and South Korea to address these needs.

Specialized tools like Taptilo and Polly/Annie are crucial for teaching Braille to students with visual impairment. These tools provide a more interactive and engaging learning experience for students, which can help them learn Braille more effectively. Taptilo is a Braille learning device that uses a modular design to teach Braille in a fun and interactive way. It has a variety of features such as audio feedback, games, and quizzes that can help students learn Braille more effectively\footnote{\raggedright \href{https://www.taptilo.com/ }{Taptilo. (n.d.). Taptilo. Retrieved December 19, 2023} \href{https://www.taptilo.com/ }{https://www.taptilo.com/ }}. Polly and Annie are two Braille teaching tools that use a combination of hardware and software to teach Braille to students. They use a variety of interactive games and activities to help students learn Braille more effectively\footnote{\raggedright \href{https://www.thinkerbelllabs.com/}{Thinkerbell Labs. (n.d.). Polly. Retrieved December 19, 2023} \href{https://www.thinkerbelllabs.com/}{https://www.thinkerbelllabs.com/}}.

In addition to providing a more engaging learning experience, specialized tools like Taptilo and Polly/Annie can also help students learn Braille more quickly. These tools are designed to be intuitive and easy to use, which can help students learn Braille more quickly than traditional methods. Additionally, these tools can provide students with immediate feedback on their progress, which can help them identify areas where they need to improve.

Finally, specialized tools like Taptilo and Polly/Annie can help students with visual impairment become more independent. By learning Braille more effectively and quickly, students can become more independent in their daily lives. They can read books, take notes, and communicate with others more easily, which can help them lead more fulfilling lives.
\textit{Table \ref{tab:table15}} lists current available options for braille instructional devices.

\pagebreak\begin{flushleft} \pagebreak 
 
\begin{longtable}[]{@{}
 >{\raggedright\arraybackslash}m{.3\textwidth}
 >{\raggedright\arraybackslash}m{.1\textwidth}
 >{\raggedright\arraybackslash}b{.6\textwidth}@{}
 }
 \toprule
 
 \textbf{Equipment} & \textbf{Cost} & \textbf{Manufacturer} \\
 \midrule
 \endhead \hline \\
 \multicolumn{3}{r}{\textbf{Continued on next page}}
 \endfoot \endlastfoot
 Braille Doodle & \$85 & Touchpad Pro Foundation \\ \cdashline{1-3}
 Braille Teach & \$150 & Braille Teach \\ \cdashline{1-3}
 BrailleBlox & \$85 \footnote{\raggedright Requires purchase of a LeapFrog Fridge Phonics base, \textasciitilde\$20} & BrailleBot \\ \cdashline{1-3}
 BrailleBuzz & \$99 & APH \\ \cdashline{1-3}
 BrailleCoach & \$1,095 & Logan Tech \\ \cdashline{1-3}
 Feelif Creator & \$2,200 & Feelif Technology \\ \cdashline{1-3}
 Feelif Pro & \$3,595 & Feelif Technology \\ \cdashline{1-3}
 Mountbatten Braille Tutor & \$5,495 & Harpo \\ \cdashline{1-3}
 Polly\footnote{\raggedright Called ``Annie" outside the Unites States} & \$999 & APH \break Thinkerbell Labs \\ \cdashline{1-3}
 Read Read & \$645 & EdVar Tech \\ \cdashline{1-3}
 SMART Brailler & \$2,195 & Perkins \\ \cdashline{1-3}
 Taptilo & \$1,349 & HIMS\break OHFA Tech \\[1.0em]\hline
 \caption[Braille Education Device]{Braille Education Device}\label{tab:table15}
\end{longtable}\clearpage \end{flushleft}