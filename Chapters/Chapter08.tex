\chapter{Navigating Independence: The Essential Role of Accessible Daily Living Technology in Empowering Visually Impaired Students for Success and Safety}\label{accessible-gps-mapping}

In the pursuit of independence and safety, orientation and mobility training holds a pivotal place in the educational journey of visually impaired students. In this dynamic landscape, accessible GPS equipment emerges as a technological beacon, offering a transformative bridge to mobility, autonomy, and enhanced safety. This chapter explores the indispensable role that accessible GPS tools play in empowering visually impaired students for success, ensuring safe navigation through the world, and fostering a sense of confidence in their daily lives.

The quest for independence is intricately tied to the ability to navigate and explore the surrounding environment. For visually impaired students, this journey is often met with challenges that extend beyond the typical obstacles encountered in education. Accessible GPS equipment becomes a critical ally, providing not only the means to explore the world independently but also enhancing safety through reliable navigational assistance.

As we delve into this chapter, we will explore the functionalities of accessible GPS devices tailored to the unique needs of visually impaired users. From real-time audible directions to haptic feedback systems, these tools extend beyond standard navigation, creating a multi-sensory experience that empowers students to traverse their surroundings confidently. The importance of this technology is accentuated during orientation and mobility training, where students learn not only to navigate physical spaces but also to develop crucial skills for safety and situational awareness.

Beyond the practicalities of navigation, the impact of accessible GPS equipment on student success cannot be overstated. These tools contribute to broader educational goals by fostering a sense of independence, reducing reliance on external assistance, and instilling a foundational skill set for safe and self-assured mobility.

Through this exploration, it becomes clear that accessible GPS equipment is not merely a tool for navigation; it is a catalyst for empowerment and safety. Through orientation and mobility training, we ensure that visually impaired students can embark on their educational journeys with a sense of autonomy, confidence, and, above all, safety.

\section{Accessible GPS Hardware and Software}\label{accessible-gps-mapping-hardware}
When purchasing an accessible GPS unit for the blind, it is important to consider the following factors to ensure safe navigation and crossing of streets:
\begin{itemize}
 \item \emph{Audible signals}: The GPS unit should provide audible signals to indicate when it is safe to cross the street. This feature allows blind pedestrians to cross the road at the right time, more quickly and safely while maintaining their orientation throughout the crossing\footnote{\href{http://www.inclusivecitymaker.com/pedestrian-safety-visually-impaired-blind-people/}{Inclusive City Maker. (n.d.). Pedestrian safety: Are your crossings safe for the visually impaired? Retrieved December 19, 2023}}.
 \item \emph{Compatibility}: The GPS unit should be compatible with other assistive technology devices, such as screen readers and braille displays\footnote{\href{http://www.afb.org/blindness-and-low-vision/using-technology/smartphone-gps-navigation-people-visual-impairments}{American Foundation for the Blind. (n.d.). Smartphone GPS navigation. Retrieved December 19, 2023}}.
 \item \emph{Portability}: Portable GPS units are ideal for blind pedestrians who need to move around the city. They should be lightweight and easy to carry.
 \item \emph{Battery life}: Battery life is an important consideration for portable GPS units. The battery should last long enough to get through a day without needing to be recharged.
 \item \emph{Ease of use}: The GPS unit should be easy to use and adjust. It should have large buttons and controls that are easy to locate and operate.
 \item \emph{AI integration}: Modern GPS systems now incorporate artificial intelligence to provide enhanced contextual guidance and real-time environmental awareness.
 \item \emph{Indoor navigation}: Many current systems now support indoor navigation capabilities where traditional GPS signals are not available.
\end{itemize}
These considerations will help ensure that blind pedestrians have access to the tools they need to navigate and cross streets safely. \emph{Table \ref{tab:chapter8:accessible-gps-hardware}} lists current available accessible GPS hardware devices and applications.

\tagpdfsetup{table/header-rows={1}}
\centering
\begin{longtblr}[
  caption = {Accessible GPS hardware and software: model, function, and company},
  label = {tab:chapter8:accessible-gps-hardware},
  note = {Available GPS navigation devices and applications designed for visually impaired users, including specialized features like haptic feedback, audio output, and AI integration (Updated 2025)}
]{
  colspec = {X[l] X[l] X[l]},
  rowhead = 1,
  hlines,
  stretch = 1.5
}
Model & Function & Company \\
Stellar Trek & GPS & Humanware \\
Victor Reader Trek & GPS + Digital Audio Player & Humanware \\
Wayband & GPS (Haptic Output) & WearWorks \\
Envision Glasses & AI-Powered GPS + Object Recognition & Envision \\
Glidance Glide & Self-Guided Mobility Assistant & Glidance \\
Audiom & Audio-Based Navigation & Audiom \\
BlindSquare & Smartphone GPS App & MIPsoft \\
Lazarillo & Free GPS Navigation App & Lazarillo \\
Nearby Explorer & GPS Navigation App & American Printing House \\
MyWay Classic & Comprehensive GPS App & Swiss Federation of the Blind \\
OKO AI Copilot & AI Traffic Signal Recognition & OKO \\
Voice Vista & 3D Audio GPS (Microsoft) & Microsoft \\
\end{longtblr}

\section{Accessible Technology for Daily Living}\label{ind-living}
Auditory feedback technology is essential for blind people to live independently and complete daily tasks. It provides a way for the visually impaired to interact with their environment and receive information that they would otherwise miss. Modern assistive technology has evolved significantly with the integration of artificial intelligence, providing more sophisticated and contextual assistance than ever before.

The integration of AI-powered assistance has revolutionized daily living technology for the visually impaired. These systems now offer on-demand guidance, contextual help, and enhanced environmental awareness through advanced algorithms and machine learning capabilities. For example, AI-powered tools can now provide real-time object recognition, text reading, and scene description, dramatically expanding the independence of visually impaired users.

Smart assistive navigation systems now combine voice-over technology with advanced object detection capabilities, offering guidance through auditory feedback and tactile input upon object recognition. These systems can identify people, animals, crosswalks, pavements, and uneven terrain, providing comprehensive environmental awareness that extends far beyond traditional mobility aids.

In addition to navigation, modern assistive technology supports learning and skill development. Recent advances in haptic feedback technology continue to enhance Braille learning, while AI-powered applications provide personalized learning experiences tailored to individual needs and preferences.

The global assistive technologies market for visually impaired individuals is projected to reach \$12.31 billion by 2029, reflecting the growing recognition of these technologies' importance and the continued innovation in this field. This growth demonstrates the expanding availability and sophistication of tools designed to support independent living for the visually impaired community.

\subsection{Accessible Home Technology}\label{ind-living-tools}
When purchasing household items modified to give audio feedback for the blind, it is important to consider the following factors to ensure that they can access activities of daily living\footnote{\href{http://www.allaboutvision.com/resources/adapting-the-home-better-blindness-accessibility/}{All About Vision. (n.d.). Adapting your home for better blindness accessibility. Retrieved December 19, 2023}}:
\begin{itemize}
 \item \emph{Audible feedback}: Household items should provide audible feedback to the user to ensure that they are being used correctly and safely.
 \item \emph{Compatibility}: The item should be compatible with other assistive technology devices, such as screen readers and braille displays.
 \item \emph{Ease of use}: The item should be easy to use and adjust. It should have large buttons and controls that are easy to locate and operate.
 \item \emph{Portability}: Portable items are ideal for blind users who need to move around the house. They should be lightweight and easy to carry.
 \item \emph{Cost}: The cost of the item should be reasonable and within the user's budget.
 \item \emph{Smart home integration}: Modern devices should integrate with smart home systems and voice assistants for enhanced automation and control.
 \item \emph{AI capabilities}: Advanced devices now incorporate artificial intelligence for improved functionality and personalized assistance.
\end{itemize}
These considerations will help ensure that blind users have access to the tools they need to perform activities of daily living safely and independently.

\emph{Table \ref{tab:chapter8:accessible-home-technology}} shows a range of technology available for blind/visually impaired people designed to facilitate independent living\footnote{Prices updated from current market research and major vendors of products intended to facilitate independent living skills (2025)}.

\tagpdfsetup{table/header-rows={1}}
\centering
\begin{longtblr}[
  caption = {Accessible home technology: model and cost (Updated 2025)},
  label = {tab:chapter8:accessible-home-technology},
  note = {Comprehensive list of accessible household devices with audio feedback for independent living, including medical, kitchen, measurement tools, and AI-powered devices}
]{
  colspec = {X[l] X[l]},
  rowhead = 1,
  row{1} = {font=\bfseries},
  hlines,
  stretch = 1.5
}
Model & Cost \\
Infrared Talking Thermometer & \$50 \\
Liquid Level Indicator & \$15 \\
PenFriend Voice Labelling System & \$190 (Extra 418 labels: \$35) \\
Talking First Aid Guide & \$40 \\
Talking Indoor/Outdoor Thermometer & \$20 \\
Talking Kitchen Scale & \$45 \\
Talking Measuring Tape & \$160 \\
Talking Meat Thermometer & \$45 \\
Talking Timer Clock & \$20 \\
Talking Watch & \$25 \\
Talking Weighing Scale & \$45 \\
Talking Pulse Oximeter & \$40 \\
Talking Scale (Body Weight) & \$85 \\
Talking Blood Pressure Monitor & \$150 \\
Talking Pill System & \$85 \\
Talking Blood Glucose Meter & \$45 \\
WayLink Scanner & \$140 (Extra 25 magnets: \$45) \\
Smart Talking Thermostat & \$120 \\
AI-Powered Voice Assistant Device & \$80 \\
Talking Color Identifier & \$30 \\
Smart Talking Doorbell & \$150 \\
Talking Barcode Scanner & \$200 \\
AI Reading Assistant (Mobile App) & \$10/month \\
Smart Talking Smoke Detector & \$75 \\
Talking Currency Reader & \$180 \\
\end{longtblr}



\subsection{Emerging Technologies and Future Directions}\label{emerging-tech}
The landscape of accessible technology continues to evolve rapidly, with several emerging technologies showing significant promise for enhancing independence and quality of life for visually impaired individuals:

\begin{itemize}
 \item \emph{AI-Powered Wearables}: Advanced smart glasses and wearable devices that provide real-time environmental description, object recognition, and contextual assistance through artificial intelligence.
 \item \emph{Indoor Navigation Systems}: Sophisticated applications that enable safe navigation within buildings where GPS signals are unavailable, using smartphone sensors and specialized mapping data.
 \item \emph{Haptic Feedback Devices}: Advanced tactile feedback systems that provide navigation guidance and environmental information through touch and vibration.
 \item \emph{Voice-Activated Smart Home Integration}: Comprehensive home automation systems that respond to voice commands and provide audio feedback for all household functions.
 \item \emph{Machine Learning Personalization}: Adaptive technologies that learn user preferences and behaviors to provide increasingly personalized assistance over time.
\end{itemize}

These emerging technologies represent the next frontier in accessible technology, offering unprecedented levels of independence and integration with everyday life. As these technologies continue to develop and become more affordable, they will play an increasingly important role in empowering visually impaired individuals to achieve their full potential in education, employment, and daily living.
\begin{thebibliography}{99}
\bibitem{InclusiveCityMaker2023} Inclusive City Maker. (n.d.). Pedestrian safety: Are your crossings safe for the visually impaired? Retrieved December 19, 2023, from \url{http://www.inclusivecitymaker.com/pedestrian-safety-visually-impaired-blind-people/}
\bibitem{AFBGPS2023} American Foundation for the Blind. (n.d.). Smartphone GPS navigation. Retrieved December 19, 2023, from \url{http://www.afb.org/blindness-and-low-vision/using-technology/smartphone-gps-navigation-people-visual-impairments}
\bibitem{AllAboutVision2023} All About Vision. (n.d.). Adapting your home for better blindness accessibility. Retrieved December 19, 2023, from \url{http://www.allaboutvision.com/resources/adapting-the-home-better-blindness-accessibility/}
\bibitem{MarketResearch2025} Prices updated from current market research and major vendors of products intended to facilitate independent living skills (2025).
\end{thebibliography}
