\hypertarget{ios-devices}{}\chapter[\raggedright Transformative Tablets:\hfill\break 
Pioneering Success for Visually Impaired Students Through\hfill\break Innovative Apps]{Transformative Tablets: Pioneering Success for Visually Impaired Students Through Innovative Apps}\label{ios-devices}\extramarks{Vision Department Technology Needs}{Chapter 2: Transformative Tablets}
\noindent\makebox[\linewidth]{\rule{\linewidth}{0.4pt}}
{\let\clearpage\relax\localtableofcontents\let\clearpage\relax\locallistoftables}\newpage

In an era where technology shapes the landscape of education, tablets have emerged as transformative tools, providing visually impaired students with unprecedented access to knowledge and fostering independence in their academic journeys\footnote{\raggedright I am omitting iPhone and Android phones from this document as the purchase of student phones is beyond the purview of a school district. However, iOS apps are provided as many of these are available on both Tablets and Phones and training students to use the technology on their personal device is often necessary, particularly within the auspice of Orientation \& Mobility instruction}. Within the realm of tablets, both iPad and Android devices stand as beacons of innovation, offering not only user-friendly interfaces but also a diverse array of applications specifically tailored to bridge the accessibility gap. This chapter embarks on a compelling exploration of how tablets, in tandem with purpose-built apps, are not just tools but catalysts for success in the educational odyssey of visually impaired students.

The tactile elegance of tablets goes beyond mere convenience; it represents a paradigm shift in the way students interact with information. For visually impaired learners, tablets serve as dynamic portals, offering a multi-sensory approach to engagement. Through the lens of this chapter, we will unravel the unique functionalities of both iPad and Android tablets, delving into their respective strengths and contributions to an inclusive educational experience.

Apps, the lifeblood of these devices, play an instrumental role in transforming tablets into personalized learning companions. From screen readers that convert text to speech with remarkable precision to magnification apps that enhance visual content, the ecosystem of applications available empowers visually impaired students to navigate the digital realm with confidence. Tablets, coupled with innovative apps, are not mere gadgets; they represent a dynamic force propelling visually impaired students toward success; underscoring the indispensable role these tools play in shaping an educational landscape where every student, regardless of visual abilities, can seize the opportunities that lie ahead.

\pagebreak\hypertarget{tablet-considerations}{}\section{Tablet Considerations}\label{tab:tablelet-considerations}

When selecting a tablet for students with visual impairments to access their schoolwork, careful consideration must be given to the device's accessibility features to ensure an inclusive and conducive learning environment. Essential considerations include the tablet's compatibility with screen readers and magnification tools, ensuring that these assistive technologies seamlessly integrate with the device's operating system. Additionally, evaluating the availability and effectiveness of built-in accessibility features such as VoiceOver for iOS or TalkBack for Android is crucial\footnote{\raggedright Traditionally, assistive technology for the blind has focused on the iPad line since the Android line had historically lagged behind the Apple products for accessibility features prior to 2020 so accessibility apps have favored the iOS/iPadOS architecture. However, groups are emerging that specifically instruct users of Android devices how to access and use accessibility settings \textit{cf.}, \href{http://www.youtube.com/channel/UCvEM-SmpwElNALldhp8hG1g}{Blind Android Users}}. The tablet's tactile features, size, and weight should also be taken into account to accommodate students' specific needs. High contrast and customizable color settings, as well as text-to-speech functionalities, are vital components that enhance readability. Furthermore, the tablet's compatibility with a variety of educational apps designed with accessibility in mind is paramount. By prioritizing these considerations, educators and administrators can empower students with visual impairments to engage with their schoolwork independently and efficiently, fostering a more inclusive learning experience.

One frequently overlooked challenge in using tablets for individuals with visual impairments is the potential for visual fatigue. Recent research, such as the study by Pakdee and Sengsoon (2021)\footnote{\raggedright \textit{cf}., 
 \href{http://www.researchgate.net/publication/352764109\_Immediate\_Effects\_of\_Different\_Screen\_Sizes\_on\_Visual\_Fatigue\_in\_Video\_Display\_Terminal\_Users}{Pakdee, S., \& Sengsoon, P. (2021). Immediate Effects of Different Screen Sizes on Visual Fatigue in Video Display Terminal Users. \textit{Iranian Rehabilitation Journal, 19(2)}, 137-1461. DOI:10.32598/irj.19.2.1108.2}}, reveals that opting for a slightly larger device can mitigate visual fatigue, particularly for those engaged in visually demanding tasks. This consideration becomes even more pertinent for individuals with visual impairments.

While the iPad Pro2 and Samsung Galaxy Tab 9 tablets are often lauded for their increased brightness, it's crucial not to prioritize brightness as a major factor. Research suggests that boosting brightness can exacerbate visual fatigue. Instead, emphasis should be placed on the larger screen's enhanced resolution and expanded visual area, facilitating efficient use of Zoom functions. This becomes especially significant when aiming to teach students to adeptly navigate assistive technology without relying solely on pinch zooming, a feature that may not consistently function within all applications.

Another critical consideration involves contrast ratios. For students with photophobia, adjusting luminance levels to lower settings can significantly enhance clarity of text and images. This nuanced approach to tablet selection is pivotal in creating an accessible and comfortable learning environment for students with visual impairments.

For individuals with visual impairments, the efficacy of these devices relies heavily on factors that go beyond mere functionality. One crucial aspect that significantly impacts the accessibility of tablets for visually impaired students is the contrast ratio. The contrast ratio, representing the difference in luminance between the brightest and darkest elements on a screen, plays a pivotal role in ensuring that individuals with visual impairments can effectively engage with educational content. In a school setting, where tablets are increasingly utilized for various learning activities, understanding and prioritizing contrast ratio becomes paramount in fostering an inclusive and enriching educational environment for all students, regardless of their visual abilities.


\pagebreak \hypertarget{tablet-options}{}\section{Tablet Options}\label{tab:tablet-options}
When choosing an Android Tablet or iPad for a student with visual impairments, several factors must be considered to ensure that the student receives free and appropriate public education. The first factor to consider is the screen contrast ratio. A high contrast ratio is essential for students with visual impairments as it makes it easier for them to read text and view images on the screen. For Android Tablets, the W3C recommends a contrast ratio of at least 4.5:1 for small text and 3.0:1 for large text\footnote{\raggedright \href{https://support.google.com/accessibility/android/answer/7158390?hl=en}{Google. (n.d.). Color contrast - Android Accessibility Help. Retrieved December 19, 2023}}. On the other hand, Apple devices have an “Increase Contrast” feature that can be turned on to make text and other elements more visible\footnote{\raggedright \href{https://www.imore.com/how-increase-contrast-visual-accessibility-iphone-and-ipad}{iMore. (n.d.). How to increase contrast for visual accessibility on iPhone and iPad. Retrieved December 19, 2023}}.

The second factor to consider is the size of the screen. A larger screen is beneficial for students with visual impairments as it allows them to view text and images more clearly. Tablets usually have larger screens than smartphones, making them a better choice for students with visual impairments\footnote{\raggedright \href{https://www.afb.org/blindness-and-low-vision/using-technology/cell-phones-tablets-mobile/smartphone-or-tablet-which}{American Foundation for the Blind. (n.d.). Smartphone or Tablet: Which is Best for You? Retrieved December 19, 2023}}. However, it is important to note that larger screens come at the expense of portability. Therefore, it is essential to find a balance between screen size and portability.

The third factor to consider is the availability of accessible apps. Both Android and iOS devices have built-in accessibility features such as screen readers, magnifiers, and high contrast modes \footnote{\raggedright \href{https://www.aao.org/eye-health/tips-prevention/low-vision-impairment-apps-tech-assistive-devices }{American Academy of Ophthalmology. (n.d.). 30 Apps, Devices and Technologies for People With Vision Impairments. Retrieved December 19, 2023}}\fnsep\footnote{\raggedright \href{https://www.afb.org/blindness-and-low-vision/using-technology/cell-phones-tablets-mobile/apple-ios-iphone-and-ipad }{American Foundation for the Blind. (n.d.). Apple iOS for iPhone and iPad: Considerations for Users with Visual Impairments. Retrieved December 19, 2023}}. Additionally, there are several apps available that are specifically designed for students with visual impairments. For example, the “Lookout” app for Android provides spoken feedback about things around you, while the “Be My Eyes” app connects visually-impaired people with sighted volunteers through a live video call\footnote{\raggedright \href{https://www.aao.org/eye-health/tips-prevention/technology-apps-devices-children-blind-low-vision}{American Academy of Ophthalmology. (n.d.). Technology Tools for Children with Low Vision. Retrieved December 19, 2023}}. It is important to ensure that the device has access to these apps to ensure that the student can receive free and appropriate public education.
\textit{Table \ref{tab:table9}} describes current tablet computers that are available for students with visual impairments.

\pagebreak 
 
\begin{longtable}[]{@{}
 >{\raggedright\arraybackslash}m{.35\textwidth}
 >{\raggedright\arraybackslash}m{.35\textwidth}
 >{\raggedright\arraybackslash}m{.35\textwidth}@{}
 }
 \toprule
 
 \textbf{Tablet} & \textbf{Cost} & \textbf{Screen Size} \\
 \midrule
 \endhead \hline \\
 \multicolumn{3}{r}{\textbf{Continued on Next Page}} \endfoot
 \endlastfoot
 \multicolumn{3}{l}{\textbf{AndroidOS 13+ Tablets}}\\ \cdashline{1-3}
 Acer Iconia Tab P10 & \$199 & 10.4 \\ \cdashline{1-3}
 Alldocube iPlay 50 mini Pro NFE & \$209 & 8.4 \\ \cdashline{1-3}
 Alldocube iPlay 60 & \$319 & 10.95 \\ \cdashline{1-3}
 Blackview OSCAL Pad 16 & \$150 & 10.5 \\ \cdashline{1-3}
 Doogee U10 Pro & \$139 & 10.1 \\ \cdashline{1-3}
 Galaxy Tab A9 & \$179 & 8.7 \\ \cdashline{1-3}
 Galaxy Tab A9+ & \$239 & 11 \\ \cdashline{1-3}
 Google Pixel Tablet & \$499 & 10.95 \\ \cdashline{1-3}
 Honor Tablet Pad 9 & 223 & 12.1 \\ \cdashline{1-3}
 Hyundai HyTab 7 & \$59 & 7 \\ \cdashline{1-3}
 Lenovo Tab Extreme & \$949 & 14.5 \\ \cdashline{1-3}
 Lenovo Tab M10 5G & \$441 & 10.61 \\ \cdashline{1-3}
 Lenovo Tab M10 Plus (3rd Gen) & \$149 & 10.6 \\ \cdashline{1-3}
 Lenovo Tab M7 (3rd gen.) & \$99 & 7 \\ \cdashline{1-3}
 Lenovo Tab M8 (4th Gen) & \$109 & 8 \\ \cdashline{1-3}
 Lenovo Tab M9 & \$149 & 9 \\ \cdashline{1-3}
 Lenovo Tab P11 (2nd gen) & \$269 & 11.5 \\ \cdashline{1-3}
 Lenovo Tab P11 Plus & \$209 & 11 \\ \cdashline{1-3}
 Lenovo Tab P11 Pro (2nd Gen) & \$339 & 11.2 \\ \cdashline{1-3}
 Lenovo Tab P12 & \$379 & 12.7 \\ \cdashline{1-3}
 Lenovo Tab P12 Pro & \$629 & 12.6 \\ \cdashline{1-3}
 Lenovo Xiaoxin Pad 2024 & \$228 & 11 \\ \cdashline{1-3}
 Lenovo Yoga Tab 11 & \$269 & 11 \\ \cdashline{1-3}
 Lenovo Yoga Tab 13 & \$589 & 13 \\ \cdashline{1-3}
 Nokia T10 & \$169 & 8 \\ \cdashline{1-3}
 Nokia T20 & \$199 & 10.4 \\ \cdashline{1-3}
 Nokia T21 & \$279 & 10.36 \\ \cdashline{1-3}
 OnePlus Pad & \$479 & 11.61 \\ \cdashline{1-3}
 Oppo Pad Air 2 & \$259 & 11.35 \\ \cdashline{1-3}
 Oscal Pad 18 & \$319 & 11 \\ \cdashline{1-3}
 Razer Edge & \$399 & 6.8 \\ \cdashline{1-3}
 Redmi Pad & \$219 & 10.61 \\ \cdashline{1-3}
 Redmi Pad SE & \$225 & 11 \\ \cdashline{1-3}
 Samsung Galaxy Tab A7 Lite & \$119 & 8.7 \\ \cdashline{1-3}
 Samsung Galaxy Tab A8 & \$179 & 10.5 \\ \cdashline{1-3}
 Samsung Galaxy Tab Active3 & \$439 & 8 \\ \cdashline{1-3}
 Samsung Galaxy Tab S7 FE & \$529 & 12.4 \\ \cdashline{1-3}
 Samsung Galaxy Tab S9 & \$799 & 11 \\ \cdashline{1-3}
 Samsung Galaxy Tab S9 FE & \$449 & 10.9 \\ \cdashline{1-3}
 Samsung Galaxy Tab S9 FE+ & \$599 & 12.4 \\ \cdashline{1-3}
 Samsung Galaxy Tab S9 Ultra & \$1199 & 14.6 \\ \cdashline{1-3}
 Samsung Galaxy Tab S9+ & \$999 & 12.4 \\ \cdashline{1-3}
 Teclast P30T & \$176 & 10.1 \\ \cdashline{1-3}
 Teclast T60 & \$225 & 12 \\ \cdashline{1-3}
 UMIDIGI A15 Tab & \$999 & 11 \\ \cdashline{1-3}
 Vivo Pad 2 & \$349 & 12.1 \\ \cdashline{1-3}
 Xiaomi Pad 6 & \$399 & 11 \\ \cdashline{1-3}
 Xiaomi Pad 6 Pro & \$589 & 11 \\ \cdashline{1-3}
 ZTE Nubia Pad 3D & \$1099 & 12.4 \\ \cdashline{1-3}
 \multicolumn{3}{l}{\textbf{iPadOS Tablets}}\\ \cdashline{1-3}
 Apple iPad 10.2 & \$269 & 10.2 \\ \cdashline{1-3}
 Apple iPad 10.9 & \$449 & 10.9 \\ \cdashline{1-3}
 Apple iPad Air 5 & \$599 & 10.9 \\ \cdashline{1-3}
 Apple iPad Pro 11 & \$799 & 11 \\ \cdashline{1-3}
 Apple iPad Pro 12.9 & \$1099 & 12.9 \\ \cdashline{1-3}
 Apple iPad mini 6 & \$499 & 8.3 \\ \cdashline{1-3}
 \multicolumn{3}{l}{\textbf{Window OS Tablets}}\\ \cdashline{1-3}
 Alldocube & \$669 & 12.6 \\ \cdashline{1-3}
 Asus ROG Flow Z13 & \$1799 & 13.4 \\ \cdashline{1-3}
 Asus ROG Flow Z13 (2023) & \$1799 & 13.4 \\ \cdashline{1-3}
 Asus Vivobook 13 Slate OLED & \$749 & 13.3 \\ \cdashline{1-3}
 Dell XPS 13 & \$1099 & 13 \\ \cdashline{1-3}
 Huawei MateBook E & \$1419 & 12.6 \\ \cdashline{1-3}
 Lenovo Yoga Duet & \$970 & 13 \\ \cdashline{1-3}
 Microsoft Surface Book 3 13.5 & \$799 & 13.5 \\ \cdashline{1-3}
 Microsoft Surface Book 3 15 & \$1159 & 15 \\ \cdashline{1-3}
 Microsoft Surface Go 3 & \$399 & 10.5 \\ \cdashline{1-3}
 Microsoft Surface Go 3 & \$629 & 10.5 \\ \cdashline{1-3}
 Microsoft Surface Go 4 & \$579 & 10.5 \\ \cdashline{1-3}
 Microsoft Surface Pro 9 & \$979 & 13 \\ \cdashline{1-3}
 \multicolumn{3}{l}{\textbf{ChromeOS Tablets}}\\ \cdashline{1-3}
 Acer Chromebook Tab 10 & \$299 & 9.7 \\ \cdashline{1-3}
 Asus Chromebook CM3 & \$369 & 10.5 \\ \cdashline{1-3}
 Asus Chromebook Tablet CT100 & \$299 & 9.7 \\ \cdashline{1-3}
 Fydetab Duo & \$630 & 12.35 \\ \cdashline{1-3}
 Google Pixel Slate & \$499 & 12.3 \\ \cdashline{1-3}
 HP Chromebook x2 & \$799 & 12.3 \\ \cdashline{1-3}
 HP Chromebook x2 11 & \$599 & 11 \\ \cdashline{1-3}
 Lenovo 10e Chromebook Tablet & \$259 & 10.1 \\ \cdashline{1-3}
 Lenovo Chromebook Duet & \$229 & 10.1 \\ \cdashline{1-3}
 Lenovo Chromebook Duet 3 & \$349 & 11 \\ \cdashline{1-3}
 Lenovo Chromebook Duet 5 & \$379 & 13.3 \\[1.0em]\hline
 \caption[Tablet Options]{Tablet Options Organized by OS}\label{tab:table9}
\end{longtable}\clearpage

\pagebreak \hypertarget{tablet-recommend}{}\section{Tablet Recommendations}\label{tab:tablet-recommend}
\textit{Table \ref{tab:table91}} describes current tablet computers that are available for students with visual impairments.

\pagebreak 
 
\begin{longtable}[]{@{}
 >{\raggedright\arraybackslash}m{.35\textwidth}
 >{\raggedright\arraybackslash}m{.35\textwidth}
 >{\raggedright\arraybackslash}m{.35\textwidth}@{}
 }
 \toprule
 
 \textbf{Tablet} & \textbf{Cost} & \textbf{Screen Size} \\
 \midrule
 \endhead \hline \\
 \multicolumn{3}{r}{\textbf{Continued on Next Page}} \endfoot
 \endlastfoot
 \multicolumn{3}{l}{\textbf{AndroidOS 13+ Tablets}}\\ \cdashline{1-3}
 Google Pixel Tablet & \$499 & 10.95 \\ \cdashline{1-3}
 Lenovo Tab P12 Pro & \$629 & 12.6 \\ \cdashline{1-3}
 Samsung Galaxy Tab S9 & \$799 & 11 \\ \cdashline{1-3}
 \rowcolor{red!10} Samsung Galaxy Tab S9 Ultra & \$1199 & 14.6 \\ \cdashline{1-3}
 \rowcolor{red!10} Samsung Galaxy Tab S9+ & \$999 & 12.4 \\ \cdashline{1-3}
 \multicolumn{3}{l}{\textbf{iPadOS Tablets}}\\ \cdashline{1-3}
 Apple iPad 10.9 & \$449 & 10.9 \\ \cdashline{1-3}
 \rowcolor{red!10} Apple iPad Air 5 & \$599 & 10.9 \\ \cdashline{1-3}
 Apple iPad Pro 11 & \$799 & 11 \\ \cdashline{1-3}
 \rowcolor{red!10} Apple iPad Pro 12.9 & \$1099 & 12.9 \\ \cdashline{1-3}
 Apple iPad mini 6 & \$499 & 8.3 \\ \cdashline{1-3}
 \multicolumn{3}{l}{\textbf{Windows OS Tablets}}\\ \cdashline{1-3}
 Microsoft Surface Go 3 & \$399 & 10.5 \\ \cdashline{1-3}
 Microsoft Surface Go 3 & \$629 & 10.5 \\ \cdashline{1-3}
 \rowcolor{red!10} Microsoft Surface Go 4 & \$579 & 10.5 \\ \cdashline{1-3}
 \rowcolor{red!10} Microsoft Surface Pro 9 & \$979 & 13 \\ \cdashline{1-3}
 \multicolumn{3}{l}{\textbf{ChromeOS Tablets}}\\ \cdashline{1-3}
 \rowcolor{red!10} Google Pixel Slate & \$499 & 12.3 \\ \cdashline{1-3}
 Lenovo IdeaPad Duet 3 & \$349 & 11 \\ \cdashline{1-3}
 Lenovo IdeaPad Duet 5 & \$379 & 13.3 \\[1.0em]\hline
 \caption[Tablet Recommendations]{Tablet Recommendations Organized by OS. Preferred option is highlighted in light red. }\label{tab:table91}
\end{longtable}\clearpage


\pagebreak
\hypertarget{tablet-apps}{}\section{Mobile Applications}\label{tab:tablelet-apps}
Mobile apps run on tablets are becoming increasingly important for students with visual impairments to access a free and appropriate public education. These apps can provide students with access to digital content, assistive technology, and other tools that can help them succeed in their studies. High-quality mobile apps can help students with visual impairments access the same educational materials as their sighted peers and participate fully in the curriculum. They can also help improve literacy skills, comprehension, and productivity. In this section, we will explore the importance of high-quality mobile apps for students with visual impairments and discuss some of the best apps available on the market today. \textit{Table \ref{tab:table10}} gives a list of current apps available for use with students with visual impairments.

\pagebreak 
 
\begin{longtable}[]{@{}
 >{\raggedright\arraybackslash}m{.3\textwidth}
 >{\raggedright\arraybackslash}m{.1\textwidth}
 >{\raggedright\arraybackslash}m{.35\textwidth}@{}
 >{\raggedright\arraybackslash}b{.25\textwidth}@{}
 }
 \toprule
 \textbf{App} & \textbf{Cost} & \textbf{Function} & \textbf{OS} \\
 \midrule
 \endhead \hline \\
 \multicolumn{4}{r}{\textbf{Continued on Next Page}} 
 \endfoot
 \endlastfoot
 \multicolumn{4}{l}{\textbf{Accessibility Training/Auditory Games}} \\ \cdashline{1-4} 
CosmoBally in Space & free & Train VoiceOver Gestures & iOS/iPadOS \\ \cdashline{1-4}
Ballyland Magic Plus & \$3.99 & Train VoiceOver Gestures & iOS/iPadOS \\ \cdashline{1-4}
Ballyland Rotor & \$2.99 & Train VoiceOver rotor & iOS/iPadOS \\ \cdashline{1-4}
Ballyland Stay Still Squeaky! & \$2.99 & Train VoiceOver Gestures & iOS/iPadOS \\ \cdashline{1-4}
Blindfold Games Launcher & free\footnote{\raggedright Games purchased separately} & Sonic Games & iOS/iPadOS \\ \cdashline{1-4}
Blindfold Tap and Swipe & free & Train VoiceOver Gestures & iOS/iPadOS \\ \cdashline{1-4}
ObjectiveEd Games & free\footnotemark[\value{footnote}] & Sonic Games & iOS/iPadOS \\ \cdashline{1-4}
VO Lab & \$4.99 & Train VoiceOver Gestures & iOS/iPadOS \\ \cdashline{1-4} 
Screenreader & free & Train Accessibility Gestures & iOS/iPadOS\break Android 13+ \\ \cdashline{1-4} 
 \multicolumn{4}{l}{\textbf{Cortical Vision Impairment}} \\ \cdashline{1-4}
Art of Glow & free & CVI-based Vision Training & iOS/iPadOS \\ \cdashline{1-4}
Big Band Patterns & \$34.99 & CVI-based Vision Training & iOS/iPadOS \\ \cdashline{1-4}
Big Bang Pictures & \$34.99 & CVI-based Vision Training & iOS/iPadOS \\ \cdashline{1-4}
CVI Connect & \$10/mo & CVI-based Vision Training & iOS/iPadOS \\ \cdashline{1-4}
CVI Connect Pro & free\footnote{\raggedright Annual Price Per Enrolled Student: 1-5=\$300\quad6-10=\$250\quad11-15=\$200\quad16-19=\$150} & CVI-based Vision Training & iOS/iPadOS \\ \cdashline{1-4}
CVI Toddler Visual Eye Train & free & CVI-based Vision Training & iOS/iPadOS \\ \cdashline{1-4}
CVI Training (Color) & free & CVI-based Vision Training & iOS/iPadOS \\ \cdashline{1-4}
CVI Training (Human face) & free & CVI-based Vision Training & iOS/iPadOS \\ \cdashline{1-4}
CVI Training (Pattern) & free & CVI-based Vision Training & iOS/iPadOS \\ \cdashline{1-4}
CVI Training (Recognition) & free & CVI-based Vision Training & iOS/iPadOS \\ \cdashline{1-4}
CVI Training (Visual Tracking) & free & CVI-based Vision Training & iOS/iPadOS \\ \cdashline{1-4}
Dexteria VMI & \$5.99 & CVI-based Vision Training & iOS/iPadOS \\ \cdashline{1-4}
Doodle Kids & free & CVI-based Vision Training & iOS/iPadOS \\ \cdashline{1-4}
EDA Play & \$4.99 & CVI-based Vision Training & iOS/iPadOS \\ \cdashline{1-4}
EDA Play ELIS & \$2.99 & CVI-based Vision Training & iOS/iPadOS \\ \cdashline{1-4}
EDA Play PAULI & \$2.99 & CVI-based Vision Training & iOS/iPadOS \\ \cdashline{1-4}
EDA Play TOBY & free & CVI-based Vision Training & iOS/iPadOS \\ \cdashline{1-4}
EDA Play TOM & free & CVI-based Vision Training & iOS/iPadOS \\ \cdashline{1-4}
EyeMove & free & CVI-based Vision Training & iOS/iPadOS \\ \cdashline{1-4}
Fludity HD & free & CVI-based Vision Training & iOS/iPadOS \\ \cdashline{1-4}
Little Bear Sees & \$4.99 & CVI-based Vision Training & iOS/iPadOS \\ \cdashline{1-4}
P.O.V. Spatial Reasoning & \$3.99 & CVI-based Vision Training & iOS/iPadOS \\ \cdashline{1-4}
Peekaboo Barn & \$2.99 & CVI-based Vision Training & iOS/iPadOS \\ \cdashline{1-4}
Sensory Electra & free & CVI-based Vision Training & iOS/iPadOS \\ \cdashline{1-4}
Sensory Light Box & \$3.99 & CVI-based Vision Training & iOS/iPadOS \\ \cdashline{1-4}
Tap-n-See Now & \$2.99 & CVI-based Vision Training & iOS/iPadOS \\ \cdashline{1-4}
VO Lab & free & CVI-based Vision Training & iOS/iPadOS \\ \cdashline{1-4}
Visual Attention Therapy Lite & free & CVI-based Vision Training & iOS/iPadOS \\ \cdashline{1-4}
 \multicolumn{4}{l}{\textbf{Audiobook/Reading}} \\ \cdashline{1-4}
Audible & free\footnote{\raggedright requires books to be purchased} & Audiobook & iOS/iPadOS\break AndroidOS 13+ \\ \cdashline{1-4}
BARD Mobile & free\footnote{\raggedright requires account with local affiliate State Library for the Blind} & e-Book & iOS/iPadOS\break AndroidOS 13+ \\ \cdashline{1-4}
Bookshare Reader & free & DAISY Reader & iOS/iPadOS \\ \cdashline{1-4}
Dolphin EasyReader & free & DAISY Reader & iOS/iPadOS\break AndroidOS 13+ \\ \cdashline{1-4}
KNFB Reader\break(rebranded OneStepReader) & \$99.99 & OCR/Reading & iOS/iPadOS\break AndroidOS 13+ \\ \cdashline{1-4}
Kindle & free\footnotemark[12] & e-Book & iOS/iPadOS\break AndroidOS 13+ \\ \cdashline{1-4}
Libby & free\footnotemark[13] & Audiobook & iOS/iPadOS\break AndroidOS 13+ \\ \cdashline{1-4}
VoiceDream Reader & free\footnote{\raggedright requires \$79.99/yr subscription and additional \$4.99 for each premium voice} & DAISY Reader & iOS/iPadOS \\ \cdashline{1-4}
 \multicolumn{4}{l}{\textbf{Productivity/Schoolwork/Optical Character Recognition}} \\ \cdashline{1-4}
Aiko & free & AI Speech to text & iOS/iPadOS \\ \cdashline{1-4}
Ballyland Code 1: Say Hello & \$2.99 & Auditory Coding & iOS/iPadOS \\ \cdashline{1-4}
Ballyland Code 2: Give Rotor & \$2.99 & Auditory Codings & iOS/iPadOS \\ \cdashline{1-4}
Ballyland Code 3: Pick Up & \$2.99 & Auditory Coding & iOS/iPadOS \\ \cdashline{1-4}
Clusiv & free\footnote{\raggedright Full access provided through Vocational Rehabilitation} & Online learning platform\footnote{\raggedright Clusiv is an online learning platform for the blind and visually impaired that teaches occupational training, technology skills, and educational courses to empower employment} & iOS/iPadOS \\ \cdashline{1-4}
Clusive & free & Online learning platform\footnote{\raggedright Clusive is an open-source online learning platform for the blind and visually impaired} & iOS/iPadOS \\ \cdashline{1-4}
Code Quest & free & Auditory Coding & iOS/iPadOS \\ \cdashline{1-4}
Desmos Graphing Calculator & free & Accessible Graphing & iOS/iPadOS \break AndroidOS 13+ \\ \cdashline{1-4}
Desmos Scientific Calculator & free & Accessible Scientific Calculator & iOS/iPadOS \break AndroidOS 13+ \\ \cdashline{1-4}
Envision AI & free & OCR & iOS/iPadOS\break AndroidOS 13+ \\ \cdashline{1-4}
GoodNotes & free\footnote{\raggedright with in app purchases} & Scan \& Markup Documents & iOS/iPadOS \break AndroidOS 13+ \\ \cdashline{1-4}
KNFB Reader\break(rebranded OneStepReader) & \$99.99 & OCR/Reading & iOS/iPadOS\break AndroidOS 13+ \\ \cdashline{1-4}
My Board Buddy & free & local view of class blackboard & iOS/iPadOS \\ \cdashline{1-4}
Notability & free\footnotemark[16] & Scan \& Markup Documents & iOS/iPadOS \break AndroidOS 13+ \\ \cdashline{1-4}
QuickScanner & free\footnote{\raggedright a paid subscription removes ads and allows saving of documents} & OCR & iOS/iPadOS\break AndroidOS 13+ \\ \cdashline{1-4}
SeeingAI & free & Talking Camera & iOS/iPadOS\break AndroidOS 13+ \\ \cdashline{1-4}
TapTapSee & free & Talking Camera & iOS/iPadOS\break AndroidOS 13+ \\ \cdashline{1-4}
Voice Aloud Reader & free\footnote{\raggedright Premium Voices=\$9.99/mo, Pro Version=\$8.99/mo, OCR 800=\$1.99/mo, OCR 100=\$0.99/mo} & OCR/Reading & iOS/iPadOS\break Android 13+ \\ \cdashline{1-4}
\multicolumn{4}{l}{\textbf{Orientation \& Mobility / Navigation}} \\ \cdashline{1-4}
Apple Maps & free& Turn by Turn Navigation& iOS/iPadOS \\ \cdashline{1-4}
BlindBat & free& Echolocation for the blind\footnote{\raggedright Echolocation requires specific training} & iOS/iPadOS \\ \cdashline{1-4}
BlindSquare & \$39.99 & GPS Navigation & iOS/iPadOS \\ \cdashline{1-4}
Clew & free & Indoor navigation\footnote{\raggedright Augmented Reality is used to identify turns and stairways} & iOS/iPadOS\break AndroidOS 13+ \\ \cdashline{1-4}
Eyedar & free & Echolocation & iOS \\ \cdashline{1-4}
GoodMaps Explore & free & Turn by Turn Navigation\break Indoor navigation & iOS/iPadOS \\ \cdashline{1-4}
GoodMaps Outdoors & free & Turn by Turn Navigation & iOS/iPadOS \\ \cdashline{1-4}
Google Maps & free & Turn by Turn Navigation & iOS/iPadOS\break AndroidOS 13+ \\ \cdashline{1-4}
HapticNav & free\footnotemark[16] & Haptic GPS navigation & iOS/iPadOS\break AndroidOS 13+ \\ \cdashline{1-4}
Lazarillo & free & GPS navigation & iOS/iPadOS \\ \cdashline{1-4}
Moovit & free\footnotemark[16] & Local Public Transit & iOS/iPadOS\break AndroidOS 13+ \\ \cdashline{1-4}
Musical Cane Game & free & White Cane Training & iPadOS \\ \cdashline{1-4}
Oko & free & Smart Camera\break Traffic lights/traffic & iOS/iPadOS\break AndroidOS 13+ \\ \cdashline{1-4}
Seeing Eye GPS & free\footnote{\raggedright Monthly subscription \$6} & Turn by Turn Navigation & iOS/iPadOS\break AndroidOS 13+ \\ \cdashline{1-4}
VoiceVista & free & Auditory Identification of Surroundings & iOS \\ \cdashline{1-4}
Waymap & free & Turn by Turn Navigation\break Indoor navigation & iOS/iPadOS\break AndroidOS 13+ \\ \cdashline{1-4}
WeWalk & free\footnotemark[16] & GPS Navigation & iOS/iPadOS \\ \cdashline{1-4}
XploreNinja & \$39.99 & GPS Navigation & AndroidOS 13+ \\ \cdashline{1-4} 
 \multicolumn{4}{l}{\textbf{Independent Living Skills}} \\\cdashline{1-4}
CashReader & free\footnote{\raggedright requires a subscription to remove limited scans/day, \$2/mo, \$12/yr, \&30/lifetime} & Scan and Identify Paper money & iOS/iPadOS\break Android13 + \\ \cdashline{1-4}
Menus4All & free\footnote{\raggedright requires a subscription, \$3/mo or \$30/yr} & Accessible Restaurant Menus & iOS/iPadOS \\ \cdashline{1-4}
Zuzanka & free\footnote{\raggedright requires a subscription to remove limited scans/day, \$5/mo, \$35/yr, \&80/lifetime} & Expiration Date\break Barcode Scanner & iOS/iPadOS \\[1.0em]\hline
 \caption[Mobile/Tablet Apps]{Mobile/Tablet Apps}\label{tab:table10}
\end{longtable}\clearpage