\chapter{Refreshable Braille Displays}\label{ch3:braille}
\raggedright

\begin{raggedright}
	\textbf{Accessibility\index{accessibility} Note:} This chapter provides a comprehensive overview of refreshable Braille displays\index{braille display}, notetakers\index{notetaker}, and educational devices for students with visual impairments\index{visual impairment}. The content has been structured for clarity, navigation, and accessibility\index{accessibility}, with semantic markup and descriptive context for all tables and lists\index{Markdown!lists}.
\end{raggedright}

\section{~~Braille Notetakers and Laptops}\label{ch3:sec:notetakers-laptops}
Braille notetakers\index{notetaker!braille notetaker} are specialized devices that combine a refreshable Braille display\index{braille display} with a Braille keyboard and note-taking\index{Markdown!note-taking} software. They are essential tools for literacy, allowing students to read and write in Braille\index{braille} efficiently \supercite{Holbrook2006, Presley2012, PerkinsNoteTaking, TeachingVI}. When paired with a laptop, a Braille notetaker can serve as a powerful Braille terminal for accessing a wider range of applications and online content \supercite{Kelly2011, Day2021}.

\section{~~Braille Notetaker/Laptop Recommendations}\label{ch3:sec:notetaker-laptop-recs}
The choice of a Braille notetaker and laptop combination depends on the student's individual needs, the educational context, and budget considerations. The following table provides a comparison of leading Braille notetakers and their suitability for use with different laptop\index{laptop} operating systems.

\subsubsection{Table \ref{ch3:tab:braille-notetaker-laptop-recommendations}}
\begingroup
\fontsize{10pt}{12pt}\selectfont
\tagpdfsetup{table/header-rows={1}}
\begin{longtblr}[
		caption = {\gls{braille} \gls{notetaker}/\gls{laptop} Recommendations},
		label = {ch3:tab:braille-notetaker-laptop-recommendations},
		note = {This table provides a comparative overview of leading Braille notetakers and their compatibility with different laptop operating systems, highlighting key features relevant to students with visual impairments.}
	]{
		colspec = {X[l] X[l] X[l] X[l]},
		rowhead = 1,
		row{1} = {font=\normalfont},
		hlines,
	}
	\toprule
	Notetaker\index{notetaker} Model                    & Key Features                                                                                             & Laptop OS Compatibility                                      & Price Range       \\
	\midrule
	BrailleNote Touch+ \supercite{HumanWareBrailleNote} & Android\index{operating system!Android}-based, Google Play Store access, 32 Braille\index{braille} cells & Windows, macOS, ChromeOS\index{operating system!ChromeOS}    & \$5,500 - \$6,000 \\
	BrailleSense 6 \supercite{HIMSBrailleSense}         & Android 10, 32 Braille cells, advanced productivity apps\index{apps}                                     & Windows, macOS, ChromeOS                                     & \$5,000 - \$5,500 \\
	Orbit Reader 20 \supercite{OrbitReader20}           & Affordable, 20 Braille\index{braille} cells, standalone reader for SD cards                              & Windows\index{operating system!Windows}, macOS, iOS, Android & \$500 - \$600     \\
	\bottomrule
\end{longtblr}
\normalsize


\section{~~Refreshable Braille Displays}\label{ch3:sec:refreshable-braille}
Refreshable Braille displays\index{braille display} connect to computers and mobile devices to provide tactile output of screen content. They are essential for students who are Braille readers, enabling them to access digital text in real-time \supercite{Presley2012, Kamei-Hannan2012, PerkinsBrailleDisplay}. The choice of display size often depends on portability needs and the type of tasks the student will be performing.

\subsection{14-20 cell Refreshable Braille Displays}\label{ch3:ssec:14-20-cell}
Smaller Braille displays are highly portable and are ideal for use with smartphones and tablets\index{tablet}. They are excellent for reading on the go, sending text messages, and navigating mobile apps\index{apps} \supercite{Kamei-Hannan2012, BrailleMarketResearch}.

\begingroup
\fontsize{10pt}{12pt}\selectfont
\tagpdfsetup{table/header-rows={1}}
\begin{longtblr}[
		caption = {14-20 Cell Refreshable Braille Displays},
		label = {ch3:tab:14-20-cell-displays},
		note = {This table provides a selection of recommended 14-20 cell Braille displays, highlighting their key features relevant to students with visual impairments.}
	]{
		colspec = {X[l] X[l] X[l]},
		rowhead = 1,
		row{1} = {font=\normalfont},
		hlines,
	}
	\toprule
	Model                                                                 & Key Features                                                           & Price Range       \\
	\midrule
	Focus 14 Blue \supercite{FocusBlue}                                   & 14 Braille\index{braille} cells, 8-dot keyboard, Bluetooth             & \$1,300 - \$1,500 \\
	Orbit Reader 20 \supercite{OrbitReader20}                             & 20 Braille\index{braille} cells, standalone reader, affordable         & \$500 - \$600     \\
	APH\index{braille embosser!APH} Chameleon 20 \supercite{APHChameleon} & 20 Braille cells, 8-dot keyboard, notetaker\index{notetaker} functions & \$1,500 - \$1,700 \\
	\bottomrule
\end{longtblr}
\normalsize


\subsection{32-40 cell Refreshable Braille Displays}\label{ch3:ssec:32-40-cell}
Mid-size Braille displays\index{braille display} offer a good balance between portability and reading comfort. They provide a longer line of Braille, which can enhance reading fluency and reduce the need for frequent panning \supercite{Wall2003, Holbrook2006, Kamei-Hannan2012}. They are well-suited for use with laptops and for more intensive reading and writing tasks.

\begingroup
\fontsize{10pt}{12pt}\selectfont
\tagpdfsetup{table/header-rows={1}}
\begin{longtblr}[
		caption = {32-40 Cell Refreshable Braille Displays},
		label = {ch3:tab:32-40-cell-displays},
		note = {This table provides a selection of recommended 32-40 cell Braille displays, highlighting their key features relevant to students with visual impairments.}
	]{
		colspec = {X[l] X[l] X[l]},
		rowhead = 1,
		row{1} = {font=\normalfont},
		hlines,
	}
	\toprule
	Model                                         & Key Features                                                  & Price Range       \\
	\midrule
	Focus 40 Blue \supercite{FocusBlue}           & 40 Braille cells, 8-dot keyboard, \gls{bluetooth}             & \$2,800 - \$3,200 \\
	Brailliant BI 40X \supercite{BrailliantBI40X} & 40 Braille cells, notetaker\index{notetaker} functions, USB-C & \$3,000 - \$3,500 \\
	APH Mantis Q40 \supercite{APHMantis}          & 40 Braille cells, QWERTY keyboard, notetaker functions        & \$2,500 - \$2,800 \\
	\bottomrule
\end{longtblr}
\normalsize


\section{~~Multiple Line Braille Displays/Tablets}\label{ch3:sec:multi-line}
Multi-line Braille displays represent a significant advancement in Braille technology\index{technology}, allowing for the presentation of multiple lines of Braille\index{braille} at once. This can greatly enhance the reading experience, particularly for complex content like tables, math equations, and spatial information \supercite{Behrmann2012, Lueck2016, TactileSkillsDevelopment}.

\begingroup
\fontsize{10pt}{12pt}\selectfont
\tagpdfsetup{table/header-rows={1}}
\begin{longtblr}[
		caption = {Multiple Line Braille Displays/Tablets},
		label = {ch3:tab:multi-line-displays},
		note = {This table provides a selection of innovative multi-line Braille displays, highlighting their key features relevant to students with visual impairments.}
	]{
		colspec = {X[l] X[l] X[l]},
		rowhead = 1,
		row{1} = {font=\normalfont},
		hlines,
	}
	\toprule
	Model                                 & Key Features                                                                                                & Price Range       \\
	\midrule
	Canute 360 \supercite{Canute360}      & 9 lines of 40 cells, reads BRF files                                                                        & \$2,500 - \$3,000 \\
	Graphiti \supercite{OrbitGraphiti}    & 8 lines of 60 cells, tactile graphics\index{tactile graphics} display                                       & \$10,000+         \\
	Monarch \supercite{APHMonarch}        & 10 lines of 32 cells, \gls{tactile} graphics and text                                                       & \$5,000 - \$6,000 \\
	Dot Pad X \supercite{visionaiddotpad} & 10 lines of 32 cells with additional 20 cell braille line, dynamic tactile graphics\index{tactile graphics} & \$5,995           \\
	\bottomrule
\end{longtblr}
\normalsize


\section{~~Braille Education Devices}\label{ch3:sec:braille-ed-devices}
These devices are specifically designed to support early Braille literacy\index{braille literacy} and the development of Braille skills. They often incorporate gamified learning and interactive feedback to engage young learners \supercite{Lueck2016, Holbrook2006, ThinkerbellLabs}.

\begingroup
\fontsize{10pt}{12pt}\selectfont
\tagpdfsetup{table/header-rows={1}}
\begin{longtblr}[
		caption = {\gls{brailleeducation} Devices},
		label = {ch3:tab:braille-education-devices},
		note = {This table provides a selection of devices designed for Braille education, highlighting their key features relevant to students with visual impairments.}
	]{
		colspec = {X[l] X[l] X[l]},
		rowhead = 1,
		row{1} = {font=\normalfont},
		hlines,
	}
	\toprule
	Model                                                                          & Key Features                                         & Price Range             \\
	\midrule
	APH Code Jumper\index{braille education!Code Jumper} \supercite{APHCodeJumper} & Teaches coding concepts through physical pods        & \$500 - \$600           \\
	LEGO Braille Bricks \supercite{LEGOBricks}                                     & Teaches Braille alphabet and numbers through play    & Not for individual sale \\
	Taptilo \supercite{Taptilo}                                                    & Smart Braille learning device with interactive games & \$3,000 - \$3,500       \\
	\bottomrule
\end{longtblr}
\normalsize
