\chapter{Assistive Technology Considerations}\label{trouble3}

\noindent
\textbf{Accessibility Note:} This appendix provides an overview of best practices and frameworks for considering and assessing assistive technology (AT) for students with visual impairments. The structure and content have been enhanced for clarity, navigation, and accessibility.

Assistive technology is an essential component of ensuring that students with visual impairments receive a free and appropriate public education. However, it is important to use a valid assistive technology assessment before providing assistive technology to a student. A valid assessment can help identify the specific needs of the student and determine the most appropriate assistive technology solutions. This can help ensure that the student receives the right tools to succeed in their studies. Additionally, a valid assessment can help ensure that the student receives the appropriate accommodations and modifications to their educational program.

By using all the data available, educators can make informed decisions about the most appropriate assistive technology solutions for each student. This can help ensure that the student receives the right tools to succeed in their studies  \cite{AEMCenter}. It is important to note that convenience should not be a factor when making decisions about assistive technology. The focus should always be on what is best for the student.

It is also essential to use all the data available to guide decision making when providing assistive technology to a student. This includes data from the student, their family, and their educators. By using all the data available, educators can make informed decisions about the most appropriate assistive technology solutions for each student. This can help ensure that the student receives the right tools to succeed in their studies \cite{AEMCenter}. It is important to note that convenience should not be a factor when making decisions about assistive technology. The focus should always be on what is best for the student.

The landscape of assistive technology continues to evolve rapidly, with emerging technologies such as AI-powered devices showing significant promise. Recent developments include advanced wearable devices that combine artificial intelligence with real-time environmental feedback, offering unprecedented levels of independence for students with visual impairments. However, the fundamental principles of proper assessment and individualized selection remain constant regardless of technological advances.

Using a valid assistive technology assessment and all available data to guide decision making can help ensure that students with visual impairments receive the appropriate assistive technology solutions to succeed in their studies. This can help improve their academic performance and increase their chances of success in school. Additionally, it can help students with visual impairments become more independent in their daily lives. By providing students with the tools they need to access information and communicate with others, assistive technology can help them become more self-sufficient and less reliant on others\footnote{\raggedright \href{https://www.wati.org/free-publications/assistive-technology-consideration-to-assessment/}{Wisconsin Assistive Technology Initiative. (2010). Assistive technology consideration to assessment. Retrieved December 19, 2023}}.

Finally, it is important to note that the use of assistive technology is not a one-size-fits-all solution. The technology needs of students with visual impairments can vary widely depending on their individual needs and abilities. Therefore, it is important to work with students, families, and educators to identify the most appropriate assistive technology solutions for each student. By doing so, we can help ensure that students with visual impairments have the tools they need to succeed in school and beyond.

\section{SETT Framework}\label{trouble41}

\noindent
\textbf{Context:} The SETT Framework is a foundational model for evaluating and justifying assistive technology choices for students with disabilities, including those who are blind or visually impaired. The following description uses semantic markup for clarity and accessibility.

The \href{https://www.joyzabala.com/links-resources}{SETT Framework} \cite{ZabalaSETT, MNSETT} is a widely used method for evaluating assistive technology (AT) needs for students with disabilities. The acronym SETT stands for:

\begin{description}
  \item[Student] Characteristics, strengths, and needs of the individual learner.
  \item[Environment] The settings in which the student learns and interacts.
  \item[Task] The specific activities or tasks the student needs to accomplish.
  \item[Tools] The devices, services, and strategies that support the student in performing tasks.
\end{description}

The framework emphasizes the importance of understanding the student's characteristics, the environments in which they learn and grow, and the tasks required to be an active learner in those environments before identifying a system of tools that enables the student to actively engage in the tasks. A team-based collaborative assessment of needs will lead to determining the most promising system of tools for the student, with a consideration of the environments this learner is in \cite{Hollingshead2020}.

Recent applications of the SETT Framework have been particularly valuable in remote learning environments, a development that gained prominence during the COVID-19 pandemic and continues to be relevant as educational delivery models become more flexible. The framework's emphasis on environmental considerations has proven especially important as students now learn across multiple environments, from traditional classrooms to home settings and virtual spaces.

The SETT Framework is particularly relevant to evaluating and justifying assistive technology choices for the blind. For example, the framework can be used to identify the specific needs of a blind student, such as the need for a screen reader or a braille display. The framework can also help identify the specific tasks that the student needs to be able to do, such as reading textbooks or accessing online resources. By considering the student's characteristics, the environments in which they learn, and the tasks required to be an active learner in those environments, the framework can help identify the most appropriate assistive technology tools for the student \cite{Zabala2005, Zabala2018}.

The SETT Framework continues to be recognized as a foundational model for assistive technology decision-making, with ongoing research and applications demonstrating its continued relevance in contemporary educational settings. The framework's flexibility allows it to accommodate emerging technologies while maintaining its core focus on student-centered assessment and decision-making.

\section{Assistive Technology Assessments}\label{trouble4}
There are several assistive technology assessments available for use with blind or visually impaired people.

\noindent
\textbf{Context:} The following is a curated list of some of the most current and comprehensive assessment tools and resources for evaluating assistive technology needs. Each resource is referenced in the bibliography and is provided here in a semantic list for accessibility.

\begin{itemize}
  \item \textbf{Assistive Technology Checklist for Assessment}~ \cite{SnowChecklist}: A practical checklist for evaluating AT needs, available from Teaching Students with Visual Impairments.
  \item \textbf{Assistive Technology Assessment for Students Who Are Blind or Visually Impaired}~ \cite{TeachingAssessment}: A detailed assessment guide from Teaching Students with Visual Impairments.
  \item \textbf{Basic Technology Assessment Template}~ \cite{PerkinsTemplate}: A template from Perkins School for the Blind to guide technology assessments.
  \item \textbf{Assistive Technology for Students Who Are Blind or Visually Impaired: A Guide to Assessment}~ \cite{PresleyGuide}: A comprehensive book on AT assessment.
  \item \textbf{Assessing Students' Needs for Assistive Technology}~ \cite{WATIAssessing}: A resource from the Wisconsin Assistive Technology Initiative.
  \item \textbf{Assistive Technology for Students who are Visually Impaired}~ \cite{ISBEAssistive}: Guidance from the Illinois State Board of Education.
  \item \textbf{A Resource Guide to Assistive Technology for Students with Visual Impairments}~ \cite{QIATGuide}: A guide from the QIAT Community.
  \item \textbf{Assistive Technology - Instructional Services}~ \cite{CSBInstruction}: Resources from the California School for the Blind.
  \item \textbf{Assistive Technology for Blind or Low Vision Participants}~ \cite{MIUSATips}: Tips and resources from MIUSA.
\end{itemize}

Recent developments in assistive technology assessment include increased attention to the psychosocial impacts of assistive technologies, with new research protocols focusing on understanding how these tools affect the emotional and social well-being of users. Additionally, there has been growing recognition of the need for culturally responsive assessment practices that consider the diverse backgrounds and experiences of students with visual impairments.

The field continues to evolve with the integration of artificial intelligence and machine learning technologies, requiring assessments to consider not only traditional assistive technologies but also emerging AI-powered solutions that offer new possibilities for independence and academic success. Assessment teams are increasingly called upon to evaluate these newer technologies alongside established tools to ensure comprehensive and forward-thinking technology recommendations.

\section{References and Further Reading}
\addcontentsline{toc}{section}{References and Further Reading}
\begin{thebibliography}{99}
\bibitem{AEMCenter} AEM Center. (n.d.). Assistive Technology. Retrieved December 19, 2023. Available at: \url{https://aem.cast.org/learn/assistive-technology}.
\bibitem{WATI2010} Wisconsin Assistive Technology Initiative. (2010). Assistive technology consideration to assessment. Retrieved December 19, 2023. Available at: \url{https://www.wati.org/free-publications/assistive-technology-consideration-to-assessment/}.
\bibitem{ZabalaSETT} SETT Framework Resources. Available at: \url{https://www.joyzabala.com/links-resources}.
\bibitem{MNSETT} Minnesota Department of Administration. (n.d.). SETT Framework / Guide to Assistive Technology. [Webpage]. Minnesota's State Portal. Available at: \url{https://mn.gov/admin/at/learning/prek-12/sett-framework.jsp}.
\bibitem{Hollingshead2020} Hollingshead, A., Zabala, J., \& Carson, J. (2020). The SETT Framework and Evaluating Assistive Technology Remotely. Council for Exceptional Children. Available at: \url{https://exceptionalchildren.org/blog/sett-framework-and-evaluating-assistive-technology-remotely}.
\bibitem{Zabala2005} Zabala, J. (2005). Ready, SETT, go! Getting started with the SETT framework. Closing The Gap, 24(6), 1-8. Available at: \url{https://www.joyzabala.com/uploads/1/0/9/0/109073507/ready_sett_go.pdf}.
\bibitem{Zabala2018} Zabala, J. (2018). SETTing the stage for success: Building success through effective selection and use of assistive technology systems. In Handbook of Research on Integrating Technology Into Contemporary Language Learning and Teaching (pp. 1-22). IGI Global. Available at: \url{https://www.researchgate.net/profile/Joy-Zabala/publication/237798275_SETTing_the_Stage_for_Success_Building_Success_through_Effective_Selection_and_Use_of_Assistive_Technology_Systems/links/56452b9208aef646e6cc24d9/SETTing-the-Stage-for-Success-Building-Success-through-Effective-Selection-and-Use-of-Assistive-Technology-Systems.pdf?origin=publication_detail}.
\bibitem{SnowChecklist} Snow, A. (n.d.). Assistive Technology Checklist for Assessment. Teaching Students with Visual Impairments. Retrieved July 2025. Available at: \url{https://www.teachingvisuallyimpaired.com/assistive-technology-assessment.html}.
\bibitem{TeachingAssessment} Teaching Students with Visual Impairments. (n.d.). Assistive Technology Assessment for Students Who Are Blind or Visually Impaired. Retrieved July 2025. Available at: \url{http://www.teachingvisuallyimpaired.com/uploads/1/4/1/2/14122361/at_assessment_revised.pdf}.
\bibitem{PerkinsTemplate} Perkins School for the Blind. (n.d.). Basic Technology Assessment Template. Retrieved July 2025. Available at: \url{https://www.perkins.org/sites/elearning.perkinsdev1.org/files/Basic\%20Technology\%20Assessment\%20Template\_0\_0.docx}.
\bibitem{PresleyGuide} Presley, I., \& D'Andrea, F. M. (2024). Assistive Technology for Students Who Are Blind or Visually Impaired: A Guide to Assessment (Updated Edition). American Foundation for the Blind.
\bibitem{WATIAssessing} Wisconsin Assistive Technology Initiative. (2024). Assessing Students' Needs for Assistive Technology. Retrieved July 2025. Available at: \url{https://www.wati.org/free-publications/assessing-students-needs-for-assistive-technology/}.
\bibitem{ISBEAssistive} Illinois State Board of Education. (2024). Assistive Technology for Students who are Visually Impaired. Retrieved July 2025. Available at: \url{https://www.isbe.net/Documents/Assistive-Technology-Students-Visually-Impaired.pdf}.
\bibitem{SETTResources} SETT Framework Resources. Available at: \url{https://www.joyzabala.com/links-resources}.
\bibitem{QIATGuide} QIAT Community. (2024). A Resource Guide to Assistive Technology for Students with Visual Impairments. Retrieved July 2025. Available at: \url{https://www.qiat.org/docs/resourcebank/TEBO_VI_Resource_Guide.pdf}.
\bibitem{CSBInstruction} California School for the Blind. (2024). Assistive Technology - Instructional Services. Retrieved July 2025. Available at: \url{https://www.csb-cde.ca.gov/instruction/assistivetech/index.aspx}.
\bibitem{MIUSATips} MIUSA. (2025). Assistive Technology for Blind or Low Vision Participants. Retrieved July 2025. Available at: \url{https://miusa.org/resource/tip-sheets/assistivetechnologyforblind/}.
\end{thebibliography}
