\chapter{Troubleshooting Screenreader \& Magnifier Performance}
\glsreset{ocr}\glsreset{icr}\glsreset{tts}\glsreset{llm}\glsreset{uia}\glsreset{msaa}\glsreset{pdfua}\glsreset{api}\glsreset{cpu}
\label{app1:troubleshooting}

\begin{raggedright}
	This appendix provides accessible troubleshooting guidance for screenreader\index{screen reader} and magnifier performance\index{troubleshooting!screenreader performance}. Each section is structured for logical \gidx{navigation}{navigation} and screen reader clarity, with context added before lists\index{Markdown!lists} and resources.
\end{raggedright}

\section{~~Clearing System Cache}
\label{app1:cache}
It is often recommended to users of screenreaders to maintain a habit of clearing the browser and system cache\index{troubleshooting!clearing cache} (s) in order to optimize performance of their laptop\index{laptop}. However, the impact of this practice depends on your \gidx{hardware}{hardware}. The following explanation provides context for both SSD and spinning hard drive users \cite{SystemOptimizationGuides, Microsoft2023WindowsPerformance}:

In addition, SSDs have a limited number of write cycles. Every time data is written to an SSD, it uses up one of these write cycles. Clearing the cache causes more data to be written to the SSD, which can reduce the lifespan of the drive. This is because when the cache is cleared, the computer must download the data again, which requires writing the data to the SSD. This can cause unnecessary wear and tear on the drive and reduce its lifespan.

\begin{description}
	\item[SSD Users:] Clearing the cache can actually slow down the response of a computer with an SSD. This is because the cache stores frequently accessed data, such as images and scripts\index{PDF!scripts}, so that they can be loaded quickly. When the cache is cleared, the computer must download this data again, which can slow down the response time.
	\item[Spinning Hard Drive Users:] In contrast, spinning hard drives are slower than SSDs and can benefit from clearing the cache. This is because spinning hard drives have to physically move a read/write head to access data, which can take longer than reading data from an SSD.
\end{description}

\section{~~Slow Responsiveness}
\label{app1:response}
When a \gidx{screenreader}{screen reader} like JAWS or NVDA is not responding to input or is taking a long time to report changes on the screen, there are several things you can try to resolve the issue. The following steps are presented in a logical order for troubleshooting\index{troubleshooting} \cite{Fowler2011ScreenReaderLatency, Smith2022}:

\begin{description}
	\item[Restart Devices:] First, try restarting the \gidx{screenreader}{screen reader} and the computer. This can help clear any temporary issues that may be causing the problem.
	\item[Update Software\index{software}:] If this does not work, try updating the screen reader to the latest version. Screen readers are updated regularly to fix bugs and improve performance. As of 2025, both JAWS\index{screen reader!JAWS} and NVDA continue to receive regular updates with enhanced features and performance improvements \cite{JAWSWhatsNew, turn0search9}.
	\item[New Features:] JAWS 2025 includes new features like FS Companion, an AI\index{AI} assistant that helps users learn the \gidx{software}{software} more effectively, and enhanced language detection capabilities. NVDA\index{accessibility!NVDA} has introduced AI-powered image descriptions and automatic add-on updates in recent versions, which can help reduce barriers for users worldwide. These updates often include performance optimizations that may resolve responsiveness issues \cite{JAWSAILabeler, NarratorImageDescriptions}.
	\item[Adjust Settings:] If the problem persists, try adjusting the settings of the screen reader. Some screen readers\index{screen reader} have settings that can be adjusted to improve performance. For example, you can adjust the verbosity level to reduce the amount of information that is read out loud. You can also adjust the speed of the screen reader to make it faster or slower. For JAWS, use Insert + J to access settings, and for NVDA, use Insert + N to open the settings dialog. It's best not to make other changes unless you are absolutely sure about what they do.
	\item[Free System Resources:] Memory usage can also affect screen reader performance. Close unnecessary applications and browser tabs to free up system resources. Additionally, ensure that your computer meets the minimum system requirements for your screen reader. NVDA requires Windows\index{operating system!Windows} 8.1 or later, while JAWS\index{screen reader!JAWS} has similar requirements for optimal performance \cite{NVDARequirements, JAWSRequirements}.
	\item[Contact Support:] Finally, if none of these steps work, you may need to contact the manufacturer of the screen reader for further assistance. They may be able to provide additional troubleshooting\index{troubleshooting} steps or help you diagnose the problem. It's important to remember that screen readers are complex pieces of software and may require specialized knowledge to troubleshoot. By following these steps, you can help ensure that your screen reader is working properly and providing you with the \gidx{accessibility}{accessibility} you need \cite{FreedomScientificJAWS, NVAccess, MicrosoftAccessibility}.
\end{description}

\section{~~Browser Compatibility Issues}
\label{app1:browser}
\gls{screenreader} work differently across various web browsers, and compatibility issues can sometimes cause performance problems. The following tips can help address browser-related issues \cite{WebAIMSurvey}:

\begin{description}
	\item[Browser Choice:] Chrome and Firefox generally offer the best compatibility with modern screen readers, while Edge has improved significantly in recent years. Some websites may work better with specific browser and screen reader combinations.
	\item[Basic Troubleshooting\index{troubleshooting}:] If you're experiencing issues with a particular website, try switching to a different browser. Clear your browser cache and cookies, and ensure that JavaScript is enabled, as many accessibility features depend on it. Some screen readers\index{screen reader} also offer browser-specific settings that can be adjusted for optimal performance.
	\item[Updates and Known Issues:] Regular browser updates are important for maintaining compatibility with screen readers\index{screen reader}. However, sometimes new browser versions can introduce temporary compatibility issues. If you notice problems after a browser update, check the screen reader manufacturer's website for known issues and workarounds.
\end{description}

\section{~~AI and \gidx{machinelearning}{Machine Learning} Enhancements}
\label{app1:ai}
Modern screen readers are increasingly incorporating \gls{AI} and \gls{machinelearning} technologies to improve user experience. The following points summarize key enhancements and \gls{troubleshooting} tips \cite{MicrosoftAIAccessibility, Kim2023}:

\begin{description}
	\item[AI Features:] NVDA\index{accessibility!NVDA}'s AI-powered image descriptions can automatically describe images on web pages, reducing the need for manual alt-text. JAWS\index{screen reader!JAWS} 2025 includes FS Companion, which uses AI\index{AI} to help users learn the software more effectively and troubleshoot common issues \cite{NarratorImageDescriptions, JAWSAILabeler}.
	\item[Connectivity Requirements:] These AI features may require internet connectivity to function properly. If you're experiencing issues with AI-powered features, check your internet connection and ensure that your screen reader has permission to access online services. Some organizations may have firewall restrictions that prevent these features from working properly.
\end{description}

\section{~~Official Support\index{troubleshooting!official support} Contact}
\label{app1:report}
For direct assistance, contact the official support channels for your screen reader or operating system\index{operating system}. The following list provides accessible contact options:

\begin{description}
	\item[JAWS/Fusion\gidx{magnification}{magnification}\index{magnification!Fusion}:] You can submit a technical support request online, call 800-444-4443 or 727-803-8000 weekdays during business hours, or visit \href{https://support.freedomscientific.com/support}{Freedom Scientific\index{video magnifier!Freedom Scientific} Support} \cite{FreedomScientificJAWS}. For sales inquiries, contact \href{mailto:info@vispero.com}{info@vispero.com}.
	\item[Dolphin Products:] You can contact Dolphin's technical support team by emailing \href{mailto:support@yourdolphin.com}{Dolphin Support} \cite{DolphinScreenreaderRequirements}.
	\item[NVDA:] You can submit a bug report or request support by visiting \href{https://www.nvaccess.org/}{NV Access} or emailing \href{mailto:info@nvaccess.org}{NVDA Support Desk} \cite{NVAccess}. NVDA\index{accessibility!NVDA} can be downloaded free of charge, and the organization relies on donations to continue development.
	\item[Windows\index{operating system!Windows}:] You can contact Microsoft\index{tablet!Microsoft}'s technical support team by visiting \href{https://support.microsoft.com/}{Microsoft Support} \cite{MicrosoftAccessibility, DisabilityAnswerDesk}.
	\item[VoiceOver (Apple):] Contact Apple\index{tablet!Apple} Support through their website or the Apple Support app for assistance with VoiceOver\index{screen reader!VoiceOver} on Mac, iOS, and iPadOS devices \cite{AppleAccessibility}.
\end{description}

\section{~~Community Support via Groups and Forums}
\label{app1:community}
Online communities and discussion groups can often provide faster responses than official customer support. These communities consist of experienced users, developers, and \gidx{accessibility}{accessibility} professionals who can share practical solutions and workarounds. Members of these communities are often experts in their field and can provide quick and accurate answers to questions.

\paragraph{Accessible Community Resources:}
The following categorized list introduces relevant online communities for \gidx{visualimpairment}{visual impairment} accessibility needs. Each group is presented with a brief description for \gidx{screenreader}{screen reader} clarity.

\begin{description}
	\item[JAWS / Fusion\gidx{magnification}{magnification}\index{magnification!Fusion}:]
	      \begin{itemize}
		      \item \href{https://groups.io/g/jfw/}{The JAWS for Windows Support List}
		      \item \href{https://groups.io/g/jfw-users/}{JFW Users List}
		      \item \href{https://groups.io/g/jawsdiscussion/}{JAWS\index{screen reader!JAWS} Discussion}
		      \item \href{https://groups.io/g/jawslite/}{JAWS Lite}
		      \item \href{https://groups.io/g/jawsscripting/}{JAWS Scripting}
	      \end{itemize}
	\item[NVDA:]
	      \begin{itemize}
		      \item \href{https://NVDA.groups.io/g/NVDA/}{NVDA Group}
		      \item \href{https://NVDA-addons.groups.io/g/NVDA-addons}{NVDA Addons Group}
		      \item \href{https://NVDA.groups.io/g/chat/}{Chat Subgroup of the NVDA\index{accessibility!NVDA} Group}
		      \item \href{https://groups.io/g/NVDA-devel/}{NVDA Development}
		      \item \href{https://groups.io/g/NVDAdiscussion/}{NVDA Discussion}
		      \item \href{https://groups.io/g/NVDAhelp/}{NVDA Help}
	      \end{itemize}
	\item[Windows / General Accessibility\index{accessibility}:]
	      \begin{itemize}
		      \item \href{https://winaccess.groups.io/g/winaccess}{Windows Access with Screen Readers\index{screen reader}}
	      \end{itemize}
	\item[General Technology\index{technology} (Screen Readers Discussed Frequently):]
	      \begin{itemize}
		      \item \href{https://groups.io/g/blindtechdiscuss/}{Blind Tech Discuss}
		      \item \href{https://groups.io/g/tech-for-blind}{Tech For Blind}
		      \item \href{https://groups.io/g/blindadtech}{BlindADTech}
		      \item \href{https://groups.io/g/blind-techies/}{Blind Techies}
	      \end{itemize}
	\item[Social Media and Modern Platforms:]
	      \begin{itemize}
		      \item Reddit: r/Blind and r/\gls{accessibility} communities
		      \item Discord servers for accessibility and \gidx{assistivetechnology}{assistive technology}
		      \item Facebook groups for specific screen reader users
	      \end{itemize}
\end{description}

\section{~~Professional Testing and Development Resources}
\label{app1:testing}
For developers and accessibility professionals, there are specialized resources and tools available for testing screen reader compatibility. The following resources are introduced for accessible \gidx{navigation}{navigation}:

\begin{description}
	\item[AssistivLabs:] Provides cloud-based access to real screen readers, magnifiers, and other assistive technologies\index{assistive technology} for testing purposes. This allows developers to test their applications with actual screen readers rather than relying on simulations.
	\item[TPGi (The Paciello Group):] Maintains technical resources and knowledge bases for accessibility professionals. Their ARC Platform includes KnowledgeBase, which provides up-to-date information on screen reader compatibility and testing techniques \cite{TPGiARC}.
\end{description}

\section{~~Staying Current with Updates}
\label{app1:updates}
Screen reader\index{screen reader} technology evolves rapidly, with new features and compatibility improvements released regularly. The following tips will help you stay informed and ensure continued \gidx{accessibility}{accessibility}:

\begin{description}
	\item[Update Frequency:] Both JAWS\index{screen reader!JAWS} and NVDA release annual major updates, while smaller updates and patches are distributed throughout the year. It's important to stay informed about these updates through official channels, community discussions, and accessibility news sources \cite{JAWSWhatsNew, turn0search9}.
	\item[Newsletters and Surveys:] Consider subscribing to newsletters from screen reader manufacturers and accessibility organizations to stay informed about new features, known issues, and best practices. The WebAIM Screen Reader User Survey\index{WebAIM!Screen Reader User Survey}, conducted periodically, provides valuable insights into current usage patterns and preferences in the screen reader community \cite{WebAIMSurvey}.
	\item[Verify Information:] While community support\index{troubleshooting!official support} can be invaluable, it's important to verify information from unofficial sources. Official documentation and support channels should be your primary reference for critical issues or when making important decisions about \gidx{screenreader}{screen reader} configuration and usage.
\end{description}
test

