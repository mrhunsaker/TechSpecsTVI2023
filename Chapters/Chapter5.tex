\chapter{Shaping Knowledge: The Imperative Role of 3D Printed Materials in Fostering Hands-On Literacy for Visually Impaired Students}\label{d-printers}

In the realm of education, the power of hands-on experience is unparalleled. For visually impaired students, the journey toward literacy and comprehension takes on a unique dimension—one that is enriched and transformed through the tactile exploration of 3D printed materials. This chapter explores the indispensable role that 3D printed materials play in providing a tangible, tactile bridge to knowledge. These innovative creations facilitate hands-on engagement with concepts and serve as catalysts for literacy, fostering success for visually impaired students across a diverse spectrum of subjects.

The need for tangible exploration is paramount, especially when conceptualizing abstract ideas or interacting with physical entities is integral to the learning process. Traditional educational materials often rely on visual cues that pose challenges for students with visual impairments. 3D printed materials transcend the limitations of traditional teaching tools and enhance literacy by providing a multisensory gateway to understanding.

From historical artifacts to mathematical models, 3D printed materials transform abstract concepts into tangible, touchable entities. These creations allow visually impaired students to feel, explore, and internalize knowledge in a manner that aligns with their unique learning styles.

Hands-on learning with 3D printed materials fosters comprehension, empowerment, and curiosity. These tools democratize access to knowledge and enhance the educational journey for visually impaired students.

\section{3D Printers}\label{d-print-equipment}
When selecting a 3D printer for students with visual impairments, it is important to consider the following features:

\begin{itemize}
    \item \emph{Tactile printing:} The printer should produce 3D models that are tactile and easily understood by students with visual impairments.
    \item \emph{High resolution:} The printer should produce high-resolution models with fine details.
    \item \emph{Ease of use:} The printer should be easy to use, set up, and maintain.
    \item \emph{Compatibility:} The printer should be compatible with a wide range of software and file formats.
    \item \emph{Cost:} The printer should be affordable and within the school or institution's budget.
\end{itemize}

3D printing can help visually impaired students learn a variety of disciplines such as engineering, manufacturing, food, art, and health.\footnote{\href{http://files.eric.ed.gov/fulltext/EJ1247154.pdf}{Karbowski, C. F. (2020). See3D: 3D Printing for People Who Are Blind. Journal of Science Education for Students with Disabilities, 23(1), n1.}} 3D printed models can benefit both blind and sighted students, allowing for multisensory learning and independence.\footnote{\href{http://www.matterhackers.com/articles/3d-printed-educational-models-for-the-visually-impaired}{MatterHackers. (2017). 3D printed educational models for the visually impaired. MatterHackers}}

Table \ref{tab:chapter5:3d-printer-comparison} lists current available 3D printers.

\tagpdfsetup{table/header-rows={1}}
\centering
\begin{longtblr}[
  caption = {Comparison of 3D printers: model, cost, print bed size, filament size, and manufacturer},
  label = {tab:chapter5:3d-printer-comparison},
  note = {Detailed comparison of entry to mid-range 3D printers suitable for educational use, including print specifications and pricing. *Prices as of 2025-04-08; expect 30-40\% increase.}
]{
  colspec = {X[l] X[l] X[l] X[l] X[l]},
  rowhead = 1,
  hlines,
  stretch = 1.5,
}
Model & Cost & Print Bed Size & Filament Size & Manufacturer \\
Ender 3 Max Neo & \$359 & 300x300x320mm & 1.75mm & Creality \\
Ender 5 Plus & \$579 & 350x350x400mm & 1.75mm & Creality \\
K1 & \$599 & 220x220x256mm & 1.75mm & Creality \\
Kobra Max & \$569 & 450x400x400mm & 1.75mm & Anycubic \\
Kobra Plus & \$499 & 300x300x350mm & 1.75mm & Anycubic \\
M5C & \$399 & 220x220x250mm & 1.75mm & AnkerMake \\
Mini+ & \$459 & 180x180x180mm & 1.75mm & Prusa \\
Neptune 3 Max & \$470 & 420x420x500mm & 1.75mm & Elegoo \\
Neptune 4 Pro & \$330 & 225x225x265mm & 1.75mm & Elegoo \\
Anycubik Kobra S1 Combo & \$749.99 & --- & 1.75mm & Anycubic \\
Artillery M1 Pro & \$349.00 & --- & 1.75mm & Artillery \\
Elegoo Centauri Carbon & \$299.99 & --- & 1.75mm & Elegoo \\
\end{longtblr}


\tagpdfsetup{table/header-rows={1}}
\centering
\begin{longtblr}[
  caption = {Additional 3D printers: model, cost, print bed size, filament size, and manufacturer.},
  label = {tab:chapter5:3d-printer-comparison-2},
  note{} = {Premium and specialized 3D printers with advanced features for educational institutions. Includes printers with enclosures/environmental control.}
]{
  colspec = {X[l] X[l] X[l] X[l] X[l]},
  rowhead = 1,
  hlines,
  stretch = 1.5,
}
Model & Cost & Print Bed Size & Filament Size & Manufacturer \\
P1P 3D Printer & \$699 & 256×256×256mm & 1.75mm & Bambu \\
P1S 3D Printer (Combo) & \$949.99 & 256×256×256mm & 1.75mm & Bambu \\
X1C Carbon Combo & \$1,199.99 & 256×256×256mm & 1.75mm & Bambu \\
H2D Combo & \$2,199 & --- & 1.75mm & Bambu \\
H2D Combo (Laser/Cricut) & \$2,799 & --- & 1.75mm & Bambu \\
Prusa Core ONE & \$1,199.99 & --- & 1.75mm & Prusa \\
Creality K2 Plus & \$1,499.99 & --- & 1.75mm & Creality \\
Sidewinder X2 & \$399 & 300x300x396mm & 1.75mm & Artillery \\
A1 3D Printer & \$559 & 256×256×256mm & 1.75mm & Bambu \\
\end{longtblr}

\section{Web Resources for 3D Print Files and Accessibility}\label{3d-print-web-resources}

\emph{Designed For VI Specifically}
\begin{itemize}
    \item BTactile, Benetech ImageShare, Median Augenbit, Tactiles
\end{itemize}

\emph{Math Curricula}
\begin{itemize}
    \item Nonscriptum Calculus, Geometry, Trigonometry
\end{itemize}

\emph{Astronomy/Physics}
\begin{itemize}
    \item 3D Opal, Astrokit, NASA, Roving Bits Constellations, Tactile Universe
\end{itemize}

\emph{Biology}
\begin{itemize}
    \item 3D Biology, NIH 3D Print Collections/Models
\end{itemize}

\emph{General User-Uploaded 3D Print File Collections}
\begin{itemize}
    \item 3D Warehouse, Cults 3D, GCTrader, GrabCad, Instructables, My Mini Factory, Pinshape, Printables, Sketchfab, Thingiverse, Traceparts, Turbo Squid, YouMagine
\end{itemize}

\emph{3D File Search Aggregators}
\begin{itemize}
    \item 3D Export, 3D Find It, 3D Print Shelf, 3DPea, 3DSourced, 3devo, 3dmdb, Creazilla, Free3d, MakerOnline, MakerWorld, Mito3D, Open 3D Model, Open3dSea, Pinshape, STL Finder, STLBase, STLRepo, SeekSTL, Serev3D, SketchFab, Thangs3D, Trofp, Yeggi
\end{itemize}

\emph{AI 3D Model Generation}
\begin{itemize}
    \item 3D AI Maker, Cube by CSM, Luma LLabs Genie, Meshcapade, Meshy.ai, Sline Design, Sloyd
\end{itemize}

\emph{Professional Groups}
\begin{itemize}
    \item AT Makers, Makers Making Change, Volksswitch
\end{itemize}

\emph{Visually Impaired Education and Accessibility Resources}
\begin{itemize}
    \item 3D Print Accessibility ListServ, Accessible 3D, Accessible Graphics, MatterHackers, Oklahoma ABLE Tech, See3D, Solid Print 3D
\end{itemize}


\section{3D Printer Materials}\label{d-printer-materials}
3D printing creates three-dimensional objects from computer-aided design (CAD) files. The process involves depositing materials layer by layer to build a shape.\footnote{\href{http://www.3ds.com/make/solutions/industries/3d-printing-education}{Dassault Systèmes. (n.d.). 3D printing in education. Retrieved December 19, 2023}} To use a 3D printer in an educational environment, you need:

\begin{itemize}
    \item \emph{3D printer}: Available in various sizes, from benchtop to large-format, including models with enclosures/environmental control for improved reliability.
    \item \emph{Filament}: The material used to create the 3D object (e.g., PLA, TPU, ABS, PETG, etc.).\footnote{\href{http://www.techlearning.com/buying-guides/best-3d-printers-for-schools}{Tech \& Learning. (2023). Best 3D printers for schools. Retrieved December 19, 2023}}
    \item \emph{Computer}: Required to create the 3D model using CAD software.
    \item \emph{CAD software}: Used to create the 3D model.
    \item \emph{Slicing software}: Converts the 3D model into a format the printer can understand and generates the G-code for printing.\footnote{\href{http://www.teachthought.com/technology/ways-3d-printing-can-be-used-in-education/}{TeachThought. (2021). 10 ways 3D printing can be used in education. Retrieved December 19, 2023}}
\end{itemize}

\textbf{3D Printer Filament (PLA) and Color Resources}

\emph{FilamentColors} is a color checking program for popular PLA vendors, providing Hex codes for reproducible color accuracy. Not all vendors are available, but the list is growing.

\textit{Prices are for 1kg/2.2lb basic PLA, default with spool unless noted. Refills require a spool. Bambu Labs AMS System compatibility prioritized. Prices as of 2025-04-08; tariffs may increase non-US supplier costs by 10-45\%.}

\textbf{Non-US Suppliers:}
\begin{itemize}
    \item Bambu Labs: \$22 (\$17 with 4+ rolls) with spool; \$19 (\$14 with 4+ kg) for refills
    \item Creality: \$15 Soleyin Ultra PLA; \$17 Ender Fast PLA
    \item Dymanism: \$30
    \item ELEGOO: \$13
    \item eSun: \$17
    \item Sunlu: \$18
    \item MicroCenter Inland PLA: \$19
\end{itemize}

\textbf{Manufactured in the USA (no major tariff impact):}
Most US PLA is sourced from Natureworks LLC (Ingeo Line).
\begin{itemize}
    \item 3D Fuel: \$25
    \item 3D Innovators: \$20
    \item 3DXTech: \$32
    \item American Filament: \$25 (\$12 500g refill)
    \item Atomic Filament: \$30
    \item AtraxiaArt: \$24+
    \item Blendmaker: \$16
    \item COEX LLC: \$24
    \item Fila Cube: \$23
    \item Filamatrix: \$21
    \item Filastruder: \$9 PLA, \$11 PLA Pro
    \item Fusion Filaments: \$29
    \item Gizmo Dorks: \$23
    \item Greengate 3D: \$31
    \item Hatchbox: \$32
    \item IC3D: \$29
    \item iiiD Max (3D Max): \$21
    \item Jinos Filament: \$22 (\$17/16 spool pack)
    \item Keene Village Plastics: \$30
    \item MakeShaper: \$29
    \item Marlon Precision 3D Filaments: \$23
    \item Matterhackers: \$18+
    \item Monofilament Direct: \$25
    \item Numakers: \$20
    \item Overture 3D: \$23
    \item Paramount 3D: \$22 (\$19/8pack)
    \item Polar Filament: \$18
    \item Polymaker: \$20
    \item PolySonic: \$25
    \item PolyTerra: \$20
    \item PrintBed: \$25
    \item Printed Solid: \$24
    \item Printerior: \$28
    \item ProtoPasta: \$19
    \item Push Plastic: \$24
    \item RepKord: \$40 (\$10/1 pound)
    \item Splice 3D: \$15/spool (bulk: \$12 w/4+, \$10 w/8+, \$9 w/24+)
    \item Toner Plastics: \$22
    \item VoxelPLA: \$16
    \item ZYLTech: \$17
\end{itemize}

Table \ref{tab:table20} lists materials needed to use the 3D printers shown in Table \ref{tab:chapter5:3d-printer-comparison}.

\tagpdfsetup{table/header-rows={1}}
\centering
\begin{longtblr}[
  caption = {3D Printer Materials},
  label = {tab:table20},
  note = {Essential consumable materials and tools required for 3D printing in educational settings. See above for detailed filament vendor/pricing.}
]{
  colspec = {X[l] X[l] X[l]},
  rowhead = 1,
  hlines,
  stretch = 1.5,
}
Item & Cost & Vendor \\
1.75mm filament (see above) & \$13--\$40/kg & Multiple (Bambu, Elegoo, 3D Fuel, etc.) \\
3D Print Tool Kit & \$58.00 & HIJIRH \\
Assorted Sandpaper (48 pcs) & \$7.00 & Vicien \\
Glue Sticks (30 pack) & \$10.00 & Amazon Basics \\
Painter's Tape (2" width 12 Pack) & \$43.00 & Amazon \\
\end{longtblr}

\section{3D Printer Software}\label{d-printer-materials-program}
3D printing software allows users to create, edit, and slice 3D models. These programs enable users to design models, slice them into layers, and generate G-code for the printer.

\textbf{Resources for Programs to Create 3D Models}

\textit{Free:}
\begin{itemize}
    \item 3D Slash: Free web version, fun UI.
    \item BRL-CAD: Advanced solid modeling, used by U.S. military.
    \item Blender: Open-source, steep learning curve, complex models.
    \item DesignSpark Mechanical: Free mechanical CAD, rapid prototyping.
    \item FreeCAD: Open-source parametric modeler.
    \item OpenSCAD: Script-based modeling for programmers.
    \item SketchUp: Balance of usability and functionality.
    \item Tinkercad: Browser-based, beginner-friendly, block-building.
    \item Wings3D: Open-source polygon modeler.
\end{itemize}

\textit{Education Plans:}
\begin{itemize}
    \item Fusion 360: Free for personal/startups, cloud-based, advanced features.
    \item Shapr3D: Multi-device, free and pro (\$299/yr).
\end{itemize}

\textit{Professional:}
\begin{itemize}
    \item 3DS Max: \$1,545/yr, animation and modeling.
    \item Cinema 4D: \$720/yr or \$3,945 perpetual.
    \item Inventor: \$1,985/yr, mechanical design.
    \item Maya: \$1,545/yr, animation.
    \item Modo: \$599/yr or \$1,799 perpetual.
    \item Rhino3D: \$995, NURB modeling.
    \item SolidWorks: \$1,295/yr or \$3,995 perpetual.
\end{itemize}

\textbf{3D Print Slicing Programs}
\begin{itemize}
    \item 3DPrinterOS, Bambu Studio (default for Bambu Labs), IdeaMaker, KISSlicer, OctoPrint, Orcaslicer, Repetier, Simplify3D, Slic3r, Ultimaker Cura.
\end{itemize}

Table \ref{tab:table201} lists software and their functions.

\tagpdfsetup{table/header-rows={1}}
\centering
\begin{longtblr}[
  caption = {3D Printer Software and Functions},
  label = {tab:table201},
  note = {Available software tools for 3D modeling and printing, categorized by function and cost}
]{
  colspec = {X[l] X[l] X[l]},
  rowhead = 1,
  hlines,
  stretch = 1.5,
}
Program & Cost & Function \\
Fusion 360 & Free\footnote{for advanced features \$60/month} & Generate 3D file \\
FreeCAD & \$0 & Generate 3D file \\
SolidWorks & \$4,000/yr & Generate 3D file \\
TinkerCAD & \$0 & Generate 3D file \\
SketchUp Free & Free & Generate 3D file \\
Blender & \$0 & Generate 3D file \\
Rhino 6 & \$995/\$195 student & Generate 3D file \\
Cura & \$0 & Slice \& Print 3D Model \\
Slic3r & \$0 & Slice \& Print 3D Model \\
PrusaSlicer & \$0 & Slice \& Print 3D Model \\
Simplify3D & \$149 & Slice \& Print 3D Model \\
Meshmixer & \$0 & Slice \& Fix 3D Print Files \\
Meshlab & \$0 & Slice \& Fix 3D Print Files \\
\end{longtblr}
