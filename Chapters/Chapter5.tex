\hypertarget{d-printers}{}\chapter[\raggedright Shaping Knowledge:\hfill\break  The Imperative Role of 3D Printed Materials in Fostering Hands-On\hfill\break  Literacy for Visually Impaired Students]{Shaping Knowledge: The Imperative Role of 3D Printed Materials in Fostering Hands On Literacy for Visually Impaired Students}\label{d-printers}
\extramarks{Vision Department Technology Needs}{Shaping Knowledge}
{\let\clearpage\relax\localtableofcontents\let\clearpage\relax\locallistoftables}\newpage
In the realm of education, the power of hands-on experience is unparalleled. For visually impaired students, the journey toward literacy and comprehension takes on a unique dimension—one that is enriched and transformed through the tactile exploration of 3D printed materials. This chapter embarks on a captivating exploration of the indispensable role that 3D printed materials play in providing a tangible, tactile bridge to knowledge. These innovative creations not only facilitate hands-on engagement with concepts but stand as catalysts for literacy, fostering success for visually impaired students across a diverse spectrum of subjects.

The need for tangible exploration is paramount, especially when conceptualizing abstract ideas or interacting with physical entities is integral to the learning process. Traditional educational materials often rely on visual cues that pose challenges for students with visual impairments. Enter 3D printed materials—a technological marvel that transcends the limitations of traditional teaching tools. This chapter delves into the transformative impact of these creations, spotlighting their role in enhancing literacy by providing a multisensory gateway to understanding.

From intricate historical artifacts to complex mathematical models, 3D printed materials have the power to transform abstract concepts into tangible, touchable entities. Through this chapter, we will explore how these creations transcend the boundaries of traditional education, offering visually impaired students the opportunity to feel, explore, and internalize knowledge in a manner that aligns with their unique learning styles.

The immersive nature of hands-on learning with 3D printed materials not only fosters comprehension but also instills a sense of empowerment and curiosity. Given the diverse applications of these innovative tools, it is clear that they are not merely educational aids but agents of transformation, democratizing access to knowledge and enhancing the educational journey for visually impaired students. This chapter seeks to underscore the necessity of 3D printed materials in shaping the hands-on literacy experience, ensuring that visually impaired students can grasp the intricacies of the world around them with confidence, curiosity, and a sense of empowerment. 

\pagebreak \hypertarget{d-print-equipment}{}\section{3D Printers}\label{d-print-equipment}
When selecting a 3D printer for students with visual impairments, it is important to consider the following features:

Tactile printing: The printer should be capable of producing 3D models that are tactile and can be easily understood by students with visual impairments. The models should have a clear texture and be easy to touch and feel.

High resolution: The printer should be able to produce high-resolution models with fine details. This is important for creating models that are accurate and easy to understand.

Ease of use: The printer should be easy to use and operate. It should have a simple interface and be easy to set up and maintain.

Compatibility: The printer should be compatible with a wide range of software and file formats. This is important for creating and printing models from a variety of sources.

Cost: The printer should be affordable and within the budget of the school or institution. This is important for ensuring that all students have access to the technology they need to access a free and appropriate public education.

According to a study by Kietzmann et al., 3D printing can help visually impaired students learn a variety of disciplines such as engineering, manufacturing, food, art, and health. Additionally, 3D printed models can be beneficial for students who are blind and sighted, which allows for students of various levels of vision to use the same tactile learning tools\footnote{\raggedright \href{http://files.eric.ed.gov/fulltext/EJ1247154.pdf}{Karbowski, C. F. (2020). See3D: 3D Printing for People Who Are Blind. Journal of Science Education for Students with Disabilities, 23(1), n1.} \url{http://files.eric.ed.gov/fulltext/EJ1247154.pdf}}. Neal McKenzie, an Assistive Technology Specialist for the Visually Impaired Department at the Sonoma County Office of Education, uses 3D printing technology to make education more accessible and comfortable for his students. He recommends using 3D printing to create tactile math graphing systems, Braille learning tactile games, and other functional tools that help visually impaired students be more independent and access specific concepts or assignments\footnote{\raggedright \raggedright\href{http://www.matterhackers.com/articles/3d-printed-educational-models-for-the-visually-impaired}{MatterHackers. (2017). 3D printed educational models for the visually impaired. MatterHackers} \hfill\break\url{http://www.matterhackers.com/articles/3d-printed-educational-models-for-the-visually-impaired}}. 

Table \ref{tab:table19} lists current available 3D printers.

\pagebreak 
\large\textbf{Table \ref{tab:table19}}\normalfont 
\begin{longtable}[]{@{}
	>{\raggedright\arraybackslash}m{.2\textwidth}
	>{\raggedright\arraybackslash}m{.08\textwidth}
	>{\raggedright\arraybackslash}m{.2\textwidth}
	>{\raggedright\arraybackslash}m{.2\textwidth}
	>{\raggedright\arraybackslash}b{.2\textwidth}@{}
	}
	\toprule

	\textbf{Model}  & \textbf{Cost} & P\textbf{Print Bed Size} & \textbf{Filament Size} & \textbf{Manufacturer} \\
	\midrule
	\endhead \hline                                                                                             \\
	\multicolumn{5}{r}{\textbf{Continued on Next Page}} \endfoot
	\endlastfoot
Ender 3 Max Neo & \$359         & 300x300x320mm            & 1.75mm                 & Creality              \\ \cdashline{1-5}
Ender 5 Plus    & \$579         & 350x350x400mm            & 1.75mm                 & Creality              \\ \cdashline{1-5}
Kobra Max       & \$569         & 450x400x400mm            & 1.75mm                 & Anycubic              \\ \cdashline{1-5}
Kobra Plus      & \$499         & 300x300x350mm            & 1.75mm                 & Anycubic              \\ \cdashline{1-5}
M5C             & \$399         & 220x220x250mm            & 1.75mm                 & AnkerMake             \\ \cdashline{1-5}
Mini+           & \$459         & 180x180x180mm            & 1.75mm                 & Prusa                 \\ \cdashline{1-5}
Neptune 3 Max   & \$470         & 420x420x500mm            & 1.75mm                 & Elegoo                \\ \cdashline{1-5}
Neptune 4 Pro   & \$330         & 225x225x265mm            & 1.75mm                 & Elegoo                \\ \cdashline{1-5}
P1P 3D Printer  & \$699         & 256×256×256mm            & 1.75mm                 & Bambu                 \\ \cdashline{1-5}
P1S 3D Printer  & \$949         & 256×256×256mm            & 1.75mm                 & Bambu                 \\ \cdashline{1-5}
Sidewinder X2   & \$399         & 300x300x396mm            & 1.75mm                 & Artillery             \\[1.0em]\hline
	\caption{ 3D Printers }\label{tab:table19}
\end{longtable}

\pagebreak
\hypertarget{d-printer-materials}{}\section{3D Printer Materials}\label{d-printer-materials}
3D printing is a production technique that creates a three-dimensional object from a computer-aided design (CAD) file. The process involves depositing one or more materials, typically plastic, metal, wax or composite, layer by layer to build a shape\footnote{\raggedright \href{http://www.3ds.com/make/solutions/industries/3d-printing-education}{Dassault Systèmes. (n.d.). 3D printing in education. Retrieved December 19, 2023} \url{http://www.3ds.com/make/solutions/industries/3d-printing-education}}. To use a 3D printer in an educational environment, you would need the following materials:

\begin{itemize}[leftmargin=*]
\item \textbf{3D printer}: 3D printers come in various sizes, from small enough to fit on a benchtop to large-format industrial machines. Large printers can produce bigger objects, but the machines take up more space and cost significantly more than benchtop printers\footnotemark[\value{footnote}].
\item \textbf{Filament}: Filament is the material used to create the 3D object. The most common filament materials are PLA, TPU, ABS, CPE, PETG, nGen, INOVA-1800, HIPS, HT, t-glase, Alloy 910, Polyamide, Nylon 645, Polycarbonate, PC-Max, PC+PBT, PC-ABS Alloy, PCTPE, and more\footnote{\raggedright \href{http://www.techlearning.com/buying-guides/best-3d-printers-for-schools}{Tech \& Learning. (2023). Best 3D printers for schools. Retrieved December 19, 2023} \url{http://www.techlearning.com/buying-guides/best-3d-printers-for-schools}}.
\item \textbf{Computer}: A computer is required to create the 3D model using CAD software. The computer sends the model to the 3D printer to create the object.
\item \textbf{CAD software}: CAD software is used to create the 3D model. There are many free and paid CAD software options available.
\item \textbf{Slicing software}: Slicing software is used to convert the 3D model into a format that the 3D printer can understand. The software slices the model into layers and generates the G-code that the printer uses to create the object\footnote{\raggedright \href{http://www.teachthought.com/technology/ways-3d-printing-can-be-used-in-education/}{TeachThought. (2021). 10 ways 3D printing can be used in education. Retrieved December 19, 2023} \url{http://www.teachthought.com/technology/ways-3d-printing-can-be-used-in-education/}}.
\end{itemize}
Table \ref{tab:table20} lists materials needed in order to use the 3d printers shown in Table \ref{tab:table19}.

\pagebreak 
\large\textbf{Table \ref{tab:table20}}\normalfont 
\begin{longtable}[]{@{}
	>{\raggedright\arraybackslash}m{.5\textwidth}
	>{\raggedright\arraybackslash}m{.2\textwidth}
	>{\raggedright\arraybackslash}b{.3\textwidth}@{}
	}
	\toprule
	\textbf{Item}                     & \textbf{Cost}             & \textbf{Vendor} \\
	\midrule
	\endhead \hline                                                                 \\
	\multicolumn{3}{r}{\textbf{Continued on Next Page}} \endfoot
	\endlastfoot
1.75mm filament                   & \$12.00/kg & Elegoo          \\ \cdashline{1-3}
3D Print Tool Kit                 & \$58.00                   & HIJIRH          \\ \cdashline{1-3}
Assorted Sandpaper (48 pcs)       & \$7.00                    & Vicien          \\ \cdashline{1-3}
Glue Sticks (30 pack)             & \$10.00                   & Amazon Basics   \\ \cdashline{1-3}
Painter's Tape (2" width 12 Pack) & \$43.00                   & Amazon          \\[1.0em]\hline
	\caption{ 3D Printer Materials }\label{tab:table20}
\end{longtable}
