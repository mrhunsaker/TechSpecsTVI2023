\chapter{Instructional Programs \& Materials}\label{appx4}

The Individuals with Disabilities Education Improvement Act (IDEIA) 2004 mandates that students with disabilities receive a free and appropriate public education (FAPE) in the least restrictive environment possible\footnote{\raggedright \href{http://sites.ed.gov/idea/statuteregulations/}{20 U.S.C. § 1400, et.}}. To ensure that blind and low vision students have access to FAPE, there is a need for evidence-based specialized curriculum to teach screenreader usage, magnification usage, accessible typing programs, and accessible coding curricula to teach tech skills to blind/low vision students.

Screen readers are software programs that allow blind and visually impaired users to read the text that is displayed on a computer screen with a speech synthesizer or braille display\footnote{\raggedright \href{https://www.pathstoliteracy.org/resource/introduction-screen-reader-instruction/}{Paths to Literacy. (n.d.). Introduction to Screen Reader Instruction. Retrieved July 2025}}. Magnification software enlarges the text and images on the screen for low vision users\footnote{\raggedright \href{https://www.afb.org/blindness-and-low-vision/using-technology/assistive-technology-videos/magnification-software}{American Foundation for the Blind. (n.d.). Magnification Software. Retrieved July 2025}}. Accessible typing programs help students with disabilities learn to type using adaptive technology. Accessible coding curricula teach blind and low vision students how to code using specialized software that is designed to be accessible to them\footnote{\raggedright \href{https://www.freecodecamp.org/news/helping-blind-people-learn-to-code-c47c68d4a237/}{FreeCodeCamp.org (2018). Helping blind people learn to code. Retrieved July 2025}}.

Evidence-based specialized curriculum for teaching these skills is important because it ensures that students with disabilities have access to the same educational opportunities as their peers. It also helps to ensure that students with disabilities are able to develop the skills they need to succeed in the workforce. By providing students with disabilities with the tools they need to succeed, we can help to create a more inclusive society where everyone has the opportunity to reach their full potential.

\section{Accessible Touch Typing Instruction}\label{appx5}

Learning to touch type is an essential skill for anyone who spends a significant amount of time typing. Touch typing can help you type more efficiently and accurately, which can save you time and reduce the risk of repetitive strain injuries. By using all ten fingers to type without looking at the keyboard, you can significantly increase your typing speed and reduce the number of errors you make. This can help you complete your typing tasks faster and with greater accuracy. Additionally, touch typing can help you use keystroke shortcuts more smoothly, which can help you navigate your computer more quickly and efficiently. Screen readers can also be used more effectively when you are able to touch type, as you can focus on the content being read rather than the keyboard. In summary, learning to touch type can help you become a more efficient and fluent typist, as well as improve your ability to navigate your computer and screen readers.

\emph{You may be thinking: My blind child has a Braille device. Why does she need to learn to type?}

Even if your child has a Braille device such as the Braillenote Touch, typing is essential. The computer is the mainstream device that your child will need in order to be productive in school and in the workplace. When I meet a new blind student, parents often tell me, ``My child needs to learn to use a screen reader.'' The first question I ask is, ``Does your child know how to type?'' In order to use a screen reader such as JAWS effectively, you have to be able to type accurately. Braille is important, too, and it definitely has its uses in technology. But I believe that typing is as important as Braille.

Typing allows blind students to use mainstream devices. They can use a laptop or desktop computer, or they can connect a keyboard to a tablet. When I use my iPhone and type in text messages, my keyboarding skills help me use the screen, even without a Braille display.

\emph{-- Treva Olivero} \href{https://nfb.org/images/nfb/publications/fr/fr40/1/fr400103.htm}{National Federation of the Blind. (2019). The Braille Monitor, January 1997. Retrieved July 2025}

There are a number of options available for teaching touch typing skills to students with visual impairments.
\begin{itemize}
 \item \href{https://www.accessibyte.com/typio-online-page/}{Typio}\footnote{\raggedright This product was specifically developed for use with the blind and features 130 guided lessons with audio feedback}
 \item \href{https://www.sonokids.org/ballyland-early-learning/ballyland-keyboarding/}{Ballyland Keyboarding}\footnote{\raggedright Specifically designed for blind and visually impaired students}
 \item \href{https://nelowvision.com/product/typeability-typing-and-computer-tutor-program-for-the-blind-and-visually-impaired/}{TypeAbility}\footnote{\raggedright Teaches the entire computer keyboard in 99 user-friendly lessons, updated as of 2025}
 \item \href{https://saomaicenter.org/en/smsoft/smtt}{Sao Mai Typing Tutor}\footnote{\raggedright Specifically designed for blind users}
 \item \href{https://www.cfb.state.nm.us/apps/}{Keystroke}\footnote{\raggedright Developed by the Commission for the Blind}
 \item \href{https://typer.aphtech.org/}{APH Typer Online}\footnote{\raggedright Formerly known as Talking Typer, maintained by APH}
 \item \href{https://www.typingclub.com/}{Typing Club}\footnote{\raggedright Mainstream typing program with accessibility features}
 \item \href{https://www.readandspell.com/us/typing-for-the-blind}{Touch-Type Read and Spell (TTRS)}\footnote{\raggedright This resource has been specifically shown to be effective for blind students through \href{https://www.readandspell.com/sites/default/files/Research/greenrich\_report\_-ttrs\_for\_visually\_imparied.pdf}{independent research}}
 \item \href{https://kaz-type.com/visualimpairment}{KAZ Typing Software}\footnote{\raggedright Specifically designed for visually impaired and blind users with voice prompts}
\end{itemize}

\section{AndroidOS/iOS/iPadOS Gesture Training}\label{appx6}

Learning VoiceOver and TalkBack gestures on tablets and phones is essential for users with visual impairments. VoiceOver is a screen reader that comes pre-installed on Apple devices, while TalkBack is a screen reader that comes pre-installed on Android devices\footnote{\raggedright \href{https://www.boia.org/blog/understanding-how-people-with-disabilities-use-mobile-devices}{Bureau of Internet Accessibility. (n.d.). Understanding How People With Disabilities Use Mobile Devices. Bureau of Internet Accessibility. }}\footnote{\raggedright \href{https://www.boia.org/blog/google-talkback-an-overview-of-androids-free-screen-reader}{Bureau of Internet Accessibility. (n.d.). Google TalkBack: An Overview of Android's Free Screen Reader. Bureau of Internet Accessibility.}}. Both screen readers include gesture-based controls and braille keyboard support. While these screen readers are useful tools, they depend on accurate text alternatives for non-text content. Learning VoiceOver and TalkBack gestures can help users navigate their devices more efficiently and effectively\footnote{\raggedright \href{https://support.apple.com/guide/iphone/turn-on-and-practice-voiceover-iph3e2e415f/ios}{Apple. (2022, December 20). Turn on and Practice VoiceOver. Apple Support.}}. For instance, TalkBack gestures can help users navigate and perform frequent actions on their Android devices, such as moving to the next item on the screen, selecting an item, and activating screen search.\footnote{\raggedright \href{https://support.google.com/accessibility/android/answer/6151827?hl=en}{Google. (n.d.). Use TalkBack on your Android device. Google.}} Similarly, VoiceOver gestures can help users navigate and perform frequent actions on their Apple devices, such as opening the app switcher, accessing the control center, and activating Siri. Competency with VoiceOver and TalkBack gestures can enable users to access the same activities as their peers, manage eye fatigue, and use good posture and a good viewing distance.

\begin{itemize}
 \begin{itemize}
  \item \href{https://screenreader.app/}{ScreenReader App}\footnote{\raggedright Users are invited to add any missing information to either \href{https://github.com/appt-org/screenreader-android}{screenreader-android} for Android TalkBack or \href{https://github.com/appt-org/screenreader-ios}{screenreader-ios} for VoiceOver}
  \item \href{https://www.sonokids.org/ballyland-early-learning/ballyland-game-apps/}{Ballyland Apps}
  \item \href{https://srt.csb-cde.ca.gov/}{The Screen Reader Training Website}\footnote{\raggedright This targets VoiceOver, but can be used for TalkBack with assistance}
  \item \href{https://hadley.edu/workshops/listen-with-talkback-series}{Listen with TalkBack Series from Hadley}
  \item \href{https://hadley.edu/workshops/listen-with-voiceover-series}{Listen with VoiceOver Series from Hadley}
 \end{itemize}

\hypertarget{appx7}{}\section[Screenreader Training]{Screenreader Training}\label{appx7}
Learning advanced methods of navigating the computer with a screen reader such as JAWS, Windows Narrator, or NVDA is essential for users with visual impairments. Recent developments in 2024-2025 show that NVDA continues to gain popularity, with NVDA 2025.1 introducing remote access capabilities and enhanced performance. JAWS 2025 includes significant performance optimizations and better compatibility with resource-intensive applications. While arrow keys and Tab can be useful for basic navigation, advanced methods can provide more efficient and comprehensive navigation. For instance, JAWS provides a feature called "Virtual Cursor" that allows users to navigate web pages and documents by line, word, character, or even by paragraph\footnote{\raggedright \href{https://www.freedomscientific.com/products/software/jaws}{Freedom Scientific. (n.d.). JAWS Screen Reader. Freedom Scientific.}}. Similarly, Windows Narrator provides a feature called "Scan Mode" that allows users to navigate web pages and documents by headings, links, tables, and landmarks.\footnote{\raggedright \href{https://support.microsoft.com/en-us/windows/narrator-user-guide-4b2e6b3f-1d6d-8a5c-4f6d2a3b3d6f}{Microsoft. (2022, December 31). Narrator User Guide. Microsoft. }}\footnote{\raggedright \href{https://support.microsoft.com/en-us/windows/use-a-screen-reader-to-navigate-windows-11-5f8a9e7c-7d3e-2d5a-0f5c-5f9b5b8a7a3d}{Microsoft. (2022, December 31) Use a screen reader to navigate Windows 11. Microsoft.}}. NVDA provides a feature called "Object Navigation" that allows users to navigate web pages and documents by objects such as buttons, checkboxes, and text fields\footnote{\raggedright \href{https://www.nvaccess.org/files/nvda/documentation/userGuide.html\#toc3.1}{NV Access. (2022, December 31). NVDA User Guide. NV Access.}} . Learning advanced methods of navigation can help users save time and effort, and increase productivity. It is important to note that while screen readers can be helpful, they should not replace other assistive technologies such as screen magnifiers. Therefore, it is important to learn advanced methods of navigating the computer with a screen reader to take full advantage of its benefits.
\begin{itemize}
 \item \href{https://www.freedomscientific.com/SurfsUp/}{Surf's Up}\footnote{\raggedright Offline version available for download as a zipped file \href{https://support.freedomscientific.com/SurfsUp/ZIP\_Surfs\_Up.zip}{at this link}}
 \item \href{https://srt.csb-cde.ca.gov/}{The Screen Reader Training Website}\footnote{\raggedright This site is an update to the Surf's Up curriculum undertaken by the California School for the Blind that has been expanded to cover NVDA, JAWS, and VoiceOver}
 \item \href{https://hadley.edu/workshops/windows-narrator-series}{Windows Narrator Series from Hadley}
 \item \href{https://hadley.edu/workshops/nvda-screen-reader-series}{NVDA Series from Hadley}
 \item \href{https://carroll.org/the-windows-screen-reader-primer-all-the-basics-and-more-second-edition/}{Windows Screen Reader Primer}\footnote{\raggedright in 2nd Ed. as of \today}\footnote{\raggedright This primer covers use of Windows Narrator, NVDA, and JAWS}
 \item \href{https://www.blind.training/}{Access Technology Institute, LLC. Courses}\footnote{\raggedright Sells training, textbooks, and subscription-based content about JAWS and NVDA}
 \item \href{https://www.nvaccess.org/product/nvda-productivity-bundle/}{NVDA Training Materials}\footnote{\raggedright Includes Basic Screenreader Training and Specific Training for Outlook, Word, Excel, and PowerPoint use with NVDA}
 \item \href{https://support.freedomscientific.com/Training/JAWS-Basic-Training.zip}{JAWS Basic Training}
 \item \href{https://eyetvision.org/screen-reader-curriculum-landing-page/\#wwt2}{Working with Text from eyeTvision}\footnote{\raggedright Covers NVDA, JAWS, and ChromeVox Screenreaders}
 \item \href{https://eyetvision.org/screen-reader-curriculum-landing-page/\#bin2}{Basic Internet Navigation from eyeTvision}\footnote{\raggedright Covers NVDA, JAWS, and ChromeVox Screenreaders}
 \item \href{https://shop.nbp.org/products/windows-screen-reader-keystroke-compendium-2024-update}{Windows Screen Reader Keystroke Compendium 2024}\footnote{\raggedright Comprehensive keystroke reference for JAWS, NVDA, and Narrator updated for 2024}
\end{itemize}

\hypertarget{appx11}{}\section[Screen Magnifier Training]{Screen Magnifier Training}\label{appx11}
Specialized screen magnification software like ZoomText, Fusion, Windows Magnifier, and Dolphin SuperNova are designed to provide a more comprehensive and customizable experience than the built-in magnification tools. While the built-in magnification tools can be useful for basic tasks, they may not be sufficient for users with more complex needs\footnote{\raggedright \href{https://www.boia.org/blog/screen-magnifiers-who-and-how-they-help}{Bureau of Internet Accessibility. (n.d.). Screen Magnifiers: Who and How They Help. Bureau of Internet Accessibility. }}. Specialized software can provide features such as color filtering, cursor enhancements, and text-to-speech capabilities\footnote{\raggedright \href{https://www.perkins.org/resource/getting-started-screen-magnification/}{Perkins School for the Blind. (2022, August 17). Getting started with screen magnification. Retrieved July 2025}}. Additionally, specialized software can help users manage eye fatigue, use good posture and a good viewing distance, and access the same activities as their peers. Competency with specialized screen magnification software can enable students to succeed in postsecondary education and jobs\footnote{\raggedright \href{https://www.afb.org/blindness-and-low-vision/using-technology/screen-magnification}{American Foundation for the Blind. (2022, August 17). Screen Magnification. American Foundation for the Blind.}}. It is important to note that while specialized screen magnification software can be helpful, it should not replace other assistive technologies such as screen readers. Therefore, it is important to learn how to use specialized screen magnification software to take full advantage of its benefits\footnote{\raggedright \href{https://nelowvision.com/introduction-to-screen-reading-and-magnification-software-a-comprehensive-guide-to-assistive-technology-assessment/}{Low Vision Center. (n.d.). Introduction to Screen Reading and Magnification Software: A Comprehensive Guide to Assistive Technology Assessment. Low Vision Center. }}.
\begin{itemize}
 \begin{itemize}
  \item \href{https://support.freedomscientific.com/teachers/resources/ZoomText\_resources.zip}{ZoomText Resources from Freedom Scientific}
  \item \href{https://support.freedomscientific.com/teachers/resources/Fusion\_resources.zip}{Fusion Resources from Freedom Scientific}
  \item \href{https://yourdolphin.com/support/tutorials}{Dolphin Supernova Training Materials}
 \end{itemize}

\hypertarget{appx10}{}\section[Braille Display Use]{Braille Display Use}\label{appx10}
Learning how to use a refreshable braille display is essential for emerging braille readers. Refreshable Braille Displays are peripheral devices that display braille characters, usually by raising and lowering dots through holes in a flat surface. Users can input braille using either the 6 or 8 key Perkins-style braille keyboard or, more recently, a QWERTY keyboard. While it may be tempting to use only the minimum functions of a braille display, being explicitly taught how to use it can provide many benefits. For instance, it can help improve finger strength and isolated finger control, which are crucial for writing\footnote{\raggedright \href{https://www.perkins.org/resource/benefits-using-braille-display-emerging-readers/}{Perkins School for the Blind. (n.d.). Benefits of Using a Braille Display with Emerging Readers. Retrieved July 2025}}. Additionally, using a braille display can help emerging readers with tactile discrimination and make it easier to read. Furthermore, pairing a braille display with a computer, tablet, or smartphone can provide instant auditory feedback as the student types, which can help with motivation. Recent developments in 2024-2025 include innovations in haptic feedback technology and improvements in braille display connectivity.
\begin{itemize}
\item \href{https://view.officeapps.live.com/op/view.aspx?src=https\%3A\%2F\%2Fwww.wssb.wa.gov\%2Fsites\%2Fdefault\%2Ffiles\%2F2021-10\%2FUsing\%2520APH\%2520Mantis\%2520Q40.docx&wdOrigin=BROWSELINK}{APH Mantis Q40 Braille Display \& Notetaker from Washington School for the Blind}
\item \href{https://view.officeapps.live.com/op/view.aspx?src=https\%3A\%2F\%2Fwww.wssb.wa.gov\%2Fsites\%2Fdefault\%2Ffiles\%2F2023-07\%2FUsing\%2520APH\%2520Chameleon\%252020.docx&wdOrigin=BROWSELINK}{APH Chameleon 20 Braille Display \& Notetaker from Washington School for the Blind}
\item \href{https://drive.google.com/drive/folders/1V\_hXjrsDeKUbNImA6Q77joADQbqMKKKl}{BrailleSense 6 Training from WCBVI AT}
\item \href{https://drive.google.com/drive/folders/10HeixUb4E21nPLCStmnrsxLVehKThPP}{BrailleSense 6 Training from California School of the Blind}
\item \href{https://drive.google.com/drive/folders/1OKBBdjbbD6asrE4dYyP7do9EWvY--5wf}{BrailleNote Touch Plus Training from California School of the Blind}
\item \href{https://eyetvision.org/}{Diving Into Braille Displays from eyeTvision}
\end{itemize}

\hypertarget{appx8}{}\section[Accessible Coding Curricula]{Accessible Coding Curricula}\label{appx8}
It is possible for blind students to learn computer programming. In fact, there are many resources available to help them learn, with significant developments in 2024-2025 making coding more accessible than ever before. Recent initiatives include partnerships between Code.org and accessible programming language developers, as well as new AI-powered tools for accessibility testing. For instance, the Perkins School for the Blind provides information on Quorum, an accessible programming language, as well as other resources and information related to blind programmers and coders\footnote{\raggedright \href{https://www.perkins.org/stories/blind-programmers-and-coders}{Perkins School for the Blind. (n.d.). Blind programmers and coders. Perkins School for the Blind.}}. Additionally, EarSketch, a platform designed to teach students to code in Python or JavaScript through music and creative discovery, continues to be adapted for blind and visually impaired youth\footnote{\raggedright \href{https://earsketch.gatech.edu/}{Georgia Tech. (2022, August 24). EarSketch. Georgia Tech}}. Microsoft has also developed Code Jumper, a coding language for children who are blind or visually impaired, which is comprised of modular, physical pieces that students can string together to create code\footnote{\raggedright \href{https://www.microsoft.com/en-us/research/project/code-jumper/}{Microsoft. (n.d.). Code Jumper. Microsoft.} }. New accessible tools such as KaiBot and enhanced CodeQuest applications are being integrated into mainstream K-12 education. It's worth noting that blind people can be successful software developers, with 1 out of every 200 software developers being blind\footnote{\raggedright \href{https://www.freecodecamp.org/news/how-blind-people-code-fdb64e3bf5c/}{FreeCodeCamp. (2017, November 14). How blind people code. FreeCodeCamp. }}. With the right resources and support, blind students can learn computer programming and pursue a career in software development.\footnote{\raggedright \emph{cf}., \href{https://files.eric.ed.gov/fulltext/EJ1207407.pdf}{Hadwen-Bennett, Alex \& Sentance, Sue \& Morrison, Cecily. (2018). Making Programming Accessible to Learners with Visual Impairments: A Literature Review. International Journal of Computer Science Education in Schools. 2. 10.21585/ijcses.v2i2.25.}}\footnote{\raggedright \href{https://www.mdpi.com/2071-1050/12/13/5320}{Alotaibi, Hind \& Al-Khalifa, Hend \& AlSaeed, Duaa. (2020). Teaching Programming to Students with Vision Impairment: Impact of Tactile Teaching Strategies on Student's Achievements and Perceptions. Sustainability. 12. 10.3390/su12135320.}}

The following list contains the current list of accessible coding options available for students with visual impairment/blindness, updated with 2024-2025 developments.
\begin{itemize}
 \begin{itemize}
  \item \href{https://zersiax.github.io/}{APH Connect Center Coding Course taught by Florian Beijers}\footnote{\raggedright Florian Beijers is a blind computer programmer who ran this course for APH. This link goes to the archived site that contains links to the lectures and archives all the course materials.}
  \item \href{https://codehs.com/}{CodeHS}\footnote{\raggedright Mainstream coding platform with enhanced accessibility features}
  \item \href{https://www.codecademy.com/}{Code Academy}\footnote{\raggedright Popular coding platform with screen reader compatibility}
  \item \href{https://codejumper.com/}{APH CodeJumper}\footnote{\raggedright A Microsoft \href{https://docs.microsoft.com/en-us/learn/modules/code-jumper-inclusive-physical-coding-language/}{Training Module} is available to teach CodeJumper to teachers}\footnote{\raggedright Specifically Designed by Microsoft and APH for use by the blind}
  \item \href{https://www.aph.org/product/code-quest-for-ipad-iphone/}{Code Quest}\footnote{\raggedright Accessible coding app designed for blind students}
  \item \href{https://www.aph.org/product/accessible-code-and-go-mouse/}{APH Code \& Go Mouse}\footnote{\raggedright Physical coding toy designed for blind students}
  \item \href{https://earsketch.gatech.edu/landing/}{EarSketch}\footnote{\raggedright \href{https://www.teachers.earsketch.org/learn}{Freely available Teaching Resources for Teachers}}
  \item \href{https://code.org/accessibility}{Code.org Accessibility Resources}\footnote{\raggedright Comprehensive documentation and accessible Hour of Code lessons, updated 2024-2025}
  \item \href{https://quorumlanguage.com/}{Quorum Programming Language}\footnote{\raggedright Accessible programming language designed for blind students, partnered with Code.org}
  \item KaiBot\footnote{\raggedright New accessible robotics platform for K-12 coding education}
 \end{itemize}

\section{Emerging Technologies and Future Directions}\label{appx12}

The landscape of assistive technology for students with visual impairments continues to evolve rapidly. Recent developments in 2024-2025 include significant advances in AI-powered accessibility tools, haptic feedback devices, and innovative approaches to digital content access. These emerging technologies promise to further enhance educational opportunities for blind and low vision students.

Key developments include:
\begin{itemize}
 \begin{itemize}
  \item AI-powered accessibility testing tools such as Axe DevTools AI for automated web content accessibility
  \item Enhanced haptic feedback devices for accessing digital content and touchscreen interfaces
  \item Innovations in braille technology and display connectivity
  \item Integration of accessible coding tools into mainstream K-12 curricula
  \item Development of more sophisticated screen reader remote access capabilities
  \item Advances in voice-controlled interfaces and natural language processing for educational applications
 \end{itemize}

Educational institutions and assistive technology providers continue to collaborate on developing more inclusive and accessible learning environments. The emphasis on universal design principles ensures that accessibility improvements benefit all students, not just those with visual impairments. As these technologies mature, they will provide even more opportunities for students with visual impairments to participate fully in educational and professional activities.
