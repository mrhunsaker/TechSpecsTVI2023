\chapter{Assistive Technology Considerations}\label{trouble3}
Assistive technology is an essential component of ensuring that students with visual impairments receive a free and appropriate public education. However, it is important to use a valid assistive technology assessment before providing assistive technology to a student. A valid assessment can help identify the specific needs of the student and determine the most appropriate assistive technology solutions. This can help ensure that the student receives the right tools to succeed in their studies\footnote{\raggedright \href{https://aem.cast.org/nimas-nimac/nimas-nimac}{AEM Center. (n.d.). NIMAS \& NIMAC. Retrieved December 19, 2023}}. Additionally, a valid assessment can help ensure that the student receives the appropriate accommodations and modifications to their educational program\footnote{\raggedright \href{https://daisy.org/about\_us/what-is-daisy/ }{DAISY Consortium. (n.d.). What is DAISY? Retrieved December 19, 2023}}.

It is also essential to use all the data available to guide decision making when providing assistive technology to a student. This includes data from the student, their family, and their educators. By using all the data available, educators can make informed decisions about the most appropriate assistive technology solutions for each student. This can help ensure that the student receives the right tools to succeed in their studies\footnote{\raggedright \href{https://aem.cast.org/learn/assistive-technology}{AEM Center. (n.d.). Assistive Technology. Retrieved December 19, 2023}}. It is important to note that convenience should not be a factor when making decisions about assistive technology. The focus should always be on what is best for the student.

Using a valid assistive technology assessment and all available data to guide decision making can help ensure that students with visual impairments receive the appropriate assistive technology solutions to succeed in their studies. This can help improve their academic performance and increase their chances of success in school. Additionally, it can help students with visual impairments become more independent in their daily lives. By providing students with the tools they need to access information and communicate with others, assistive technology can help them become more self-sufficient and less reliant on others\footnote{\raggedright \href{https://www.wati.org/free-publications/assistive-technology-consideration-to-assessment/}{Wisconsin Assistive Technology Initiative. (2010). Assistive technology consideration to assessment. Retrieved December 19, 2023}}.

Finally, it is important to note that the use of assistive technology is not a one-size-fits-all solution. The technology needs of students with visual impairments can vary widely depending on their individual needs and abilities. Therefore, it is important to work with students, families, and educators to identify the most appropriate assistive technology solutions for each student. By doing so, we can help ensure that students with visual impairments have the tools they need to succeed in school and beyond.

\section{SETT Framework}\label{trouble41}
The  \href{https://www.joyzabala.com/links-resources}{SETT Framework}\footnote{\raggedright click on ``resources/SETT Downloads'' and see items below}\footnote{\raggedright \emph{cf.}, \href{https://mn.gov/admin/at/getting-started/ready-sett-go.jsp}{Minnesota Department of Administration. (n.d.). Ready, SETT, Go! Getting started with the SETT framework. [Webpage]. Minnesota’s State Portal}} is a widely used method for evaluating assistive technology (AT) needs for students with disabilities. The acronym SETT stands for Student, Environment, Task, and Tools. The framework emphasizes the importance of understanding the student’s characteristics, the environments in which they learn and grow, and the tasks required to be an active learner in those environments before identifying a system of tools that enables the student to actively engage in the tasks. A team-based collaborative assessment of needs will lead to determining the most promising system of tools for the student, with a consideration of the environments this learner is in\footnote{\raggedright \href{https://exceptionalchildren.org/blog/sett-framework-and-evaluating-assistive-technology-remotely}{Hollingshead, A., Zabala, J., \& Carson, J. (2020). The SETT Framework and Evaluating Assistive Technology Remotely. Council for Exceptional Children.} }.

The SETT Framework is particularly relevant to evaluating and justifying assistive technology choices for the blind. For example, the framework can be used to identify the specific needs of a blind student, such as the need for a screen reader or a braille display. The framework can also help identify the specific tasks that the student needs to be able to do, such as reading textbooks or accessing online resources. By considering the student’s characteristics, the environments in which they learn, and the tasks required to be an active learner in those environments, the framework can help identify the most appropriate assistive technology tools for the student\footnote{\raggedright \href{https://www.joyzabala.com/uploads/1/0/9/0/109073507/ready_sett_go.pdf}{Zabala, J. (2005). Ready, SETT, go! Getting started with the SETT framework. Closing The Gap, 24(6), 1-8.} }\footnote{\raggedright \href{https://www.researchgate.net/profile/Joy-Zabala/publication/237798275_SETTing_the_Stage_for_Success_Building_Success_through_Effective_Selection_and_Use_of_Assistive_Technology_Systems/links/56452b9208aef646e6cc24d9/SETTing-the-Stage-for-Success-Building-Success-through-Effective-Selection-and-Use-of-Assistive-Technology-Systems.pdf?origin=publication_detail}{Zabala, J. (2018). SETTing the stage for success: Building success through effective selection and use of assistive technology systems. In Handbook of Research on Integrating Technology Into Contemporary Language Learning and Teaching (pp. 1-22). IGI Global}}.

The SETT Framework is a valuable tool for evaluating and justifying assistive technology choices for students with disabilities. The framework emphasizes the importance of understanding the student’s characteristics, the environments in which they learn and grow, and the tasks required to be an active learner in those environments before identifying a system of tools that enables the student to actively engage in the tasks. By using the SETT Framework, educators and stakeholders can make informed decisions that empower individuals with disabilities to achieve their full potential.

\section{Assistive Technology Assessments}\label{trouble4}
There are several assistive technology assessments available for use with blind or visually impaired people. Here are some of the available assessments:
\begin{itemize}
 \item \href{https://www.teachingvisuallyimpaired.com/assistive-technology-assessment.html}{Snow, A. (n.d.). Assistive Technology Checklist for Assessment. Retrieved December 19, 2023}
 \item \href{http://www.teachingvisuallyimpaired.com/uploads/1/4/1/2/14122361/at\_assessment\_revised.pdf}{Teaching Students with Visual Impairments. (n.d.). Assistive Technology Assessment for Students Who Are Blind or Visually Impaired. Retrieved December 19, 2023}
 \item \href{https://www.perkins.org/sites/elearning.perkinsdev1.org/files/Basic\%20Technology\%20Assessment\%20Template\_0\_0.docx}{Perkins School for the Blind. (n.d.). Basic Technology Assessment Template. Retrieved December 19, 2023}
 \item Presley, I., \& Siu, T. (2012). Assistive Technology for Students Who Are Blind or Visually Impaired: A Guide to Assessment. American Foundation for the Blind.
 \item \href{https://www.wati.org/free-publications/assessing-students-needs-for-assistive-technology/}{Wisconsin Assistive Technology Initiative}
 \item \href{https://mdelio.org/blind-visually-impaired/educator-support/assistive-technology-guidelines}{MDE-LIO Assistive Technology Guidelines}
 \item \href{https://www.joyzabala.com/links-resources}{SETT Framework}\footnote{\raggedright click on ``resources/SETT Downloads'' and see items below}\footnote{\raggedright \emph{cf.}, \href{https://mn.gov/admin/at/getting-started/ready-sett-go.jsp}{Minnesota Department of Administration. (n.d.). Ready, SETT, Go! Getting started with the SETT framework. [Webpage]. Minnesota’s State Portal}}
\end{itemize}
