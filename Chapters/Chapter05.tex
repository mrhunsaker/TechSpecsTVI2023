\chapter{Shaping Knowledge: The Imperative Role of 3D Printed Materials in Fostering Hands-On Literacy for Visually Impaired Students}\label{d-printers}

In the realm of education, the power of hands-on experience is unparalleled. For visually impaired students, the journey toward literacy and comprehension takes on a unique dimension—one that is enriched and transformed through the tactile exploration of 3D printed materials. This chapter explores the indispensable role that 3D printed materials play in providing a tangible, tactile bridge to knowledge. These innovative creations facilitate hands-on engagement with concepts and serve as catalysts for literacy, fostering success for visually impaired students across a diverse spectrum of subjects.

The need for tangible exploration is paramount, especially when conceptualizing abstract ideas or interacting with physical entities is integral to the learning process. Traditional educational materials often rely on visual cues that pose challenges for students with visual impairments. 3D printed materials transcend the limitations of traditional teaching tools and enhance literacy by providing a multisensory gateway to understanding.

From historical artifacts to mathematical models, 3D printed materials transform abstract concepts into tangible, touchable entities. These creations allow visually impaired students to feel, explore, and internalize knowledge in a manner that aligns with their unique learning styles.

Hands-on learning with 3D printed materials fosters comprehension, empowerment, and curiosity. These tools democratize access to knowledge and enhance the educational journey for visually impaired students.

\section{3D Printers}\label{d-print-equipment}
When selecting a 3D printer for students with visual impairments, it is important to consider the following features:

\begin{itemize}
    \item \emph{Tactile printing:} The printer should produce 3D models that are tactile and easily understood by students with visual impairments.
    \item \emph{High resolution:} The printer should produce high-resolution models with fine details.
    \item \emph{Ease of use:} The printer should be easy to use, set up, and maintain.
    \item \emph{Compatibility:} The printer should be compatible with a wide range of software and file formats.
    \item \emph{Cost:} The printer should be affordable and within the school or institution's budget.
    \item \emph{Safety features:} Enclosed designs and automatic bed leveling for safer operation in educational environments.
    \item \emph{Reliability:} Consistent performance with minimal maintenance requirements.
\end{itemize}

3D printing can help visually impaired students learn a variety of disciplines such as engineering, manufacturing, food, art, and health~\cite{Karbowski2020}. 3D printed models can benefit both blind and sighted students, allowing for multisensory learning and independence~\cite{MatterHackers2017}.

Table \ref{tab:chapter5:3d-printer-comparison} lists current available 3D printers with updated pricing.

\tagpdfsetup{table/header-rows={1}}
\centering
\begin{longtblr}[
  caption = {Comparison of 3D printers: model, cost, print bed size, filament size, and manufacturer},
  label = {tab:chapter5:3d-printer-comparison},
  note = {Detailed comparison of entry to mid-range 3D printers suitable for educational use, including print specifications and pricing. *Prices as of July 2025; tariffs have affected pricing significantly.}
]{
  colspec = {X[l] X[l] X[l] X[l] X[l]},
  rowhead = 1,
  hlines,
  stretch = 1.5
}
Model & Cost & Print Bed Size & Filament Size & Manufacturer \\
Bambu A1 Mini & \$249 & 180x180x180mm & 1.75mm & Bambu Lab \\
Elegoo Neptune 3 Pro & \$170 & 225x225x280mm & 1.75mm & Elegoo \\
Creality Ender-3 V3 KE & \$269 & 220x220x250mm & 1.75mm & Creality \\
Ender 3 Max Neo & \$389 & 300x300x320mm & 1.75mm & Creality \\
Ender 5 Plus & \$629 & 350x350x400mm & 1.75mm & Creality \\
Creality K1 & \$649 & 220x220x256mm & 1.75mm & Creality \\
Anycubic Kobra Max & \$619 & 450x400x400mm & 1.75mm & Anycubic \\
Anycubic Kobra Plus & \$549 & 300x300x350mm & 1.75mm & Anycubic \\
AnkerMake M5C & \$429 & 220x220x250mm & 1.75mm & AnkerMake \\
Prusa Mini+ & \$499 & 180x180x180mm & 1.75mm & Prusa \\
Elegoo Neptune 3 Max & \$520 & 420x420x500mm & 1.75mm & Elegoo \\
Elegoo Neptune 4 Pro & \$380 & 225x225x265mm & 1.75mm & Elegoo \\
Anycubic Kobra S1 Combo & \$829 & 220x220x270mm & 1.75mm & Anycubic \\
Artillery Sidewinder X2 & \$449 & 300x300x396mm & 1.75mm & Artillery \\
\end{longtblr}

\tagpdfsetup{table/header-rows={1}}
\centering
\begin{longtblr}[
  caption = {Premium 3D printers: model, cost, print bed size, filament size, and manufacturer.},
  label = {tab:chapter5:3d-printer-comparison-2},
  note = {Premium and specialized 3D printers with advanced features for educational institutions. Includes printers with enclosures and environmental control. Prices reflect current market conditions.}
]{
  colspec = {X[l] X[l] X[l] X[l] X[l]},
  rowhead = 1,
  hlines,
  stretch = 1.5
}
Model & Cost & Print Bed Size & Filament Size & Manufacturer \\
Bambu P1P & \$749 & 256×256×256mm & 1.75mm & Bambu Lab \\
Bambu P1S (Combo) & \$699 & 256×256×256mm & 1.75mm & Bambu Lab \\
Bambu X1C Carbon (Combo) & \$1,299 & 256×256×256mm & 1.75mm & Bambu Lab \\
Bambu A1 & \$599 & 256×256×256mm & 1.75mm & Bambu Lab \\
Prusa MK4 & \$829 & 250×210×220mm & 1.75mm & Prusa \\
Prusa MK4 Kit & \$749 & 250×210×220mm & 1.75mm & Prusa \\
Prusa Core ONE & \$1,299 & 250×220×270mm & 1.75mm & Prusa \\
Creality K2 Plus & \$1,299 & 350×350×400mm & 1.75mm & Creality \\
Creality K2 Plus (with CFS) & \$1,499 & 350×350×400mm & 1.75mm & Creality \\
\end{longtblr}

\section{Web Resources for 3D Print Files and Accessibility}\label{3d-print-web-resources}

\emph{Designed For VI Specifically}
\begin{itemize}
    \item BTactile, Benetech ImageShare, Median Augenbit, Tactiles, See3D, Accessible3D
\end{itemize}

\emph{Math Curricula}
\begin{itemize}
    \item Nonscriptum Calculus, Geometry, Trigonometry, Tactile Math Models
\end{itemize}

\emph{Astronomy/Physics}
\begin{itemize}
    \item 3D Opal, Astrokit, NASA, Roving Bits Constellations, Tactile Universe, ESA 3D Models
\end{itemize}

\emph{Biology}
\begin{itemize}
    \item 3D Biology, NIH 3D Print Collections/Models, Tactile Anatomy Models
\end{itemize}

\emph{General User-Uploaded 3D Print File Collections}
\begin{itemize}
    \item Printables, Thingiverse, My Mini Factory, Cults 3D, Thangs, MakerWorld, GrabCad, Instructables, Pinshape, Sketchfab, 3D Warehouse, Traceparts, Turbo Squid, YouMagine
\end{itemize}

\emph{3D File Search Aggregators}
\begin{itemize}
    \item Thangs3D, Yeggi, 3D Find It, 3D Print Shelf, 3DSourced, STL Finder, STLBase, MakerOnline, Mito3D, Open 3D Model, SeekSTL, Creazilla, Free3d
\end{itemize}

\emph{AI 3D Model Generation}
\begin{itemize}
    \item Meshy.ai, Luma AI Genie, Sloyd, Kaedim, Spline AI, Masterpiece Studio, 3D AI Studio
\end{itemize}

\emph{Professional Groups}
\begin{itemize}
    \item AT Makers, Makers Making Change, Enabling the Future, Thingiverse Assistive Technology
\end{itemize}

\emph{Visually Impaired Education and Accessibility Resources}
\begin{itemize}
    \item See3D, Accessible3D, MatterHackers Education, Braille Institute, American Foundation for the Blind, Paths to Literacy, 3D Print Accessibility Community
\end{itemize}

\section{3D Printer Materials}\label{d-printer-materials}
3D printing creates three-dimensional objects from computer-aided design (CAD) files. The process involves depositing materials layer by layer to build a shape~\cite{DassaultEducation}. To use a 3D printer in an educational environment, you need:

\begin{itemize}
    \item \emph{3D printer}: Available in various sizes, from benchtop to large-format, including models with enclosures/environmental control for improved reliability.
    \item \emph{Filament}: The material used to create the 3D object (e.g., PLA, TPU, ABS, PETG, etc.)~\cite{TechLearning2023}.
    \item \emph{Computer}: Required to create the 3D model using CAD software.
    \item \emph{CAD software}: Used to create the 3D model.
    \item \emph{Slicing software}: Converts the 3D model into a format the printer can understand and generates the G-code for printing~\cite{TeachThought2021}.
\end{itemize}

\emph{3D Printer Filament (PLA) and Color Resources}

\emph{FilamentColors} is a color checking program for popular PLA vendors, providing Hex codes for reproducible color accuracy. Not all vendors are available, but the list is growing.

\textit{Prices are for 1kg/2.2lb basic PLA, default with spool unless noted. Refills require a spool. Current pricing reflects market conditions as of July 2025, including tariff impacts on non-US suppliers.}

\emph{International Suppliers (prices affected by tariffs):}
\begin{itemize}
    \item Bambu Labs: \$28 (\$23 with 4+ rolls) with spool; \$25 (\$20 with 4+ kg) for refills
    \item Creality: \$22 Soleyin Ultra PLA; \$25 Ender Fast PLA
    \item ELEGOO: \$18
    \item eSun: \$24
    \item Sunlu: \$25
    \item Polymaker: \$26
    \item OVERTURE: \$22
\end{itemize}

\begin{thebibliography}{99}

\bibitem{Karbowski2020} Karbowski, C. F. (2020). See3D: 3D Printing for People Who Are Blind. \textit{Journal of Science Education for Students with Disabilities, 23}(1), n1. Available at: \url{http://files.eric.ed.gov/fulltext/EJ1247154.pdf}.

\bibitem{MatterHackers2017} MatterHackers. (2017). 3D printed educational models for the visually impaired. Available at: \url{http://www.matterhackers.com/articles/3d-printed-educational-models-for-the-visually-impaired}.

\bibitem{DassaultEducation} Dassault Systèmes. (n.d.). 3D printing in education. Retrieved December 19, 2023. Available at: \url{http://www.3ds.com/make/solutions/industries/3d-printing-education}.

\bibitem{TechLearning2023} Tech \& Learning. (2023). Best 3D printers for schools. Retrieved December 19, 2023. Available at: \url{http://www.techlearning.com/buying-guides/best-3d-printers-for-schools}.

\bibitem{TeachThought2021} TeachThought. (2021). 10 ways 3D printing can be used in education. Retrieved December 19, 2023. Available at: \url{http://www.teachthought.com/technology/ways-3d-printing-can-be-used-in-education/}.

\end{thebibliography}

\emph{Manufactured in the USA (minimal tariff impact):}
Most US PLA is sourced from Natureworks LLC (Ingeo Line).
\begin{itemize}
    \item Polar Filament: \$19 (Basic PLA), \$22 (Premium colors)
    \item 3D Fuel: \$27
    \item American Filament: \$27 (\$14 500g refill)
    \item Atomic Filament: \$32
    \item Hatchbox: \$24
    \item MatterHackers: \$20+
    \item Overture 3D: \$25
    \item Polymaker: \$22
    \item ProtoPasta: \$21
    \item Push Plastic: \$26
    \item Printed Solid: \$26
    \item Filastruder: \$11 PLA, \$13 PLA Pro
    \item Splice 3D: \$17/spool (bulk: \$14 w/4+, \$12 w/8+, \$11 w/24+)
    \item ZYLTech: \$19
    \item Gizmo Dorks: \$25
    \item IC3D: \$31
    \item Keene Village Plastics: \$32
    \item Numakers: \$22
    \item Paramount 3D: \$24 (\$21/8pack)
    \item VoxelPLA: \$18
    \item Toner Plastics: \$24
\end{itemize}

Table \ref{tab:table20} lists materials needed to use the 3D printers shown in Table \ref{tab:chapter5:3d-printer-comparison}.

\tagpdfsetup{table/header-rows={1}}
\centering
\begin{longtblr}[
  caption = {3D Printer Materials},
  label = {tab:table20},
  note = {Essential consumable materials and tools required for 3D printing in educational settings. Prices updated for July 2025 market conditions.}
]{
  colspec = {X[l] X[l] X[l]},
  rowhead = 1,
  hlines,
  stretch = 1.5
}
Item & Cost & Vendor \\
1.75mm filament (see above) & \$18--\$45/kg & Multiple (Bambu, Elegoo, Polar, 3D Fuel, etc.) \\
3D Print Tool Kit & \$65.00 & HIJIRH, Amazon \\
Assorted Sandpaper (48 pcs) & \$9.00 & Vicien, Amazon \\
Glue Sticks (30 pack) & \$12.00 & Amazon Basics \\
Painter's Tape (2" width 12 Pack) & \$48.00 & Amazon \\
Build Surface Cleaner & \$8.00 & Various \\
Nozzle Cleaning Kit & \$15.00 & Various \\
\end{longtblr}

\section{3D Printer Software}\label{d-printer-materials-program}
3D printing software allows users to create, edit, and slice 3D models. These programs enable users to design models, slice them into layers, and generate G-code for the printer.

\emph{Resources for Programs to Create 3D Models}

\textit{Free:}
\begin{itemize}
    \item Tinkercad: Browser-based, beginner-friendly, block-building interface.
    \item Fusion 360: Free for personal/educational use, professional features.
    \item FreeCAD: Open-source parametric modeler, improving rapidly.
    \item Blender: Open-source, steep learning curve, excellent for complex models.
    \item SketchUp Free: Web-based, good balance of usability and functionality.
    \item OpenSCAD: Script-based modeling, ideal for programmers.
    \item BRL-CAD: Advanced solid modeling, used by U.S. military.
    \item Wings3D: Open-source polygon modeler.
    \item 3D Slash: Fun, voxel-based interface.
\end{itemize}

\textit{Education Plans:}
\begin{itemize}
    \item Fusion 360: Free for students/educators, \$60/month otherwise.
    \item SolidWorks: Educational licenses available, \$99/year for students.
    \item Inventor: Educational pricing available.
    \item Shapr3D: Educational discounts available.
\end{itemize}

\textit{Professional:}
\begin{itemize}
    \item SolidWorks: \$1,395/yr or \$4,195 perpetual.
    \item Inventor: \$2,085/yr.
    \item Rhino3D: \$1,095, \$195 student.
    \item 3DS Max: \$1,645/yr.
    \item Maya: \$1,645/yr.
    \item Cinema 4D: \$770/yr or \$4,195 perpetual.
    \item Modo: \$639/yr or \$1,909 perpetual.
\end{itemize}

3D Print Slicing Programs
\begin{itemize}
    \item Bambu Studio: Default for Bambu Lab printers, advanced features.
    \item PrusaSlicer: Free, excellent for most FDM printers.
    \item Ultimaker Cura: Free, widely compatible.
    \item OrcaSlicer: Free, community-developed with advanced features.
    \item Simplify3D: \$149, commercial slicer with support.
    \item IdeaMaker: Free, by Raise3D.
    \item Repetier-Host: Free, includes slicing and printer control.
    \item KISSlicer: Free version available.
    \item 3DPrinterOS: Cloud-based slicing and management.
\end{itemize}

Table \ref{tab:table201} lists software and their functions with updated pricing.

\tagpdfsetup{table/header-rows={1}}
\centering
\begin{longtblr}[
  caption = {3D Printer Software and Functions},
  label = {tab:table201},
  note = {Available software tools for 3D modeling and printing, categorized by function and cost. Prices updated for July 2025.}
]{
  colspec = {X[l] X[l] X[l]},
  rowhead = 1,
  hlines,
  stretch = 1.5
}
Program & Cost & Function \\
Tinkercad & Free & Generate 3D file \\
Fusion 360 & Free (Education)/\$60/month & Generate 3D file \\
FreeCAD & Free & Generate 3D file \\
SolidWorks & \$1,395/yr/\$99 student & Generate 3D file \\
SketchUp Free & Free & Generate 3D file \\
Blender & Free & Generate 3D file \\
Rhino 7 & \$1,095/\$195 student & Generate 3D file \\
Bambu Studio & Free & Slice \& Print 3D Model \\
PrusaSlicer & Free & Slice \& Print 3D Model \\
Ultimaker Cura & Free & Slice \& Print 3D Model \\
OrcaSlicer & Free & Slice \& Print 3D Model \\
Simplify3D & \$149 & Slice \& Print 3D Model \\
Meshmixer & Free & Fix \& Modify 3D Print Files \\
Meshlab & Free & Fix \& Modify 3D Print Files \\
Netfabb & Free (Basic) & Fix \& Modify 3D Print Files \\
\end{longtblr}

\section{Recent Developments in 3D Printing for Accessibility}\label{recent-developments}

The field of 3D printing for visually impaired education continues to evolve rapidly. Recent developments include improved tactile design guidelines, better integration with assistive technologies, and enhanced software accessibility features. Educational institutions are increasingly recognizing the transformative potential of 3D printing technology for creating inclusive learning environments.

Key trends include:
\begin{itemize}
    \item Integration of AI tools for automatic generation of tactile models
    \item Development of specialized software with screen reader compatibility
    \item Increased collaboration between accessibility organizations and 3D printing communities
    \item Growing libraries of pre-designed accessible educational models
    \item Enhanced safety features in educational 3D printers
\end{itemize}

The market growth in 3D printing technology continues to drive down costs while improving accessibility, making these tools more available to educational institutions serving visually impaired students.
