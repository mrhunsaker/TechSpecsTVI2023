\chapter{Accessible Educational Programs \& Materials}\label{app4:instructional-programs}
\glsreset{ocr}\glsreset{icr}\glsreset{tts}\glsreset{llm}\glsreset{uia}\glsreset{msaa}\glsreset{pdfua}\glsreset{api}\glsreset{cpu}

\section{~~Accessible Touch Typing Instruction}\label{app4:typing-instruction}

\noindent
\textbf{Context:} Touch typing is a foundational skill for efficient computer use, particularly for \gidx{screenreader}{screen reader} users. The following programs are designed to be accessible and effective for learners with visual impairments\index{visual impairment}.

\begin{itemize}
	\item \textbf{Talking Typer:} A popular \gidx{software}{software} program that provides audible\index{audiobook!Audible} feedback for typing instruction, now available for both Windows and iOS\index{operating system!iOS} platforms. It includes structured lessons, practice drills, and games to engage learners of all ages \cite{TypeAbility2025}.
	\item \textbf{TypingClub:} An online platform that offers a comprehensive, accessible typing curriculum. It is compatible with screen readers\index{screen reader} and provides a structured learning path from basic keystrokes to advanced typing skills \cite{TypingClub2025}.
	\item \textbf{APH\gidx{brailleembosser}{braille embosser}\index{braille embosser!APH} Typer Online:} A free, web-based typing tutor from the American Printing House for the Blind that is designed for users with low vision and is also accessible to screen reader users \cite{APH2025}.
	\item \textbf{Accessibyte:} Offers a suite of cloud-based apps\index{apps}, including Typio, an accessible typing tutor that is designed to be fun and engaging for students \cite{Typio2025}.
\end{itemize}

Recent advancements in typing instruction include the integration of gamification and adaptive learning technologies, which can personalize the learning experience and maintain student engagement. Additionally, there is a growing emphasis on teaching typing on a variety of devices, including tablets\index{tablet} and smartphones, to reflect the diverse \gidx{technology}{technology} landscape that students will encounter.

\section{~~AndroidOS/iOS/iPadOS Gesture Training}\label{app4:gesture-training}

\noindent
\textbf{Context:} Proficiency with touchscreen gestures is essential for navigating modern mobile devices. The following resources provide structured training for VoiceOver\index{screen reader!VoiceOver} (iOS/iPadOS) and TalkBack\index{screen reader!TalkBack} (Android) gestures.

\begin{itemize}
	\item \textbf{Apple's VoiceOver Getting Started Guide:} Comprehensive documentation and built-in practice modules on iOS\index{operating system!iOS} and iPadOS devices that allow users to learn and practice VoiceOver gestures in a controlled environment.
	\item \textbf{Google's TalkBack Tutorial:} An interactive tutorial included in the Android Accessibility\index{accessibility} Suite that guides users through the essential TalkBack gestures for \gidx{navigation}{navigation} and interaction.
	\item \textbf{Hadley:} Offers a wide range of free workshops on using iOS and Android\index{operating system!Android} devices with screen readers\index{screen reader}, including detailed instruction on gestures and navigation.
	\item \textbf{AppleVis:} A community-driven website for blind and low-vision users of Apple\index{tablet!Apple} products, offering forums, guides, and tutorials on VoiceOver gestures and app \gidx{accessibility}{accessibility}.
\end{itemize}

\section[Screenreader Training]{Screenreader\index{screen reader} Training}\label{app4:screenreader-training}

\noindent
\textbf{Context:} Effective screen reader use is a critical skill for accessing digital information. The following resources offer comprehensive training for popular screen readers like JAWS, NVDA, and VoiceOver.

\begin{itemize}
	\item \textbf{Freedom Scientific\index{video magnifier!Freedom Scientific} Training:} Offers a variety of free and paid training resources for JAWS\index{screen reader!JAWS}, including webinars, tutorials, and certification programs.
	\item \textbf{NV Access:} Provides official documentation and community-based support for NVDA\index{accessibility!NVDA}. Third-party organizations like NVDA Certified Expert also offer structured training courses.
	\item \textbf{Hadley:} Provides an extensive library of free workshops on using JAWS, NVDA, and VoiceOver\index{screen reader!VoiceOver} for a wide range of tasks, from basic \gidx{navigation}{navigation} to advanced web browsing.
	\item \textbf{American Foundation for the Blind (AFB):} Offers articles, guides, and webinars on various aspects of screen reader use and digital \gidx{accessibility}{accessibility}\index{digital accessibility}.
\end{itemize}

\section[Screen Magnifier Training]{Screen Magnifier Training}\label{app4:magnifier-training}

\noindent
\textbf{Context:} For students with low vision, screen magnifiers are essential tools for accessing visual information on a computer. The following resources provide training on using \gidx{software}{software} like ZoomText and the built-in magnifiers in Windows\index{operating system!Windows} and macOS.

\begin{itemize}
	\item \textbf{Freedom Scientific Training:} Provides tutorials and webinars for ZoomText\gidx{magnification}{magnification}\index{magnification!ZoomText} and Fusion, covering features such as magnification levels, color enhancements, and screen reading capabilities.
	\item \textbf{Microsoft\index{tablet!Microsoft}'s Windows Magnifiermagnification\index{magnification!Windows Magnifier} Guide:} Official documentation on how to use the built-in Magnifier in Windows, including keyboard shortcuts and customization options.
	\item \textbf{Apple\index{tablet!Apple}'s Zoom Accessibility Guide:} Detailed instructions on using the Zoommagnification\index{magnification!Zoom} feature in macOS and iOS\index{operating system!iOS} to magnify screen content.
	\item \textbf{Hadley:} Offers workshops on using screen magnification tools to read, write, and access digital content effectively.
\end{itemize}

\section[\gidx{brailledisplay}{Braille Display} Use]{Braille Display Use}\label{app4:braille-display-use}

\noindent
\textbf{Context:} Refreshable braille displays provide tactile access to digital text, which is crucial for literacy and STEM\index{STEM} education. The following resources support the development of \gidx{brailledisplay}{braille display} skills.

\begin{itemize}
	\item \textbf{Paths to Literacy:} A collaboration between Perkins School for the Blind and Texas School for the Blind and Visually Impaired, offering articles, strategies, and lesson plans for teaching \gidx{brailledisplay}{braille display} use.
	\item \textbf{National Braille Press:} Publishes books and resources on \gidx{brailleliteracy}{braille literacy}, including guides on using braille \gidx{technology}{technology}.
	\item \textbf{Braille Authority of North America\index{BANA} (BANA):} Provides official guidance on \gidx{braille}{braille} codes and standards, which is essential for understanding how digital text is translated into braille.
	\item \textbf{Manufacturer Training:} Companies like HumanWare, HIMS, and Orbit Research\index{DAISY!Orbit Research} offer tutorials and support for their specific braille display models.
\end{itemize}

\section[Accessible Coding Curricula]{Accessible Coding Curricula}\label{app4:coding-curricula}

\noindent
\textbf{Context:} Coding and computer science are increasingly important fields, and accessible curricula are essential for including students with visual impairments\index{visual impairment}. The following resources are designed to teach coding in an accessible way.

\begin{itemize}
	\item \textbf{Code.org:} Has made significant strides in making its computer science curriculum accessible to \gidx{screenreader}{screen reader} users, offering block-based and text-based coding lessons.
	\item \textbf{Quorum Programming Language:} A programming language designed to be accessible from the ground up, with a syntax that is easy to read and understand with a screen reader.
	\item \textbf{Blocks4All:} An accessible block-based coding environment that allows students to learn programming concepts using a screen reader.
	\item \textbf{Apple\index{tablet!Apple}'s Swift Playgrounds:} An app for iPad and Mac that teaches coding in an interactive and accessible way, with support for VoiceOver\index{screen reader!VoiceOver}.
\end{itemize}

\section{~~Emerging Technologies and Future Directions}\label{app4:emerging-tech}

\noindent
\textbf{Context:} The field of \gidx{assistivetechnology}{assistive technology} is constantly evolving. It is important for educators and students to stay informed about emerging technologies and their potential applications.

\begin{itemize}
	\item \textbf{Artificial Intelligence\index{AI} (AI):} AI-powered tools, such as Seeing AI and Lookout by Google\index{tablet!Google}, are providing real-time descriptions of the visual world, which has significant implications for education and \gidx{independence}{independence}.
	\item \textbf{Virtual and Augmented Reality (VR/AR):} While still an emerging area, accessible VR and AR applications have the potential to offer new ways of learning and interacting with digital content.
	\item \textbf{Wearable Technology\index{technology}:} Smart glasses and other wearable devices are being developed to provide \gidx{navigation}{navigation} assistance, object recognition, and other forms of support for people with visual impairments\index{visual impairment}.
	\item \textbf{Haptic Feedback\index{haptic feedback}:} Advanced haptic technologies are being explored to create more immersive and informative tactile experiences, which could enhance digital art, data visualization, and more.
\end{itemize}

