\chapter{Instructional Programs \& Materials}\label{appx4}

\noindent
\textbf{Accessibility Note:} This appendix provides an overview of instructional programs and materials designed for students with visual impairments. The content and structure have been enhanced for clarity, navigation, and accessibility, including context for lists and resources.

The Individuals with Disabilities Education Improvement Act (IDEIA) 2004 mandates that students with disabilities receive a free and appropriate public education (FAPE) in the least restrictive environment possible \cite{IDEA2004}. To ensure that blind and low vision students have access to FAPE, there is a need for evidence-based specialized curriculum to teach screenreader usage, magnification usage, accessible typing programs, and accessible coding curricula to teach tech skills to blind/low vision students.

Screen readers are software programs that allow blind and visually impaired users to read the text that is displayed on a computer screen with a speech synthesizer or braille display \cite{PathsToLiteracy}. Magnification software enlarges the text and images on the screen for low vision users \cite{AFBMagnification}. Accessible typing programs help students with disabilities learn to type using adaptive technology. Accessible coding curricula teach blind and low vision students how to code using specialized software that is designed to be accessible to them \cite{FreeCodeCamp2018}.

Evidence-based specialized curriculum for teaching these skills is important because it ensures that students with disabilities have access to the same educational opportunities as their peers. It also helps to ensure that students with disabilities are able to develop the skills they need to succeed in the workforce. By providing students with disabilities with the tools they need to succeed, we can help to create a more inclusive society where everyone has the opportunity to reach their full potential.

\section{Accessible Touch Typing Instruction}\label{appx5}

Learning to touch type is an essential skill for anyone who spends a significant amount of time typing. Touch typing can help you type more efficiently and accurately, which can save you time and reduce the risk of repetitive strain injuries. By using all ten fingers to type without looking at the keyboard, you can significantly increase your typing speed and reduce the number of errors you make. This can help you complete your typing tasks faster and with greater accuracy. Additionally, touch typing can help you use keystroke shortcuts more smoothly, which can help you navigate your computer more quickly and efficiently. Screen readers can also be used more effectively when you are able to touch type, as you can focus on the content being read rather than the keyboard. In summary, learning to touch type can help you become a more efficient and fluent typist, as well as improve your ability to navigate your computer and screen readers.

\emph{You may be thinking: My blind child has a Braille device. Why does she need to learn to type?}

Even if your child has a Braille device such as the Braillenote Touch, typing is essential. The computer is the mainstream device that your child will need in order to be productive in school and in the workplace. When I meet a new blind student, parents often tell me, ``My child needs to learn to use a screen reader.'' The first question I ask is, ``Does your child know how to type?'' In order to use a screen reader such as JAWS effectively, you have to be able to type accurately. Braille is important, too, and it definitely has its uses in technology. But I believe that typing is as important as Braille.

Typing allows blind students to use mainstream devices. They can use a laptop or desktop computer, or they can connect a keyboard to a tablet. When I use my iPhone and type in text messages, my keyboarding skills help me use the screen, even without a Braille display.

\emph{-- Treva Olivero}  \cite{Olivero1997}

A variety of accessible touch typing programs are available to support students with visual impairments in developing essential keyboarding skills. The following list highlights some of the most widely used and accessible options, each referenced in the bibliography for further exploration:

\begin{itemize}
 \item \href{https://www.accessibyte.com/typio-online-page/}{Typio} \cite{Typio2025}: An accessible online typing program designed specifically for blind and visually impaired students.
 \item \href{https://www.sonokids.org/ballyland-early-learning/ballyland-keyboarding/}{Ballyland Keyboarding} \cite{Ballyland2025}: A program focused on early learning and keyboarding skills for children with visual impairments.
 \item \href{https://nelowvision.com/product/typeability-typing-and-computer-tutor-program-for-the-blind-and-visually-impaired/}{TypeAbility} \cite{TypeAbility2025}: A comprehensive typing and computer tutor program for blind and visually impaired users.
 \item \href{https://saomaicenter.org/en/smsoft/smtt}{Sao Mai Typing Tutor} \cite{SaoMai2025}: A free typing tutor developed for people with visual impairments.
 \item \href{https://www.cfb.state.nm.us/apps/}{Keystroke} \cite{Keystroke2025}: An accessible typing program from the New Mexico Commission for the Blind.
 \item \href{https://typer.aphtech.org/}{APH Typer Online} \cite{APH2025}: An online typing program from the American Printing House for the Blind.
 \item \href{https://www.typingclub.com/}{Typing Club} \cite{TypingClub2025}: A mainstream typing program with accessibility features for blind and low vision users.
 \item \href{https://www.readandspell.com/us/typing-for-the-blind}{Touch-Type Read and Spell (TTRS)} \cite{TTRS2025}: A literacy and typing program accessible to blind and visually impaired students.
 \item \href{https://kaz-type.com/visualimpairment}{KAZ Typing Software} \cite{KAZ2025}: Typing software with a dedicated version for users with visual impairments.
\end{itemize}



\section{AndroidOS/iOS/iPadOS Gesture Training}\label{appx6}

Learning VoiceOver and TalkBack gestures on tablets and phones is essential for users with visual impairments. VoiceOver is a screen reader that comes pre-installed on Apple devices, while TalkBack is a screen reader that comes pre-installed on Android devices \cite{BOIADevices,BOIATalkBack}. Both screen readers include gesture-based controls and braille keyboard support. While these screen readers are useful tools, they depend on accurate text alternatives for non-text content. Learning VoiceOver and TalkBack gestures can help users navigate their devices more efficiently and effectively \cite{AppleVoiceOver}. For instance, TalkBack gestures can help users navigate and perform frequent actions on their Android devices, such as moving to the next item on the screen, selecting an item, and activating screen search. \cite{GoogleTalkBack} Similarly, VoiceOver gestures can help users navigate and perform frequent actions on their Apple devices, such as opening the app switcher, accessing the control center, and activating Siri. Competency with VoiceOver and TalkBack gestures can enable users to access the same activities as their peers, manage eye fatigue, and use good posture and a good viewing distance.

\noindent
\textbf{Context:} The following resources provide accessible training and practice for learning gestures on Android and iOS/iPadOS devices. Each resource is referenced in the bibliography for further information.

\begin{itemize}
  \item \href{https://screenreader.app/}{ScreenReader App} \cite{ScreenReaderApp}: An interactive app for practicing and mastering screen reader gestures on both iOS and Android devices.
  \item \href{https://www.sonokids.org/ballyland-early-learning/ballyland-game-apps/}{Ballyland Apps}: A suite of accessible games and apps designed to teach gesture skills to children with visual impairments.
  \item \href{https://srt.csb-cde.ca.gov/}{The Screen Reader Training Website} \cite{SRTVoiceOver}: An online resource offering structured lessons for learning VoiceOver and TalkBack gestures.
  \item \href{https://hadley.edu/workshops/listen-with-talkback-series}{Listen with TalkBack Series from Hadley}: A series of audio workshops focused on using TalkBack on Android devices.
  \item \href{https://hadley.edu/workshops/listen-with-voiceover-series}{Listen with VoiceOver Series from Hadley}
 \end{itemize}

\hypertarget{appx7}{}\section[Screenreader Training]{Screenreader Training}\label{appx7}
Learning advanced methods of navigating the computer with a screen reader such as JAWS, Windows Narrator, or NVDA is essential for users with visual impairments. Recent developments in 2024-2025 show that NVDA continues to gain popularity, with NVDA 2025.1 introducing remote access capabilities and enhanced performance. JAWS 2025 includes significant performance optimizations and better compatibility with resource-intensive applications. While arrow keys and Tab can be useful for basic navigation, advanced methods can provide more efficient and comprehensive navigation. For instance, JAWS provides a feature called "Virtual Cursor" that allows users to navigate web pages and documents by line, word, character, or even by paragraph \cite{JAWSFS}. Similarly, Windows Narrator provides a feature called "Scan Mode" that allows users to navigate web pages and documents by headings, links, tables, and landmarks. \cite{MSNarratorGuide,MSWin11ScreenReader} NVDA provides a feature called "Object Navigation" that allows users to navigate web pages and documents by objects such as buttons, checkboxes, and text fields \cite{NVDAGuide}. Learning advanced methods of navigation can help users save time and effort, and increase productivity. It is important to note that while screen readers can be helpful, they should not replace other assistive technologies such as screen magnifiers. Therefore, it is important to learn advanced methods of navigating the computer with a screen reader to take full advantage of its benefits.

\noindent
\textbf{Context:} The following resources provide structured training and support for learning to use screen readers efficiently. Each resource is referenced in the bibliography for further details.

\begin{itemize}
 \item \href{https://www.freedomscientific.com/SurfsUp/}{Surf's Up} \cite{SurfsUp,SurfsUpOffline}: An interactive online tutorial for learning to browse the web with JAWS and other screen readers.
 \item \href{https://srt.csb-cde.ca.gov/}{The Screen Reader Training Website} \cite{SRTUpdate}: A comprehensive site with lessons and exercises for JAWS, NVDA, and VoiceOver.
 \item \href{https://hadley.edu/workshops/windows-narrator-series}{Windows Narrator Series from Hadley}: A series of workshops focused on using Windows Narrator for computer access.
 \item \href{https://hadley.edu/workshops/nvda-screen-reader-series}{NVDA Series from Hadley}: Workshops and tutorials for learning NVDA screen reader skills.
 \item \href{https://carroll.org/the-windows-screen-reader-primer-all-the-basics-and-more-second-edition/}{Windows Screen Reader Primer} \cite{WSRPrimer2ndEd,WSRPrimerCoverage}: A comprehensive guide to screen reader basics and advanced techniques.
 \item \href{https://www.blind.training/}{Access Technology Institute, LLC. Courses} \cite{ATITraining}: Online courses covering a range of access technology topics, including screen readers.
 \item \href{https://www.nvaccess.org/product/nvda-productivity-bundle/}{NVDA Training Materials} \cite{NVDATraining}: Official training materials and productivity guides for NVDA users.
 \item \href{https://support.freedomscientific.com/Training/JAWS-Basic-Training.zip}{JAWS Basic Training}
 \item \href{https://eyetvision.org/screen-reader-curriculum-landing-page/\#wwt2}{Working with Text from eyeTvision} \cite{EyetvisionWWT}
 \item \href{https://eyetvision.org/screen-reader-curriculum-landing-page/\#bin2}{Basic Internet Navigation from eyeTvision} \cite{EyetvisionBIN}
 \item \href{https://shop.nbp.org/products/windows-screen-reader-keystroke-compendium-2024-update}{Windows Screen Reader Keystroke Compendium 2024} \cite{WSRKeystrokeCompendium}
\end{itemize}

\hypertarget{appx11}{}\section[Screen Magnifier Training]{Screen Magnifier Training}\label{appx11}
Specialized screen magnification software like ZoomText, Fusion, Windows Magnifier, and Dolphin SuperNova are designed to provide a more comprehensive and customizable experience than the built-in magnification tools. While the built-in magnification tools can be useful for basic tasks, they may not be sufficient for users with more complex needs \cite{BOIAScreenMagnifiers}. Specialized software can provide features such as color filtering, cursor enhancements, and text-to-speech capabilities \cite{PerkinsScreenMagnification}. Additionally, specialized software can help users manage eye fatigue, use good posture and a good viewing distance, and access the same activities as their peers. Competency with specialized screen magnification software can enable students to succeed in postsecondary education and jobs \cite{AFBScreenMagnification}. It is important to note that while specialized screen magnification software can be helpful, it should not replace other assistive technologies such as screen readers. Therefore, it is important to learn how to use specialized screen magnification software to take full advantage of its benefits \cite{LowVisionCenter}.

\noindent
\textbf{Context:} The following resources provide training and support for learning to use specialized screen magnification software. Each resource is referenced in the bibliography for further details.

\begin{itemize}
  \item \href{https://support.freedomscientific.com/teachers/resources/ZoomText\_resources.zip}{ZoomText Resources from Freedom Scientific}: Downloadable training materials and guides for ZoomText, a leading screen magnification and screen reading software.
  \item \href{https://support.freedomscientific.com/teachers/resources/Fusion\_resources.zip}{Fusion Resources from Freedom Scientific}: Training resources for Fusion, which combines ZoomText and JAWS for users who need both magnification and speech.
  \item \href{https://yourdolphin.com/support/tutorials}{Dolphin Supernova Training Materials}: Online tutorials and support for Dolphin SuperNova, a suite of magnification and screen reading tools.
 \end{itemize}

\hypertarget{appx10}{}\section[Braille Display Use]{Braille Display Use}\label{appx10}
Learning how to use a refreshable braille display is essential for emerging braille readers. Refreshable Braille Displays are peripheral devices that display braille characters, usually by raising and lowering dots through holes in a flat surface. Users can input braille using either the 6 or 8 key Perkins-style braille keyboard or, more recently, a QWERTY keyboard. While it may be tempting to use only the minimum functions of a braille display, being explicitly taught how to use it can provide many benefits. For instance, it can help improve finger strength and isolated finger control, which are crucial for writing \cite{PerkinsBrailleDisplay}. Additionally, using a braille display can help emerging readers with tactile discrimination and make it easier to read. Furthermore, pairing a braille display with a computer, tablet, or smartphone can provide instant auditory feedback as the student types, which can help with motivation. Recent developments in 2024-2025 include innovations in haptic feedback technology and improvements in braille display connectivity.

\noindent
\textbf{Context:} The following resources provide training and support for learning to use refreshable braille displays. Each resource is referenced in the bibliography for further details.

\begin{itemize}
\item \href{https://view.officeapps.live.com/op/view.aspx?src=https\%3A\%2F\%2Fwww.wssb.wa.gov\%2Fsites\%2Fdefault\%2Ffiles\%2F2021-10\%2FUsing\%2520APH\%2520Mantis\%2520Q40.docx&wdOrigin=BROWSELINK}{APH Mantis Q40 Braille Display \& Notetaker from Washington School for the Blind}: A user guide for the APH Mantis Q40, covering setup and daily use.
\item \href{https://view.officeapps.live.com/op/view.aspx?src=https\%3A\%2F\%2Fwww.wssb.wa.gov\%2Fsites\%2Fdefault\%2Ffiles\%2F2023-07\%2FUsing\%2520APH\%2520Chameleon\%252020.docx&wdOrigin=BROWSELINK}{APH Chameleon 20 Braille Display \& Notetaker from Washington School for the Blind}: A user guide for the APH Chameleon 20, including tips for students and teachers.
\item \href{https://drive.google.com/drive/folders/1V\_hXjrsDeKUbNImA6Q77joADQbqMKKKl}{BrailleSense 6 Training from WCBVI AT}: Training materials for the BrailleSense 6, provided by Wisconsin Center for the Blind and Visually Impaired.
\item \href{https://drive.google.com/drive/folders/10HeixUb4E21nPLCStmnrsxLVehKThPP}{BrailleSense 6 Training from California School of the Blind}: Additional BrailleSense 6 training resources from CSB.
\item \href{https://drive.google.com/drive/folders/1OKBBdjbbD6asrE4dYyP7do9EWvY--5wf}{BrailleNote Touch Plus Training from California School of the Blind}: Training materials for the BrailleNote Touch Plus device.
\item \href{https://eyetvision.org/}{Diving Into Braille Displays from eyeTvision}: A collection of lessons and resources for learning to use braille displays in educational and daily contexts.
\end{itemize}

\hypertarget{appx8}{}\section[Accessible Coding Curricula]{Accessible Coding Curricula}\label{appx8}
For instance, the Perkins School for the Blind provides information on Quorum, an accessible programming language, as well as other resources and information related to blind programmers and coders \cite{PerkinsQuorum}. Additionally, EarSketch, a platform designed to teach students to code in Python or JavaScript through music and creative discovery, continues to be adapted for blind and visually impaired youth \cite{EarSketchGT}. Microsoft has also developed Code Jumper, a coding language for children who are blind or visually impaired, which is comprised of modular, physical pieces that students can string together to create code \cite{MSCodeJumper}. New accessible tools such as KaiBot and enhanced CodeQuest applications are being integrated into mainstream K-12 education. It's worth noting that blind people can be successful software developers, with 1 out of every 200 software developers being blind \cite{FreeCodeCamp2017}. With the right resources and support, blind students can learn computer programming and pursue a career in software development. \cite{HadwenBennett2018,Alotaibi2020}

\noindent
\textbf{Context:} The following resources provide accessible coding curricula and tools for students with visual impairments or blindness. Each resource is referenced in the bibliography for further details.

\begin{itemize}
  \item \href{https://zersiax.github.io/}{APH Connect Center Coding Course taught by Florian Beijers} \cite{BeijersAPH}: An accessible online coding course designed for blind and visually impaired learners.
  \item \href{https://codehs.com/}{CodeHS} \cite{CodeHS}: A mainstream coding platform with accessibility features and curricula for K-12 students.
  \item \href{https://www.codecademy.com/}{Code Academy} \cite{CodeAcademy}: An online platform offering accessible coding lessons in multiple programming languages.
  \item \href{https://codejumper.com/}{APH CodeJumper} \cite{APHCodeJumper,APHCodeJumperMS}: A physical coding kit developed for children who are blind or visually impaired, using tactile and auditory feedback.
  \item \href{https://www.aph.org/product/code-quest-for-ipad-iphone/}{Code Quest} \cite{CodeQuest}: An accessible coding game for iPad and iPhone that teaches programming concepts through interactive play.
  \item \href{https://www.aph.org/product/accessible-code-and-go-mouse/}{APH Code \& Go Mouse} \cite{CodeGoMouse}: A hands-on coding tool for young learners, designed to be accessible for students with visual impairments.
  \item \href{https://earsketch.gatech.edu/landing/}{EarSketch} \cite{EarSketchLanding,EarSketchTeaching}: A platform that teaches coding through music creation, with accessibility features for blind and low vision students.
  \item \href{https://code.org/accessibility}{Code.org Accessibility Resources} \cite{CodeOrgAccessibility}: Accessibility resources and guidance for using Code.org's curriculum with students who are blind or visually impaired.
  \item \href{https://quorumlanguage.com/}{Quorum Programming Language} \cite{QuorumLanguage}: An accessible programming language and curriculum specifically designed for blind and visually impaired students.
  \item KaiBot \cite{KaiBot}: An accessible educational robot and coding platform with tactile and auditory feedback for blind and low vision learners.
\end{itemize}

\section{Emerging Technologies and Future Directions}\label{appx12}

The landscape of assistive technology for students with visual impairments continues to evolve rapidly. Recent developments in 2024-2025 include significant advances in AI-powered accessibility tools, haptic feedback devices, and innovative approaches to digital content access. These emerging technologies promise to further enhance educational opportunities for blind and low vision students.

Key developments include:
\begin{itemize}
  \item AI-powered accessibility testing tools such as Axe DevTools AI for automated web content accessibility
  \item Enhanced haptic feedback devices for accessing digital content and touchscreen interfaces
  \item Innovations in braille technology and display connectivity
  \item Integration of accessible coding tools into mainstream K-12 curricula
  \item Development of more sophisticated screen reader remote access capabilities
  \item Advances in voice-controlled interfaces and natural language processing for educational applications
 \end{itemize}

Educational institutions and assistive technology providers continue to collaborate on developing more inclusive and accessible learning environments. The emphasis on universal design principles ensures that accessibility improvements benefit all students, not just those with visual impairments. As these technologies mature, they will provide even more opportunities for students with visual impairments to participate fully in educational and professional activities.

\begin{thebibliography}{99}
\bibitem{BOIADevices} Bureau of Internet Accessibility. Assistive Devices for the Visually Impaired. Available at: \url{https://www.boia.org/blog/assistive-devices-for-the-visually-impaired} [Accessed: July 2025].
\bibitem{BOIATalkBack} Bureau of Internet Accessibility. How to Use Android TalkBack. Available at: \url{https://www.boia.org/blog/how-to-use-android-talkback} [Accessed: July 2025].
\bibitem{AppleVoiceOver} Apple Inc. VoiceOver for iOS and macOS. Available at: \url{https://www.apple.com/accessibility/voiceover/} [Accessed: July 2025].
\bibitem{GoogleTalkBack} Google. TalkBack Screen Reader. Available at: \url{https://support.google.com/accessibility/android/answer/6007100} [Accessed: July 2025].
\bibitem{Olivero1997} National Federation of the Blind. (2019). The Braille Monitor, January 1997. Retrieved July 2025. Available at: \url{https://nfb.org/images/nfb/publications/fr/fr40/1/fr400103.htm}.
\bibitem{Typio2025} Accessibyte. Typio Online Page. Available at: \url{https://www.accessibyte.com/typio-online-page/}.
\bibitem{Ballyland2025} Sonokids. Ballyland Keyboarding. Available at: \url{https://www.sonokids.org/ballyland-early-learning/ballyland-keyboarding/}.
\bibitem{TypeAbility2025} TypeAbility. Typing and Computer Tutor Program for the Blind and Visually Impaired. Available at: \url{https://nelowvision.com/product/typeability-typing-and-computer-tutor-program-for-the-blind-and-visually-impaired/}.
\bibitem{SaoMai2025} Sao Mai Center for the Blind. Sao Mai Typing Tutor. Available at: \url{https://saomaicenter.org/en/smsoft/smtt}.
\bibitem{Keystroke2025} Commission for the Blind. Keystroke. Available at: \url{https://www.cfb.state.nm.us/apps/}.
\bibitem{APH2025} APH Typer Online. Available at: \url{https://typer.aphtech.org/}.
\bibitem{TypingClub2025} Typing Club. Available at: \url{https://www.typingclub.com/}.
\bibitem{TTRS2025} Touch-Type Read and Spell (TTRS). Available at: \url{https://www.readandspell.com/us/typing-for-the-blind}.
\bibitem{KAZ2025} KAZ Typing Software. Available at: \url{https://kaz-type.com/visualimpairment}.
\bibitem{IDEA2004} U.S. Department of Education. Individuals with Disabilities Education Act (IDEA), 20 U.S.C. § 1400, et seq. Available at: \url{http://sites.ed.gov/idea/statuteregulations/}.
\bibitem{PathsToLiteracy} Paths to Literacy. Introduction to Screen Reader Instruction. Available at: \url{https://www.pathstoliteracy.org/resource/introduction-screen-reader-instruction/}.
\bibitem{AFBMagnification} American Foundation for the Blind. Magnification Software. Available at: \url{https://www.afb.org/blindness-and-low-vision/using-technology/assistive-technology-videos/magnification-software}.
\bibitem{FreeCodeCamp2018} FreeCodeCamp.org. Helping blind people learn to code. (2018). Available at: \url{https://www.freecodecamp.org/news/helping-blind-people-learn-to-code-c47c68d4a237/}.
\bibitem{ScreenReaderApp} ScreenReader App. User-contributed information for Android TalkBack and iOS VoiceOver. Available at: \url{https://screenreader.app/}.
\bibitem{SRTVoiceOver} The Screen Reader Training Website. VoiceOver and TalkBack curriculum. Available at: \url{https://srt.csb-cde.ca.gov/}.
\bibitem{JAWSFS} Freedom Scientific. JAWS Screen Reader. Available at: \url{https://www.freedomscientific.com/products/software/jaws}.
\bibitem{MSNarratorGuide} Microsoft. Narrator User Guide. (2022, December 31). Available at: \url{https://support.microsoft.com/en-us/windows/narrator-user-guide-4b2e6b3f-1d6d-8a5c-4f6d2a3b3d6f}.
\bibitem{MSWin11ScreenReader} Microsoft. Use a screen reader to navigate Windows 11. (2022, December 31). Available at: \url{https://support.microsoft.com/en-us/windows/use-a-screen-reader-to-navigate-windows-11-5f8a9e7c-7d3e-2d5a-0f5c-5f9b5b8a7a3d}.
\bibitem{NVDAGuide} NV Access. NVDA User Guide. (2022, December 31). Available at: \url{https://www.nvaccess.org/files/nvda/documentation/userGuide.html#toc3.1}.
\bibitem{SurfsUp} Freedom Scientific. Surf's Up. Available at: \url{https://www.freedomscientific.com/SurfsUp/}.
\bibitem{SurfsUpOffline} Freedom Scientific. Surf's Up Offline Version. Available at: \url{https://support.freedomscientific.com/SurfsUp/ZIP_Surfs_Up.zip}.
\bibitem{SRTUpdate} The Screen Reader Training Website. Expanded curriculum for NVDA, JAWS, and VoiceOver. Available at: \url{https://srt.csb-cde.ca.gov/}.
\bibitem{WSRPrimer2ndEd} Carroll Center for the Blind. Windows Screen Reader Primer, 2nd Edition. Available at: \url{https://carroll.org/the-windows-screen-reader-primer-all-the-basics-and-more-second-edition/}.
\bibitem{WSRPrimerCoverage} Carroll Center for the Blind. Windows Screen Reader Primer Coverage. Covers use of Windows Narrator, NVDA, and JAWS.
\bibitem{ATITraining} Access Technology Institute, LLC. Courses and training for JAWS and NVDA. Available at: \url{https://www.blind.training/}.
\bibitem{NVDATraining} NV Access. NVDA Productivity Bundle. Available at: \url{https://www.nvaccess.org/product/nvda-productivity-bundle/}.
\bibitem{EyetvisionWWT} eyeTvision. Working with Text. Covers NVDA, JAWS, and ChromeVox Screenreaders. Available at: \url{https://eyetvision.org/screen-reader-curriculum-landing-page/#wwt2}.
\bibitem{EyetvisionBIN} eyeTvision. Basic Internet Navigation. Covers NVDA, JAWS, and ChromeVox Screenreaders. Available at: \url{https://eyetvision.org/screen-reader-curriculum-landing-page/#bin2}.
\bibitem{WSRKeystrokeCompendium} National Braille Press. Windows Screen Reader Keystroke Compendium 2024. Available at: \url{https://shop.nbp.org/products/windows-screen-reader-keystroke-compendium-2024-update}.
\bibitem{PerkinsBrailleDisplay} Perkins School for the Blind. Benefits of Using a Braille Display with Emerging Readers. Available at: \url{https://www.perkins.org/resource/benefits-using-braille-display-emerging-readers/}.
\bibitem{PerkinsQuorum} Perkins School for the Blind. Blind programmers and coders. Available at: \url{https://www.perkins.org/stories/blind-programmers-and-coders}.
\bibitem{EarSketchGT} Georgia Tech. EarSketch. (2022, August 24). Available at: \url{https://earsketch.gatech.edu/}.
\bibitem{MSCodeJumper} Microsoft. Code Jumper. Available at: \url{https://www.microsoft.com/en-us/research/project/code-jumper/}.
\bibitem{FreeCodeCamp2017} FreeCodeCamp.org. How blind people code. (2017, November 14). Available at: \url{https://www.freecodecamp.org/news/how-blind-people-code-fdb64e3bf5c/}.
\bibitem{HadwenBennett2018} Hadwen-Bennett, Alex \& Sentance, Sue \& Morrison, Cecily. (2018). Making Programming Accessible to Learners with Visual Impairments: A Literature Review. International Journal of Computer Science Education in Schools. 2. 10.21585/ijcses.v2i2.25. Available at: \url{https://files.eric.ed.gov/fulltext/EJ1207407.pdf}.
\bibitem{Alotaibi2020} Alotaibi, Hind \& Al-Khalifa, Hend \& AlSaeed, Duaa. (2020). Teaching Programming to Students with Vision Impairment: Impact of Tactile Teaching Strategies on Student's Achievements and Perceptions. Sustainability. 12. 10.3390/su12135320. Available at: \url{https://www.mdpi.com/2071-1050/12/13/5320}.
\bibitem{BeijersAPH} APH Connect Center Coding Course taught by Florian Beijers. Archived lectures and materials. Available at: \url{https://zersiax.github.io/}.
\bibitem{CodeHS} CodeHS. Mainstream coding platform with enhanced accessibility features. Available at: \url{https://codehs.com/}.
\bibitem{CodeAcademy} Code Academy. Popular coding platform with screen reader compatibility. Available at: \url{https://www.codecademy.com/}.
\bibitem{APHCodeJumper} APH CodeJumper. Coding language for blind/VI children. Available at: \url{https://codejumper.com/}.
\bibitem{APHCodeJumperMS} Microsoft. Code Jumper Training Module. Available at: \url{https://docs.microsoft.com/en-us/learn/modules/code-jumper-inclusive-physical-coding-language/}.
\bibitem{CodeQuest} APH. Code Quest for iPad/iPhone. Accessible coding app. Available at: \url{https://www.aph.org/product/code-quest-for-ipad-iphone/}.
\bibitem{CodeGoMouse} APH. Code \& Go Mouse. Physical coding toy for blind students. Available at: \url{https://www.aph.org/product/accessible-code-and-go-mouse/}.
\bibitem{EarSketchLanding} Georgia Tech. EarSketch Landing Page. Available at: \url{https://earsketch.gatech.edu/landing/}.
\bibitem{EarSketchTeaching} Georgia Tech. EarSketch Teaching Resources. Available at: \url{https://www.teachers.earsketch.org/learn}.
\bibitem{CodeOrgAccessibility} Code.org. Accessibility Resources. Available at: \url{https://code.org/accessibility}.
\bibitem{QuorumLanguage} Quorum Programming Language. Accessible programming language for blind students. Available at: \url{https://quorumlanguage.com/}.
\bibitem{KaiBot} KaiBot. Accessible robotics platform for K-12 coding education.
\bibitem{BOIAScreenMagnifiers} Bureau of Internet Accessibility. Screen Magnifiers: Who and How They Help. Available at: \url{https://www.boia.org/blog/screen-magnifiers-who-and-how-they-help}.
\bibitem{PerkinsScreenMagnification} Perkins School for the Blind. Getting started with screen magnification. Available at: \url{https://www.perkins.org/resource/getting-started-screen-magnification/}.
\bibitem{AFBScreenMagnification} American Foundation for the Blind. Screen Magnification. Available at: \url{https://www.afb.org/blindness-and-low-vision/using-technology/screen-magnification}.
\bibitem{LowVisionCenter} Low Vision Center. Introduction to Screen Reading and Magnification Software: A Comprehensive Guide to Assistive Technology Assessment. Available at: \url{https://nelowvision.com/introduction-to-screen-reading-and-magnification-software-a-comprehensive-guide-to-assistive-technology-assessment/}.
\end{thebibliography}
