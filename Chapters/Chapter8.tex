\chapter{Navigating Independence: The Essential Role of Accessible Daily Living Technology in Empowering Visually Impaired Students for Success and Safety}\label{accessible-gps-mapping}

In the pursuit of independence and safety, orientation and mobility training holds a pivotal place in the educational journey of visually impaired students. In this dynamic landscape, accessible GPS equipment emerges as a technological beacon, offering a transformative bridge to mobility, autonomy, and enhanced safety. This chapter explores the indispensable role that accessible GPS tools play in empowering visually impaired students for success, ensuring safe navigation through the world, and fostering a sense of confidence in their daily lives.

The quest for independence is intricately tied to the ability to navigate and explore the surrounding environment. For visually impaired students, this journey is often met with challenges that extend beyond the typical obstacles encountered in education. Accessible GPS equipment becomes a critical ally, providing not only the means to explore the world independently but also enhancing safety through reliable navigational assistance.

As we delve into this chapter, we will explore the functionalities of accessible GPS devices tailored to the unique needs of visually impaired users. From real-time audible directions to haptic feedback systems, these tools extend beyond standard navigation, creating a multi-sensory experience that empowers students to traverse their surroundings confidently. The importance of this technology is accentuated during orientation and mobility training, where students learn not only to navigate physical spaces but also to develop crucial skills for safety and situational awareness.

Beyond the practicalities of navigation, the impact of accessible GPS equipment on student success cannot be overstated. These tools contribute to broader educational goals by fostering a sense of independence, reducing reliance on external assistance, and instilling a foundational skill set for safe and self-assured mobility.

Through this exploration, it becomes clear that accessible GPS equipment is not merely a tool for navigation; it is a catalyst for empowerment and safety. Through orientation and mobility training, we ensure that visually impaired students can embark on their educational journeys with a sense of autonomy, confidence, and, above all, safety.

\section{Accessible GPS Hardware}\label{accessible-gps-mapping-hardware}
When purchasing an accessible GPS unit for the blind, it is important to consider the following factors to ensure safe navigation and crossing of streets:
\begin{itemize}
 \item \emph{Audible signals}: The GPS unit should provide audible signals to indicate when it is safe to cross the street. This feature allows blind pedestrians to cross the road at the right time, more quickly and safely while maintaining their orientation throughout the crossing\footnote{\href{http://www.inclusivecitymaker.com/pedestrian-safety-visually-impaired-blind-people/}{Inclusive City Maker. (n.d.). Pedestrian safety: Are your crossings safe for the visually impaired? Retrieved December 19, 2023}}.
 \item \emph{Compatibility}: The GPS unit should be compatible with other assistive technology devices, such as screen readers and braille displays\footnote{\href{http://www.afb.org/blindness-and-low-vision/using-technology/smartphone-gps-navigation-people-visual-impairments}{American Foundation for the Blind. (n.d.). Smartphone GPS navigation. Retrieved December 19, 2023}}.
 \item \emph{Portability}: Portable GPS units are ideal for blind pedestrians who need to move around the city. They should be lightweight and easy to carry.
 \item \emph{Battery life}: Battery life is an important consideration for portable GPS units. The battery should last long enough to get through a day without needing to be recharged.
 \item \emph{Ease of use}: The GPS unit should be easy to use and adjust. It should have large buttons and controls that are easy to locate and operate.
\end{itemize}
These considerations will help ensure that blind pedestrians have access to the tools they need to navigate and cross streets safely. \emph{Table \ref{tab:chapter8:accessible-gps-hardware}} lists current available accessible GPS hardware devices.

\tagpdfsetup{table/header-rows={1}}
\centering
\begin{longtblr}[
  caption = {Accessible GPS hardware: model, function, and company},
  label = {tab:chapter8:accessible-gps-hardware},
  note = {Available GPS navigation devices designed for visually impaired users, including specialized features like haptic feedback and audio output}
]{
  colspec = {X[l] X[l] X[l]},
  rowhead = 1,
  hlines,
  stretch = 1.5,
}
Model & Function & Company \\
Stellar Trek & GPS & Humanware \\
Victor Reader Trek & GPS + Digital Audio Player & Humanware \\
Wayband & GPS (Haptic Output) & WearWorks \\
\end{longtblr}

\section{Accessible Technology for Daily Living}\label{ind-living}
Auditory feedback technology is essential for blind people to live independently and complete daily tasks. It provides a way for the visually impaired to interact with their environment and receive information that they would otherwise miss. For example, an auditory-based tool can be used to support totally blind people to check the lights in an autonomous and relatively simple way\footnote{\href{http://link.springer.com/article/10.1007/s12652-020-01944-w}{Leporini, B., Rosellini, M., \& Forgione, N. (2020). Designing assistive technology for getting more independence for blind people when performing everyday tasks: an auditory-based tool as a case study. Journal of Ambient Intelligence and Humanized Computing, 11(6), 6107-6123.}}. This is just one example of how technology can be used to help the blind. Other examples include haptic feedback, which can be used to provide tactile feedback to the user, and voice recognition software, which can be used to control devices and appliances. These technologies can help the visually impaired to navigate their environment, communicate with others, and perform tasks that would otherwise be difficult or impossible.

By providing auditory feedback, technology can help the blind to live more independently and improve their quality of life. For instance, auditory-based tools can be used to support totally blind people to check the lights in an autonomous and relatively simple way\footnotemark[\value{footnote}]. This tool can be used to detect the presence of light and provide feedback to the user through sound. The study also proposed an idea that can be exploited in other application cases that use light feedback\footnote{\href{http://https://www.acb.org/content/accessible-pedestrian-signals-aps/}{American Council of the Blind. (n.d.). Accessible pedestrian signals (APS). Retrieved December 19, 2023}}. This is just one example of how technology can be used to help the blind. Other examples include haptic feedback, which can be used to provide tactile feedback to the user, and voice recognition software, which can be used to control devices and appliances. These technologies can help the visually impaired to navigate their environment, communicate with others, and perform tasks that would otherwise be difficult or impossible.

In addition to the benefits mentioned above, auditory feedback technology can also help the blind to learn new skills and improve their education. For example, a study published in Frontiers in Neuroscience showed how haptic feedback can be used to help blind people learn Braille\footnote{\href{http://www.frontiersin.org/articles/10.3389/fnins.2020.00528/full}{Fleury, M., Lioi, G., Barillot, C., \& Lécuyer, A. (2020). A Survey on the Use of Haptic Feedback for Brain-Computer Interfaces and Neurofeedback. Frontiers in Neuroscience, 14. doi.org/10.3389/fnins.2020.00528}}. The study found that haptic feedback can help the user to learn Braille faster and more accurately than traditional methods. This is just one example of how technology can be used to help the blind to learn new skills and improve their education. By providing auditory feedback, haptic feedback, and voice recognition software, technology can help the visually impaired to live more independently, improve their quality of life, and learn new skills.

\subsection{Accessible Home Technology}\label{ind-living-tools}
When purchasing household items modified to give audio feedback for the blind, it is important to consider the following factors to ensure that they can access activities of daily living\footnote{\href{http://www.allaboutvision.com/resources/adapting-the-home-better-blindness-accessibility/}{All About Vision. (n.d.). Adapting your home for better blindness accessibility. Retrieved December 19, 2023}}:
\begin{itemize}
 \item \emph{Audible feedback}: Household items should provide audible feedback to the user to ensure that they are being used correctly and safely.
 \item \emph{Compatibility}: The item should be compatible with other assistive technology devices, such as screen readers and braille displays.
 \item \emph{Ease of use}: The item should be easy to use and adjust. It should have large buttons and controls that are easy to locate and operate.
 \item \emph{Portability}: Portable items are ideal for blind users who need to move around the house. They should be lightweight and easy to carry.
 \item \emph{Cost}: The cost of the item should be reasonable and within the user’s budget.
\end{itemize}
These considerations will help ensure that blind users have access to the tools they need to perform activities of daily living safely and independently.

\emph{Table \ref{tab:chapter8:accessible-home-technology}} shows a range of technology available for blind/visually impaired people designed to facilitate independent living\footnote{Prices from either \href{http://www.braillebookstore.com/}{The Braille Bookstore} or \href{http://www.maxiaids.com/}{Maxi-Aids}, two major vendors of products intended to facilitate independent living skills}.

\tagpdfsetup{table/header-rows={1}}
\centering
\begin{longtblr}[
  caption = {Accessible home technology: model and cost.},
  label = {tab:chapter8:accessible-home-technology},
  note = {Comprehensive list of accessible household devices with audio feedback for independent living, including medical, kitchen, and measurement tools}
]{
  colspec = {X[l] X[l]},
  rowhead = 1,
  row{1} = {font=\bfseries},
  hlines,
  stretch = 1.5,
}
Model & Cost \\
Infrared Talking thermometer & \$45 \\
Liquid Level Indicator & \$10 \\
PenFriend Voice Labelling System & \$170 (Extra 418 labels: \$30) \\
Talking First Aid Guide & \$35 \\
Talking Indoor/Outdoor Thermometer & \$15 \\
Talking Kitchen Scale & \$35 \\
Talking Measuring Tape & \$145 \\
Talking Meat Thermometer & \$40 \\
Talking Timer Clock & \$15 \\
Talking Watch & \$15 \\
Talking Weighing Scale & \$35 \\
Talking Pulse Oximeter & \$32 \\
Talking Scale (Body Weight) & \$70 \\
Talking Blood Pressure Monitor & \$135 \\
Talking Pill System & \$70 \\
Talking Blood Glucose Meter & \$38 \\
WayLink Scanner & \$125 (Extra 25 magnets: \$40) \\
\end{longtblr}
