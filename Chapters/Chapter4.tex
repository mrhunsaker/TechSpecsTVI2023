\chapter{Empowering Minds: The Crucial Role of High-Quality Braille Embossers in Unlocking STEM Literacy for Visually Impaired Students}\label{generation}

In the ever-evolving realms of Science, Technology, Engineering, and Mathematics (STEM), the pursuit of literacy takes on a particularly intricate form. For visually impaired students, the challenges are multifaceted, but with the advent of high-quality braille embossers, a transformative bridge has been constructed. This chapter explores the indispensable role that high-quality braille embossers play in shaping the educational narrative of visually impaired students, especially in the critical domains of Math and STEM. These devices, with their ability to translate complex symbols and notations into tangible braille and tactile graphics, foster literacy, comprehension, and success in STEM fields.

The crux of this exploration lies in recognizing the nuanced requirements of visually impaired students pursuing education in Math and STEM disciplines. Traditional print materials, laden with intricate diagrams, mathematical symbols, and graphs, pose formidable challenges for learners with visual impairments. High-quality braille embossers bridge this gap, converting abstract mathematical concepts and scientific data into tangible formats, empowering students to actively engage with and comprehend the intricacies of STEM subjects.

Embossed tactile graphics break down the barriers to understanding complex mathematical equations, graphical representations, and scientific concepts, ultimately fostering a sense of autonomy and empowerment among visually impaired students. By providing access to the visual nuances inherent in STEM fields, these devices pave the way for literacy, comprehension, and active participation, ensuring that visually impaired students can unlock the full spectrum of opportunities in Math and STEM disciplines.

\section{Braille Embossers}\label{embossers}
Having access to a high-quality braille embosser is essential for students with visual impairments to receive a free and appropriate public education. Braille embossers are printers that produce braille text and tactile graphics on paper. They are used to create braille copies of textbooks, worksheets, and other educational materials. High-quality embossers produce sharp, clear braille that is easy to read and tactile graphics that are easy to interpret. This is important because it allows students with visual impairments to access the same educational materials as their sighted peers. Braille embossers also allow students to create their own braille notes and written work, which can help improve their literacy skills and independence. By providing students with visual impairments access to high-quality braille embossers, we can help ensure that they have the tools they need to succeed in their studies and beyond. \emph{Table \ref{tab:chapter4:braille-embossers}} lists current available embossers\footnote{I am only focusing on 11x11.5'' braille paper size as US Letter size is impractical for braille}.

\tagpdfsetup{table/header-rows={1}}
\centering
\begin{longtblr}[
  caption = {Braille embosser comparison: machine, capability, and company},
  label = {tab:chapter4:braille-embossers},
  note = {Comprehensive comparison of current braille embossers, highlighting their graphics capabilities and interpoint braille features for educational use}
]{
  colspec = {X[l] X[l] X[l]},
  rowhead = 1,
  rowhead = 1,
  hlines,
  stretch = 1.5,
}
Machine & Capability & Company \\
APH PageBlaster (old Model Index-D V4) & Simple Graphics, Interpoint Braille & APH, Index Braille \\
Basic-D V5 & Simple Graphics, Interpoint Braille & Index Braille \\
BrailleTrac 120 & Simple Graphics, Interpoint Braille & Irie-AT \\
Juliet 120 & Simple Graphics, Interpoint Braille & ETS, Humanware \\
ViewPlus Columbia & Complex Graphics, Interpoint Braille & ViewPlus \\
ViewPlus Rogue (old ViewPlus Max) & Complex Graphics, Interpoint Braille & ViewPlus \\
\end{longtblr}

\section{High Resolution Tactile Graphics}\label{tactile-graphics-high-resolution-complex-graphics}
There are some historical challenges that have befallen blind students that rely on tactile graphics and braille.
\begin{itemize}
 \item Historically, by the time students with visual impairments enter school, they have not received enough instruction in the development and use of their tactile skills or had enough opportunities to touch and explore their world.\footnote{\href{http://www.tsbvi.edu/tx-senseabilities/issues/fall-winter-2016/the-development-of-tactile-skills}{Adkins, A., Sewell, D., \& Cleveland, J. (2016). The Development of Tactile Skills. TX \emph{SenseAbilities, Fall/Winter}.}}
 \item Tactile Graphicacy requires the ability to access, comprehend, and produce tactile graphics or raised line drawings. This requires:
   \begin{itemize}
     \item Fine motor sensitivity and dexterity
     \item Efficient use of carefully constructed knowledge
     \item Variety of tactile-cognitive strategies
   \end{itemize}
 \item Students have to develop a perception that there are different kinds of symbolic information on a page with different kinds of meaning
 \item Students have to develop an ability to discriminate between different tactile surfaces and to draw meaning from them
 \item These are \emph{not} inherent or natural for braille readers as they require:
   \begin{itemize}
     \item Explicit attention
     \item Education
     \item Careful, systematic building of tactile exploratory and interpretive skills
   \end{itemize}
\end{itemize}

There are a number of benefits to having access to accessible tactile graphics in the classroom. These include:
\begin{itemize}
 \item Provides a focus for attention and perception
 \item Builds pathways to retain and memorize information
 \item Natural destination for conversation and social interaction
 \item Pictures invite and motivate a learner's curiosity and engagement
\end{itemize}
\emph{Table \ref{tab:table17}} lists current available embossers and other devices for creation of high resolution tactile graphics.

\tagpdfsetup{table/header-rows={1}}
\centering
\begin{longtblr}[
  caption = {High resolution tactile graphics embossers: machine and company.},
  label = {tab:table17},
  note{} = {Specialized embossers for high-resolution tactile graphics production, listing available models.}
]{
  colspec = {X[l] X[l]},
  rowhead = 1,
  hlines,
  stretch = 1.5,
}
Machine & Company \\
APH PixBlaster (old Model ViewPlus Columbia) & APH, ViewPlus \\
Basic-D V5 & Index Braille \\
EZ-Form Brailon Duplicator & American Thermoform \\
PIAF tactile embosser & Humanware \\
Swell Form Machine & American Thermoform \\
ViewPlus Columbia & ViewPlus \\
ViewPlus Delta & ViewPlus \\
ViewPlus Elite & ViewPlus \\
ViewPlus Premier & ViewPlus \\
\end{longtblr}

\section{Tactile Graphic Supplies}\label{tactile-paper}
\emph{Table \ref{tab:table18}} lists materials needed to use with the graphics devices shown in \emph{Table \ref{tab:table17}}.

\tagpdfsetup{table/header-rows={1}}
\centering
\begin{longtblr}[
  caption = {Paper supplies for Tactile Graphics Generation},
  label = {tab:table18},
  note = {Available paper supplies and media for different tactile graphics devices.}
]{
  colspec = {X[l] X[l]},
  rowhead = 1,
  hlines,
  stretch = 1.5,
}
Paper / Medium & Company \\
Brailon Thermoform Paper (for EZ-Form Duplicator) & American Thermoform \\
Swell Touch Paper (for Swell Form Machine) & American Thermoform \\
Tangible Magic capsule Paper (for PIAF tactile embosser) & Humanware \\
Tractor-Feed Braille Paper (for embossers) & APH \\
\end{longtblr}
