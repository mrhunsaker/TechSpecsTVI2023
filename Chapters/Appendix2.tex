\hypertarget{trouble2}{}\chapter[\hfill\break\raggedright Troubleshooting Braille Notebooks and Displays]{Troubleshooting Braille Notebooks and Displays}\label{trouble2}
\extramarks{Vision Department Technology Needs}{Appendix B: Troubleshooting Braille Notebooks and Displays}
\noindent\makebox[\linewidth]{\rule{\linewidth}{0.4pt}}
{\let\clearpage\relax\localtableofcontents}\newpage
\hypertarget{notebook2}{}\section{Braille Notetakers}\label{notebook}
If your Braille notetaker is not responding to user input, there are several things you can try to troubleshoot the issue. First, ensure that the device is properly charged and turned on. If the device is still not working, try restarting it by holding down the power button for a few seconds. If the problem persists, check the settings of the device to ensure that it is configured to work with your specific screen reader software. If none of these steps work, consult the user manual for your specific Braille notetaker or contact the manufacturer for further assistance.

If you are using a BrailleNote Touch Plus, you can also try resetting the device to its original settings by starting it in recovery mode\footnote{\raggedright \href{http://www.humanware.com/microsite/bntouch/faq.php}{Humanware. (n.d.). BrailleNote Touch - Frequently Asked Questions. Retrieved December 18, 2023}}. This will restore the device to its factory settings and may help to resolve any issues with the device. If you are still having trouble, you can try restoring the device to its default settings or adjusting the settings to be optimized for testing. If you are using a BrailleSense 6, you can try updating the firmware to the latest version and re-initialize factory defaults\footnote{\raggedright \href{http://hims-inc.com/wp-content/uploads/2023/11/BrailleSense-6-User-Manual-V2-1.pdf}{BrailleSense 6 User Manual, Chapter3.9, page 68}}. This will restore the device to its factory settings and may help to resolve any issues with the device. If you are still having trouble, you can try restoring the device to its default settings or adjusting the settings to be optimized for testing. If you are using a BrailleSense 6, you can try updating the firmware to the latest version and re-initialize factory defaults\footnotemark[\value{footnote}]. This may help to resolve any issues with the device and improve its performance.

In summary, if your Braille notetaker is not responding to user input, there are several steps you can take to troubleshoot the issue. These include checking the battery and power settings, restarting the device, checking the software settings, and consulting the user manual or contacting the manufacturer for further assistance. By following these steps, you can help ensure that your Braille notetaker is working properly and that you can continue to use it to access information and complete tasks.  
\pagebreak \hypertarget{display2}{}\section{Braille Displays}\label{display2}
If your refreshable braille display is not responding with the computer screen reader, there are several things you can try to troubleshoot the issue. First, ensure that the braille display is properly connected to the computer and that the drivers are installed correctly. If the display is still not working, try restarting the computer and the screen reader software. If the problem persists, check the settings of the screen reader software to ensure that it is configured to work with the braille display. If none of these steps work, consult the user manual for your specific braille display or contact the manufacturer for further assistance.
\pagebreak \hypertarget{report2}{}\section{Official Support Contact}\label{report2}
\begin{itemize}[leftmargin=*]
	\item HIMS: You can submit a technical support request, call 512-837-2000 weekdays between 8:30 AM and 5:30 PM CST, or send an email to \href{mailto:support@hims-inc.com}{HIMS Technical Support}  
	\item Humanware: You can submit a technical support request, call 1-800-722-3393 weekdays between 8:30 Am and 7:00 PM ET, or fill out the \href{http://store.humanware.com/hus/contact/}{Customer Support Contact Form} and select ``Technical Support'' as the subject. 
	\item Orbit Research: You can submit a technical support request, call 1-888-606-7248 from 9:00 AM - 5:00 PM ET, or send an email to \href{mailto:techsupport@orbitresearch.com}{Orbit Research Technical Support}  
	\item Eurobraille: You can submit a technical support request, call 331 55 26 91 00 (company is located in Madrid, France. Customer service speaks French, Spanish, and English) from 2:30AM - 6:00AM/7:30-11:30 AM ET or send an email to \href{mailto:econtact@eurobraille.fr}{Technical Support}  
	\item Nippon Telesoft: You can email \href{mailto:ts-email@telesoft.co.jp}{Nippon Telesoft Technical Service}  
	\item Nattiq Technologies: You can email \href{mailto:info@nattiq.com}{Nattiq Technologies Technical Support}  
	\item Notey the NoteTaker: You can search for technical support at \href{http://notey-project.com/2023/03/31/notey-forum-tech-support/}{The Notey the Notetaker Support Forum}. The individual components are subject to the support provided by the various companies. 
	\item Bristol Braille: You can email \href{mailto:support@bristolbraille.org}{Bristol Braille Technical Support}  
	\item Freedom Scientific: You can submit a technical support request, call 727-803-8600 weekdays 8:30Am-7:00PM ET, or fill out the \href{http://support.freedomscientific.com/Forms/TechSupport}{Freedom Scientific Technical Support Contact Page}  
	\item APH: You can submit a technical support request, call 800-223-1839 weekdays from 8:00AM - 8:00PM ET or send an email to \href{mailto:cs@aph.org}{APH Customer Service Support}  
	\item VisioBraille: You can submit a technical support request by emailing \href{mailto:service@visiobraille.de}{VisioBraille GmbH Service Department}  
	\item Help Tech: You can submit a technical support request by filling out the \href{http://www.helptech.eu/contact}{Help Tech Service Request Form}  
	\item Optelec: You can submit a technical support request by filling out the \href{http://in.optelec.com/dealers/contactform}{Optelec Customer Service Contact Form}  
\end{itemize}

\pagebreak \hypertarget{listserv2}{}\section{Community Support via ListServ}\label{listserv2}
Sometimes asking a listserv that talks about refreshable braille displays may give faster responses than contacting official customer support. This is because listservs are online communities where people with similar interests can share information and help each other out. Members of these communities are often experts in their field and can provide quick and accurate answers to questions. In contrast, customer support teams may have to follow a set of procedures and protocols before they can provide assistance. This can take time and may not always result in a satisfactory resolution. Additionally, customer support teams may not be available 24/7, whereas listservs are often active around the clock. However, it’s important to remember that listservs are not official sources of information and the advice given may not always be accurate or up-to-date. It’s always a good idea to verify information before acting on it.
\begin{itemize}[leftmargin=*]
	\item Braille Displays
	      \begin{itemize}[leftmargin=2em]
	      	\item \href{http://groups.io/g/Brailliant-BI-X-USERS/}{Brailliant-BI-X Users Support List}
	      	\item \href{http://groups.io/g/braille-display-users}{Braille Display Users}  
	      	\item \href{http://groups.io/g/braillenote}{BrailleNote Users}  
	      	\item \href{http://groups.io/g/nlsEReader/messages}{NLS e-Reader}  
	      	\item \href{http://groups.io/g/orbit-reader}{Orbit Reader Discussion group}  
	      	\item \href{http://www.freelists.org/list/braillecell}{Refreshable Braille \& tactile Graphics Devices}  
	      	\item \href{http://www.freelists.org/list/braille-sense}{Braille Sense Discussion}  
	      	\item \href{http://www.freelists.org/list/aphmantischameleonuser}{APH Mantis \& Chameleon User}  
	      	\item \href{http://www.freelists.org/list/aph\_dynamictactiledisplay\_announce}{Dynamic Tactile Display Announcements (APH Monarch)}  
	      	\item \href{http://groups.io/g/hims-notetakers-chat}{HIMS Notetakers}  
	      \end{itemize}
	\item General Technology (Braille Displays Discussed Frequently)
	      \begin{itemize}[leftmargin=2em]
	      	\item \href{http://groups.io/g/blindtechdiscuss/messages}{Blind tech Discuss}  
	      	\item \href{http://groups.io/g/tech-for-blind}{Tech For Blind}  
	      	\item \href{http://groups.io/g/blindadtech}{BlindADTech}  
	      	\item \href{http://groups.io/g/blind-techies/messages}{Blind Techies}  
	      \end{itemize}
\end{itemize}