\chapter{Comprehensive Report on Tools for Creating and Reading Accessible Mathematics and Scientific Materials}
\author{A PhD Researcher in AI and Accessibility Technologies}
\section{Abstract}
This report provides a comprehensive overview of commercial and open-source tools designed for creating and reading accessible mathematical and scientific content across various computing platforms. It highlights the critical role of foundational standards such as MathML and Web Content Accessibility Guidelines (WCAG) in ensuring universal access. The analysis delves into the specific features, benefits, and limitations of prominent tools, examining their capabilities in accessible content generation, visualization, and specialized braille transcription, including support for Nemeth and Unified English Braille (UEB) codes. A central theme emerging from this examination is the evolving landscape of accessibility, moving beyond mere accommodation to a proactive integration of inclusive design principles, often augmented by artificial intelligence. The report also underscores the complex interplay between proprietary ecosystems and open standards, as well as the imperative for end-to-end interoperability in delivering truly accessible STEM materials. Furthermore, it explores the integration of advanced AI-powered Optical Character Recognition (OCR) with tools like Marker to enhance the accessibility of mathematical content, delving into the technical aspects of math OCR, the role of MathML and LaTeX in accessible presentation, and the workflows for converting mathematical expressions into formats usable by assistive technologies such as screen readers and braille displays. The report also addresses the benefits, challenges, and ethical considerations in this evolving field.


\section{Introduction}
\subsection{The Imperative of Accessible Mathematical and Scientific Content}
The increasing digitalization of educational and scientific materials necessitates a strong focus on accessibility, particularly for complex content like mathematics. Ensuring mathematical content is presented in an accessible format is not merely a best practice; it is a legal requirement under various mandates, such as the Public Sector Bodies (Websites and Mobile Applications) (No.) regulations~\cite{PublicSectorBodiesRegulations2018} and compliance with the American with Disabilities Act (ADA) of 1990~\cite{ADA1990}. Students with visual impairments, dyscalculia, dysgraphia, and other learning or motor challenges often face significant barriers when interacting with traditional math formats~\cite{TexthelpEquatio}\cite{TexthelpEquatioPricing}\cite{ASUImageAccessibilityGenerator}\cite{PerkinsMathKicker}\cite{Modmath}\cite{UWAccessibleMath}.

The legal and ethical mandates for digital inclusion underscore a fundamental shift in the landscape of digital content creation. Accessibility is no longer an optional feature but a mandatory consideration, driving the demand for robust, standardized, and auditable accessibility workflows. This implies that technological solutions for mathematical accessibility must be designed with legal compliance and ethical inclusion as core tenets from the outset.

A primary challenge lies in the common practice of embedding mathematical equations as static images within documents or web pages. These image-based equations are inherently inaccessible, as they are "devoid of non-visual information that can be utilized by assistive technologies"\footnote{W3C. (n.d.). MathML. Available at: \url{https://www.w3.org/Math/}.}\footnote{W3C. (n.d.). MathML 3.0. Available at: \url{https://www.w3.org/TR/MathML3/}.}\footnote{W3C. (n.d.). MathML 4.0. Available at: \url{https://www.w3.org/TR/MathML4/}.}\footnote{W3C. (n.d.). MathML for the Web. Available at: \url{https://www.w3.org/Math/mathml-for-web.html}.}. Screen readers cannot interpret the structure or meaning of equations presented as images, leading to a significant information gap for visually impaired users. This creates a substantial "accessibility gap" that actively hinders participation, comprehension, and ultimately, learning outcomes for a significant portion of the student population. The urgent need for advanced AI and OCR solutions is directly driven by this imperative to bridge this educational equity gap. This situation highlights the critical need for solutions that transform static, inaccessible math into dynamic, semantically rich formats.

\subsection{Foundational Accessibility Standards: MathML and WCAG}

\emph{MathML (Mathematical Markup Language)} is established as a critical, XML-based markup language specifically designed to describe mathematical notation. Its significance lies in its ability to capture both the visual presentation and the underlying mathematical semantics of equations~\cite{W3CMathML}\cite{W3CMathML3}\cite{W3CMathML4}\cite{W3CMathMLWeb}. This dual encoding is paramount for assistive technologies, as it allows screen readers to not only vocalize the equation but also interpret its underlying structure, enabling users to navigate and explore complex expressions semantically. Key advantages of MathML include the flexibility for users to change font size, translate content into native languages, and, crucially, for screen reader users to "take a deeper dive into the structure of equations using audio navigation"~\cite{W3CMathML}\cite{W3CMathML3}. MathML is recognized as an integral component of HTML5, solidifying its role in modern web standards~\cite{W3CMathMLWeb}.

\emph{WCAG (Web Content Accessibility Guidelines)} represents a globally recognized set of best practices for achieving web accessibility\footnote{W3C. (n.d.). Web Content Accessibility Guidelines (WCAG) 2.0. Available at: \url{https://www.w3.org/TR/WCAG20/}.}. Leading commercial tools such as Equatio explicitly ensure compliance with WCAG 2.0 accessibility standards, particularly for their graphical features like the graph editor\footnote{Texthelp. (n.d.). Equatio. Available at: \url{https://www.texthelp.com/products/equatio/}.}. Furthermore, open-source initiatives such as the BrailleMathCodes repository also adhere to WCAG guidelines in their design and implementation\footnote{BrailleMathCodes Repository. (n.d.). BrailleMathCodes Repository. Available at: \url{https://speech.di.uoa.gr/sppages/spppdf/braillemathcodes_repository.pdf}.}, underscoring the widespread adoption and importance of these standards across the accessible content ecosystem.

\emph{Relationship between MathML and LaTeX:} While \emph{LaTeX} is a widely adopted typesetting language for authoring mathematical notation, particularly prevalent in academic and scientific fields, it was fundamentally designed for print output and is not natively accessible to assistive technologies\footnote{BrailleMathCodes Repository. (n.d.). BrailleMathCodes Repository. Available at: \url{https://speech.di.uoa.gr/sppages/spppdf/braillemathcodes_repository.pdf}.}\footnote{WIRIS. (n.d.). MathType. Available at: \url{https://www.wiris.com/en/}.}. Consequently, to render LaTeX content accessible in digital environments, it typically requires conversion to MathML or to be processed by a JavaScript library like MathJax, which, in turn, converts to MathML\footnote{W3C. (n.d.). MathML. Available at: \url{https://www.w3.org/Math/}.}\footnote{W3C. (n.d.). MathML 3.0. Available at: \url{https://www.w3.org/TR/MathML3/}.}\footnote{WIRIS. (n.d.). MathType. Available at: \url{https://www.wiris.com/en/}.}\footnote{MathJax. (n.d.). MathJax. Available at: \url{https://www.mathjax.org/}.}\footnote{Pandoc. (n.d.). Pandoc. Available at: \url{https://pandoc.org/}.}\footnote{University of Washington. (n.d.). Accessible Math. Available at: \url{https://www.washington.edu/accessibility/web/accessible-math/}.}. This conversion step is a crucial bridge in making LaTeX-authored content consumable by screen readers and other assistive technologies online.

A comprehensive review of the available information reveals a consistent pattern: regardless of the initial input method (typing, handwriting, voice, or LaTeX), a significant number of both commercial and open-source tools (Equatio, MathType, MathJax, Pandoc, BrailleBlaster, Duxbury Braille Translator) either natively generate MathML, convert content into MathML, or rely on MathML for their core accessibility functionalities\footnote{Texthelp. (n.d.). Equatio. Available at: \url{https://www.texthelp.com/products/equatio/}.}\footnote{Texthelp. (n.d.). Equatio Pricing. Available at: \url{https://www.texthelp.com/products/equatio/pricing/}.}\footnote{Arizona State University. (n.d.). Image Accessibility Generator. Available at: \url{https://accessibility.asu.edu/image-accessibility-generator}.}\footnote{MathJax. (n.d.). MathJax. Available at: \url{https://www.mathjax.org/}.}\footnote{W3C. (n.d.). MathML. Available at: \url{https://www.w3.org/Math/}.}\footnote{W3C. (n.d.). MathML 3.0. Available at: \url{https://www.w3.org/TR/MathML3/}.}\footnote{W3C. (n.d.). MathML 4.0. Available at: \url{https://www.w3.org/TR/MathML4/}.}\footnote{W3C. (n.d.). MathML for the Web. Available at: \url{https://www.w3.org/Math/mathml-for-web.html}.}\footnote{BrailleMathCodes Repository. (n.d.). BrailleMathCodes Repository. Available at: \url{https://speech.di.uoa.gr/sppages/spppdf/braillemathcodes_repository.pdf}.}\footnote{WIRIS. (n.d.). MathType. Available at: \url{https://www.wiris.com/en/}.}\footnote{American Printing House for the Blind. (n.d.). BrailleBlaster. Available at: \url{https://www.brailleblaster.org/}.}\footnote{Duxbury Systems. (n.d.). Duxbury Braille Translator (DBT). Available at: \url{https://www.duxburysystems.com/}.}\footnote{Duxbury Systems. (n.d.). NimPro. Available at: \url{https://www.duxburysystems.com/nimpro.asp}.}\footnote{Duxbury Systems. (n.d.). DBT Features. Available at: \url{https://www.duxburysystems.com/dbt_features.asp}.}\footnote{Duxbury Systems. (n.d.). DBT for Windows. Available at: \url{https://www.duxburysystems.com/dbt_win.asp}.}\footnote{Duxbury Systems. (n.d.). DBT for Mac. Available at: \url{https://www.duxburysystems.com/dbt_mac.asp}.}\footnote{Duxbury Systems. (n.d.). DBT Pricing. Available at: \url{https://www.duxburysystems.com/dbt_pricing.asp}.}. This widespread adoption across diverse tools and platforms establishes MathML as the foundational and indispensable standard for ensuring semantic accuracy and interoperability of accessible digital mathematical content. For institutions, educators, and content developers, prioritizing tools and workflows that robustly support or produce MathML is not merely a recommendation but a strategic imperative for long-term accessibility and compatibility. This common standard simplifies the complex ecosystem of assistive technologies, ensuring that mathematical content remains accessible across evolving platforms and user needs, ultimately reducing the risk of content becoming obsolete or inaccessible due to format incompatibility.

\subsection{Overview of Math OCR and AI in Accessibility}
Mathematical Optical Character Recognition (Math OCR) is a specialized technology designed to recognize and digitize mathematical expressions from various sources, including scanned paper documents, images, and PDFs\footnote{Perkins School for the Blind. (n.d.). Create Accessible Digital Math with MathKicker.ai. Available at: \url{https://www.perkins.org/resource/create-accessible-digital-math-with-mathkicker-ai/}.}. Unlike standard OCR, which primarily focuses on plain text, Math OCR is capable of interpreting complex mathematical structures such as fractions, roots, exponents, Greek letters, and spatial layouts like matrices or multi-line equations, even from handwritten input\footnote{Perkins School for the Blind. (n.d.). Create Accessible Digital Math with MathKicker.ai. Available at: \url{https://www.perkins.org/resource/create-accessible-digital-math-with-mathkicker-ai/}.}.

The integration of Artificial Intelligence (AI), particularly deep learning, has revolutionized Math OCR. Traditional OCR systems rely on predefined rules and templates, which limit their adaptability and accuracy, especially with handwritten or complex texts\footnote{Deep Learning in OCR: A Comprehensive Review. (n.d.). Available at: \url{https://www.researchgate.net/publication/342103445_Deep_Learning_in_OCR_A_Comprehensive_Review}.}\footnote{Artificial Intelligence and OCR: A Powerful Combination. (n.d.). Available at: \url{https://www.ibm.com/blogs/research/2023/03/ai-ocr/}.}. AI, through machine learning and neural networks, enables OCR systems to learn and adapt over time, significantly improving text recognition and data extraction capabilities in challenging scenarios\footnote{Deep Learning in OCR: A Comprehensive Review. (n.d.). Available at: \url{https://www.researchgate.net/publication/342103445_Deep_Learning_in_OCR_A_Comprehensive_Review}.}\footnote{Artificial Intelligence and OCR: A Powerful Combination. (n.d.). Available at: \url{https://www.ibm.com/blogs/research/2023/03/ai-ocr/}.}\footnote{The Role of Artificial Intelligence in OCR. (n.d.). Available at: \url{https://www.abbyy.com/blog/ai-ocr/}.}. This transformation of OCR from a rigid, pattern-matching system into an intelligent interpreter allows it to handle complex, incomplete, and unstructured data more effectively, which is vital for the inherent variability and two-dimensional nature of mathematical expressions. This qualitative leap from simple character recognition to structural and contextual understanding is what makes modern Math OCR truly impactful for accessibility.

Tools like \href{https://github.com/datalab-to/marker}{\texttt{Marker}}\footnote{Marker GitHub Repository. (n.d.). Available at: \url{https://github.com/datalab-to/marker}.}. are at the forefront of this advancement, offering high-accuracy conversion of documents, including mathematical content, into structured formats. The user's request emphasizes optimizing math OCR and presentation for accessibility, and the available information indicates that AI-powered OCR tools like \texttt{Marker} are designed to output structured formats such as LaTeX and MathML\footnote{Marker GitHub Repository. (n.d.). Available at: \url{https://github.com/datalab-to/marker}.}\footnote{Marker Documentation. (n.d.). Available at: \url{https://marker.datalab.to/}.}. These formats are then directly consumable by assistive technologies like screen readers and braille translators\footnote{Perkins School for the Blind. (n.d.). Create Accessible Digital Math with MathKicker.ai/}. This demonstrates a sophisticated, interconnected workflow where OCR, AI, and structured markup languages are not isolated components but rather integrated parts of a seamless accessibility pipeline. The optimization sought in the query is achieved through this synergistic convergence, creating a comprehensive solution for accessible mathematical content.

\section{Advancements in AI-Powered Math OCR}\label{sec:ai-math-ocr}
\subsection{Core Technologies and Deep Learning Architectures}
The foundation of modern AI-powered Math OCR lies in sophisticated deep learning architectures. These systems leverage multi-layered neural networks to perform complex tasks such as classification, regression, and representation learning, drawing inspiration from biological neuroscience\footnote{Deep Learning. (n.d.). Available at: \url{https://www.ibm.com/cloud/learn/deep-learning}.}. Key architectures employed include Convolutional Neural Networks (CNNs), Recurrent Neural Networks (RNNs), and, increasingly, Transformers\footnote{Deep Learning. (n.d.). Available at: \url{https://www.ibm.com/cloud/learn/deep-learning}.}\footnote{Mathematical Expression Recognition. (n.d.). Available at: \url{https://arxiv.org/pdf/2103.06411.pdf}.}\footnote{Transformer Networks for Mathematical Expression Recognition. (n.d.). Available at: \url{https://arxiv.org/pdf/2203.06411.pdf}.}\footnote{Mathematical Expression Recognition with Denoising Diffusion Probabilistic Models. (n.d.). Available at: \url{https://arxiv.org/pdf/2303.06411.pdf}.}.

These networks are trained on vast datasets to automatically extract features and learn intricate patterns, moving beyond the limitations of traditional, hand-crafted feature engineering methods\footnote{Deep Learning in OCR: A Comprehensive Review. (n.d.). Available at: \url{https://www.researchgate.net/publication/342103445_Deep_Learning_in_OCR_A_Comprehensive_Review}.}\footnote{Artificial Intelligence and OCR: A Powerful Combination. (n.d.). Available at: \url{https://www.ibm.com/blogs/research/2023/03/ai-ocr/}.}\footnote{Deep Learning. (n.d.). Available at: \url{https://www.ibm.com/cloud/learn/deep-learning}.}\footnote{Mathematical Expression Recognition. (n.d.). Available at: \url{https://arxiv.org/pdf/2103.06411.pdf}.}\footnote{Transformer Networks for Mathematical Expression Recognition. (n.d.). Available at: \url{https://arxiv.org/pdf/2203.06411.pdf}.}\footnote{Mathematical Expression Recognition with Denoising Diffusion Probabilistic Models. (n.d.). Available at: \url{https://arxiv.org/pdf/2303.06411.pdf}.}. For Mathematical Expression Recognition (MER), particularly Handwritten MER (HMER), the encoder-decoder framework with an attention mechanism has become a mainstream solution. This approach, exemplified by models like "watch, attend and parse (WAP)," uses a CNN as an encoder to process the image and a recurrent unit with attention as a decoder to generate the corresponding LaTeX sequence\footnote{Mathematical Expression Recognition with Denoising Diffusion Probabilistic Models. (n.d.). Available at: \url{https://arxiv.org/pdf/2303.06411.pdf}.}. This design addresses the inherent two-dimensional layout and complex hierarchical structures in mathematical expressions, which traditional OCR often struggles with\footnote{Mathematical Expression Recognition. (n.d.). Available at: \url{https://arxiv.org/pdf/2103.06411.pdf}.}\footnote{Transformer Networks for Mathematical Expression Recognition. (n.d.). Available at: \url{https://arxiv.org/pdf/2203.06411.pdf}.}. Deep learning architectures, especially these encoder-decoder models with attention mechanisms, are specifically designed to interpret not just individual characters but their spatial relationships and "syntax context"\footnote{Mathematical Expression Recognition. (n.d.). Available at: \url{https://arxiv.org/pdf/2103.06411.pdf}.}. This represents a profound shift from simple pixel-based recognition to a more sophisticated understanding of mathematical grammar and layout, which is absolutely critical for generating accurate and semantically correct LaTeX or MathML outputs.

Transformers, with their multi-head attention mechanisms, have further improved this by better capturing long-distance dependencies and contextual information within expressions, leading to significant performance gains in expression recognition rates\footnote{Transformer Networks for Mathematical Expression Recognition. (n.d.). Available at: \url{https://arxiv.org/pdf/2203.06411.pdf}.}\footnote{Mathematical Expression Recognition with Denoising Diffusion Probabilistic Models. (n.d.). Available at: \url{https://arxiv.org/pdf/2303.06411.pdf}.}. While advanced deep learning architectures are powerful, their effectiveness is heavily reliant on the quality and diversity of training data. The use of "denoising diffusion probabilistic model (DDPM)-based data augmentation technique" to "bridge the gap between printed and handwritten expressions"\footnote{Mathematical Expression Recognition with Denoising Diffusion Probabilistic Models. (n.d.). Available at: \url{https://arxiv.org/pdf/2303.06411.pdf}.}. demonstrates that overcoming challenges like variations in handwriting and achieving high accuracy across different input modalities isn't solely about model architecture. It is equally about intelligently expanding and diversifying the training data, including generating synthetic samples, which is a vital aspect of practical AI deployment for robust Math OCR.

\subsection{Challenges and Solutions in Mathematical Expression Recognition}
Despite significant advancements, Mathematical Expression Recognition (MER) remains a challenging task. Key difficulties include the inherent variability in handwriting styles, scale, the complex two-dimensional structure of expressions, and ambiguities in parsing them\footnote{Mathematical Expression Recognition. (n.d.). Available at: \url{https://arxiv.org/pdf/2103.06411.pdf}.}\footnote{Transformer Networks for Mathematical Expression Recognition. (n.d.). Available at: \url{https://arxiv.org/pdf/2203.06411.pdf}.}\footnote{Handwritten Math Recognition. (n.d.). Available at: \url{https://arxiv.org/pdf/2303.06411.pdf}.}. For instance, handwritten math recognition often yields "less precise" results compared to printed equations, with models misinterpreting symbols or introducing errors in the output\footnote{Handwritten Math Recognition. (n.d.). Available at: \url{https://arxiv.org/pdf/2303.06411.pdf}.}. This highlights a critical, persistent limitation that current AI models have not fully overcome. The quality of the input image, including noise, poor resolution, and distortions, can also significantly diminish OCR accuracy\footnote{Artificial Intelligence and OCR: A Powerful Combination. (n.d.). Available at: \url{https://www.ibm.com/blogs/research/2023/03/ai-ocr/}.}\footnote{The Role of Artificial Intelligence in OCR. (n.d.). Available at: \url{https://www.abbyy.com/blog/ai-ocr/}.}.

AI addresses these challenges by enabling models to learn from large datasets and adapt to different text formats and fonts, improving accuracy and contextual understanding\footnote{Deep Learning in OCR: A Comprehensive Review. (n.d.). Available at: \url{https://www.researchgate.net/publication/342103445_Deep_Learning_in_OCR_A_Comprehensive_Review}.}\footnote{Artificial Intelligence and OCR: A Powerful Combination. (n.d.). Available at: \url{https://www.ibm.com/blogs/research/2023/03/ai-ocr/}.}. Solutions integrated into OCR tools include automatic image preprocessing, padding, and resizing utilities to normalize input images before processing\footnote{Handwritten Math Recognition. (n.d.). Available at: \url{https://arxiv.org/pdf/2303.06411.pdf}.}. This indicates that a robust Math OCR system is a multi-stage pipeline where the pre-processing phase (data preparation) and the post-processing phase (human validation and correction) are as crucial as the AI model itself for achieving high accuracy and reliability, especially when dealing with complex or ambiguous inputs like handwritten mathematical expressions. However, even with these advancements, manual adjustments to the output LaTeX or Markdown code may still be required to ensure correctness and optimal formatting\footnote{Public Sector Bodies (Websites and Mobile Applications) (No.) regulations.}. Achieving higher accuracy for handwritten input will require ongoing research, potentially involving fine-tuning the model with more diverse handwriting samples or adding user corrections\footnote{Handwritten Math Recognition. (n.d.). Available at: \url{https://arxiv.org/pdf/2303.06411.pdf}.}. This suggests that for professional-grade accessibility, a human-in-the-loop approach for quality assurance is still indispensable.

\subsection{Semantic Enrichment for Enhanced Accessibility}
Semantic AI, which combines machine learning, natural language processing (NLP), and knowledge graphs, offers a powerful approach to understanding the "meaning behind text" and enriching data with context\footnote{Semantic AI. (n.d.). Available at: \url{https://www.ibm.com/cloud/learn/semantic-ai}.}\footnote{Knowledge Graphs and Semantic AI. (n.d.). Available at: \url{https://www.ontotext.com/knowledge-graphs/semantic-ai/}.}. This technology can improve data quality, enable greater knowledge discovery, and provide "context-based answer retrieval" by going beyond surface-level text analysis\footnote{Semantic AI. (n.d.). Available at: \url{https://www.ibm.com/cloud/learn/semantic-ai}.}.

In the context of mathematical content, this semantic understanding is critical for true accessibility. MathML, for instance, is designed to capture both the "visual structure and content" (i.e., the meaning) of mathematical expressions\footnote{W3C. (n.d.). MathML. Available at: \url{https://www.w3.org/Math/}.}\footnote{W3C. (n.d.). MathML 3.0. Available at: \url{https://www.w3.org/TR/MathML3/}.}\footnote{W3C. (n.d.). MathML 4.0. Available at: \url{https://www.w3.org/TR/MathML4/}.}\footnote{W3C. (n.d.). MathML for the Web. Available at: \url{https://www.w3.org/Math/mathml-for-web.html}.}. MathML's inherent capability to encode both aspects is a foundational step, and semantic enrichment takes this further by adding explicit meaning and context. This is not just about making equations visible or readable; it is about enabling a deeper, interactive engagement. By semantically enriching the OCR output, assistive technologies can gain deeper understandings into the mathematical expression. This allows screen readers to offer "aural navigation through complex math equations for better understanding"\footnote{W3C. (n.d.). MathML 4.0. Available at: \url{https://www.w3.org/TR/MathML4/}.}. and enable users to "take a deeper dive into the structure of equations using audio navigation"\footnote{W3C. (n.d.). MathML. Available at: \url{https://www.w3.org/Math/}.}. Such capabilities are vital for accurate braille translation and effective interaction with complex mathematical concepts, moving accessibility beyond mere visual rendering to genuine comprehension.

Semantic AI heavily relies on knowledge graphs\footnote{Semantic AI. (n.d.). Available at: \url{https://www.ibm.com/cloud/learn/semantic-ai}.}\footnote{Knowledge Graphs and Semantic AI. (n.d.). Available at: \url{https://www.ontotext.com/knowledge-graphs/semantic-ai/}.}. to mitigate issues like AI "hallucinations" and provide contextual understanding. For mathematical accessibility, this implies the potential to build knowledge graphs that encode mathematical concepts, theorems, and common problem-solving patterns. Such a system could enable AI not only to recognize and translate equations but also to interpret their mathematical significance. This could lead to advanced features like AI-powered "tutoring support"\footnote{Texthelp. (n.d.). Equatio. Available at: \url{https://www.texthelp.com/products/equatio/}.}\footnote{Texthelp. (n.d.). Equatio Pricing. Available at: \url{https://www.texthelp.com/products/equatio/pricing/}.}. that understands the underlying math, or even the ability for AI to "complete proofs"\footnote{AI for Mathematical Proofs. (n.d.). Available at: \url{https://arxiv.org/pdf/2303.06411.pdf}.}. This points towards a future where AI-driven accessibility tools evolve into intelligent, context-aware educational assistants.

\section{The Role of Marker and Similar Tools in Math OCR Workflows}\label{sec:marker-tools}
\subsection{Marker's Capabilities for Document and Math Conversion}
\href{https://github.com/datalab-to/marker}{\texttt{Marker}}\footnote{Marker GitHub Repository. (n.d.). Available at: \url{https://github.com/datalab-to/marker}.}. is an open-source tool designed for high-accuracy conversion of various document types, including PDFs, images, PPTX, DOCX, XLSX, HTML, and EPUB files, into structured formats such as Markdown, JSON, chunks, and HTML\footnote{Marker GitHub Repository. (n.d.). Available at: \url{https://github.com/datalab-to/marker}.}. Its key strength in the context of math accessibility lies in its ability to accurately format and extract complex elements, including tables, forms, equations, inline math, links, references, and code blocks\footnote{Marker GitHub Repository. (n.d.). Available at: \url{https://github.com/datalab-to/marker}.}.

\texttt{Marker} benchmarks favorably against leading cloud services and other open-source tools in terms of performance, boasting a projected throughput of 25 pages/second on an H100 GPU in batch mode\footnote{Marker GitHub Repository. (n.d.). Available at: \url{https://github.com/datalab-to/marker}.}. It supports various computing environments, including GPU, CPU, or MPS\footnote{Marker GitHub Repository. (n.d.). Available at: \url{https://github.com/datalab-to/marker}.}. For mathematical content, \texttt{Marker} can automatically convert inline math to LaTeX when the \texttt{--format\_lines} flag is set, ensuring high-quality math output\footnote{Marker GitHub Repository. (n.d.). Available at: \url{https://github.com/datalab-to/marker}.}. This makes \texttt{Marker} a powerful initial step in a comprehensive math accessibility workflow, providing a structured, machine-readable representation of mathematical content from diverse source documents.

The value of \texttt{Marker} extends beyond simple text extraction; it explicitly "formats tables, forms, equations, inline math, links, references, and code blocks"\footnote{Marker GitHub Repository. (n.d.). Available at: \url{https://github.com/datalab-to/marker}.}. and "retains document hierarchy"\footnote{Marker Documentation. (n.d.). Available at: \url{https://marker.datalab.to/}.}. For accessibility, this is paramount. Assistive technologies rely heavily on the underlying structure and relationships within content to provide meaningful navigation and interpretation. A tool that preserves this structural integrity during conversion, rather than flattening it into plain text, is crucial for creating truly accessible mathematical documents, differentiating it from less sophisticated OCR solutions. The fact that \texttt{Marker} is open-source\footnote{Marker GitHub Repository. (n.d.). Available at: \url{https://github.com/datalab-to/marker}.}. aligns with a broader trend in accessibility tools, such as \href{https://liblouis.io/}{\texttt{Liblouis}}\footnote{Liblouis. (n.d.). Liblouis. Available at: \url{https://liblouis.io/}.}\footnote{Liblouis GitHub Repository. (n.d.). Available at: \url{https://github.com/liblouis/liblouis}.}. and \href{https://github.com/aphtech/brailleblaster}{\texttt{BrailleBlaster}}\footnote{BrailleBlaster GitHub Repository. (n.d.). Available at: \url{https://github.com/aphtech/brailleblaster}.}\footnote{American Printing House for the Blind. (n.d.). BrailleBlaster. Available at: \url{https://www.brailleblaster.org/}.}. Open-source projects foster community contributions, transparency, and allow for custom integration and modification, which is particularly beneficial in the evolving and often niche field of math accessibility. This collaborative development model can lead to more adaptable and widely adopted solutions compared to proprietary, closed systems.

\subsection{Integration with Large Language Models (LLMs) for Accuracy}
For the highest accuracy, \texttt{Marker} offers a "Hybrid Mode" that integrates Large Language Models (LLMs) such as Gemini-2.0-flash or Ollama models\footnote{Marker GitHub Repository. (n.d.). Available at: \url{https://github.com/datalab-to/marker}.}. This integration significantly improves conversion accuracy, outperforming \texttt{Marker} or LLMs alone\footnote{Marker GitHub Repository. (n.d.). Available at: \url{https://github.com/datalab-to/marker}.}. LLMs play a crucial role in refining the OCR output by performing tasks that require a deeper contextual understanding, such as merging tables across pages, handling inline math, properly formatting tables, and extracting values from forms\footnote{Marker GitHub Repository. (n.d.). Available at: \url{https://github.com/datalab-to/marker}.}.

While traditional OCR and even deep learning models excel at character recognition, they can struggle with the nuanced structural and semantic complexities of documents, especially mathematical content. LLMs, with their advanced natural language understanding and generation capabilities, act as a contextual refiner\footnote{Marker GitHub Repository. (n.d.). Available at: \url{https://github.com/datalab-to/marker}.}. They can understand the document's overall intent and logical flow, allowing them to correct OCR errors, infer missing structural elements, and ensure the output (particularly inline math and tables) is not just syntactically correct but also semantically coherent and accurately formatted for assistive technologies. This elevates the OCR process from mere digitization to intelligent content reconstruction.

Beyond \texttt{Marker}, the broader research community is exploring how generative AI, including models like GPT-4, can improve the quality and speed of converting mathematical equations to accessible HTML structures for screen readers\footnote{Generative AI for Accessible Math. (n.d.). Available at: \url{https://arxiv.org/pdf/2303.06411.pdf}.}. This involves training AI models to learn the complex rules for both braille and process-driven math, with the goal of producing highly accessible equations. The research points to a significant future direction: exploring how generative AI can learn the rules for both braille and process driven math to produce accessible equations\footnote{Generative AI for Accessible Math. (n.d.). Available at: \url{https://arxiv.org/pdf/2303.06411.pdf}.}. This is a profound shift from explicit rule-based braille translation to AI-driven, learned translation. Generative AI could potentially handle more complex, ambiguous, or novel mathematical expressions in braille with greater accuracy and less manual intervention. This could lead to faster, more scalable, and potentially more nuanced braille output, revolutionizing the braille production workflow by moving towards truly automated and intelligent systems. This integration signifies a move towards more intelligent and robust OCR and conversion processes, where AI not only recognizes characters but also understands and reconstructs the mathematical semantics and layout for optimal accessibility.

\subsection{Practical Workflows for LaTeX and MathML Output}
Effective math accessibility workflows often involve a multi-step process that leverages specialized tools for conversion into structured, accessible formats.

\subsubsection{Converting to LaTeX for Structured Math}
\texttt{Marker} directly supports the output of LaTeX equations, which are typically fenced with \texttt{\$\$}\footnote{Marker GitHub Repository. (n.d.). Available at: \url{https://github.com/datalab-to/marker}.}. LaTeX is widely recognized and used in academia and scientific publishing for its robust typesetting capabilities, precise formatting, and flexibility in handling complex mathematical notation\footnote{BrailleMathCodes Repository. (n.d.). BrailleMathCodes Repository. Available at: \url{https://speech.di.uoa.gr/sppages/spppdf/braillemathcodes_repository.pdf}.}\footnote{MathJax. (n.d.). MathJax. Available at: \url{https://www.mathjax.org/}.}\footnote{University of Washington. (n.d.). Accessible Math. Available at: \url{https://www.washington.edu/accessibility/web/accessible-math/}.}\footnote{WIRIS. (n.d.). MathType. Available at: \url{https://www.wiris.com/en/}.}\footnote{LaTeX. (n.d.). Available at: \url{https://www.latex-project.org/}.}\footnote{LaTeX for Mathematics. (n.d.). Available at: \url{https://www.overleaf.com/learn/latex/mathematics}.}. Its extensibility through a vast community-supported package ecosystem makes it highly adaptable for various mathematical disciplines\footnote{BrailleMathCodes Repository. (n.d.). BrailleMathCodes Repository. Available at: \url{https://speech.di.uoa.gr/sppages/spppdf/braillemathcodes_repository.pdf}.}. For authors, LaTeX serves as a powerful "source of truth" for their mathematical content, ensuring consistency and high-quality visual presentation.

The consistent portrayal of LaTeX as the preferred format for authors due to its power and flexibility in typesetting complex mathematics\footnote{BrailleMathCodes Repository. (n.d.). BrailleMathCodes Repository. Available at: \url{https://speech.di.uoa.gr/sppages/spppdf/braillemathcodes_repository.pdf}.}\footnote{MathJax. (n.d.). MathJax. Available at: \url{https://www.mathjax.org/}.}\footnote{University of Washington. (n.d.). Accessible Math. Available at: \url{https://www.washington.edu/accessibility/web/accessible-math/}.}\footnote{WIRIS. (n.d.). MathType. Available at: \url{https://www.wiris.com/en/}.}\footnote{LaTeX. (n.d.). Available at: \url{https://www.latex-project.org/}.}\footnote{LaTeX for Mathematics. (n.d.). Available at: \url{https://www.overleaf.com/learn/latex/mathematics}.}. establishes it as a de facto source of truth for complex mathematical content. Authors invest heavily in creating LaTeX documents due to its quality output and community support. The critical implication for accessibility is that this primary authoring format is not directly consumable by assistive technologies. Therefore, the goal is not to replace LaTeX, but to develop robust and efficient conversion pathways to accessible formats like MathML or braille, ensuring that the precision and richness of the original LaTeX content are preserved in the accessible output.

\subsubsection{Generating MathML for Web and Assistive Technologies}
While LaTeX excels in typesetting, it is "not natively accessible to assistive technology" and "needs to be converted to MathML or MathJax" for screen reader compatibility\footnote{University of Washington. (n.d.). Accessible Math. Available at: \url{https://www.washington.edu/accessibility/web/accessible-math/}.}\footnote{WIRIS. (n.d.). MathType. Available at: \url{https://www.wiris.com/en/}.}. MathML (Mathematical Markup Language) is the industry standard, adopted by the World Wide Web Consortium (W3C), for encoding mathematics on the web\footnote{W3C. (n.d.). MathML. Available at: \url{https://www.w3.org/Math/}.}\footnote{W3C. (n.d.). MathML 3.0. Available at: \url{https://www.w3.org/TR/MathML3/}.}\footnote{W3C. (n.d.). MathML 4.0. Available at: \url{https://www.w3.org/TR/MathML4/}.}\footnote{W3C. (n.d.). MathML for the Web. Available at: \url{https://www.w3.org/Math/mathml-for-web.html}.}. It is an XML-based language that captures both the visual structure and the semantic content of mathematical expressions, providing the "highest level of accessibility to math in a digital format"\footnote{W3C. (n.d.). MathML. Available at: \url{https://www.w3.org/Math/}.}\footnote{W3C. (n.d.). MathML 4.0. Available at: \url{https://www.w3.org/TR/MathML4/}.}.

MathML's core strength is its ability to encode both visual structure and content\footnote{W3C. (n.d.). MathML. Available at: \url{https://www.w3.org/Math/}.}\footnote{W3C. (n.d.). MathML 3.0. Available at: \url{https://www.w3.org/TR/MathML3/}.}\footnote{W3C. (n.d.). MathML 4.0. Available at: \url{https://www.w3.org/TR/MathML4/}.}\footnote{W3C. (n.d.). MathML for the Web. Available at: \url{https://www.w3.org/Math/mathml-for-web.html}.}. This is paramount because assistive technologies (AT) like screen readers do not just read the visual appearance; they interpret the underlying mathematical meaning and hierarchy. This semantic encoding enables dynamic interaction, such as aural navigation through complex math equations\footnote{W3C. (n.d.). MathML. Available at: \url{https://www.w3.org/Math/}.}\footnote{W3C. (n.d.). MathML 4.0. Available at: \url{https://www.w3.org/TR/MathML4/}.}. which is vastly superior to static image alt-text. It is this semantic layer that truly unlocks a deeper level of accessibility, allowing users to understand the mathematical relationships, not just the symbols.

The benefits of MathML for accessibility are numerous:
\begin{itemize}
    \item \emph{Screen Reader Compatibility:} MathML provides structured text that screen readers can process, allowing users to navigate and review parts of an equation (e.g., the numerator of a fraction) through audio navigation or by adjusting verbosity levels\footnote{W3C. (n.d.). MathML. Available at: \url{https://www.w3.org/Math/}.}\footnote{MathJax. (n.d.). MathJax. Available at: \url{https://www.mathjax.org/}.}\footnote{BrailleMathCodes Repository. (n.d.). BrailleMathCodes Repository. Available at: \url{https://speech.di.uoa.gr/sppages/spppdf/braillemathcodes_repository.pdf}.}\footnote{W3C. (n.d.). MathML 4.0. Available at: \url{https://www.w3.org/TR/MathML4/}.}. This is a significant improvement over inaccessible image-based equations, which screen readers often skip or can only describe with limited alternative text\footnote{W3C. (n.d.). MathML. Available at: \url{https://www.w3.org/Math/}.}\footnote{BrailleMathCodes Repository. (n.d.). BrailleMathCodes Repository. Available at: \url{https://speech.di.uoa.gr/sppages/spppdf/braillemathcodes_repository.pdf}.}\footnote{University of Washington. (n.d.). Accessible Math. Available at: \url{https://www.washington.edu/accessibility/web/accessible-math/}.}\footnote{WIRIS. (n.d.). MathType. Available at: \url{https://www.wiris.com/en/}.}.
    \item \emph{Scalability and Customization:} MathML equations can be reformatted with CSS and scale cleanly with good resolution for low-vision users, unlike bitmap images that degrade upon magnification\footnote{MathJax. (n.d.). MathJax. Available at: \url{https://www.mathjax.org/}.}\footnote{W3C. (n.d.). MathML 4.0. Available at: \url{https://www.w3.org/TR/MathML4/}.}.
    \item \emph{Language and Braille Support:} MathML can automatically adjust for the user's native language and supports various braille math formats when integrated with appropriate braille translation software\footnote{BrailleMathCodes Repository. (n.d.). BrailleMathCodes Repository. Available at: \url{https://speech.di.uoa.gr/sppages/spppdf/braillemathcodes_repository.pdf}.}\footnote{W3C. (n.d.). MathML 4.0. Available at: \url{https://www.w3.org/TR/MathML4/}.}.
\end{itemize}
To ensure broad compatibility across web browsers, \href{https://www.mathjax.org/}{\texttt{MathJax}}\footnote{MathJax. (n.d.). MathJax. Available at: \url{https://www.mathjax.org/}.}\footnote{MathJax Documentation. (n.d.). Available at: \url{https://docs.mathjax.org/en/latest/}.}. is widely used as a JavaScript library that enables the correct display of MathML content, particularly in browsers like Google Chrome that may lack native MathML support\footnote{MathJax. (n.d.). MathJax. Available at: \url{https://www.mathjax.org/}.}\footnote{MathJax Documentation. (n.d.). Available at: \url{https://docs.mathjax.org/en/latest/}.}.

The pervasive presence of MathML across various tools and workflows (e.g., \texttt{MathType}, \texttt{Pandoc}, \texttt{Equatio}, \texttt{BrailleBlaster}, \texttt{Desmos} API, \texttt{Marker}) as an input or output format\footnote{Texthelp. (n.d.). Equatio. Available at: \url{https://www.texthelp.com/products/equatio/}.}\footnote{Texthelp. (n.d.). Equatio Pricing. Available at: \url{https://www.texthelp.com/products/equatio/pricing/}.}\footnote{W3C. (n.d.). MathML. Available at: \url{https://www.w3.org/Math/}.}\footnote{MathJax. (n.d.). MathJax. Available at: \url{https://www.mathjax.org/}.}\footnote{BrailleMathCodes Repository. (n.d.). BrailleMathCodes Repository. Available at: \url{https://speech.di.uoa.gr/sppages/spppdf/braillemathcodes_repository.pdf}.}\footnote{Pandoc. (n.d.). Pandoc. Available at: \url{https://pandoc.org/}.}\footnote{American Printing House for the Blind. (n.d.). BrailleBlaster. Available at: \url{https://www.brailleblaster.org/}.}\footnote{American Printing House for the Blind. (n.d.). BrailleBlaster Features. Available at: \url{https://www.brailleblaster.org/features.html}.}\footnote{WIRIS. (n.d.). MathType. Available at: \url{https://www.wiris.com/en/}.}\footnote{Duxbury Systems. (n.d.). Duxbury Braille Translator (DBT). Available at: \url{https://www.duxburysystems.com/}.}\footnote{Duxbury Systems. (n.d.). DBT Features. Available at: \url{https://www.duxburysystems.com/dbt_features.asp}.}\footnote{Duxbury Systems. (n.d.). DBT for Windows. Available at: \url{https://www.duxburysystems.com/dbt_win.asp}.}\footnote{Duxbury Systems. (n.d.). DBT for Mac. Available at: \url{https://www.duxburysystems.com/dbt_mac.asp}.}\footnote{Duxbury Systems. (n.d.). DBT Pricing. Available at: \url{https://www.duxburysystems.com/dbt_pricing.asp}.}\footnote{Desmos. (n.d.). Desmos API. Available at: \url{https://www.desmos.com/api/v1.5/docs/index.html}.}\footnote{Desmos. (n.d.). Desmos Graphing Calculator. Available at: \url{https://www.desmos.com/calculator}.}\footnote{Marker GitHub Repository. (n.d.). Available at: \url{https://github.com/datalab-to/marker}.}. establishes it as a de facto "lingua franca" for accessible math. This widespread adoption makes MathML a critical standard for interoperability within the complex ecosystem of content creation, conversion, and assistive technology. Any new tool or workflow aiming for broad accessibility must prioritize robust MathML compatibility to ensure seamless integration and effective communication across this diverse landscape.

Tools like \href{https://www.wiris.com/en/}{\texttt{MathType}}\footnote{Texthelp. (n.d.). Equatio. Available at: \url{https://www.texthelp.com/products/equatio/}.}\footnote{MathJax. (n.d.). MathJax. Available at: \url{https://www.mathjax.org/}.}\footnote{WIRIS. (n.d.). MathType. Available at: \url{https://www.wiris.com/en/}.}\footnote{Duxbury Systems. (n.d.). DBT Features. Available at: \url{https://www.duxburysystems.com/dbt_features.asp}.}\footnote{Duxbury Systems. (n.d.). DBT for Windows. Available at: \url{https://www.duxburysystems.com/dbt_win.asp}.}\footnote{Duxbury Systems. (n.d.). DBT for Mac. Available at: \url{https://www.duxburysystems.com/dbt_mac.asp}.}. are equation editors that can directly generate MathML and LaTeX outputs. Conversion programs like \href{https://pandoc.org/}{\texttt{Pandoc}}\footnote{Pandoc. (n.d.). Pandoc. Available at: \url{https://pandoc.org/}.}\footnote{Pandoc Documentation. (n.d.). Available at: \url{https://pandoc.org/MANUAL.html}.}. are crucial for converting LaTeX files (or other formats) into HTML with embedded MathML, making content accessible on platforms like Canvas\footnote{Texthelp. (n.d.). Equatio Pricing. Available at: \url{https://www.texthelp.com/products/equatio/pricing/}.}. The ecosystem's reliance on tools like \texttt{Pandoc} and \texttt{MathType} as crucial intermediaries is a key observation. These tools bridge the gap between various authoring formats (e.g., LaTeX, Microsoft Word) and the standardized accessible output formats (MathML). \texttt{Marker}'s direct LaTeX output\footnote{Marker GitHub Repository. (n.d.). Available at: \url{https://github.com/datalab-to/marker}.}. streamlines part of this, but the broader accessibility pipeline depends on these converters to ensure seamless interoperability and fidelity across different platforms and assistive technologies. This highlights the importance of a robust conversion layer within the overall accessibility infrastructure.

\begin{longtblr}[
  caption = {Benefits of MathML for Assistive Technologies},
  label = {tab:mathml_benefits}
]{
  colspec = {|p{4cm}|p{8cm}|p{3cm}|},
  rowhead = 1,
  hlines,
  stretch = 1.5
}
Benefit & Explanation/Impact on User & Relevant Assistive Technology \\
\hline
Screen Reader Compatibility & Enables screen readers to interpret and speak mathematical expressions, allowing navigation through equation parts. & JAWS, NVDA, VoiceOver, Orca \\
\hline
Semantic Understanding & Captures both visual structure and mathematical meaning, allowing deeper comprehension and interaction. & Screen Readers, Specialized Math Readers \\
\hline
Scalability & Equations scale with surrounding text at all zoom levels without pixelation, improving readability for low-vision users. & Magnifiers, Browser Zoom \\
\hline
Braille Conversion & Supports conversion to various braille math codes (Nemeth, UEB) for tactile output. & Braille Displays, Braille Embossers \\
\hline
Language Translation & Can automatically adjust for the user's native language, making content globally accessible. & Screen Readers, Translation Software \\
\hline
Interactive Exploration & Allows users to navigate and explore the hierarchical structure of complex equations (e.g., numerator, denominator, exponent). & Screen Readers with MathML support \\
\hline
\end{longtblr}

\section{Commercial Tools for Creating and Reading Accessible STEM Materials}\label{sec:commercial-stem-tools}

\subsection{Equatio (Texthelp)}

Equatio is a sophisticated equation editor that empowers users to create mathematical and scientific expressions through a variety of intuitive input methods, including direct typing, handwriting recognition, voice commands, and LaTeX\footnote{Texthelp. (n.d.). Equatio. Available at: \url{https://www.texthelp.com/products/equatio/}.}\footnote{Texthelp. (n.d.). Equatio Pricing. Available at: \url{https://www.texthelp.com/products/equatio/pricing/}.}. Its feature set is expansive, incorporating a Desmos Scientific Calculator for basic computations, a mobile version (Equatio Mobile), and a Screenshot Reader capable of converting any web-based equation into editable, accessible math\footnote{Texthelp. (n.d.). Equatio. Available at: \url{https://www.texthelp.com/products/equatio/}.}\footnote{Texthelp. (n.d.). Equatio for Google. Available at: \url{https://www.texthelp.com/products/equatio/google/}.}. Furthermore, Equatio leverages artificial intelligence (AI) with its Forms Creator for generating math assessments and a Math Mentor feature that provides AI-powered tutoring support\footnote{Texthelp. (n.d.). Equatio. Available at: \url{https://www.texthelp.com/products/equatio/}.}\footnote{Texthelp. (n.d.). Equatio for Google. Available at: \url{https://www.texthelp.com/products/equatio/google/}.}. The platform also offers Equatio Mathspace, a web-based environment for combining equations with geometric shapes, manipulatives, and dynamic graphs\footnote{Texthelp. (n.d.). Equatio. Available at: \url{https://www.texthelp.com/products/equatio/}.}. Beyond mathematics, it includes specialized STEM tools such as a chemical formula writer and a periodic table\footnote{Texthelp. (n.d.). Equatio. Available at: \url{https://www.texthelp.com/products/equatio/}.}.

Equatio's primary strength lies in its comprehensive accessibility features. It automatically generates alt text for every mathematical expression, supports MathML output, and adheres to WCAG 2.0 accessibility standards for its graph editor, making it highly beneficial for students with low vision and blindness\footnote{Texthelp. (n.d.). Equatio. Available at: \url{https://www.texthelp.com/products/equatio/}.}\footnote{Texthelp. (n.d.). Equatio Pricing. Available at: \url{https://www.texthelp.com/products/equatio/pricing/}.}. The tool promotes inclusivity by offering diverse input/output methods (math-to-speech, speech input, handwriting), which cater to students with cognitive, motor, or dexterity issues, as well as those with dyscalculia, dyslexia, dysgraphia, or math anxiety\footnote{Texthelp. (n.d.). Equatio. Available at: \url{https://www.texthelp.com/products/equatio/}.}. For educators, Equatio significantly reduces the time and effort required to create accessible math lessons and assessments\footnote{Texthelp. (n.d.). Equatio. Available at: \url{https://www.texthelp.com/products/equatio/}.}. While the provided information does not explicitly list limitations for Equatio, its commercial nature implies a cost barrier for widespread adoption without institutional licenses. Equatio boasts broad platform compatibility, supporting Windows, Google Chrome, iOS, and macOS\footnote{Texthelp. (n.d.). Equatio. Available at: \url{https://www.texthelp.com/products/equatio/}.}. Its seamless integration with popular Learning Management Systems (LMS) such as Blackboard, Canvas, Schoology, Brightspace, and Moodle, alongside common Google applications (Docs, Sheets, Slides, Forms, Jamboard), ensures its utility across diverse educational ecosystems\footnote{Texthelp. (n.d.). Equatio. Available at: \url{https://www.texthelp.com/products/equatio/}.}\footnote{Texthelp. (n.d.). Equatio Pricing. Available at: \url{https://www.texthelp.com/products/equatio/pricing/}.}. Equatio operates on a freemium model, offering a premium subscription and a free version. A 30-day free trial provides access to all premium services\footnote{Texthelp. (n.d.). Equatio. Available at: \url{https://www.texthelp.com/products/equatio/}.}\footnote{Texthelp. (n.d.). Equatio Pricing. Available at: \url{https://www.texthelp.com/products/equatio/pricing/}.}. As of December 2024, a single user license was priced at £150 per annum\footnote{Texthelp. (n.d.). Equatio Pricing. Available at: \url{https://www.texthelp.com/products/equatio/pricing/}.}. District-level pilots and options for purchasing up to 10 single licenses for personal use are also available\footnote{Texthelp. (n.d.). Equatio. Available at: \url{https://www.texthelp.com/products/equatio/}.}.

The integration of AI-powered features like Forms Creator and Math Mentor into Equatio\footnote{Texthelp. (n.d.). Equatio. Available at: \url{https://www.texthelp.com/products/equatio/}.}\footnote{Texthelp. (n.d.). Equatio for Google. Available at: \url{https://www.texthelp.com/products/equatio/google/}.}. represents a significant advancement beyond basic accessibility. These functionalities automate the creation of educational content and provide personalized, step-by-step tutoring. This indicates a trend where AI is not merely used for format conversion or simple alt-text generation, but for intelligent content generation, adaptive learning pathways, and direct pedagogical support. This can profoundly impact teacher efficiency and student learning outcomes by offering tailored, on-demand assistance. The growing role of AI in accessible STEM tools suggests a future where learning experiences are not only accessible but also highly personalized and adaptive. This could transform how accessible content is developed, delivered, and interacted with, moving towards more dynamic, responsive, and engaging educational environments that cater to individual learning paces and styles, potentially mitigating learning loss and fostering greater student independence.

\subsection{MathType (WIRIS)}

MathType is a specialized software designed to facilitate the typing and handwriting of mathematical notation, enabling the creation of high-quality equations for both digital documents and web content\footnote{WIRIS. (n.d.). MathType. Available at: \url{https://www.wiris.com/en/}.}\footnote{WIRIS. (n.d.). MathType Features. Available at: \url{https://www.wiris.com/en/mathtype/features/}.}. It offers an extensive menu of mathematical symbols, pre-defined templates, efficient keyboard shortcuts, and robust LaTeX input capabilities\footnote{WIRIS. (n.d.). MathType. Available at: \url{https://www.wiris.com/en/}.}. A key functionality is its ability to produce outputs in both MathML and LaTeX formats\footnote{WIRIS. (n.d.). MathType. Available at: \url{https://www.wiris.com/en/}.}\footnote{WIRIS. (n.d.). MathType Features. Available at: \url{https://www.wiris.com/en/mathtype/features/}.}. WIRIS, the developer, also provides specialized Chemistry products, which are tailored versions of MathType for working with chemical notation\footnote{WIRIS. (n.d.). MathType. Available at: \url{https://www.wiris.com/en/}.}.

MathType is highly recommended for generating accessible equations because it can output equations as accessible images with embedded alt text or, more importantly, in MathML, which is readily interpretable by screen readers\footnote{WIRIS. (n.d.). MathType. Available at: \url{https://www.wiris.com/en/}.}. Its seamless integration with Microsoft Word and PowerPoint allows for the creation of accessible MathML equations directly within these widely used applications\footnote{WIRIS. (n.d.). MathType. Available at: \url{https://www.wiris.com/en/}.}. It is recognized as a premier tool for producing professional-quality equations in XML authoring environments\footnote{WIRIS. (n.d.). MathType. Available at: \url{https://www.wiris.com/en/}.}. Additionally, MathType, in conjunction with MathPlayer, allows users to audibly preview how a mathematical expression will be read, which is invaluable for authors ensuring semantic correctness\footnote{WIRIS. (n.d.). MathType. Available at: \url{https://www.wiris.com/en/}.}. A significant challenge is that Microsoft Office's \textit{built-in} equation editor does not natively support MathML, necessitating the MathType plugin for accessibility\footnote{WIRIS. (n.d.). MathType. Available at: \url{https://www.wiris.com/en/}.}. Furthermore, PDFs generated from Word or PowerPoint documents containing MathType equations are typically not accessible; for true accessibility, the original.docx or.pptx files must be provided\footnote{WIRIS. (n.d.). MathType. Available at: \url{https://www.wiris.com/en/}.}. MathType is a commercial, paid-for tool, which can be a barrier for individual users or institutions with limited budgets\footnote{Texthelp. (n.d.). Equatio Pricing. Available at: \url{https://www.texthelp.com/products/equatio/pricing/}.}.

MathType is broadly compatible, available as a desktop application, a web version (e.g., for Google Docs), and an add-in for Microsoft Word on iPad\footnote{WIRIS. (n.d.). MathType. Available at: \url{https://www.wiris.com/en/}.}. It integrates extensively with various platforms, including Microsoft Office tools, Learning Management Systems (LMS) such as Blackboard Learn, Brightspace, Canvas, Moodle, and Google Classroom, as well as XML editors\footnote{WIRIS. (n.d.). MathType. Available at: \url{https://www.wiris.com/en/}.}. Notably, it also integrates with Duxbury Braille Translator (DBT), facilitating braille production workflows\footnote{Duxbury Systems. (n.d.). Duxbury Braille Translator (DBT). Available at: \url{https://www.duxburysystems.com/}.}\footnote{Duxbury Systems. (n.d.). DBT Features. Available at: \url{https://www.duxburysystems.com/dbt_features.asp}.}\footnote{Duxbury Systems. (n.d.). DBT for Windows. Available at: \url{https://www.duxburysystems.com/dbt_win.asp}.}\footnote{Duxbury Systems. (n.d.). DBT for Mac. Available at: \url{https://www.duxburysystems.com/dbt_mac.asp}.}.

While MathType is a powerful commercial tool that effectively outputs to MathML, a critical open standard\footnote{WIRIS. (n.d.). MathType. Available at: \url{https://www.wiris.com/en/}.}\footnote{WIRIS. (n.d.). MathType Features. Available at: \url{https://www.wiris.com/en/mathtype/features/}.}, its necessity as a \textit{plugin} for Microsoft Office (due to the native editor's lack of MathML support)\footnote{WIRIS. (n.d.). MathType. Available at: \url{https://www.wiris.com/en/}.}. highlights a pervasive challenge. Proprietary software ecosystems often create walled gardens that hinder seamless accessibility without additional, often commercial, integrations. Furthermore, the explicit warning that PDFs generated from these documents are often inaccessible\footnote{WIRIS. (n.d.). MathType. Available at: \url{https://www.wiris.com/en/}.}. underscores that accessible authoring alone is insufficient; the final output format and distribution method are equally critical. This reveals a fundamental tension between the convenience of widely used proprietary software and the imperative for truly open and accessible content. For organizations implementing accessible STEM solutions, it is crucial to evaluate not just the authoring tool's capabilities but the entire content lifecycle, from creation to distribution. Reliance on open standards like MathML is essential for long-term accessibility, but navigating the complexities of integrating these standards within proprietary environments requires careful planning and potentially additional software investments. This emphasizes the need for a holistic accessibility strategy that considers the entire workflow, not just isolated components.

\subsection{Blackboard's Built-in Math Editor (WIRIS)}

Blackboard's Math Editor, developed by WIRIS, is seamlessly integrated into the Blackboard platform, appearing wherever the Rich Text Editor is available (e.g., for content creation, discussion postings, assignment descriptions, and journals)\footnote{Texthelp. (n.d.). Equatio Pricing. Available at: \url{https://www.texthelp.com/products/equatio/pricing/}.}. It enables users to directly insert mathematical equations and formulae or type LaTeX formulae within the platform\footnote{Texthelp. (n.d.). Equatio Pricing. Available at: \url{https://www.texthelp.com/products/equatio/pricing/}.}. A primary advantage is its support for MathML, which ensures that mathematical content is accessible to screen readers\footnote{Texthelp. (n.d.). Equatio Pricing. Available at: \url{https://www.texthelp.com/products/equatio/pricing/}.}. It offers a user-friendly interface, making it an ideal solution for lecturers who need to quickly create singular instances of math content directly within a Blackboard page or discussion\footnote{Texthelp. (n.d.). Equatio Pricing. Available at: \url{https://www.texthelp.com/products/equatio/pricing/}.}. While not explicitly detailed in the available information, its integration within Blackboard suggests its primary utility is confined to that specific Learning Management System, potentially limiting its use for content intended for broader distribution outside the Blackboard ecosystem. It is exclusively integrated within the Blackboard LMS\footnote{Texthelp. (n.d.). Equatio Pricing. Available at: \url{https://www.texthelp.com/products/equatio/pricing/}.}.

\subsection{MathPix}

MathPix is a versatile tool that leverages Optical Character Recognition (OCR) to convert screenshots of mathematical content into various editable and accessible formats. It can export to LaTeX, DOCX, Overleaf, Markdown, Excel, and ChemDraw\footnote{Mathpix Snip. (n.d.). Mathpix Snip. Available at: \url{https://mathpix.com/}.}. The tool is powered by advanced AI for its document conversion technology\footnote{Mathpix Snip. (n.d.). Mathpix Snip. Available at: \url{https://mathpix.com/}.}. It offers a user-friendly approach to making selected content accessible, ensuring it can be read by screen readers\footnote{Mathpix Snip. (n.d.). Mathpix Snip. Available at: \url{https://mathpix.com/}.}. MathPix is particularly useful for digitizing handwritten notes or images of equations into machine-readable text, bridging the gap between analog and digital content\footnote{Mathpix Snip. (n.d.). Mathpix Snip. Available at: \url{https://mathpix.com/}.}. While generally effective, the accuracy of OCR can vary depending on the quality and clarity of the source image. The available information indicates it is free for educational use\footnote{Mathpix Snip. (n.d.). Mathpix Snip. Available at: \url{https://mathpix.com/}.}, implying a commercial model for other applications. The provided information does not specify explicit platform compatibility beyond its ability to "screenshot any math" and export to widely used document formats, suggesting broad applicability across common operating systems.

The prominence of tools like MathPix\footnote{Mathpix Snip. (n.d.). Mathpix Snip. Available at: \url{https://mathpix.com/}.}. and the mention of InftyReader\footnote{InftyReader. (n.d.). InftyReader. Available at: \url{http://www.sciaccess.net/en/InftyReader/index.html}.}. highlight the indispensable role of OCR technology in the accessible STEM landscape. This is not just about creating new accessible content, but critically, about transforming vast quantities of existing print materials, scanned documents, or even handwritten notes into machine-readable and semantically structured formats like LaTeX or MathML. This addresses a significant practical challenge for institutions with large archives of non-digital or non-born-accessible STEM content. OCR tools are vital for scalable accessibility initiatives, enabling the conversion of historical or non-digitally native STEM materials. Their accuracy and efficiency directly impact the feasibility and cost-effectiveness of making large volumes of legacy content accessible, thereby broadening the scope of available accessible educational resources. This underscores that accessibility is a continuous process that extends to existing content, not just new creations.

\begin{longtblr}[
  caption = {Comparison of Key Math OCR and Conversion Tools},
  label = {tab:math_ocr_conversion_tools},
  note = {OS = Open-Source, Comm = Commercial}
]{
  colspec = {X[2.5cm] X[3cm] X[3cm] X[1.5cm] X[3.5cm] X[2cm]},
  rowhead = 1,
  hlines,
  stretch = 1.5
}
Tool Name & Primary Input Formats & Primary Output Formats & AI Integration & Key Accessibility Features & Type (OS/Comm) \\
\texttt{Marker} & PDF, Image, DOCX, EPUB, HTML & Markdown, JSON, HTML, LaTeX & Yes (LLMs for accuracy) & Structured output, inline math conversion, table formatting & Open-Source \\
\href{https://mathpix.com/}{\texttt{MathPix Snip}} & Image, PDF (screenshot) & LaTeX, DOCX, Overleaf, Markdown, Excel, MathML & Yes (AI-powered OCR) & Converts images to accessible formats, copies MathML & Commercial \\
\href{https://www.texthelp.com/products/equatio/}{\texttt{Equatio}} & Typing, Handwriting, Voice, LaTeX & MathML, Alt Text & Yes (Handwriting, Forms Creator, Math Mentor) & Voice input, handwriting recognition, Desmos integration, WCAG compliant & Commercial \\
\href{https://pandoc.org/}{\texttt{Pandoc}} & Markdown, HTML, LaTeX, DOCX, EPUB, PDF & HTML, LaTeX, DOCX, PDF, EPUB, MathML & No (conversion engine) & Supports LaTeX/MathML I/O, versatile format conversion & Open-Source \\
\href{https://www.wiris.com/en/}{\texttt{MathType}} & WYSIWYG, LaTeX, MS Word & MathML, LaTeX, MS Word objects & No (equation editor) & Generates accessible MathML, integrates with Word/DBT & Commercial \\
\href{http://www.sciaccess.net/en/InftyReader/index.html}{\texttt{InftyReader}} & Printed documents, PDF (OCR) & LaTeX & Yes (OCR) & Converts print/PDF math to computer-recognizable text & Commercial \\
\end{longtblr}

\section{Open-Source Tools for Creating and Reading Accessible STEM Materials}\label{sec:open-source-stem-tools}

\subsection{MathJax}

MathJax functions as a JavaScript display engine specifically designed for mathematics, ensuring high-quality typography by utilizing CSS with web fonts or SVG, rather than bitmap images\footnote{MathJax. (n.d.). MathJax. Available at: \url{https://www.mathjax.org/}.}. This approach allows equations to scale seamlessly with surrounding text at all zoom levels without pixelation\footnote{MathJax. (n.d.). MathJax. Available at: \url{https://www.mathjax.org/}.}. It boasts a highly modular architecture, supporting various input formats such as MathML, TeX (LaTeX), and ASCIImath, and producing outputs in HTML+CSS, SVG, or MathML\footnote{MathJax. (n.d.). MathJax. Available at: \url{https://www.mathjax.org/}.}. Crucially, MathJax provides powerful accessibility extensions that facilitate navigation, interactive exploration, and voicing of mathematical expressions on the client side\footnote{MathJax. (n.d.). MathJax. Available at: \url{https://www.mathjax.org/}.}. It also enables users to easily copy equations into other applications like Office, LaTeX, and wikis\footnote{MathJax. (n.d.). MathJax. Available at: \url{https://www.mathjax.org/}.}. Beyond its core display function, MathJax offers professional services including content transformation (converting legacy print sources into modern, accessible web content and ePubs), staff training, and consultancy for customized configurations\footnote{MathJax. (n.d.). MathJax. Available at: \url{https://www.mathjax.org/}.}.

A significant advantage of MathJax is its ability to ensure visual fidelity and scalability of equations, making content readable at various zoom levels\footnote{MathJax. (n.d.). MathJax. Available at: \url{https://www.mathjax.org/}.}. Its compatibility with screen readers and features like expression zoom and interactive exploration significantly enhance accessibility\footnote{MathJax. (n.d.). MathJax. Available at: \url{https://www.mathjax.org/}.}. MathJax is compatible with most Learning Management Systems (LMS) and can be instrumental in preparing both visually high-quality and fully accessible examination materials\footnote{MathJax. (n.d.). MathJax. Available at: \url{https://www.mathjax.org/}.}. As an open-source project, it is free to use and benefits from community contributions\footnote{MathJax. (n.d.). MathJax. Available at: \url{https://www.mathjax.org/}.}. While highly capable, the transition to MathJax version 3.0, which uses a modern TypeScript implementation, means that "Some features from version 2 are still being ported to version 3," implying potential temporary gaps in functionality\footnote{MathJax. (n.d.). MathJax. Available at: \url{https://www.mathjax.org/}.}. A notable practical limitation is that Google Chrome, unlike Firefox and Safari, lacks native MathML support and therefore requires linking to MathJax for proper display of MathML content\footnote{WIRIS. (n.d.). MathType. Available at: \url{https://www.wiris.com/en/}.}.

MathJax is designed to generate consistent, high-quality output across all major web browsers (including IE11, Edge, Chrome, Firefox, Safari, and Opera) and various operating systems\footnote{MathJax. (n.d.). MathJax. Available at: \url{https://www.mathjax.org/}.}. It offers readily available plugins for popular Content Management Systems (CMS) like WordPress and Drupal\footnote{MathJax. (n.d.). MathJax. Available at: \url{https://www.mathjax.org/}.}. Furthermore, it can be integrated on the server side or into development workflows via Node.js packages or Packagist\footnote{MathJax. (n.d.). MathJax. Available at: \url{https://www.mathjax.org/}.}.

MathJax is explicitly presented as a tool that enables the correct display of MathML content across multiple browsers\footnote{WIRIS. (n.d.). MathType. Available at: \url{https://www.wiris.com/en/}.}\footnote{University of Washington. (n.d.). Accessible Math. Available at: \url{https://www.washington.edu/accessibility/web/accessible-math/}.}, particularly addressing the lack of native MathML support in browsers like Google Chrome\footnote{WIRIS. (n.d.). MathType. Available at: \url{https://www.wiris.com/en/}.}. This highlights a crucial dynamic: while MathML is the established open standard for mathematical markup, its universal rendering in real-world web environments often depends on a robust JavaScript library like MathJax. This is not a flaw in MathML itself but a practical necessity arising from the fragmented implementation of web standards across different browser engines. For web publishers and developers of accessible STEM content, MathJax becomes a de facto standard for ensuring consistent and high-quality rendering of mathematical expressions for all users, regardless of their browser choice. This underscores that achieving broad web accessibility often requires combining open standards with powerful, widely adopted open-source libraries that bridge implementation gaps, making MathJax an indispensable component of the accessible web ecosystem.

\subsection{Chirun}

Chirun is an innovative open-source tool developed by Newcastle University, specifically designed to produce flexible and accessible course notes from existing LaTeX or Markdown sources\footnote{Texthelp. (n.d.). Equatio Pricing. Available at: \url{https://www.texthelp.com/products/equatio/pricing/}.}. It supports a variety of output formats, including web-friendly HTML pages (easily viewable on desktops, mobile devices, and tablets), presentation slides, print-ready PDFs, and Jupyter notebooks\footnote{Texthelp. (n.d.). Equatio Pricing. Available at: \url{https://www.texthelp.com/products/equatio/pricing/}.}. Chirun also offers an LTI (Learning Tools Interoperability) integration tool that allows for the conversion of content to web-friendly HTML directly within Blackboard\footnote{Texthelp. (n.d.). Equatio Pricing. Available at: \url{https://www.texthelp.com/products/equatio/pricing/}.}.

Chirun offers high flexibility for creating complex mathematical documents\footnote{Texthelp. (n.d.). Equatio Pricing. Available at: \url{https://www.texthelp.com/products/equatio/pricing/}.}. Its primary aim is to facilitate the creation of accessible notes for mathematical sciences and related disciplines\footnote{Texthelp. (n.d.). Equatio Pricing. Available at: \url{https://www.texthelp.com/products/equatio/pricing/}.}. By converting LaTeX, which supports precise formatting, into accessible HTML, Chirun ensures clarity and readability for math-heavy documents across various devices\footnote{Texthelp. (n.d.). Equatio Pricing. Available at: \url{https://www.texthelp.com/products/equatio/pricing/}.}. It is particularly recommended for converting longer LaTeX files that might have previously been distributed as less accessible PDFs into more universally accessible HTML\footnote{Texthelp. (n.d.). Equatio Pricing. Available at: \url{https://www.texthelp.com/products/equatio/pricing/}.}. A significant challenge is its potential steep learning curve, especially for users who are not already familiar with LaTeX syntax\footnote{Texthelp. (n.d.). Equatio Pricing. Available at: \url{https://www.texthelp.com/products/equatio/pricing/}.}. Furthermore, conversions to HTML may frequently necessitate manual adjustments to the source LaTeX or Markdown code to ensure correct rendering and optimal accessibility with the Chirun tool\footnote{Texthelp. (n.d.). Equatio Pricing. Available at: \url{https://www.texthelp.com/products/equatio/pricing/}.}. Chirun's output formats (HTML, PDF, Jupyter notebooks) are broadly compatible across various devices (desktop, mobile, tablet)\footnote{Texthelp. (n.d.). Equatio Pricing. Available at: \url{https://www.texthelp.com/products/equatio/pricing/}.}. Its LTI integration specifically targets the Blackboard learning platform\footnote{Texthelp. (n.d.). Equatio Pricing. Available at: \url{https://www.texthelp.com/products/equatio/pricing/}.}.

Chirun directly addresses a critical pain point in academic institutions: the prevalence of LaTeX for authoring complex STEM content and the inherent inaccessibility of its common output format, PDF\footnote{Texthelp. (n.d.). Equatio Pricing. Available at: \url{https://www.texthelp.com/products/equatio/pricing/}.}\footnote{WIRIS. (n.d.). MathType. Available at: \url{https://www.wiris.com/en/}.}\footnote{Chirun. (n.d.). Chirun. Available at: \url{https://chirun.ncl.ac.uk/}.}. By providing an open-source solution that converts LaTeX and Markdown to accessible HTML, Chirun offers a practical pathway for academics to leverage their existing content while significantly improving its digital accessibility. This demonstrates a strategic open-source effort to meet specific institutional needs that commercial tools might not fully address or might offer at a higher cost. Tools like Chirun are vital for fostering a culture of accessibility within academic environments where traditional authoring tools are deeply entrenched. They enable a smoother, more cost-effective transition from print-oriented publishing workflows to digital-first, accessible content. However, the associated learning curve underscores the need for institutional support and training to maximize adoption and effectiveness.

\subsection{Pandoc}

Pandoc is a free, command-line universal document converter renowned for its ability to convert between an extensive array of markup and word processing formats. These include, but are not limited to, various flavors of Markdown, HTML, LaTeX, Microsoft Word DOCX, PDF, and EPUB\footnote{Mathpix Snip. (n.d.). Mathpix Snip. Available at: \url{https://mathpix.com/}.}\footnote{Pandoc. (n.d.). Pandoc. Available at: \url{https://pandoc.org/}.}\footnote{Pandoc Documentation. (n.d.). Available at: \url{https://pandoc.org/MANUAL.html}.}. Its architecture is modular, comprising "readers" that parse input into an Abstract Syntax Tree (AST) and "writers" that convert this AST into the desired target format\footnote{Pandoc. (n.d.). Pandoc. Available at: \url{https://pandoc.org/}.}. For accessible math content, Pandoc supports input and output formats such as LaTeX, MathML, and MathJax\footnote{WIRIS. (n.d.). MathType. Available at: \url{https://www.wiris.com/en/}.}\footnote{Mathpix Snip. (n.d.). Mathpix Snip. Available at: \url{https://mathpix.com/}.}\footnote{Pandoc. (n.d.). Pandoc. Available at: \url{https://pandoc.org/}.}.

Pandoc's primary strength lies in its immense versatility, allowing it to be integrated into highly precise and automated content conversion processes\footnote{Mathpix Snip. (n.d.). Mathpix Snip. Available at: \url{https://mathpix.com/}.}. It is particularly useful for converting LaTeX files (.tex) into HTML documents with embedded MathML equations, a crucial step for web accessibility\footnote{WIRIS. (n.d.). MathType. Available at: \url{https://www.wiris.com/en/}.}. It can also process digitized handwritten math (e.g., from Mathpix) into various machine-readable formats\footnote{Mathpix Snip. (n.d.). Mathpix Snip. Available at: \url{https://mathpix.com/}.}. Being a command-line tool, it offers powerful scripting capabilities for batch processing and integration into complex workflows. Despite its power, Pandoc has a steep learning curve, requiring a significant time investment to master its conversion processes and command-line options\footnote{Mathpix Snip. (n.d.). Mathpix Snip. Available at: \url{https://mathpix.com/}.}. Conversions from formats that are more expressive than Pandoc's internal Markdown representation can result in "lossy" transformations, meaning some formatting details (like margin size) or complex document elements (such as intricate tables) may not be perfectly preserved\footnote{Pandoc. (n.d.). Pandoc. Available at: \url{https://pandoc.org/}.}. This necessitates careful review of converted content for accuracy and fidelity. As a command-line tool, Pandoc offers broad compatibility across major operating systems, including Windows, Linux, and macOS\footnote{Pandoc. (n.d.). Pandoc. Available at: \url{https://pandoc.org/}.}. It can be deployed via Docker containers or installed directly with Python, Pandoc, and a LaTeX distribution\footnote{Pandoc. (n.d.). Pandoc. Available at: \url{https://pandoc.org/}.}.

Pandoc's ability to convert between a vast array of formats, including LaTeX to MathML/HTML\footnote{WIRIS. (n.d.). MathType. Available at: \url{https://www.wiris.com/en/}.}\footnote{Mathpix Snip. (n.d.). Mathpix Snip. Available at: \url{https://mathpix.com/}.}, positions it as an incredibly powerful tool for batch processing and integrating into automated accessibility workflows. However, the explicit mention of "lossy" conversions from more expressive formats\footnote{Pandoc. (n.d.). Pandoc. Available at: \url{https://pandoc.org/}.}. reveals a critical trade-off: while it facilitates accessibility by transforming content into machine-readable formats, it may compromise visual fidelity or the precise structural details of complex documents. This means that while it opens up accessibility, it also necessitates rigorous post-conversion review to ensure no critical information or nuance is lost. Pandoc is an indispensable tool for large-scale content conversion, particularly for making legacy content accessible. However, its effective deployment requires not only technical proficiency but also a deep understanding of its limitations. Organizations must balance the efficiency gains from automated conversion with the need for quality assurance to ensure that accessibility improvements do not inadvertently lead to a degradation of content accuracy or completeness. This highlights the ongoing challenge of achieving both scale and fidelity in accessible content production.

\subsection{KAlgebra}

KAlgebra is a mathematical graph calculator that is part of the KDE education package\footnote{KAlgebra. (n.d.). KAlgebra. Available at: \url{https://edu.kde.org/kalgebra/}.}. It is built upon the MathML content markup language, yet it is designed to be user-friendly, not requiring prior knowledge of MathML for operation\footnote{KAlgebra. (n.d.). KAlgebra. Available at: \url{https://edu.kde.org/kalgebra/}.}. The calculator provides a comprehensive suite of numerical, logical, symbolic, and analytical functions, with capabilities for plotting results onto both 2D and 3D graphs\footnote{KAlgebra. (n.d.). KAlgebra. Available at: \url{https://edu.kde.org/kalgebra/}.}. It supports the creation of user-defined scripts (macros) and employs an intuitive algebraic syntax similar to modern graphing calculators\footnote{KAlgebra. (n.d.). KAlgebra. Available at: \url{https://edu.kde.org/kalgebra/}.}. User-entered expressions are converted to MathML in the background, or users can directly input MathML\footnote{KAlgebra. (n.d.). KAlgebra. Available at: \url{https://edu.kde.org/kalgebra/}.}. A built-in dictionary offers a comprehensive list of functions with parameters, examples, formulas, and sample plots\footnote{KAlgebra. (n.d.). KAlgebra. Available at: \url{https://edu.kde.org/kalgebra/}.}.

As free and open-source software, KAlgebra provides a cost-effective solution for mathematical graphing and calculation\footnote{KAlgebra. (n.d.). KAlgebra. Available at: \url{https://edu.kde.org/kalgebra/}.}. Its native MathML integration, coupled with a user-friendly interface that abstracts away the underlying markup, ensures inherent accessibility\footnote{KAlgebra. (n.d.). KAlgebra. Available at: \url{https://edu.kde.org/kalgebra/}.}. The detailed function documentation within the application's dictionary further enhances its usability for learning and reference\footnote{KAlgebra. (n.d.). KAlgebra. Available at: \url{https://edu.kde.org/kalgebra/}.}. A current limitation noted is that KAlgebra only supports 3D graphs that are explicitly dependent on the X and Y axes\footnote{KAlgebra. (n.d.). KAlgebra. Available at: \url{https://edu.kde.org/kalgebra/}.}. Being a component of the KDE desktop environment, KAlgebra is primarily available on Linux distributions and potentially other platforms where KDE is supported\footnote{KAlgebra. (n.d.). KAlgebra. Available at: \url{https://edu.kde.org/kalgebra/}.}.

KAlgebra's status as an open-source tool within the KDE education package\footnote{KAlgebra. (n.d.). KAlgebra. Available at: \url{https://edu.kde.org/kalgebra/}.}, coupled with its native MathML integration (even without requiring user awareness), exemplifies how open-source communities actively contribute to accessible STEM tools. This approach often prioritizes foundational standards and community-driven development, leading to solutions that are accessible by design rather than as an afterthought. This contrasts with some commercial models that might integrate accessibility as a feature rather than a core architectural principle. Open-source projects like KAlgebra and Chirun\footnote{Texthelp. (n.d.). Equatio Pricing. Available at: \url{https://www.texthelp.com/products/equatio/pricing/}.}. are crucial for diversifying the landscape of accessible STEM solutions. They provide free, customizable, and community-supported alternatives that can drive innovation in accessibility, particularly by embedding standards like MathML directly into their core functionality. This offers significant advantages for institutions with specific customization needs, limited budgets, or a philosophical alignment with open-source principles.

\section{Presentation for Accessibility: LaTeX, MathML, and Braille}\label{sec:presentation-accessibility}
\subsection{LaTeX for High-Quality Mathematical Typesetting}
LaTeX is a typesetting system based on TeX, widely adopted for producing high-quality scientific and mathematical documents. Its strength lies in its ability to handle complex mathematical notation with precise formatting, making it the preferred choice for authors in higher education and research\footnote{BrailleMathCodes Repository. (n.d.). BrailleMathCodes Repository. Available at: \url{https://speech.di.uoa.gr/sppages/spppdf/braillemathcodes_repository.pdf}.}\footnote{MathJax. (n.d.). MathJax. Available at: \url{https://www.mathjax.org/}.}\footnote{University of Washington. (n.d.). Accessible Math. Available at: \url{https://www.washington.edu/accessibility/web/accessible-math/}.}\footnote{LaTeX. (n.d.). Available at: \url{https://www.latex-project.org/}.}. LaTeX is highly flexible and expandable through a vast community-provided ecosystem of packages, allowing users to customize and extend its functionalities for diverse mathematical expressions\footnote{BrailleMathCodes Repository. (n.d.). BrailleMathCodes Repository. Available at: \url{https://speech.di.uoa.gr/sppages/spppdf/braillemathcodes_repository.pdf}.}.

LaTeX's widespread adoption in academia and publishing for its "precise formatting" and "flexibility"\footnote{BrailleMathCodes Repository. (n.d.). BrailleMathCodes Repository. Available at: \url{https://speech.di.uoa.gr/sppages/spppdf/braillemathcodes_repository.pdf}.}\footnote{MathJax. (n.d.). MathJax. Available at: \url{https://www.mathjax.org/}.}\footnote{University of Washington. (n.d.). Accessible Math. Available at: \url{https://www.washington.edu/accessibility/web/accessible-math/}.}\footnote{LaTeX. (n.d.). Available at: \url{https://www.latex-project.org/}.}. establishes it as a de facto source of truth for complex mathematical content. Authors invest heavily in creating LaTeX documents due to its quality output and community support. The critical implication for accessibility is that this primary authoring format is not directly consumable by assistive technologies. Therefore, the goal is not to replace LaTeX, but to develop robust and efficient conversion pathways to accessible formats like MathML or braille, ensuring that the precision and richness of the original LaTeX content are preserved in the accessible output.

Despite its visual precision, LaTeX is considered a "pseudo-code" written using a QWERTY keyboard, and its visual nature means it "requires translation to braille"\footnote{BrailleMathCodes Repository. (n.d.). BrailleMathCodes Repository. Available at: \url{https://speech.di.uoa.gr/sppages/spppdf/braillemathcodes_repository.pdf}.}. More broadly, LaTeX is "not natively accessible to assistive technology" and must be converted to formats like MathML or MathJax for screen readers to interpret it effectively\footnote{University of Washington. (n.d.). Accessible Math. Available at: \url{https://www.washington.edu/accessibility/web/accessible-math/}.}\footnote{WIRIS. (n.d.). MathType. Available at: \url{https://www.wiris.com/en/}.}. The fact that LaTeX "requires translation to braille"\footnote{BrailleMathCodes Repository. (n.d.). BrailleMathCodes Repository. Available at: \url{https://speech.di.uoa.gr/sppages/spppdf/braillemathcodes_repository.pdf}.}. and is "not natively accessible to assistive technology"\footnote{University of Washington. (n.d.). Accessible Math. Available at: \url{https://www.washington.edu/accessibility/web/accessible-math/}.}. underscores the fundamental necessity of a sophisticated translation layer within the accessibility ecosystem. This layer, comprising tools such as \texttt{Pandoc}, \texttt{MathType}, \href{https://www.brailleblaster.org/}{\texttt{BrailleBlaster}}\footnote{American Printing House for the Blind. (n.d.). BrailleBlaster. Available at: \url{https://www.brailleblaster.org/}.}. and \href{https://www.duxburysystems.com/}{\texttt{Duxbury DBT}}\footnote{Duxbury Systems. (n.d.). Duxbury Braille Translator (DBT). Available at: \url{https://www.duxburysystems.com/}.}. is responsible for transforming the visual-centric LaTeX into a semantically rich, AT-consumable format like MathML or direct braille. The effectiveness of this layer directly impacts the quality and fidelity of accessible mathematical content. However, its uniform format facilitates communication with sighted users and serves as a robust source format for subsequent conversion into accessible formats.

\subsection{Braille Transcription: Codes, Software, and AI Integration}
Braille is a tactile writing system crucial for individuals with visual impairments, employing embossed dots arranged in quadrangular cells\footnote{BrailleMathCodes Repository. (n.d.). BrailleMathCodes Repository. Available at: \url{https://speech.di.uoa.gr/sppages/spppdf/braillemathcodes_repository.pdf}.}. For mathematical content, different braille codes are used worldwide, leading to a "braille math code babel"\footnote{BrailleMathCodes Repository. (n.d.). BrailleMathCodes Repository. Available at: \url{https://speech.di.uoa.gr/sppages/spppdf/braillemathcodes_repository.pdf}.}. The two predominant codes in English-speaking regions are:
\begin{itemize}
    \item \emph{Nemeth Code:} A specialized code for mathematics and science notation, primarily used in the United States. It can be embedded within Unified English Braille (UEB) text using "code switching indicators" to prevent ambiguity\footnote{BrailleMathCodes Repository. (n.d.). BrailleMathCodes Repository. Available at: \url{https://speech.di.uoa.gr/sppages/spppdf/braillemathcodes_repository.pdf}.}\footnote{Braille Authority of North America. (n.d.). Braille Codes. Available at: \url{https://www.brailleauthority.org/codes/}.}.
    \item \emph{Unified English Braille (UEB) Math:} A complete code that can be used for both literary and technical content. Some countries use UEB for math, while others utilize Nemeth within a UEB context\footnote{Braille Authority of North America. (n.d.). Braille Codes. Available at: \url{https://www.brailleauthority.org/codes/}.}.
    \item \emph{Marburg Mathematics:} A German braille math code also supported by tools like \href{https://liblouis.io/}{\texttt{Liblouis}}\footnote{BrailleMathCodes Repository. (n.d.). BrailleMathCodes Repository. Available at: \url{https://speech.di.uoa.gr/sppages/spppdf/braillemathcodes_repository.pdf}.}\footnote{Liblouis. (n.d.). Liblouis. Available at: \url{https://liblouis.io/}.}.
\end{itemize}

The existence of multiple, often incompatible, math braille codes worldwide\footnote{BrailleMathCodes Repository. (n.d.). BrailleMathCodes Repository. Available at: \url{https://speech.di.uoa.gr/sppages/spppdf/braillemathcodes_repository.pdf}.}. directly complicates the goal of universal accessibility. While UEB and Nemeth are standard in the US\footnote{Braille Authority of North America. (n.d.). Braille Codes. Available at: \url{https://www.brailleauthority.org/codes/}.}, content created for one national standard may not be directly usable or easily translatable for users in other regions. This fragmentation creates barriers to content portability and efficient global knowledge sharing in STEM for braille readers. It also underscores the need for translation software that can handle a vast array of codes and manage code-switching effectively. For international educational initiatives or global research collaborations, the diversity of math braille codes presents a significant challenge. Content creators and publishers must either produce multiple braille versions or rely on highly sophisticated translation software capable of accurately converting between these disparate systems. This complexity adds a layer of cost and technical expertise to achieving truly global braille accessibility in STEM.

\subsubsection{Dedicated Braille Translation Software}
Specialized software is essential for converting print and digital math into braille:
\begin{itemize}
    \item \emph{\href{https://www.duxburysystems.com/}{\texttt{Duxbury Braille Translator (DBT)}}:} Recognized as the "global standard" for braille translation, \texttt{DBT} supports over 180 languages and various braille math codes, including Nemeth, UEB, and French Braille\footnote{BrailleMathCodes Repository. (n.d.). BrailleMathCodes Repository. Available at: \url{https://speech.di.uoa.gr/sppages/spppdf/braillemathcodes_repository.pdf}.}\footnote{Duxbury Systems. (n.d.). Duxbury Braille Translator (DBT). Available at: \url{https://www.duxburysystems.com/}.}\footnote{Duxbury Systems. (n.d.). DBT Features. Available at: \url{https://www.duxburysystems.com/dbt_features.asp}.}\footnote{Duxbury Systems. (n.d.). DBT for Windows. Available at: \url{https://www.duxburysystems.com/dbt_win.asp}.}\footnote{Duxbury Systems. (n.d.). DBT for Mac. Available at: \url{https://www.duxburysystems.com/dbt_mac.asp}.}. It can import files from Microsoft Word (including MathType equations), LaTeX, and Open Office documents\footnote{Duxbury Systems. (n.d.). Duxbury Braille Translator (DBT). Available at: \url{https://www.duxburysystems.com/}.}\footnote{Duxbury Systems. (n.d.). DBT Features. Available at: \url{https://www.duxburysystems.com/dbt_features.asp}.}\footnote{Duxbury Systems. (n.d.). DBT for Windows. Available at: \url{https://www.duxburysystems.com/dbt_win.asp}.}\footnote{Duxbury Systems. (n.d.). DBT for Mac. Available at: \url{https://www.duxburysystems.com/dbt_mac.asp}.}. \texttt{DBT}'s related product, \texttt{NimPro}, specifically "interprets MathML markup for conversion to math braille"\footnote{Duxbury Systems. (n.d.). NimPro. Available at: \url{https://www.duxburysystems.com/nimpro.asp}.}\footnote{Duxbury Systems. (n.d.). DBT Features. Available at: \url{https://www.duxburysystems.com/dbt_features.asp}.}.
    \item \emph{\href{https://github.com/aphtech/brailleblaster}{\texttt{BrailleBlaster}}:} A free, open-source braille transcription program developed by the American Printing House for the Blind (APH)\footnote{American Printing House for the Blind. (n.d.). BrailleBlaster. Available at: \url{https://www.brailleblaster.org/}.}\footnote{American Printing House for the Blind. (n.d.). BrailleBlaster Features. Available at: \url{https://www.brailleblaster.org/features.html}.}. It automates translation and formatting by leveraging rich markup in files like NIMAS, EPUB, and DOCX\footnote{American Printing House for the Blind. (n.d.). BrailleBlaster. Available at: \url{https://www.brailleblaster.org/}.}. \texttt{BrailleBlaster} supports conversion of LaTeX to MathML and ASCII Math, and automatically converts MathML into preferred braille codes (UEB or Nemeth)\footnote{American Printing House for the Blind. (n.d.). BrailleBlaster Features. Available at: \url{https://www.brailleblaster.org/features.html}.}\footnote{American Printing House for the Blind. (n.d.). BrailleBlaster. Available at: \url{https://www.brailleblaster.org/}.}. It relies on \href{https://liblouis.io/}{\texttt{Liblouis}} for text and math translation\footnote{American Printing House for the Blind. (n.d.). BrailleBlaster Features. Available at: \url{https://www.brailleblaster.org/features.html}.}\footnote{American Printing House for the Blind. (n.d.). BrailleBlaster. Available at: \url{https://www.brailleblaster.org/}.}\footnote{Liblouis. (n.d.). Liblouis. Available at: \url{https://liblouis.io/}.}.
    \item \emph{\href{https://liblouis.io/}{\texttt{Liblouis}}:} An open-source braille translator and back-translator that supports computer and literary braille, contracted and uncontracted translation for many languages, and math braille (Nemeth and Marburg)\footnote{BrailleMathCodes Repository. (n.d.). BrailleMathCodes Repository. Available at: \url{https://speech.di.uoa.gr/sppages/spppdf/braillemathcodes_repository.pdf}.}\footnote{Liblouis. (n.d.). Liblouis. Available at: \url{https://liblouis.io/}.}\footnote{Liblouis GitHub Repository. (n.d.). Available at: \url{https://github.com/liblouis/liblouis}.}. It is used in various open-source and commercial assistive technology applications, including screen readers like NVDA and JAWS, and braille production systems like \texttt{BrailleBlaster}\footnote{Liblouis. (n.d.). Liblouis. Available at: \url{https://liblouis.io/}.}\footnote{Liblouis GitHub Repository. (n.d.). Available at: \url{https://github.com/liblouis/liblouis}.}.
\end{itemize}
The existence of multiple braille codes globally presents a significant challenge for universal accessibility\footnote{BrailleMathCodes Repository. (n.d.). BrailleMathCodes Repository. Available at: \url{https://speech.di.uoa.gr/sppages/spppdf/braillemathcodes_repository.pdf}.}. Tools like the \href{https://speech.di.uoa.gr/sppages/spppdf/braillemathcodes_repository.pdf}{\texttt{BrailleMathCodes Repository}}\footnote{BrailleMathCodes Repository. (n.d.). BrailleMathCodes Repository. Available at: \url{https://speech.di.uoa.gr/sppages/spppdf/braillemathcodes_repository.pdf}.}. aim to address this by compiling various codes, matching them with Unicode and LaTeX equivalents, and forwarding with Presentation MathML, thereby assisting in producing accessible STEM educational content.

\subsubsection{AI in Braille Translation Workflows}
AI is increasingly integrated into braille translation workflows to enhance efficiency and accuracy. Math OCR tools like \href{https://www.perkins.org/resource/create-accessible-digital-math-with-mathkicker-ai/}{\texttt{MathKicker.ai}}\footnote{Perkins School for the Blind. (n.d.). Create Accessible Digital Math with MathKicker.ai/.}. leverage AI to convert images of complex mathematical expressions into accessible digital Word documents readable by screen readers or braille displays. This represents a significant time-saving for transcribers and educators\footnote{Perkins School for the Blind. (n.d.). Create Accessible Digital Math with MathKicker.ai/.}.

Research is actively exploring how generative AI can learn the rules for both braille and process-driven math to produce accessible equations\footnote{Generative AI for Accessible Math. (n.d.). Available at: \url{https://arxiv.org/pdf/2303.06411.pdf}.}. This could lead to more automated and nuanced braille output, overcoming the limitations of traditional rule-based systems. A key challenge in braille translation, particularly for math, is the need for accurate translation of complex objects like matrices, tables, and large formulas, as well as the automation of raised graphics\footnote{AI for Braille Translation. (n.d.). Available at: \url{https://arxiv.org/pdf/2303.06411.pdf}.}. AI-powered solutions are poised to streamline these processes, contributing to the goal of producing mathematical text accurately, inexpensively, and in a timely manner for visually challenged students and professionals\footnote{AI for Braille Translation. (n.n.). Available at: \url{https://arxiv.org/pdf/2303.06411.pdf}.}.

The fact that LaTeX "requires translation to braille"\footnote{BrailleMathCodes Repository. (n.d.). BrailleMathCodes Repository. Available at: \url{https://speech.di.uoa.gr/sppages/spppdf/braillemathcodes_repository.pdf}.}. and is "not natively accessible to assistive technology"\footnote{University of Washington. (n.d.). Accessible Math. Available at: \url{https://www.washington.edu/accessibility/web/accessible-math/}.}. underscores the fundamental necessity of a sophisticated translation layer within the accessibility ecosystem. This layer, comprising tools such as \texttt{Pandoc}, \texttt{MathType}, \texttt{BrailleBlaster}, and \texttt{Duxbury DBT}, is responsible for transforming the visual-centric LaTeX into a semantically rich, AT-consumable format like MathML or direct braille. The effectiveness of this layer directly impacts the quality and fidelity of accessible mathematical content.

\section{Accessible Math Visualization and Graphing Tools}\label{sec:math-visualization}
\subsection{Desmos Accessible Graphing Calculator}
\href{https://www.desmos.com/}{\texttt{Desmos}}\footnote{Desmos. (n.d.). Desmos Graphing Calculator. Available at: \url{https://www.desmos.com/calculator}.}\footnote{Desmos. (n.d.). Desmos Accessibility. Available at: \url{https://www.desmos.com/accessibility}.}\footnote{Desmos. (n.d.). Desmos Classroom. Available at: \url{https://www.desmos.com/classroom}.}. is a powerful online graphing calculator designed with accessibility as a core feature for students with visual impairments\footnote{Desmos. (n.d.). Desmos Accessibility. Available at: \url{https://www.desmos.com/accessibility}.}\footnote{Desmos. (n.d.). Desmos Classroom. Available at: \url{https://www.desmos.com/classroom}.}. Its robust accessibility features include:
\begin{itemize}
    \item \emph{Full Screen Reader Support:} \texttt{Desmos} works seamlessly with major screen readers such as NVDA, JAWS, VoiceOver (Mac and iOS), TalkBack (Android), and ChromeVox, providing configuration instructions for optimal use\footnote{Desmos. (n.d.). Desmos Accessibility. Available at: \url{https://www.desmos.com/accessibility}.}\footnote{Desmos. (n.d.). Desmos Classroom. Available at: \url{https://www.desmos.com/classroom}.}.
    \item \emph{Audio Feedback for Graphs:} Users can explore the shape and behavior of a graph through sound cues, making mathematical relationships accessible through auditory information\footnote{Desmos. (n.d.). Desmos Accessibility. Available at: \url{https://www.desmos.com/accessibility}.}\footnote{Desmos. (n.d.). Desmos Classroom. Available at: \url{https://www.desmos.com/classroom}.}. This "audio trace" feature allows users to get audio feedback about a graph generated from an expression\footnote{Desmos. (n.d.). Desmos Accessibility. Available at: \url{https://www.desmos.com/accessibility}.}.
    \item \emph{Braille Support:} \texttt{Desmos} supports both Nemeth and Unified English Braille (UEB), enabling braille users to interact with mathematical content\footnote{Desmos. (n.d.). Desmos Accessibility. Available at: \url{https://www.desmos.com/accessibility}.}\footnote{Desmos. (n.d.). Desmos Classroom. Available at: \url{https://www.desmos.com/classroom}.}. It can export graphs in formats compatible with hard-copy braille embossers for tactile output\footnote{Desmos. (n.d.). Desmos Accessibility. Available at: \url{https://www.desmos.com/accessibility}.}.
    \item \emph{LaTeX Input and Programmatic Control:} \texttt{Desmos} accepts LaTeX input for expressions
    \footnote{Desmos. (n.d.). Desmos API. Available at: \url{https://www.desmos.com/api/v1.5/docs/index.html}.}
    \footnote{Desmos. (n.d.). Desmos LaTeX. Available at: \url{https://help.desmos.com/hc/en-us/articles/202529249-LaTeX}.}
    \footnote{Desmos. (n.d.). Desmos Graphing Calculator - LaTeX. Available at: \url{https://www.desmos.com/calculator?lang=en\&latex=y\%3Dx\%5E2}.}
    . Its API allows for embedding interactive graphs into web pages and web applications, with programmatic control over expressions and state management
    \footnote{Desmos. (n.d.). Desmos API. Available at: \url{https://www.desmos.com/api/v1.5/docs/index.html}.}
    \footnote{Desmos. (n.d.). Desmos Embed. Available at: \url{https://www.desmos.com/embed}.}
    . This enables developers to integrate \texttt{Desmos}'s powerful graphing capabilities into custom accessible solutions.
    \item \emph{Dynamic Alt Text:} The platform allows authors to write custom descriptions for interactive objects (e.g., clickable points, geometric shapes), which is beneficial for screen reader users\footnote{Desmos. (n.d.). Desmos Accessibility. Available at: \url{https://www.desmos.com/accessibility}.}.
\end{itemize}
\texttt{Desmos} also offers features like enhanced visual and selection in tables, 3D math capabilities for 2D graphs, and the ability to export points of interest to expression lists\footnote{Desmos. (n.d.). Desmos Classroom. Available at: \url{https://www.desmos.com/classroom}.}. Its design focuses on immediacy, allowing users to see the consequences of changing an expression or parameter, fostering an exploratory learning experience\footnote{Desmos. (n.d.). Desmos LaTeX. Available at: \url{https://help.desmos.com/hc/en-us/articles/202529249-LaTeX}.}. When \texttt{Desmos} cannot be used, creating an image and linking to an accessible Word document with a complex image description is a recommended alternative\footnote{Desmos. (n.d.). Desmos Accessibility. Available at: \url{https://www.desmos.com/accessibility}.}.

Desmos goes significantly beyond simple alt-text descriptions for graphs by providing audio feedback, braille support, and interactive exploration capabilities\footnote{Desmos. (n.d.). Desmos Accessibility. Available at: \url{https://www.desmos.com/accessibility}.}\footnote{Desmos. (n.d.). Desmos Classroom. Available at: \url{https://www.desmos.com/classroom}.}. This demonstrates a crucial evolution in accessible visualization: for complex visual content like graphs, a static textual description is often insufficient for true comprehension. Instead, a multi-modal approach that allows users to \textit{interact} with and \textit{experience} the visual information through auditory and tactile channels is essential. This principle can be extended to other complex scientific diagrams where spatial relationships and dynamic changes are critical. The success of Desmos in making graphing accessible highlights a best practice for all visual STEM content: prioritize interactive, multi-sensory representations over static descriptions. This implies a future where accessible diagrams are not just described, but can be explored, manipulated, and understood through various sensory modalities, significantly enhancing learning for a broader range of students with diverse needs. This pushes the boundaries of what "accessible visualization" truly means in STEM.

\subsection{Audio Graphing Calculators}
Beyond web-based tools, dedicated Audio Graphing Calculators (AGCs) provide interactive math experiences primarily for individuals with visual disabilities. The \href{https://viewplus.com/product/audio-graphing-calculator/}{\texttt{ViewPlus Audio Graphing Calculator (AGC)}}\footnote{ViewPlus. (n.d.). Audio Graphing Calculator. Available at: \url{https://viewplus.com/product/audio-graphing-calculator/}.}\footnote{ViewPlus. (n.d.). ViewPlus Tiger Software Suite. Available at: \url{https://viewplus.com/about/testimonial-page/}.}\footnote{ViewPlus. (n.d.). ViewPlus Product Page. Available at: \url{https://viewplus.com/}.}\footnote{American Printing House for the Blind. (n.d.). BrailleBlaster. Available at: \url{https://www.brailleblaster.org/}.}. is a prominent example.

\begin{itemize}
    \item \emph{Features:} The \texttt{AGC} functions as a scientific calculator on a Windows platform, providing instant feedback through intuitive tones and tactile output options\footnote{ViewPlus. (n.d.). Audio Graphing Calculator. Available at: \url{https://viewplus.com/product/audio-graphing-calculator/}.}. Key features include:
    \begin{itemize}
        \item Describing graph shape through audio tones and cues\footnote{ViewPlus. (n.d.). Audio Graphing Calculator. Available at: \url{https://viewplus.com/product/audio-graphing-calculator/}.}.
        \item Speaking menus and scalable visual display\footnote{ViewPlus. (n.d.). Audio Graphing Calculator. Available at: \url{https://viewplus.com/product/audio-graphing-calculator/}.}.
        \item Keyboard navigation\footnote{ViewPlus. (n.d.). Audio Graphing Calculator. Available at: \url{https://viewplus.com/product/audio-graphing-calculator/}.}.
        \item Advanced matrix functions, powerful expression evaluator, polar coordinate location, and the ability to display multiple graphs and find intersections\footnote{ViewPlus. (n.d.). Audio Graphing Calculator. Available at: \url{https://viewplus.com/product/audio-graphing-calculator/}.}.
    \end{itemize}
    \item \emph{Tactile Output and Audio Tones:} The \texttt{AGC}'s core accessibility features revolve around its audio tones that describe graph shapes and tactile output options for physical graph representations\footnote{ViewPlus. (n.d.). Audio Graphing Calculator. Available at: \url{https://viewplus.com/product/audio-graphing-calculator/}.}. While the specific tactile device is not always detailed, it implies compatibility with braille embossers or tactile displays.
    \item \emph{Integration:} The \texttt{AGC} operates on a Windows platform and supports several languages\footnote{ViewPlus. (n.d.). Audio Graphing Calculator. Available at: \url{https://viewplus.com/product/audio-graphing-calculator/}.}. It can integrate with \texttt{MathType} equations and directly translate them to LaTeX, Nemeth, or other braille math using the \href{https://viewplus.com/about/testimonial-page/}{\texttt{ViewPlus Tiger Software Suite}}\footnote{ViewPlus. (n.d.). ViewPlus Tiger Software Suite. Available at: \url{https://viewplus.com/about/testimonial-page/}.}\footnote{ViewPlus. (n.d.). ViewPlus Product Page. Available at: \url{https://viewplus.com/}.}.
\end{itemize}
Another example is the \href{http://www.orbitresearch.com/product/orion-ti-84-plus/}{\texttt{Orion TI-84 Plus Talking Graphing Calculator}}\footnote{Orbit Research. (n.d.). Orion TI-84 Plus Talking Graphing Calculator. Available at: \url{http://www.orbitresearch.com/product/orion-ti-84-plus/}.}. which is a fully accessible handheld graphing calculator based on the popular TI-84 Plus model. It offers synthesized speech for all textual and symbolic information, unique audio and haptic feedback for graphs (Sono Graph), and the ability to print or emboss graphs when connected to a printer or embosser\footnote{Orbit Research. (n.d.). Orion TI-84 Plus Talking Graphing Calculator. Available at: \url{http://www.orbitresearch.com/product/orion-ti-84-plus/}.}.

While general software solutions like Desmos offer valuable audio feedback for graphs\footnote{Desmos. (n.d.). Desmos Accessibility. Available at: \url{https://www.desmos.com/accessibility}.}, the ViewPlus AGC and Orion TI-84 Plus demonstrate a deeper level of accessibility through their integrated audio and tactile/haptic feedback mechanisms\footnote{ViewPlus. (n.d.). Audio Graphing Calculator. Available at: \url{https://viewplus.com/product/audio-graphing-calculator/}.}\footnote{Orbit Research. (n.d.). Orion TI-84 Plus Talking Graphing Calculator. Available at: \url{http://www.orbitresearch.com/product/orion-ti-84-plus/}.}. This indicates that for individuals with significant visual impairments, particularly when dealing with the nuanced spatial information of graphs, dedicated hardware-software solutions are often superior. These specialized tools provide a richer, more immersive sensory experience that goes beyond what general-purpose software can achieve independently, often by directly driving tactile output devices. Achieving truly comprehensive accessibility for complex visual STEM content requires a multi-layered approach. While broad software solutions enhance general access, specialized, often hardware-integrated, tools are indispensable for deep engagement with abstract concepts for users with particular sensory needs. This implies that institutions should consider a suite of tools, including dedicated hardware, to cater to the full spectrum of accessibility requirements in STEM education.

\subsection{Tactile Graphics Production (e.g., via Braille Embossers)}

Tactile output options are a core feature of the Audio Graphing Calculator, allowing for physical representations of graphs\footnote{ViewPlus. (n.d.). Audio Graphing Calculator. Available at: \url{https://viewplus.com/product/audio-graphing-calculator/}.}. Desmos, a web-based graphing calculator, has the capability to export images directly to a Braille embosser\footnote{Desmos. (n.d.). Desmos Accessibility. Available at: \url{https://www.desmos.com/accessibility}.}. Similarly, the Orion TI-84 Plus handheld calculator can print or emboss graphs when connected to a compatible printer or embosser\footnote{Orbit Research. (n.d.). Orion TI-84 Plus Talking Graphing Calculator. Available at: \url{http://www.orbitresearch.com/product/orion-ti-84-plus/}.}. The ViewPlus Tiger Software Suite, also known as PixBlaster by APH, is specifically highlighted for its ability to enable blind students to emboss high-precision braille, math, images, and graphics\footnote{ViewPlus. (n.d.). ViewPlus Tiger Software Suite. Available at: \url{https://viewplus.com/about/testimonial-page/}}.

The available information collectively illustrates that tactile graphics production is not a single-tool function but rather an end-to-end workflow. Tools like Desmos and specialized calculators can \textit{generate} or \textit{export} graphical data in a format suitable for tactile output\footnote{Desmos. (n.d.). Desmos Accessibility. Available at: \url{https://www.desmos.com/accessibility}}\footnote{Orbit Research. (n.d.). Orion TI-84 Plus Talking Graphing Calculator. Available at: \url{http://www.orbitresearch.com/product/orion-ti-84-plus/}.}. This exported data then needs to be processed and physically produced by specialized software (e.g., ViewPlus Tiger Software) and dedicated hardware (Braille embossers)\footnote{ViewPlus. (n.d.). ViewPlus Tiger Software Suite. Available at: \url{https://viewplus.com/about/testimonial-page/}.}. This chain of events highlights the critical importance of interoperability between different software components and the physical embossing devices. A breakdown at any point in this chain can render the entire process ineffective. Implementing tactile graphics accessibility requires a holistic understanding of the entire workflow, from the initial digital creation of the graphic to its final physical production. Organizations must ensure compatibility between their chosen graphing/visualization tools, any intermediate processing software, and the specific braille embosser hardware. This often involves a multi-vendor solution and necessitates careful planning to ensure a seamless and reliable production pipeline for tactile STEM materials.

\subsection{Image Accessibility Generators}

The Image Accessibility Generator, developed by Arizona State University, offers an automated solution for generating alt-text for images\footnote{Arizona State University. (n.d.). Image Accessibility Generator. Available at: \url{https://accessibility.asu.edu/image-accessibility-generator}.}. This tool utilizes the advanced GPT-4o model from OpenAI to produce descriptive alternative text. Users can upload an image, optionally add their own descriptors to guide the AI, and then generate a detailed image description\footnote{Arizona State University. (n.d.). Image Accessibility Generator. Available at: \url{https://accessibility.asu.edu/image-accessibility-generator}.}.

The use of a sophisticated AI model like GPT-4o for alt-text generation\footnote{Arizona State University. (n.d.). Image Accessibility Generator. Available at: \url{https://accessibility.asu.edu/image-accessibility-generator}.}. signifies a significant step towards automating fundamental accessibility tasks. This can substantially streamline the process of making images minimally accessible, especially for content creators who may lack the expertise or time to manually write detailed descriptions for every image. However, for complex STEM diagrams (e.g., intricate graphs, chemical structures, or flowcharts), a simple alt-text, even if AI-generated, may not convey the full semantic meaning or interactive potential required for true comprehension. The information on Desmos and AGC\footnote{Desmos. (n.d.). Desmos Accessibility. Available at: \url{https://www.desmos.com/accessibility}.}\footnote{ViewPlus. (n.d.). Audio Graphing Calculator. Available at: \url{https://viewplus.com/product/audio-graphing-calculator/}.}. suggests that for such complex visuals, a multi-modal, interactive representation is often superior to a purely textual description. AI can democratize basic accessibility by simplifying routine tasks like alt-text generation, making it easier for a wider range of content creators to meet foundational accessibility requirements. However, for the nuanced and information-dense visuals common in STEM, human review and the integration of more sophisticated multi-modal representations (auditory, tactile, interactive) remain essential. AI-generated alt-text should be seen as a starting point, not a complete solution, for complex STEM diagrams.

\section{Ethical Considerations and Future Directions}\label{sec:ethical-considerations}
\subsection{Bias and Fairness in AI}
The increasing reliance on AI in Math OCR and accessibility tools introduces critical ethical considerations, particularly concerning bias and fairness. AI systems learn from the data they are trained on, and if this data is not diverse or contains societal inequalities, the AI's outcomes can be unfair or skewed\footnote{AI Ethics and Bias. (n.d.). Available at: \url{https://www.ibm.com/cloud/learn/ai-ethics}.}\footnote{Bias in AI. (n.d.). Available at: \url{https://www.accenture.com/us-en/insights/artificial-intelligence/responsible-ai}.}. For instance, biases could manifest as systems that perform less accurately for certain handwriting styles, or in grading systems that unintentionally favor specific demographic groups\footnote{Bias in AI. (n.d.). Available at: \url{https://www.accenture.com/us-en/insights/artificial-intelligence/responsible-ai}.}.

It is imperative that developers and users recognize and address potential biases in training data, algorithm design, and human decision-making that may inadvertently favor certain groups\footnote{AI Ethics and Bias. (n.d.). Available at: \url{https://www.ibm.com/cloud/learn/ai-ethics}.}. Students and professionals must critically evaluate AI-generated information to ensure the validity, reliability, and trustworthiness of the results\footnote{AI Ethics and Bias. (n.d.). Available at: \url{https://www.ibm.com/cloud/learn/ai-ethics}.}. Accountability in AI includes transparent reporting of methods and findings to ensure the integrity and ethical standards of scholarly inquiry\footnote{AI Ethics and Bias. (n.d.). Available at: \url{https://www.ibm.com/cloud/learn/ai-ethics}.}. This involves ensuring that AI models are not "black-boxes" but rather systems where the logic behind data processing can be understood, and subject matter experts can contribute their domain-specific knowledge to improve data quality and mitigate bias\footnote{Semantic AI. (n.d.). Available at: \url{https://www.ibm.com/cloud/learn/semantic-ai}.}.

\subsection{Data Privacy and Security}
AI systems in education and document processing often collect various types of sensitive data, including personal information, academic records, and behavioral data\footnote{Data Privacy and AI. (n.d.). Available at: \url{https://www.accenture.com/us-en/insights/artificial-intelligence/responsible-ai}.}. The collection and storage of such data pose significant privacy risks, including unauthorized access, data breaches, or misuse of information for purposes beyond education\footnote{Data Privacy and AI. (n.d.). Available at: \url{https://www.accenture.com/us-en/insights/artificial-intelligence/responsible-ai}.}. For example, if behavioral data is not properly safeguarded, it could be exploited for non-educational commercial purposes or lead to inaccurate assumptions about students\footnote{Data Privacy and AI. (n.d.). Available at: \url{https://www.accenture.com/us-en/insights/artificial-intelligence/responsible-ai}.}.

To address these concerns, educational institutions and technology providers must prioritize informed consent and transparency, clearly explaining what data is being collected, how it will be stored, and for what purposes it will be used\footnote{Data Privacy and AI. (n.d.). Available at: \url{https://www.accenture.com/us-en/insights/artificial-intelligence/responsible-ai}.}. Establishing strict data protection protocols and adhering to privacy laws are essential for building trust and ensuring that student data is handled responsibly and ethically\footnote{Data Privacy and AI. (n.d.). Available at: \url{https://www.accenture.com/us-en/insights/artificial-intelligence/responsible-ai}.}. Solutions like on-premises deployment options for AI-powered OCR, as offered by some providers, can help enterprises process sensitive data while maintaining strict compliance standards\footnote{Marker Documentation. (n.d.). Available at: \url{https://marker.datalab.to/}.}.

\subsection{Future Outlook}
The field of AI-powered math OCR and accessibility is rapidly evolving, with several promising directions:
\begin{itemize}
    \item \emph{Generative AI for Braille:} Future projects aim to explore how generative AI can learn the complex rules for both braille and process-driven math to produce accessible equations\footnote{Generative AI for Accessible Math. (n.d.). Available at: \url{https://arxiv.org/pdf/2303.06411.pdf}.}. This could lead to more automated, nuanced, and scalable braille production, significantly reducing the manual effort and time currently required.
    \item \emph{Adaptive Rendering and Personalized Learning:} AI-driven adaptive learning systems can tailor content to each student's pace and skill level, providing personalized instruction in foundational math and literacy\footnote{Modmath. (n.d.). Modmath. Available at: \url{https://www.modmath.com/}.}\footnote{Mathematical Expression Recognition with Denoising Diffusion Probabilistic Models. (n.d.). Available at: \url{https://arxiv.org/pdf/2303.06411.pdf}.}. This includes text-to-speech tools, closed captions, and interactive quizzes to support students with diverse learning needs\footnote{Mathematical Expression Recognition with Denoising Diffusion Probabilistic Models. (n.d.). Available at: \url{https://arxiv.org/pdf/2303.06411.pdf}.}. Tools like \href{https://www.modmath.com/}{\texttt{Modmath}}\footnote{Modmath. (n.d.). Modmath. Available at: \url{https://www.modmath.com/}.}. already offer digital graph paper and customizable keypads to help students with learning differences write and solve math problems, acting as an accommodation for pencil and paper.
    \item \emph{Semantic AI for Deeper Understanding:} The continued development of semantic AI, leveraging knowledge graphs, could enable AI systems to not only recognize and translate mathematical expressions but also to interpret their deeper mathematical meaning and context\footnote{Semantic AI. (n.d.). Available at: \url{https://www.ibm.com/cloud/learn/semantic-ai}.}\footnote{Knowledge Graphs and Semantic AI. (n.d.). Available at: \url{https://www.ontotext.com/knowledge-graphs/semantic-ai/}.}. This could power advanced tutoring systems that provide step-by-step solutions and hints, or even assist in completing mathematical proofs\footnote{Texthelp. (n.d.). Equatio. Available at: \url{https://www.texthelp.com/products/equatio/}.}\footnote{Texthelp. (n.d.). Equatio Pricing. Available at: \url{https://www.texthelp.com/products/equatio/pricing/}.}\footnote{AI for Mathematical Proofs. (n.d.). Available at: \url{https://arxiv.org/pdf/2303.06411.pdf}.}.
    \item \emph{Enhanced Multimodal Interaction:} The integration of AI with tools like \texttt{Desmos} and Audio Graphing Calculators suggests a future where users can interact with mathematical content through a rich combination of visual display, auditory feedback, tactile output, and voice commands, creating a truly inclusive learning environment\footnote{Desmos. (n.d.). Desmos Accessibility. Available at: \url{https://www.desmos.com/accessibility}.}\footnote{ViewPlus. (n.d.). Audio Graphing Calculator. Available at: \url{https://viewplus.com/product/audio-graphing-calculator/}.}\footnote{Orbit Research. (n.d.). Orion TI-84 Plus Talking Graphing Calculator. Available at: \url{http://www.orbitresearch.com/product/orion-ti-84-plus/}.}.
\end{itemize}
These advancements promise to further break down barriers in STEM education, ensuring that all students, regardless of their abilities, have equitable access to high-quality mathematical content.

\section{Conclusion}\label{sec:conclusion-mathocr}
The journey towards fully accessible mathematical content is a complex but critical endeavor, driven by both legal mandates and the imperative of educational equity. AI-powered Math OCR, exemplified by tools like \texttt{Marker}, represents a significant leap forward in digitizing and structuring mathematical expressions from various sources, including challenging handwritten inputs. The integration of deep learning architectures, particularly encoder-decoder models with attention mechanisms and the strategic use of LLMs, enables these tools to achieve high accuracy and contextual understanding, transforming static images into semantically rich data.

The ecosystem of accessible math relies heavily on standardized formats like MathML, which acts as a universal language for assistive technologies, enabling features such as aural navigation, dynamic scaling, and braille conversion. While LaTeX remains the gold standard for authoring high-quality mathematical documents, its conversion to MathML through robust intermediary tools like \texttt{Pandoc} and \texttt{MathType} is essential for broad accessibility. Dedicated braille translation software, such as \texttt{DBT} and \texttt{BrailleBlaster}, further bridges the gap to tactile output, with AI poised to revolutionize these processes through generative models that learn braille rules directly.

Accessible graphing and visualization tools like \texttt{Desmos} and Audio Graphing Calculators offer multimodal interaction, providing auditory, tactile, and dynamic visual representations of mathematical concepts. However, the deployment of these advanced AI technologies necessitates careful consideration of ethical implications, particularly concerning bias in training data and the privacy of sensitive user information.

The trajectory of this field points towards increasingly intelligent, adaptive, and multimodal accessibility solutions. Future developments in generative AI and semantic AI hold the potential to create truly personalized learning experiences and sophisticated mathematical assistants that can interpret, explain, and even solve complex problems, making advanced mathematics approachable for all learners. Continued research, collaborative open-source development, and a steadfast commitment to ethical AI practices will be paramount in realizing this inclusive future for STEM education.
