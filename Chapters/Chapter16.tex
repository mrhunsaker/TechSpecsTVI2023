\chapter{Accessibility Auditing for Internet-Based Materials: A Comprehensive Guide}
\label{chap:accessibility-auditing}

\section{Introduction}
\label{sec:intro-audit}

Digital accessibility has become a critical consideration in web development and content creation, driven by legal requirements, ethical imperatives, and the recognition that inclusive design benefits all users. An accessibility audit is a systematic evaluation of digital content to identify barriers that prevent users with disabilities from accessing, navigating, and interacting with web-based materials effectively. This chapter provides a comprehensive guide to conducting accessibility audits, exploring available tools, and implementing effective remediation strategies.

The importance of accessibility auditing extends beyond compliance with legal frameworks such as the Americans with Disabilities Act (ADA), Section 508 of the Rehabilitation Act, and the European Accessibility Act. These audits serve as essential quality assurance processes that ensure digital content is usable by individuals with diverse abilities, including those with visual, auditory, motor, and cognitive impairments\footnote{WebAIM. (2024). \textit{Introduction to Web Accessibility}. Retrieved from https://webaim.org/intro/}.

\section{Fundamental Principles and Standards}
\label{sec:principles-standards}

\subsection{Web Content Accessibility Guidelines (WCAG)}

The Web Content Accessibility Guidelines (WCAG) 2.1, developed by the World Wide Web Consortium (W3C), serve as the international standard for web accessibility. These guidelines are organized around four fundamental principles, often referred to by the acronym POUR\footnote{World Wide Web Consortium. (2018). \textit{Web Content Accessibility Guidelines (WCAG) 2.1}. W3C Recommendation. Retrieved from https://www.w3.org/TR/WCAG21/}:

\textbf{Perceivable:} Information and user interface components must be presentable to users in ways they can perceive. This includes providing text alternatives for non-text content, captions for multimedia, and ensuring sufficient color contrast ratios.

\textbf{Operable:} User interface components and navigation must be operable by all users. This encompasses keyboard accessibility, seizure-safe content, and providing users enough time to read and use content.

\textbf{Understandable:} Information and the operation of user interface must be understandable. This involves making text readable and understandable, and making content appear and operate in predictable ways.

\textbf{Robust:} Content must be robust enough that it can be interpreted reliably by a wide variety of user agents, including assistive technologies.

WCAG 2.1 defines three levels of conformance: A (minimum level), AA (standard level), and AAA (enhanced level). Most organizations target AA conformance as it provides a balance between accessibility and practical implementation considerations\footnote{Caldwell, B., Cooper, M., Reid, L. G., \& Vanderheiden, G. (2008). \textit{Web Content Accessibility Guidelines (WCAG) 2.0}. W3C Recommendation}.

\subsection{Section 508 and Legal Frameworks}

Section 508 of the Rehabilitation Act requires federal agencies in the United States to make their electronic and information technology accessible to people with disabilities. The updated Section 508 standards, effective as of January 2018, incorporate WCAG 2.0 Level AA standards\footnote{U.S. Access Board. (2018). \textit{Information and Communication Technology (ICT) Standards and Guidelines}. 36 CFR Part 1194}.

\section{Comprehensive Audit Methodology}
\label{sec:audit-methodology}

\subsection{Planning Phase}

The planning phase establishes the foundation for an effective accessibility audit. This phase involves defining the scope of the audit, identifying target user groups, and establishing success criteria. Auditors must determine which pages, applications, or content areas will be evaluated, considering factors such as user traffic, critical functionality, and content types.

Stakeholder engagement during the planning phase ensures that the audit addresses organizational priorities and user needs. This includes consulting with disability advocacy groups, user experience teams, and content creators to understand the specific challenges users may encounter~\cite{DisabilityRightsAuditing}.

\subsection{Automated Testing Phase}

Automated testing tools can efficiently identify many accessibility issues, particularly those related to markup structure, color contrast, and missing alternative text. However, automated tools typically detect only 20-30\% of accessibility issues, making manual testing essential for comprehensive auditing~\cite{WebAIMSurvey}.

The automated testing phase should include both browser-based extensions and comprehensive scanning tools. This phase provides a baseline assessment and identifies obvious violations that can be addressed before more detailed manual testing begins.

\subsection{Manual Testing Phase}

Manual testing involves systematic evaluation of content using assistive technologies and specialized testing procedures. This phase includes keyboard navigation testing, screen reader evaluation, and cognitive load assessment. Manual testing is essential for evaluating the user experience and identifying issues that automated tools cannot detect, such as logical tab order, meaningful content structure, and contextual information.

Expert evaluators should simulate the experience of users with different types of disabilities, using assistive technologies such as screen readers, voice recognition software, and alternative input devices. This testing should occur across different browsers and devices to ensure consistent accessibility~\cite{PowerAccessibility}.

\subsection{User Testing Phase}

User testing with individuals who have disabilities provides invaluable insights into real-world accessibility challenges. This phase involves recruiting participants with diverse disabilities and observing their interactions with the content or application. User testing reveals issues that may not be apparent during expert evaluation and provides direct feedback on the effectiveness of accessibility implementations.

Participants should represent the target user base and include individuals with various types of disabilities, assistive technology preferences, and experience levels. Testing sessions should be conducted in naturalistic environments when possible, allowing participants to use their familiar assistive technologies and workflows\footnote{Petrie, H., Hamilton, F., King, N., \& Pavan, P. (2006). Remote usability evaluations with disabled people. \textit{Proceedings of the SIGCHI Conference on Human Factors in Computing Systems}, 1133-1141}.

\section{Automated Testing Tools}
\label{sec:automated-tools}

\subsection{Browser Extensions}

Browser extensions provide immediate accessibility feedback during the development and testing process. These tools integrate directly into the browser environment, allowing for real-time evaluation of web content.

\textbf{axe DevTools} is one of the most widely used browser extensions for accessibility testing. Developed by Deque Systems, axe DevTools provides comprehensive WCAG 2.1 compliance checking with detailed issue descriptions and remediation guidance. The tool identifies violations, provides best practice suggestions, and offers guided manual testing workflows\footnote{Deque Systems. (2024). \textit{axe DevTools: Accessibility Testing Tools}. Retrieved from https://www.deque.com/axe/devtools/}.

\textbf{WAVE (Web Accessibility Evaluation Tool)} is a free browser extension developed by WebAIM that provides visual feedback about the accessibility of web content. WAVE identifies accessibility and WCAG errors, facilitates manual evaluation, and provides detailed documentation about accessibility issues\footnote{WebAIM. (2024). \textit{WAVE Web Accessibility Evaluation Tool}. Retrieved from https://wave.webaim.org/}.

\textbf{Accessibility Insights for Web}, developed by Microsoft, provides automated checks and guided manual testing for web applications. The tool includes fastpass automated checks, assessment workflows for comprehensive testing, and detailed guidance for manual testing procedures\footnote{Microsoft. (2024). \textit{Accessibility Insights for Web}. Retrieved from https://accessibilityinsights.io/docs/en/web/overview/}.

\subsection{Comprehensive Scanning Platforms}

Enterprise-level scanning platforms provide automated accessibility testing for entire websites or applications, offering scalable solutions for large organizations.

\textbf{Deque WorldSpace} is a comprehensive accessibility testing platform that provides automated scanning, manual testing tools, and remediation guidance. The platform supports integration with continuous integration/continuous deployment (CI/CD) pipelines and provides detailed reporting and analytics\footnote{Deque Systems. (2024). \textit{WorldSpace: Enterprise Accessibility Testing Platform}. Retrieved from https://www.deque.com/worldspace/}.

\textbf{Siteimprove} offers automated accessibility monitoring and testing across entire digital properties. The platform provides continuous monitoring, detailed reporting, and integration with content management systems and development workflows\footnote{Siteimprove. (2024). \textit{Accessibility Testing and Monitoring}. Retrieved from https://siteimprove.com/en/accessibility/}.

\textbf{AudioEye} provides automated accessibility testing and remediation services, including real-time scanning and AI-powered fix suggestions. The platform offers both automated testing and human expert review services\footnote{AudioEye. (2024). \textit{Web Accessibility Testing and Remediation}. Retrieved from https://www.audioeye.com/}.

\subsection{Open Source Testing Tools}

Open source accessibility testing tools provide cost-effective solutions for organizations with limited budgets while offering transparency and customization opportunities.

\textbf{Pa11y} is a command-line accessibility testing tool that provides automated accessibility testing using the HTML CodeSniffer engine. Pa11y can be integrated into build processes and provides detailed reporting of accessibility issues\footnote{Pa11y. (2024). \textit{Pa11y: Command-line Accessibility Testing Tool}. Retrieved from https://pa11y.org/}.

\textbf{Lighthouse} is Google's open-source automated tool for improving web page quality, including accessibility auditing. Lighthouse provides accessibility scoring based on axe-core rules and can be run from Chrome DevTools, the command line, or as a Node module\footnote{Google. (2024). \textit{Lighthouse: Automated Tool for Improving Web Page Quality}. Retrieved from https://developers.google.com/web/tools/lighthouse}.

\textbf{Axe-core} is the open-source accessibility testing engine that powers many commercial and open-source accessibility testing tools. Axe-core provides a JavaScript API for integrating accessibility testing into custom applications and testing frameworks\footnote{Deque Systems. (2024). \textit{axe-core: Accessibility Testing Engine}. Retrieved from https://github.com/dequelabs/axe-core}.

\section{Manual Testing Methodologies}
\label{sec:manual-methods}

\subsection{Keyboard Navigation Testing}

Keyboard navigation testing ensures that all interactive elements are accessible to users who cannot use a mouse or other pointing device. This testing involves navigating through the entire interface using only the keyboard, verifying that all functionality is available and that the tab order is logical and intuitive.

Testers should verify that all interactive elements receive visible focus indicators, that tab navigation follows a logical sequence, and that keyboard shortcuts do not conflict with assistive technology commands. Special attention should be paid to complex interface elements such as dropdown menus, modal dialogs, and dynamic content updates\footnote{Thatcher, J., Burks, M. R., Heilmann, C., Henry, S. L., Kirkpatrick, A., Lauke, P. H., ... \& Urban, M. (2006). \textit{Web accessibility: Web standards and regulatory compliance}. Friends of ED}.

\subsection{Screen Reader Testing}

Screen reader testing involves evaluating content using screen reading software such as NVDA, JAWS, or VoiceOver. This testing ensures that content is properly structured for screen readers and that all information is conveyed through audio output.

Testing should include verification of heading structure, alternative text for images, form labels, table headers, and landmark regions. Testers should also evaluate the logical reading order and ensure that dynamic content changes are announced appropriately\footnote{Leporini, B., \& Paternò, F. (2004). Increasing usability when interacting through screen readers. \textit{Universal Access in the Information Society}, 3(1), 57-70}.

\subsection{Color and Contrast Testing}

Color and contrast testing ensures that content is perceivable by users with visual impairments, including color blindness and low vision. This testing involves measuring color contrast ratios and verifying that information is not conveyed through color alone.

Contrast ratios should meet WCAG requirements: 4.5:1 for normal text and 3:1 for large text at the AA level. Testing should also include simulation of different types of color blindness to ensure that color-coded information remains accessible\footnote{Ware, C. (2012). \textit{Information visualization: perception for design}. Elsevier}.

\section{Emerging AI-Powered Accessibility Tools}

\subsection{Machine Learning-Based Testing}

Artificial intelligence and machine learning technologies are increasingly being integrated into accessibility testing tools, providing more sophisticated analysis and automated remediation capabilities.

\textbf{Evinced} uses artificial intelligence to provide more accurate accessibility testing and reduce false positives. The platform analyzes user interactions and provides contextual accessibility insights that traditional rule-based tools may miss\footnote{Evinced. (2024). \textit{AI-Powered Accessibility Testing}. Retrieved from https://www.evinced.com/}.

\textbf{Stark} incorporates AI-powered features for design accessibility testing, providing real-time feedback on color contrast, typography, and layout accessibility within design tools such as Figma and Sketch\footnote{Stark. (2024). \textit{AI-Powered Design Accessibility Tools}. Retrieved from https://www.getstark.co/}.

\subsection{Automated Remediation Tools}

AI-powered automated remediation tools can identify accessibility issues and suggest or implement fixes automatically, reducing the manual effort required for accessibility improvements.

\textbf{AccessiBe} uses artificial intelligence to automatically apply accessibility modifications to websites, including screen reader adjustments, keyboard navigation improvements, and visual adjustments for users with visual impairments\footnote{AccessiBe. (2024). \textit{AI-Powered Web Accessibility Solutions}. Retrieved from https://accessibe.com/}.

\textbf{UserWay} provides AI-powered accessibility widgets and automated remediation services that can be integrated into existing websites to improve accessibility without requiring extensive code changes\footnote{UserWay. (2024). \textit{AI-Powered Accessibility Solutions}. Retrieved from https://userway.org/}.

\subsection{Natural Language Processing for Content Analysis}

Advanced AI tools are beginning to incorporate natural language processing to evaluate content readability, plain language compliance, and cognitive accessibility factors.

These tools can analyze text complexity, sentence structure, and vocabulary to identify content that may be difficult for users with cognitive disabilities to understand. Some platforms also provide suggestions for simplifying language and improving content structure\footnote{Rello, L., \& Baeza-Yates, R. (2013). Good fonts for dyslexia. \textit{Proceedings of the 15th International ACM SIGACCESS Conference on Computers and Accessibility}, 1-8}.

\section{Commercial vs. Open Source Solutions}

\subsection{Commercial Solutions}

Commercial accessibility testing solutions typically offer comprehensive support, regular updates, enterprise-level features, and professional services. These solutions often provide better integration with enterprise workflows, detailed reporting and analytics, and dedicated customer support.

The advantages of commercial solutions include professional technical support, regular updates to address new accessibility requirements, comprehensive training materials, and enterprise-grade security and compliance features. However, these solutions typically require significant financial investment and may involve vendor lock-in considerations\footnote{Lazar, J., Goldstein, D., \& Taylor, A. (2015). \textit{Ensuring digital accessibility through process and policy}. Morgan Kaufmann}.

\subsection{Open Source Solutions}

Open source accessibility testing tools provide transparency, customization opportunities, and cost-effective solutions for organizations with limited budgets. These tools often have active community support and can be modified to meet specific organizational needs.

The advantages of open source solutions include no licensing costs, full access to source code for customization, community-driven development, and freedom from vendor lock-in. However, organizations may need to invest in internal expertise for implementation and maintenance, and support is typically community-based rather than professional\footnote{Stallman, R. (2002). \textit{Free software, free society: Selected essays of Richard M. Stallman}. GNU Press}.

\subsection{Hybrid Approaches}

Many organizations adopt hybrid approaches that combine commercial and open source tools to balance cost, functionality, and support requirements. This approach might involve using open source tools for basic automated testing and commercial solutions for comprehensive auditing and expert review.

\section{Remediation Strategies and Implementation}\label{sec:remediation-implementation}

\subsection{Prioritizing Accessibility Issues}

Effective remediation requires strategic prioritization of accessibility issues based on severity, impact, and implementation complexity. Issues should be categorized using frameworks such as the WCAG conformance levels, user impact assessments, and technical complexity evaluations.

High-priority issues typically include those that completely prevent access to content or functionality, violate legal requirements, or affect large numbers of users. Medium-priority issues may include usability problems that create significant barriers but do not completely prevent access. Low-priority issues might involve minor usability improvements or enhancements that benefit specific user groups\footnote{Henry, S. L. (2007). \textit{Just ask: integrating accessibility throughout design}. ET\textbackslash Lawton}.

\subsection{Technical Remediation Approaches}

Technical remediation involves implementing code changes, content modifications, and design improvements to address identified accessibility issues. This process requires collaboration between developers, designers, and content creators to ensure comprehensive solutions.

Common technical remediation approaches include adding alternative text to images, implementing proper heading structures, ensuring keyboard accessibility, improving color contrast, and providing captions for multimedia content. More complex remediation may involve restructuring page layouts, implementing ARIA (Accessible Rich Internet Applications) attributes, or redesigning user interface components\footnote{Faulkner, S., Eicholtz, M., Arch, A., \& Wakkary, R. (2019). \textit{Accessibility handbook: making 508 compliant websites}. O'Reilly Media}.

\subsection{Content Remediation Strategies}

Content remediation focuses on improving the accessibility of textual and multimedia content through editing, restructuring, and enhancement. This process involves working with content creators to ensure that information is presented in accessible formats and that language is clear and understandable.

Content remediation strategies include simplifying complex language, adding descriptive headings, providing summaries for long documents, creating alternative formats for complex information, and ensuring that multimedia content includes appropriate captions and transcripts\footnote{Redish, J. (2012). \textit{Letting go of the words: Writing web content that works}. Morgan Kaufmann}.

\subsection{Design Remediation Principles}

Design remediation involves modifying visual design elements to improve accessibility while maintaining aesthetic appeal and brand consistency. This process requires understanding the relationship between design choices and accessibility requirements.

Design remediation principles include ensuring sufficient color contrast, providing multiple visual cues for important information, designing clear focus indicators, creating consistent navigation patterns, and ensuring that interactive elements are large enough for users with motor impairments\footnote{Krug, S. (2014). \textit{Don't make me think, revisited: A common sense approach to web usability}. New Riders}.

\section{Quality Assurance and Continuous Monitoring}\label{sec:qa-monitoring}

\subsection{Establishing Accessibility Testing Protocols}

Sustainable accessibility requires establishing ongoing testing protocols that integrate accessibility evaluation into regular development and content creation workflows. These protocols should include automated testing integration, regular manual testing schedules, and user feedback mechanisms.

Testing protocols should define roles and responsibilities, establish testing frequency, specify testing tools and methodologies, and create clear documentation and reporting procedures. Regular protocol reviews ensure that testing approaches remain current with evolving accessibility standards and organizational needs\footnote{Jaeger, P. T. (2006). Assessing section 508 compliance on federal e-government web sites: A multi-method, user-centered evaluation of accessibility for persons with disabilities. \textit{Government Information Quarterly}, 23(2), 169-190}.

\subsection{Continuous Integration and Deployment}

Integrating accessibility testing into continuous integration and deployment pipelines ensures that accessibility issues are identified and addressed early in the development process. This approach reduces the cost and complexity of remediation while preventing the introduction of new accessibility barriers.

Continuous integration approaches include automated accessibility testing in build processes, pull request accessibility checks, and deployment gates that prevent releases with critical accessibility issues. These systems should provide clear feedback to developers and include guidance for addressing identified issues\footnote{Fowler, M. (2013). \textit{Continuous delivery: reliable software releases through build, test, and deployment automation}. Addison-Wesley Professional}.

\subsection{Performance Monitoring and Analytics}

Accessibility performance monitoring involves tracking key metrics related to accessibility compliance, user experience, and remediation effectiveness. This monitoring provides insights into the success of accessibility initiatives and identifies areas for improvement.

Monitoring metrics may include accessibility compliance scores, user task completion rates, assistive technology usage patterns, and user satisfaction measures. Regular analysis of these metrics helps organizations understand the impact of accessibility improvements and guide future accessibility investments\footnote{Disability Statistics Annual Report. (2023). \textit{Annual Report on the Employment of People with Disabilities}. University of New Hampshire}.

\section{Training and Organizational Implementation}\label{sec:training-org}

\subsection{Stakeholder Education and Training}

Successful accessibility implementation requires comprehensive training for all stakeholders involved in content creation, design, and development. Training programs should be tailored to different roles and responsibilities while ensuring that all team members understand the importance of accessibility and their role in maintaining it.

Training topics should include disability awareness, accessibility standards and guidelines, testing methodologies, remediation techniques, and organizational policies and procedures. Regular refresher training ensures that teams stay current with evolving accessibility requirements and best practices\footnote{Burgstahler, S. (2015). \textit{Universal design in higher education: From principles to practice}. Harvard Education Press}.

\subsection{Policy Development and Governance}

Organizational accessibility policies provide the framework for consistent accessibility implementation across all digital properties. These policies should define accessibility standards, assign responsibilities, establish testing requirements, and create accountability mechanisms.

Effective policies include clear accessibility standards, defined roles and responsibilities, testing and compliance requirements, procurement guidelines for accessible technologies, and procedures for addressing accessibility complaints or issues. Regular policy reviews ensure that governance frameworks remain current and effective\footnote{Seeman, L., \& Cooper, M. (2015). \textit{Cognitive accessibility user research}. W3C First Public Working Draft}.

\section{Future Trends and Considerations}\label{sec:future-trends}

\subsection{Emerging Technologies and Accessibility}

As new technologies such as virtual reality, augmented reality, and voice interfaces become more prevalent, accessibility auditing methodologies must evolve to address new types of barriers and interaction patterns. These technologies present both opportunities and challenges for accessibility implementation.

Future accessibility auditing will need to consider multi-modal interfaces, spatial navigation, gesture-based interactions, and immersive content experiences. Auditors will need to develop new testing methodologies and evaluation criteria for these emerging interaction paradigms\footnote{Ye, H., Meethu, M., \& Borg, J. (2022). An exploratory study on the accessibility of virtual reality for people with disabilities. \textit{Proceedings of the 2022 CHI Conference on Human Factors in Computing Systems}, 1-12}.

\subsection{Regulatory Evolution}

Accessibility regulations continue to evolve, with new requirements and standards being developed at national and international levels. The European Accessibility Act, updated Section 508 standards, and emerging artificial intelligence regulations will influence accessibility auditing practices.

Organizations must stay informed about regulatory changes and adapt their auditing practices accordingly. This includes understanding new compliance requirements, updating testing methodologies, and ensuring that accessibility programs remain aligned with legal obligations\footnote{European Commission. (2019). \textit{European Accessibility Act}. Directive (EU) 2019/882}.

\subsection{Artificial Intelligence and Automation}

The continued development of artificial intelligence and automation technologies will significantly impact accessibility auditing practices. AI-powered tools will become more sophisticated in identifying and addressing accessibility issues, while also raising new questions about the role of human expertise in accessibility evaluation.

Future accessibility auditing will likely involve closer integration between automated tools and human expertise, with AI systems handling routine testing tasks and human experts focusing on complex usability evaluation and user experience assessment\footnote{Benton, L., Johnson, H., Ashwin, E., Brosnan, M., \& Grawemeyer, B. (2012). Developing IDEAS: supporting children with autism within a participatory design team. \textit{Proceedings of the SIGCHI Conference on Human Factors in Computing Systems}, 2599-2608}.

\section{Conclusion}\label{sec:conclusion-audit}

Accessibility auditing represents a critical component of inclusive digital design and development practices. The comprehensive methodologies, tools, and strategies outlined in this chapter provide organizations with the knowledge and resources necessary to conduct effective accessibility evaluations and implement meaningful improvements.

The field of accessibility auditing continues to evolve, driven by technological advances, regulatory changes, and growing recognition of the importance of inclusive design. Organizations that commit to comprehensive accessibility auditing practices will not only meet legal requirements but also create more usable and inclusive digital experiences for all users.

Success in accessibility auditing requires a combination of technical expertise, user-centered design principles, and organizational commitment to inclusive practices. By implementing the strategies and utilizing the tools described in this chapter, organizations can develop robust accessibility auditing capabilities that support their broader commitment to digital inclusion and equal access to information and services.

The investment in accessibility auditing pays dividends through improved user experiences, reduced legal risks, expanded market reach, and the fulfillment of ethical obligations to create inclusive digital environments. As digital technologies continue to evolve, accessibility auditing will remain an essential practice for ensuring that technological advances benefit all members of society, regardless of their abilities or disabilities.

\begin{thebibliography}{99}

\bibitem{DisabilityRightsAuditing} Disability Rights Advocates. (2019). \textit{Web Accessibility Auditing: Best Practices for Organizations}. Technical Report.

\bibitem{WebAIMSurvey} WebAIM. (2020). \textit{Screen Reader User Survey \#8 Results}. Retrieved from https://webaim.org/projects/screenreadersurvey8/.

\bibitem{PowerAccessibility} Power, C., Freire, A., Petrie, H., \& Swallow, D. (2012). Guidelines are only half of the story: accessibility problems encountered by blind users on the web. \textit{Proceedings of the SIGCHI Conference on Human Factors in Computing Systems}, 433-442.

\bibitem{DequeAxeDevTools} Deque Systems. (2024). \textit{axe DevTools: Accessibility Testing Tools}. Retrieved from https://www.deque.com/axe/devtools/.

\bibitem{WebAIMWave} WebAIM. (2024). \textit{WAVE Web Accessibility Evaluation Tool}. Retrieved from https://wave.webaim.org/.

\bibitem{MicrosoftInsights} Microsoft. (2024). \textit{Accessibility Insights for Web}. Retrieved from https://accessibilityinsights.io/docs/en/web/overview/.

\bibitem{DequeWorldSpace} Deque Systems. (2024). \textit{WorldSpace: Enterprise Accessibility Testing Platform}. Retrieved from https://www.deque.com/worldspace/.

\bibitem{SiteimproveAccessibility} Siteimprove. (2024). \textit{Accessibility Testing and Monitoring}. Retrieved from https://siteimprove.com/en/accessibility/.

\bibitem{AudioEyeTesting} AudioEye. (2024). \textit{Web Accessibility Testing and Remediation}. Retrieved from https://www.audioeye.com/.

\bibitem{Pa11yTool} Pa11y. (2024). \textit{Pa11y: Command-line Accessibility Testing Tool}. Retrieved from https://pa11y.org/.

\bibitem{GoogleLighthouse} Google. (2024). \textit{Lighthouse: Automated Tool for Improving Web Page Quality}. Retrieved from https://developers.google.com/web/tools/lighthouse.

\bibitem{DequeAxeCore} Deque Systems. (2024). \textit{axe-core: Accessibility Testing Engine}. Retrieved from https://github.com/dequelabs/axe-core.

\end{thebibliography}
