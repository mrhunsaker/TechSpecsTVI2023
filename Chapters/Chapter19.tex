\chapter{Creating Fully Accessible Documents with Markdown and its Derivatives for Individuals Who Are Blind or Visually Impaired}
\label{chap:accessible-markdown}
\addcontentsline{toc}{chapter}{Creating Fully Accessible Documents with Markdown and its Derivatives for Individuals Who Are Blind or Visually Impaired}

\section{Introduction to Accessible Markdown}
\label{sec:intro-accessible-markdown}

\subsection{Understanding Digital Accessibility for Visually Impaired Users}
Digital accessibility is a critical discipline focused on ensuring that digital content and interfaces are usable by individuals with disabilities, particularly those with low vision, blindness, and color blindness.\footnote{\url{https://reciteme.com/news/blind-accessibility/}}\footnote{\url{https://www.researchgate.net/publication/357113200_Text_to_Braille_Conversion_System}} A cornerstone of this endeavor is compatibility with assistive technologies, most notably screen readers.\footnote{\url{https://reciteme.com/news/blind-accessibility/}}\footnote{\url{https://www.testdevlab.com/blog/screen-reader-guide-to-accessibility}} These sophisticated tools function by converting digital text into synthesized speech, thereby enabling users to comprehend and navigate digital content auditorily.\footnote{\url{https://reciteme.com/news/blind-accessibility/}}\footnote{\url{https://www.testdevlab.com/blog/screen-reader-guide-to-accessibility}} Screen readers process information in a linear fashion, typically from left to right and top to bottom, which dictates how content should be structured for optimal interpretation.\footnote{\url{https://support.dsu.edu/TDClient/1796/Portal/KB/ArticleDet?ID=148297}}\footnote{\url{https://www.smashingmagazine.com/2021/09/improving-accessibility-of-markdown/}}

The fundamental principle underpinning screen reader compatibility is the use of semantic HTML markup. This means employing HTML elements for their intended programmatic meaning rather than merely for visual presentation.\footnote{\url{https://developer.mozilla.org/en-US/docs/Learn_web_development/Core/Accessibility/HTML}}\footnote{\url{https://universaldesign.ie/communications-digital/web-and-mobile-accessibility/web-accessibility-techniques/developers-introduction-and-index/provide-an-accessible-page-structure-and-layout/do-not-misuse-semantic-markup}} For instance, using \texttt{<p>} for paragraphs, \texttt{<h1>} through \texttt{<h6>} for headings, and \texttt{<ul>} or \texttt{<ol>} for lists provides a clear, machine-readable structure that screen readers can interpret and convey to the user. Without such semantic signposts, a document is perceived as an undifferentiated block of text, rendering effective navigation and comprehension virtually impossible for a screen reader user.\footnote{\url{https://developer.mozilla.org/en-US/docs/Learn_web_development/Core/Accessibility/HTML}}\footnote{\url{https://universaldesign.ie/communications-digital/web-and-mobile-accessibility/web-accessibility-techniques/developers-introduction-and-index/provide-an-accessible-page-structure-and-layout/do-not-misuse-semantic-markup}} This structural clarity, alongside sufficient color contrast, descriptive alternative text for images, and full keyboard navigability, forms the bedrock of accessible digital content.\footnote{\url{https://reciteme.com/news/blind-accessibility/}}\footnote{\url{https://www.testdevlab.com/blog/screen-reader-guide-to-accessibility}}\footnote{\url{https://developer.mozilla.org/en-US/docs/Learn_web_development/Core/Accessibility/HTML}} The underlying HTML generated from any source document is the ultimate determinant of its accessibility, highlighting the importance of understanding the semantic implications of content creation choices.

\subsection{Why Markdown for Accessibility? Benefits and Core Principles}
Markdown has emerged as a powerful tool in the creation of accessible documents due to its inherent simplicity and direct conversion to HTML.\footnote{\url{https://www.smashingmagazine.com/2021/09/improving-accessibility-of-markdown/}}\footnote{\url{https://www.smashingmagazine.com/2021/09/improving-accessibility-of-markdown/}} Its human-friendly syntax allows authors to format plain text with minimal effort, which then translates into structured HTML markup readily consumable by assistive technologies.\footnote{\url{https://www.smashingmagazine.com/2021/09/improving-accessibility-of-markdown/}} The growing adoption of Markdown presents a significant opportunity to produce content that is inherently readable by individuals who are blind or visually impaired, often requiring only minor adjustments to standard authoring practices.\footnote{\url{https://r-resources.massey.ac.nz/rmarkdown/}}

A key advantage of Markdown is its automatic generation of many fundamental accessibility requirements. For example, using hash signs (\texttt{\#}) for headings naturally creates proper heading styles in the underlying HTML, and the syntax for embedding images (\texttt{![alt text](image.png)}) inherently includes space for alternative text.\footnote{\url{https://r-resources.massey.ac.nz/rmarkdown/}} This plain-text, semantic-first approach means Markdown bypasses many common accessibility pitfalls associated with visual-first editors, such as applying bold formatting instead of true heading styles.\footnote{\url{https://universaldesign.ie/communications-digital/web-and-mobile-accessibility/web-accessibility-techniques/developers-introduction-and-index/provide-an-accessible-page-structure-and-layout/do-not-misuse-semantic-markup}} While Markdown simplifies the creation of semantically rich HTML, it is not an exhaustive or prescriptive solution.\footnote{\url{https://www.smashingmagazine.com/2021/09/improving-accessibility-of-markdown/}} Authors must still consciously apply accessibility best practices. This means that while Markdown empowers non-developers to create more accessible content with greater ease than direct HTML coding, it is not a panacea. Continuous education and adherence to accessibility guidelines remain crucial for maximizing its benefits.

\section{Basic Markdown Formatting for Accessibility}
\label{sec:markdown-formatting}

\subsection{Headings: Creating a Logical Document Structure}
Headings are arguably the most critical element for document navigation and comprehension for users of screen readers.\footnote{\url{https://reciteme.com/news/blind-accessibility/}}\footnote{\url{https://support.dsu.edu/TDClient/1796/Portal/KB/ArticleDet?ID=148297}}\footnote{\url{https://www.smashingmagazine.com/2021/09/improving-accessibility-of-markdown/}}\footnote{\url{https://www.testdevlab.com/blog/screen-reader-guide-to-accessibility}}\footnote{\url{https://developer.mozilla.org/en-US/docs/Learn_web_development/Core/Accessibility/HTML}}\footnote{\url{https://universaldesign.ie/communications-digital/web-and-mobile-accessibility/web-accessibility-techniques/developers-introduction-and-index/provide-an-accessible-page-structure-and-layout/do-not-misuse-semantic-markup}}\footnote{\url{https://ur.uni.edu/uni-brand/web-guidelines/web-accessibility}}\footnote{\url{https://docs.gitlab.com/user/markdown/}}\footnote{\url{https://www.markdowntoolbox.com/blog/markdown-best-practices-for-documentation/}}\footnote{\url{https://testpros.com/accessibility/accessibility-in-github-with-git-flavored-markdown/}}\footnote{\url{https://portal.lancaster.ac.uk/ask/checklist-latex}} Markdown's straightforward syntax, using one to six hash symbols (\texttt{\#} through \texttt{\#\#\#\#\#\#}), directly corresponds to HTML heading levels (H1 through H6), enabling the creation of a logical and hierarchical document structure.\footnote{\url{https://www.smashingmagazine.com/2021/09/improving-accessibility-of-markdown/}}\footnote{\url{https://www.markdownguide.org/basic-syntax/}} Screen reader users frequently navigate through documents by jumping from one heading to the next, using this structure as a primary roadmap to understand content organization and quickly locate specific sections.\footnote{\url{https://reciteme.com/news/blind-accessibility/}}\footnote{\url{https://support.dsu.edu/TDClient/1796/Portal/KB/ArticleDet?ID=148297}}\footnote{\url{https://www.smashingmagazine.com/2021/09/improving-accessibility-of-markdown/}}\footnote{\url{https://developer.mozilla.org/en-US/docs/Learn_web_development/Core/Accessibility/HTML}}

To ensure optimal accessibility, adherence to specific heading best practices is essential:
\begin{itemize}[noitemsep,topsep=0pt]
    \item A document should contain only one \texttt{H1} element, typically reserved for the main title, as this serves as the primary identifier for screen readers and search engines.\footnote{\url{https://support.dsu.edu/TDClient/1796/Portal/KB/ArticleDet?ID=148297}}\footnote{\url{https://ur.uni.edu/uni-brand/web-guidelines/web-accessibility}}\footnote{\url{https://docs.gitlab.com/user/markdown/}}
    \item Heading levels must be sequential; skipping levels (e.g., an \texttt{H2} followed directly by an \texttt{H4}) disrupts the logical flow and can confuse screen reader users who rely on this hierarchy for context.\footnote{\url{https://support.dsu.edu/TDClient/1796/Portal/KB/ArticleDet?ID=148297}}\footnote{\url{https://www.testdevlab.com/blog/screen-reader-guide-to-accessibility}}\footnote{\url{https://docs.gitlab.com/user/markdown/}}\footnote{\url{https://testpros.com/accessibility/accessibility-in-github-with-git-flavored-markdown/}}
    \item Headings should be nested correctly, mirroring the structure of a well-organized outline or table of contents.\footnote{\url{https://docs.gitlab.com/user/markdown/}}\footnote{\url{https://portal.lancaster.ac.uk/ask/checklist-latex}}
    \item It is imperative to use the proper Markdown heading syntax (\texttt{\#}) rather than relying on visual formatting like bolding, italics, or all capital letters to indicate a heading.\footnote{\url{https://support.dsu.edu/TDClient/1796/Portal/KB/ArticleDet?ID=148297}}\footnote{\url{https://universaldesign.ie/communications-digital/web-and-mobile-accessibility/web-accessibility-techniques/developers-introduction-and-index/provide-an-accessible-page-structure-and-layout/do-not-misuse-semantic-markup}}\footnote{\url{https://testpros.com/accessibility/accessibility-in-github-with-git-flavored-markdown/}} Screen readers interpret semantic tags, not visual styling cues, so misusing formatting can render headings invisible to assistive technology.\footnote{\url{https://universaldesign.ie/communications-digital/web-and-mobile-accessibility/web-accessibility-techniques/developers-introduction-and-index/provide-an-accessible-page-structure-and-layout/do-not-misuse-semantic-markup}}
    \item Surrounding headings with single blank lines improves readability and ensures proper rendering across different platforms.\footnote{\url{https://learn.microsoft.com/en-us/powershell/scripting/community/contributing/general-markdown?view=powershell-7.5}}
    \item Headings should be concise yet descriptive, providing enough information for users to understand the content of the section without being overly verbose.\footnote{\url{https://www.smashingmagazine.com/2021/09/improving-accessibility-of-markdown/}}\footnote{\url{https://ur.uni.edu/uni-brand/web-guidelines/web-accessibility}}
\end{itemize}
The consistent application of these rules creates a robust programmatic roadmap for screen readers, allowing for efficient navigation and comprehensive understanding of the document's content. Deviations from these guidelines, such as skipping heading levels or using non-semantic formatting, effectively break this crucial navigational functionality, making the document challenging or impossible for visually impaired users to traverse meaningfully.

\subsection{Text Emphasis: Bold, Italic, and Strikethrough}
Markdown provides straightforward syntax for text emphasis: asterisks (\texttt{**text**} or \texttt{\_\_text\_\_}) for bold, and single asterisks (\texttt{*text*} or \texttt{\_text\_}) for italics.\footnote{\url{https://www.markdownguide.org/basic-syntax/}}\footnote{\url{https://www.markdownguide.org/extended-syntax/}}\footnote{\url{https://www.markdowntoolbox.com/blog/markdown-best-practices-for-documentation/}} Combining bold and italic is achieved with three asterisks (\texttt{***text***}).\footnote{\url{https://www.markdownguide.org/basic-syntax/}} Strikethrough is indicated by wrapping text in two tildes (\texttt{\~\~text\~\~}).\footnote{\url{https://www.markdownguide.org/extended-syntax/}} While some Markdown applications may handle underscores differently, particularly in the middle of words, using asterisks for emphasis generally offers better compatibility across various Markdown processors.\footnote{\url{https://www.markdownguide.org/basic-syntax/}}

The effectiveness of text emphasis for accessibility hinges on its semantic translation. Markdown's emphasis syntax typically converts to semantic HTML tags such as \texttt{<strong>} for bold and \texttt{<em>} for italic. This semantic mapping is crucial because screen readers are designed to interpret these tags, conveying the intended emphasis to the user. For instance, a screen reader might announce "strong" or "emphasized" before reading the text, or alter its vocal tone. Conversely, if a Markdown processor were to convert emphasis merely into visual styling (e.g., a \texttt{<span>} tag with CSS for bold font-weight), the semantic meaning would be lost to assistive technologies.\footnote{\url{https://universaldesign.ie/communications-digital/web-and-mobile-accessibility/web-accessibility-techniques/developers-introduction-and-index/provide-an-accessible-page-structure-and-layout/do-not-misuse-semantic-markup}} The inherent design of Markdown, which typically generates semantic HTML for emphasis, is a significant accessibility advantage. However, authors should be aware that the underlying processor must correctly generate these semantic HTML tags for the emphasis to be truly accessible and meaningful to screen reader users.

\subsection{Lists: Ordered and Unordered}
Lists, whether bulleted (unordered) or numbered (ordered), are fundamental for improving document readability and providing essential structural cues for screen reader users.\footnote{\url{https://support.dsu.edu/TDClient/1796/Portal/KB/ArticleDet?ID=148297}}\footnote{\url{https://ur.uni.edu/uni-brand/web-guidelines/web-accessibility}}\footnote{\url{https://docs.gitlab.com/user/markdown/}} When properly formatted, screen readers accurately identify and announce these as lists, often indicating the total number of items, which provides valuable context for the user.\footnote{\url{https://support.dsu.edu/TDClient/1796/Portal/KB/ArticleDet?ID=148297}} This functionality allows visually impaired users to navigate efficiently between list items, enhancing their ability to process and comprehend information.

Markdown offers simple syntax for creating lists: bulleted lists can use hyphens (\texttt{-}), asterisks (\texttt{*}) or plus signs (\texttt{+}), while numbered lists are created with numbers followed by a period (\texttt{.}).\footnote{\url{https://www.markdownguide.org/extended-syntax/}} It is critical to use these standard Markdown syntaxes for list creation. A common pitfall is attempting to create lists using repeated tab keys or spacebars for indentation.\footnote{\url{https://support.dsu.edu/TDClient/1796/Portal/KB/ArticleDet?ID=148297}}\footnote{\url{https://universaldesign.ie/communications-digital/web-and-mobile-accessibility/web-accessibility-techniques/developers-introduction-and-index/provide-an-accessible-page-structure-and-layout/do-not-misuse-semantic-markup}} This purely visual formatting will not be recognized as a list by screen readers, rendering the list reading controls inoperative and depriving the user of the navigational and contextual benefits.\footnote{\url{https://support.dsu.edu/TDClient/1796/Portal/KB/ArticleDet?ID=148297}}\footnote{\url{https://universaldesign.ie/communications-digital/web-and-mobile-accessibility/web-accessibility-techniques/developers-introduction-and-index/provide-an-accessible-page-structure-and-layout/do-not-misuse-semantic-markup}} Ordered lists should be reserved for content where sequence or numbering is significant (e.g., steps in a procedure), while unordered lists are appropriate for non-sequential items.\footnote{\url{https://docs.gitlab.com/user/markdown/}} The native list syntax in Markdown is inherently accessible, and authors should consistently leverage it to break down complex information into digestible segments, thereby improving readability for all users, particularly those relying on screen readers.

\subsection{Blockquotes and Horizontal Rules}
Markdown includes syntax for blockquotes and horizontal rules, each serving distinct purposes in document formatting. Blockquotes are used to set off quoted text from the main content and are created by prefixing paragraphs with a greater-than symbol (\texttt{>}).\footnote{\url{https://www.markdownguide.org/basic-syntax/}} For blockquotes spanning multiple paragraphs, a \texttt{>} should be placed on the blank lines between paragraphs.\footnote{\url{https://www.markdownguide.org/basic-syntax/}} Nested blockquotes are achieved by adding additional \texttt{>} symbols (e.g., \texttt{>>}).\footnote{\url{https://www.markdownguide.org/basic-syntax/}} Screen readers typically announce blockquotes as quoted content, providing a semantic cue to the user about the nature of the text.

Horizontal rules, which provide a visual separation between sections, are created by typing three or more hyphens (\texttt{---}), asterisks (\texttt{***}) or underscores (\texttt{\_\_\_}) on a line by themselves.\footnote{\url{https://www.markdownguide.org/basic-syntax/}} While these elements offer a clean visual break in the rendered document, their semantic value for screen reader users is limited. They primarily indicate a visual separation rather than conveying significant structural meaning or navigational cues, unlike headings or lists.\footnote{\url{https://learn.microsoft.com/en-us/powershell/scripting/community/contributing/general-markdown?view=powershell-7.5}} Therefore, horizontal rules should be used sparingly and solely for visual separation, not as a substitute for proper semantic structuring elements like headings, to avoid miscommunicating document hierarchy to assistive technologies.

\section{Creating Accessible Tables in Markdown}
\label{sec:markdown-tables}

\subsection{Basic Table Syntax: Pipes and Hyphens}
Markdown provides a straightforward syntax for creating tables using pipes (\texttt{|}) to delineate columns and hyphens (\texttt{---}) to define the header row.\footnote{\url{https://www.docstomarkdown.pro/tables-in-markdown/}}\footnote{\url{https://www.markdownguide.org/extended-syntax/}} For enhanced compatibility across various Markdown processors, it is advisable to include a pipe at both ends of each row.\footnote{\url{https://www.docstomarkdown.pro/tables-in-markdown/}} Column alignment can be specified within the separator row using colons: \texttt{:---} aligns text to the left, \texttt{:---:} centers text, and \texttt{---:} aligns text to the right.\footnote{\url{https://www.docstomarkdown.pro/tables-in-markdown/}} Within table cells, basic Markdown formatting such as bolding, italics, inline code, and even images can be applied.\footnote{\url{https://www.docstomarkdown.pro/tables-in-markdown/}}

While this syntax simplifies table creation, native Markdown tables possess inherent limitations, notably the absence of support for merged cells.\footnote{\url{https://www.docstomarkdown.pro/tables-in-markdown/}} Furthermore, screen readers and other accessibility tools fundamentally rely on clearly identified headers to programmatically discern the role of each column and row.\footnote{\url{https://www.docstomarkdown.pro/tables-in-markdown/}} Consequently, tables lacking proper headers may not be fully accessible to users with disabilities.\footnote{\url{https://www.docstomarkdown.pro/tables-in-markdown/}} This highlights a critical gap: while Markdown offers an easy way to create simple tables, it often falls short in providing the rich semantic attributes (like captions, table body/header/footer sections, or explicit header-cell associations) that HTML offers for comprehensive accessibility. For any data beyond the simplest, non-critical presentation, native Markdown tables are often insufficient for achieving full accessibility, necessitating alternative approaches or robust conversion tools.

\subsection{Ensuring Table Accessibility: Headers, Captions, and Avoiding Merged/Blank Cells}
To ensure tables are fully accessible to visually impaired users, meticulous attention to semantic structure is paramount.
\begin{itemize}[noitemsep,topsep=0pt]
    \item \textbf{Table Headers}: These are indispensable for screen readers to establish programmatic associations between headers and their corresponding data cells.\footnote{\url{https://developer.mozilla.org/en-US/docs/Learn_web_development/Core/Structuring_content/Table_accessibility}}\footnote{\url{https://accessibility.oregonstate.edu/digital-accessibility/tables}}\footnote{\url{https://support.dsu.edu/TDClient/1796/Portal/KB/ArticleDet?ID=148297}} All cells functioning as headers, whether in the top row or the first column, must be explicitly marked as header cells (\texttt{<th>} in HTML) and never as data cells (\texttt{<td>}).\footnote{\url{https://developer.mozilla.org/en-US/docs/Learn_web_development/Core/Structuring_content/Table_accessibility}}\footnote{\url{https://accessibility.oregonstate.edu/digital-accessibility/tables}} This distinction enables screen readers to announce the relevant headers as a user navigates through the table, providing essential context for each data point.\footnote{\url{https://developer.mozilla.org/en-US/docs/Learn_web_development/Core/Structuring_content/Table_accessibility}}\footnote{\url{https://accessibility.oregonstate.edu/digital-accessibility/tables}}
    \item \textbf{Captions}: A descriptive caption for the table content is crucial.\footnote{\url{https://developer.mozilla.org/en-US/docs/Learn_web_development/Core/Structuring_content/Table_accessibility}} This allows blind users, in particular, to quickly grasp the table's purpose and decide whether to explore its contents in detail, rather than having to listen to every cell read aloud to understand its relevance.\footnote{\url{https://developer.mozilla.org/en-US/docs/Learn_web_development/Core/Structuring_content/Table_accessibility}} The \texttt{<caption>} HTML element is the recommended method for this, as the older \texttt{summary} attribute is deprecated.\footnote{\url{https://developer.mozilla.org/en-US/docs/Learn_web_development/Core/Structuring_content/Table_accessibility}}
    \item \textbf{Avoiding Merged/Blank Cells}: For optimal accessibility, tables should strictly avoid merged or split cells.\footnote{\url{https://support.dsu.edu/TDClient/1796/Portal/KB/ArticleDet?ID=148297}} Similarly, leaving cells, rows, or columns blank can create confusion for screen readers, which process content linearly.\footnote{\url{https://support.dsu.edu/TDClient/1796/Portal/KB/ArticleDet?ID=148297}}\footnote{\url{https://developer.mozilla.org/en-US/docs/Learn_web_development/Core/Structuring_content/Table_accessibility}} If no meaningful value exists for a cell, it is best practice to insert "N/A" (not applicable) or "None".\footnote{\url{https://support.dsu.edu/TDClient/1796/Portal/KB/ArticleDet?ID=148297}}\footnote{\url{https://docs.gitlab.com/user/markdown/}}
\end{itemize}
The inability of a visually impaired person to visually scan and infer relationships within a non-linear table structure underscores the importance of explicit table semantics. Headers and captions serve as the programmatic "signposts" and "content descriptions" that enable screen readers to articulate these relationships effectively. The strictures against merged or blank cells reinforce the need for a clear, predictable structure that screen readers can traverse without ambiguity. Therefore, authors must diligently define table headers and provide informative captions. If native Markdown syntax cannot support these semantic requirements directly (e.g., \texttt{scope} attributes), then embedding HTML or utilizing robust conversion processes becomes necessary.

\subsection{Advanced Table Structuring: Leveraging HTML for Enhanced Accessibility}
The inherent limitations of native Markdown tables, particularly their lack of support for merged cells and advanced semantic attributes, often necessitate the direct embedding of HTML within Markdown files for complex data presentation.\footnote{\url{https://www.docstomarkdown.pro/tables-in-markdown/}} This hybrid approach allows authors to leverage the full power of HTML's accessibility features.

Key HTML elements and attributes for enhancing table accessibility include:
\begin{itemize}[noitemsep,topsep=0pt]
    \item \texttt{<caption>}: As noted, this provides a concise description of the table's content.\footnote{\url{https://developer.mozilla.org/en-US/docs/Learn_web_development/Core/Structuring_content/Table_accessibility}}
    \item \texttt{<thead>}, \texttt{<tbody>}, and \texttt{<tfoot>}: These elements define the header, body, and footer sections of a table, respectively.\footnote{\url{https://developer.mozilla.org/en-US/docs/Learn_web_development/Core/Structuring_content/Table_accessibility}} While they do not directly enhance screen reader accessibility on their own, they are invaluable for applying CSS styling and layout enhancements that can significantly improve visual accessibility and overall presentation.\footnote{\url{https://developer.mozilla.org/en-US/docs/Learn_web_development/Core/Structuring_content/Table_accessibility}}
    \item \texttt{scope} attribute on \texttt{<th>}: This attribute explicitly tells screen readers whether a header applies to a row (\texttt{scope="row"}) or a column (\texttt{scope="col"}).\footnote{\url{https://developer.mozilla.org/en-US/docs/Learn_web_development/Core/Structuring_content/Table_accessibility}}\footnote{\url{https://accessibility.oregonstate.edu/digital-accessibility/tables}} This is particularly crucial for simple tables, providing clear context for each data point.\footnote{\url{https://developer.mozilla.org/en-US/docs/Learn_web_development/Core/Structuring_content/Table_accessibility}}\footnote{\url{https://accessibility.oregonstate.edu/digital-accessibility/tables}}
    \item \texttt{id} and \texttt{headers} attributes: For highly complex tables with intricate relationships between headers and data cells, these attributes create precise and explicit programmatic associations.\footnote{\url{https://developer.mozilla.org/en-US/docs/Learn_web_development/Core/Structuring_content/Table_accessibility}}
\end{itemize}
It is important to acknowledge that embedding HTML tables within Markdown can introduce compatibility issues, as not all Markdown platforms render HTML consistently, and some may even strip out embedded HTML tags.\footnote{\url{https://www.docstomarkdown.pro/tables-in-markdown/}} Consequently, thorough testing on the intended platform is always recommended to ensure correct display and accessibility.\footnote{\url{https://www.docstomarkdown.pro/tables-in-markdown/}} This reliance on HTML for advanced table features illustrates that while Markdown excels at simplifying content creation, for deeply accessible and complex data structures, it often functions as a wrapper for HTML rather than a self-sufficient solution. Authors creating complex tables in Markdown must either be proficient in the relevant HTML accessibility attributes or utilize conversion tools capable of generating these attributes automatically.

\subsection{Limitations of Native Markdown Tables for Complex Data}
The design philosophy of Markdown prioritizes simplicity and readability in its source format. This simplicity, while beneficial for general text, comes with certain trade-offs, particularly when dealing with complex data structures such as tables. The most prominent limitation is the absence of native support for merged cells, a common requirement in many detailed tables.\footnote{\url{https://www.docstomarkdown.pro/tables-in-markdown/}}

Furthermore, native Markdown syntax offers limited capabilities for directly incorporating advanced HTML attributes like \texttt{scope}, \texttt{id}, and \texttt{headers}.\footnote{\url{https://developer.mozilla.org/en-US/docs/Learn_web_development/Core/Structuring_content/Table_accessibility}} These attributes are vital for screen readers to accurately interpret the relationships within complex tables, allowing users to navigate and understand the data effectively.\footnote{\url{https://developer.mozilla.org/en-US/docs/Learn_web_development/Core/Structuring_content/Table_accessibility}} Standard Markdown also restricts the type of content that can be placed within table cells; generally, it does not support block-level elements such as headings, blockquotes, lists, horizontal rules, or most HTML tags.\footnote{\url{https://www.markdownguide.org/extended-syntax/}} While some Markdown flavors or renderers might permit limited inline formatting (e.g., images, inline code, bold/italics) within cells, full block elements are typically unsupported.\footnote{\url{https://www.docstomarkdown.pro/tables-in-markdown/}}\footnote{\url{https://www.markdownguide.org/extended-syntax/}}

This inherent trade-off means that while Markdown is excellent for quickly structuring basic text, achieving deep accessibility for complex elements often requires it to act as an interface to HTML, rather than a standalone solution. Users must understand these limitations and be prepared to either embed raw HTML or leverage advanced conversion tools to ensure highly structured and accessible content.

\begin{table}[h!]
\caption{Table 1: Basic Markdown Table Syntax vs. HTML for Enhanced Accessibility}
\label{tab:markdown-vs-html-accessibility}
\centering
\begin{tabular}{|l|l|l|l|}
\toprule
Feature/Element & Markdown Syntax (Basic) & HTML Equivalent (for Accessibility) & Purpose for Accessibility \\
\midrule
Table Header & \texttt{| Header |} \newline \texttt{|---|} & \texttt{<th>} & Identifies column/row labels for screen readers, enabling programmatic association with data cells.\footnote{\url{https://developer.mozilla.org/en-US/docs/Learn_web_development/Core/Structuring_content/Table_accessibility}}\footnote{\url{https://accessibility.oregonstate.edu/digital-accessibility/tables}} \\
\addlinespace
Table Caption & (Not natively supported) & \texttt{<caption>} & Provides a concise description of table content, allowing users to quickly understand its purpose without navigating all cells.\footnote{\url{https://developer.mozilla.org/en-US/docs/Learn_web_development/Core/Structuring_content/Table_accessibility}} \\
\addlinespace
Column/Row Scope & (Not natively supported) & \texttt{<th scope="col">} or \texttt{<th scope="row">} & Explicitly tells screen readers whether a header applies to a column or a row, clarifying data relationships.\footnote{\url{https://developer.mozilla.org/en-US/docs/Learn_web_development/Core/Structuring_content/Table_accessibility}}\footnote{\url{https://accessibility.oregonstate.edu/digital-accessibility/tables}} \\
\addlinespace
Header-Cell Association (Complex) & (Not natively supported) & \texttt{<td headers="header-id">} & For complex tables, creates precise programmatic links between data cells and multiple headers, improving navigation.\footnote{\url{https://developer.mozilla.org/en-US/docs/Learn_web_development/Core/Structuring_content/Table_accessibility}} \\
\addlinespace
Table Sections (Header/Body/Footer) & (Not natively supported) & \texttt{<thead>}, \texttt{<tbody>}, \texttt{<tfoot>} & Structurally organizes table content; useful for styling and layout that can indirectly improve accessibility.\footnote{\url{https://developer.mozilla.org/en-US/docs/Learn_web_development/Core/Structuring_content/Table_accessibility}} \\
\addlinespace
Merged Cells & (Not natively supported) & \texttt{<td colspan="X">} or \texttt{<td rowspan="Y">} & Allows cells to span multiple columns/rows; requires HTML embedding as Markdown lacks native support.\footnote{\url{https://www.docstomarkdown.pro/tables-in-markdown/}} \\
\addlinespace
Blank Cells & \texttt{| Data | | Data |} (visually blank) & Avoid blank cells; use "N/A" or "None" in HTML. & Prevents confusion for screen readers that read linearly; ensures all cells convey meaning.\footnote{\url{https://support.dsu.edu/TDClient/1796/Portal/KB/ArticleDet?ID=148297}}\footnote{\url{https://docs.gitlab.com/user/markdown/}} \\
\addlinespace
Content within Cells & Inline formatting (bold, italic, inline code, images)\footnote{\url{https://www.docstomarkdown.pro/tables-in-markdown/}} & Full HTML elements (headings, lists, blockquotes, images, etc.) & Markdown is limited to inline content; HTML allows rich, block-level content within cells for complex data.\footnote{\url{https://www.markdownguide.org/extended-syntax/}} \\
\bottomrule
\end{tabular}
\end{table}

\section{Optimizing Markdown for Note-Taking}
\label{sec:markdown-note-taking}

Markdown's plain-text foundation and structured formatting capabilities make it an advantageous choice for note-taking, particularly for individuals who are blind or visually impaired. Its linear nature aligns well with how screen readers process information, making Markdown notes inherently accessible and intuitive to navigate.\footnote{\url{https://www.perkins.org/resource/tips-and-models-for-effective-note-taking/}}\footnote{\url{https://www.teachingvisuallyimpaired.com/braille-notetakers.html}} This contrasts with more visually-oriented note-taking methods that may rely on spatial arrangements difficult for screen readers to interpret.

\subsection{Benefits for Blind/Visually Impaired Individuals}
Markdown's structure naturally supports linear note-taking, which is often the most effective method for screen reader users.\footnote{\url{https://www.perkins.org/resource/tips-and-models-for-effective-note-taking/}} The explicit syntax for headings and lists provides clear structural cues that assistive technologies can interpret and present, allowing users to quickly grasp the organization of their notes.\footnote{\url{https://www.teachingvisuallyimpaired.com/braille-notetakers.html}}\footnote{\url{https://www.markdowntoolbox.com/blog/markdown-best-practices-for-documentation/}} This is especially beneficial for students who may not be able to replicate the visual layouts of traditional note-taking methods like Cornell notes or mind maps, but can still apply the underlying conceptual strategies (e.g., identifying keywords, summarizing) in a linear Markdown format.\footnote{\url{https://www.perkins.org/resource/tips-and-models-for-effective-note-taking/}} Furthermore, Markdown notes can be easily exported to universally compatible formats such as plain text (.txt) or Word documents (.docx), which are readily consumed by braille notetakers and other assistive devices.\footnote{\url{https://www.teachingvisuallyimpaired.com/braille-notetakers.html}} This ensures that notes taken in Markdown are highly portable and adaptable to various reading modalities.

\subsection{Strategies for Accessible Note-Taking in Markdown}
To maximize the accessibility of Markdown notes, several strategies are recommended:
\begin{itemize}[noitemsep,topsep=0pt]
    \item \textbf{Consistent Use of Headings and Lists}: Employ Markdown's heading syntax (\texttt{\#}, \texttt{\#\#}, etc.) to organize notes into logical sections and subsections, and use bulleted or numbered lists for breaking down complex information or jotting down key points.\footnote{\url{https://learn.microsoft.com/en-us/powershell/scripting/community/contributing/general-markdown?view=powershell-7.5}}\footnote{\url{https://www.teachingvisuallyimpaired.com/braille-notetakers.html}} This provides a clear, navigable structure for screen readers.
    \item \textbf{Descriptive Alt Text for Images}: If images are included in notes (e.g., screenshots of diagrams), always provide concise and descriptive alternative text (\texttt{![Alt text](image.png)}). This ensures that the visual content is conveyed to those who cannot see it.\footnote{\url{https://docs.gitlab.com/user/markdown/}}\footnote{\url{https://www.markdowntoolbox.com/blog/markdown-best-practices-for-documentation/}}
    \item \textbf{Clear and Concise Language}: Use plain language, avoid excessive jargon, and expand abbreviations or acronyms upon first use. This benefits not only visually impaired users but also those with cognitive disabilities or for whom English is not a first language.\footnote{\url{https://developer.mozilla.org/en-US/docs/Learn_web_development/Core/Accessibility/HTML}}\footnote{\url{https://learn.microsoft.com/en-us/powershell/scripting/community/contributing/general-markdown?view=powershell-7.5}}
    \item \textbf{Abbreviations for Speed}: For faster note-taking, encourage the use of personal abbreviations (e.g., "2" for "two").\footnote{\url{https://www.perkins.org/resource/monster-note-taking-skills/}} While these are not universally accessible in their abbreviated form, the primary goal during live note-taking is capturing information efficiently, with the understanding that notes can be expanded or clarified later.
    \item \textbf{Exportability}: Leverage Markdown's inherent ability to be converted into various accessible formats like HTML, PDF, or DOCX.\footnote{\url{https://www.inkdrop.app/}}\footnote{\url{https://bear.app/}}\footnote{\url{https://create.uw.edu/a11y-in-action/accessible-courses/making-homework-write-up-accessible/}} This ensures flexibility for review, sharing, and conversion to braille or other specialized formats.
\end{itemize}
Markdown's structured nature naturally supports the linear way screen readers function, making it an intuitive and effective choice for note-taking for visually impaired users. The explicit and consistent use of Markdown's core formatting elements creates a predictable and navigable document structure that directly translates into a more efficient and less cognitively demanding experience for users relying on assistive technologies.

\subsection{Accessible Markdown Editors}
The choice of a Markdown editor significantly influences the note-taking experience for visually impaired users. An accessible editor provides features that facilitate interaction with the Markdown syntax and its rendered output. Key features to look for include:
\begin{itemize}[noitemsep,topsep=0pt]
    \item \textbf{Screen Reader Compatibility}: Full support for popular screen readers like JAWS, NVDA, and VoiceOver.\footnote{\url{https://daily.dev/blog/10-best-code-editors-with-accessibility-features}}
    \item \textbf{Customizable Keyboard Shortcuts}: The ability to navigate and interact with the editor solely via keyboard commands.\footnote{\url{https://daily.dev/blog/10-best-code-editors-with-accessibility-features}}
    \item \textbf{High Contrast Themes and Font Adjustments}: Options for high contrast color schemes, adjustable font sizes, and zoom capabilities to accommodate users with low vision.\footnote{\url{https://reciteme.com/news/blind-accessibility/}}\footnote{\url{https://daily.dev/blog/10-best-code-editors-with-accessibility-features}}
    \item \textbf{Clear UI and Feedback}: An intuitive and uncluttered user interface that provides clear audio or tactile feedback on actions.\footnote{\url{https://www.inkdrop.app/}}\footnote{\url{https://www.tiny.cloud/blog/accessible-rich-text-editor/}}
\end{itemize}
Several code editors and dedicated Markdown applications offer strong accessibility features:
\begin{itemize}[noitemsep,topsep=0pt]
    \item \textbf{Visual Studio Code (VS Code)}: A popular open-source editor with extensive screen reader support, customizable shortcuts, high contrast themes, and font/zoom adjustments.\footnote{\url{https://daily.dev/blog/10-best-code-editors-with-accessibility-features}}
    \item \textbf{JetBrains IntelliJ IDEA}: Offers color adjustments, customizable shortcuts, high contrast themes, and screen reader support.\footnote{\url{https://daily.dev/blog/10-best-code-editors-with-accessibility-features}}
    \item \textbf{Sublime Text}: Known for keyboard navigation, customizable themes, and font adjustments.\footnote{\url{https://daily.dev/blog/10-best-code-editors-with-accessibility-features}}
    \item \textbf{Atom}: Features high contrast themes, screen reader support, and customizable UI elements.\footnote{\url{https://daily.dev/blog/10-best-code-editors-with-accessibility-features}}
    \item \textbf{Brackets}: Supports keyboard shortcuts, high contrast themes, and screen readers.\footnote{\url{https://daily.dev/blog/10-best-code-editors-with-accessibility-features}}
    \item \textbf{Notepad++}: A free, open-source option with high contrast themes, font adjustments, and keyboard navigation.\footnote{\url{https://daily.dev/blog/10-best-code-editors-with-accessibility-features}}
    \item \textbf{Eclipse, NetBeans, PyCharm, Emacs}: Other robust code editors with various accessibility features, including screen reader support and UI customization.\footnote{\url{https://daily.dev/blog/10-best-code-editors-with-accessibility-features}}
    \item \textbf{Dedicated Markdown Editors}: Applications like EasyMDE (a JavaScript editor with toolbar buttons, spell checking, and live rendering\footnote{\url{https://github.com/Ionaru/easy-markdown-editor}}), Inkdrop (robust Markdown editor with multi-language code highlighting, organization features, and export options\footnote{\url{https://www.inkdrop.app/}}), and Bear (powerful, simple Markdown note-taking app with flexible tags, export to DOCX/PDF/HTML, and OCR search\footnote{\url{https://bear.app/}}) are designed specifically for Markdown and often incorporate accessibility considerations. TinyMCE, a rich text editor, also focuses on accessibility, including WCAG compliance checking and accessible table creation.\footnote{\url{https://www.tiny.cloud/blog/accessible-rich-text-editor/}}
\end{itemize}
The choice of a Markdown editor profoundly impacts the note-taking experience for visually impaired users. Editors that prioritize features like robust screen reader compatibility, extensive keyboard navigation, and visual customization options directly enhance a user's ability to create, edit, and manage their notes effectively. This emphasis on editor accessibility ensures that the benefits of Markdown's plain-text structure are fully realized by the target audience.

\section{Conversion to and from Braille}
\label{sec:markdown-braille}

Converting digital text to braille is a specialized process that enables individuals who are blind or visually impaired to access written information tactilely. While Markdown provides a structured plain-text source, its conversion to braille typically involves a multi-stage workflow to ensure accurate translation and proper formatting according to braille standards.

\subsection{Process Overview for Markdown to Braille}
The general braille transcription process involves several meticulous steps:
\begin{enumerate}[noitemsep,topsep=0pt]
    \item \textbf{File Submission}: Original files, including text-based formats like Markdown, Word documents, or HTML, are submitted.\footnote{\url{https://accessibleprintshop.com/the-braille-transcription-process/}}\footnote{\url{https://accessibleprintshop.com/the-braille-transcription-process/}}\footnote{\url{https://accessibleprintshop.com/the-braille-transcription-process/}}
    \item \textbf{Formatting Analysis}: A team reviews the original file to understand its layout, structure, and formatting requirements, identifying elements that need special handling in braille.\footnote{\url{https://accessibleprintshop.com/the-braille-transcription-process/}}\footnote{\url{https://accessibleprintshop.com/the-braille-transcription-process/}}\footnote{\url{https://accessibleprintshop.com/the-braille-transcription-process/}}
    \item \textbf{Braille Translation}: Specialized software, often powered by open-source braille translators like Liblouis, converts the text into braille script.\footnote{\url{https://tech.aph.org/lt/}}\footnote{\url{http://daisy.github.io/pipeline/Get-Help/User-Guide/Braille/}}\footnote{\url{https://www.aph.org/product/brailleblaster/}} This involves converting normal text into braille characters, potentially applying contractions (Contracted Braille) to save space and increase reading speed.\footnote{\url{https://inclusiveasl.com/services/accessible-media-services/braille-transcription/}}\footnote{\url{https://www.researchgate.net/publication/357113200_Text_to_Braille_Conversion_System}}
    \item \textbf{Proofreading and Quality Assurance}: Every braille file undergoes rigorous proofreading by experienced transcribers to ensure accuracy, adherence to formatting guidelines, and overall quality.\footnote{\url{https://accessibleprintshop.com/the-braille-transcription-process/}}\footnote{\url{https://accessibleprintshop.com/the-braille-transcription-process/}}\footnote{\url{https://accessibleprintshop.com/the-braille-transcription-process/}} This human review is critical due to the complexities of braille rules.
    \item \textbf{Final Production and Delivery}: Once finalized, braille materials are embossed onto paper, collated, bound (if required), and prepared for delivery.\footnote{\url{https://accessibleprintshop.com/the-braille-transcription-process/}}\footnote{\url{https://www.youtube.com/watch?v=_mnitvxt8zk}}
\end{enumerate}
For Markdown specifically, the workflow often involves an intermediate step. Tools like Pandoc can convert Markdown into plain text or other formats before feeding it into a braille translator like Liblouis.\footnote{\url{https://tech.aph.org/lt/}} Dedicated software like BrailleBlaster utilizes rich markup from formats like NIMAS, EPUB, and DOCX to automate translation and formatting, relying on Liblouis for the core translation.\footnote{\url{https://www.aph.org/product/brailleblaster/}} Online tools also exist that can translate text to braille and vice versa, offering features like font size adjustment and dark mode.\footnote{\url{https://github.com/Sebdababo/Braille-Translator}} The multi-stage nature of braille conversion, often involving intermediate formats and specialized software, highlights the complexity beyond simple text conversion.

\subsection{Formatting Considerations for Braille}
Braille is not merely a direct letter-for-letter translation; it has its own comprehensive set of formatting rules that differ significantly from print, governed by standards such as Unified English Braille (UEB) outlined by the Braille Authority of North America (BANA).\footnote{\url{https://ataem.org/Braille-Formatting}}\footnote{\url{https://www.researchgate.net/publication/357113200_Text_to_Braille_Conversion_System}} Ensuring proper braille formatting is essential for readability and navigability for tactile readers. Key considerations include:
\begin{itemize}[noitemsep,topsep=0pt]
    \item \textbf{Page Numbering}: Every braille page (except a title page) must have a page number, typically placed at the far right margin with at least three blank cells before it.\footnote{\url{http://www.brl.org/transcribers/session03/format.html}} Pages preceding the main body often use Roman numerals.\footnote{\url{http://www.brl.org/transcribers/session03/format.html}}
    \item \textbf{Running Heads}: These should be shortened to allow for at least three blank cells before the running head and three before the page number.\footnote{\url{http://www.brl.org/transcribers/session03/format.html}} A blank line should separate a running head from a subsequent heading.\footnote{\url{http://www.brl.org/transcribers/session03/format.html}}
    \item \textbf{Headings and Sub-Headings}: Headings are typically preceded and followed by one blank line and centered with at least three blank cells on either side.\footnote{\url{http://www.brl.org/transcribers/session03/format.html}} Long headings may span multiple lines.\footnote{\url{http://www.brl.org/transcribers/session03/format.html}} A crucial rule dictates that there should be at least two lines of regular text following a heading before the end of a page; if not possible, the heading must be moved to the next page.\footnote{\url{http://www.brl.org/transcribers/session03/format.html}}
    \item \textbf{Line Length}: A standard braille line is 40 cells in length, with 25 lines per page, though variations exist.\footnote{\url{http://www.brl.org/transcribers/session03/format.html}}
    \item \textbf{Omission of Images/Diagrams}: Maps, diagrams, and purely decorative pictures are generally omitted from braille, as they cannot be reproduced tactilely.\footnote{\url{http://www.brl.org/transcribers/session03/format.html}} Captions that provide unique information not found elsewhere in the text should be incorporated into the braille text.\footnote{\url{http://www.brl.org/transcribers/session03/format.html}} Some transcribers may verbally describe non-textual materials.\footnote{\url{http://www.brl.org/transcribers/session03/format.html}}
    \item \textbf{Transcriber's Notes}: Specific symbols and formatting rules exist for transcriber's notes, which provide additional context or explanations for the braille reader.\footnote{\url{http://www.brl.org/transcribers/session03/format.html}}
\end{itemize}
This distinct set of formatting rules means that braille conversion is a specialized domain. It requires explicit attention to these unique conventions to ensure the converted document is not only readable but also navigable and comprehensible for the braille user. The process goes beyond simple text conversion, involving a deep understanding of tactile reading and the specific layout requirements for braille materials.

\subsection{Challenges in Braille Conversion}
Converting Markdown, or any digital text, to braille presents several challenges:
\begin{itemize}[noitemsep,topsep=0pt]
    \item \textbf{Complexity of Braille Codes}: Braille itself is not a language but a communication modality, with different forms like uncontracted (letter-by-letter) and contracted (abbreviated) braille.\footnote{\url{https://inclusiveasl.com/services/accessible-media-services/braille-transcription/}}\footnote{\url{https://www.researchgate.net/publication/357113200_Text_to_Braille_Conversion_System}} Contracted braille, while more efficient, is harder to learn and translate due to numerous combinations of dots.\footnote{\url{https://www.researchgate.net/publication/357113200_Text_to_Braille_Conversion_System}}
    \item \textbf{Size and Cost}: Braille documents are significantly larger than their print counterparts, requiring more paper and increasing production and shipping costs.\footnote{\url{https://www.researchgate.net/publication/357113200_Text_to_Braille_Conversion_System}}
    \item \textbf{Transcription Time}: Manual transcription and rigorous proofreading are time-consuming processes, contributing to delays in making materials available.\footnote{\url{https://accessibleprintshop.com/the-braille-transcription-process/}}\footnote{\url{https://www.researchgate.net/publication/357113200_Text_to_Braille_Conversion_System}}
    \item \textbf{Loss of Visual Information}: Images, graphs, and complex visual layouts in Markdown cannot be directly translated into braille and often require verbal descriptions or tactile graphics, which adds complexity and potential loss of detail.\footnote{\url{https://inclusiveasl.com/services/accessible-media-services/braille-transcription/}}\footnote{\url{https://www.researchgate.net/publication/357113200_Text_to_Braille_Conversion_System}}\footnote{\url{http://www.brl.org/transcribers/session03/format.html}}
    \item \textbf{Mathematical Content}: Converting mathematical notation (e.g., LaTeX in RMarkdown) into accessible braille (e.g., Nemeth code) requires specialized tools and expertise.\footnote{\url{https://www.aph.org/product/brailleblaster/}}
    \item \textbf{Quality Assurance}: Automated translation tools are powerful, but human proofreading is indispensable to ensure accuracy and adherence to complex braille formatting guidelines, as errors can render content unintelligible.\footnote{\url{https://accessibleprintshop.com/the-braille-transcription-process/}}
\end{itemize}
The conversion of Markdown to braille is not a simple automated process; it is a multi-stage endeavor that requires specialized software, adherence to unique formatting standards, and often human intervention for quality assurance. This complexity underscores that braille formatting is a distinct accessibility domain, demanding explicit consideration beyond standard print or digital document design.

\section{Conversion to and from Word Files}
\label{sec:markdown-word}

The ability to convert Markdown documents to and from Microsoft Word files is highly valuable for accessibility, as Word documents are widely used and integrate well with various assistive technologies. This interoperability allows authors to leverage Markdown's simplicity while providing content in a format familiar and accessible to many visually impaired users.

\subsection{Markdown to Word Conversion}
Converting Markdown to Word (.docx) files is efficiently handled by document conversion tools, with Pandoc being a prominent example.\footnote{\url{https://allthingsopen.org/articles/pandoc-convert-markdown-documents-office-formats/}}\footnote{\url{https://pandoc.org/MANUAL.html}} Pandoc is a versatile open-source document converter capable of transforming Markdown into various formats, including Word DOCX, while preserving much of the document's structural integrity.\footnote{\url{https://allthingsopen.org/articles/pandoc-convert-markdown-documents-office-formats/}}\footnote{\url{https://pandoc.org/MANUAL.html}} For instance, a command like \texttt{pandoc -o output.docx input.md} can convert a Markdown file into a Word document.\footnote{\url{https://allthingsopen.org/articles/pandoc-convert-markdown-documents-office-formats/}} Online converters such as MConverter and CloudConvert also offer straightforward web-based solutions for Markdown to DOCX conversion, often supporting batch processing and larger file sizes.\footnote{\url{https://mconverter.eu/convert/markdown/docx/}}\footnote{\url{https://cloudconvert.com/md-to-docx}}

A significant advantage of using tools like Pandoc is their ability to parse Markdown into an abstract syntax tree (AST) before writing it to the target format.\footnote{\url{https://pandoc.org/lua-filters.html}}\footnote{\url{https://pandoc.org/MANUAL.html}} This AST-based conversion ensures that the semantic structure of the Markdown document---such as headings, lists, and tables---is largely preserved and correctly mapped to their corresponding features in the Word document.\footnote{\url{https://pandoc.org/lua-filters.html}}\footnote{\url{https://pandoc.org/MANUAL.html}} For example, Markdown headings (\texttt{\#}) will correctly translate into Word's built-in heading styles (Heading 1, Heading 2, etc.), which are crucial for screen reader navigation.\footnote{\url{https://support.dsu.edu/TDClient/1796/Portal/KB/ArticleDet?ID=148297}}\footnote{\url{https://create.uw.edu/a11y-in-action/accessible-courses/making-homework-write-up-accessible/}} Similarly, alt text provided for images in Markdown will be carried over to the Word document, making images accessible.\footnote{\url{https://pandoc.org/lua-filters.html}} This preservation of semantic fidelity is paramount for ensuring that the converted Word document remains accessible.

\subsection{Word to Markdown Conversion}
While less common, converting Word documents back into Markdown is also possible using tools like Pandoc. This process typically involves Pandoc parsing the Word document's structure and attempting to translate it into Markdown syntax. However, given Word's richer formatting capabilities compared to Markdown, this conversion can be "lossy," meaning some formatting or complex elements might not translate perfectly or might require manual adjustment.\footnote{\url{https://pandoc.org/MANUAL.html}} Complex Word features like merged cells in tables or intricate layouts may not have direct Markdown equivalents, leading to simplified or less structured output.

\subsection{Accessibility Optimization in Word Files}
The primary benefit of converting Markdown to Word for accessibility lies in Word's robust native support for assistive technologies. When Markdown is correctly structured, its semantic elements translate seamlessly into Word's built-in accessibility features:
\begin{itemize}[noitemsep,topsep=0pt]
    \item \textbf{Heading Styles}: Markdown headings become Word's native heading styles, allowing screen reader users to quickly navigate through the document's structure.\footnote{\url{https://support.dsu.edu/TDClient/1796/Portal/KB/ArticleDet?ID=148297}}\footnote{\url{https://create.uw.edu/a11y-in-action/accessible-courses/making-homework-write-up-accessible/}}
    \item \textbf{List Formatting}: Markdown lists convert to Word's bulleted or numbered lists, which screen readers announce properly, enabling users to navigate between list items and understand list context.\footnote{\url{https://support.dsu.edu/TDClient/1796/Portal/KB/ArticleDet?ID=148297}}
    \item \textbf{Alternative Text}: Alt text provided in Markdown for images is preserved, making visual content accessible to screen readers.\footnote{\url{https://support.dsu.edu/TDClient/1796/Portal/KB/ArticleDet?ID=148297}}
    \item \textbf{Table Structure}: Properly structured Markdown tables (with headers and captions) translate into accessible Word tables, allowing screen readers to interpret column and row relationships.\footnote{\url{https://support.dsu.edu/TDClient/1796/Portal/KB/ArticleDet?ID=148297}}
\end{itemize}
The use of Pandoc as a conversion bridge is critical for maintaining semantic fidelity. Its ability to process Markdown into an abstract syntax tree ensures that the underlying structural meaning, rather than just visual appearance, is transferred to the Word document. This is vital because Word's inherent accessibility features rely on proper semantic tagging. Starting with semantically rich Markdown facilitates the creation of highly accessible Word documents, highlighting the significant interoperability benefits for visually impaired users who may prefer Word for its comprehensive assistive technology integration and familiar environment.

\section{Alternative Options and Languages for Accessibility}
\label{sec:alt-options-accessibility}

While Markdown and its derivatives offer compelling advantages for creating accessible documents, a comprehensive approach requires understanding other formats and languages that can also optimize accessibility. Each option presents a unique set of pros and cons, influencing its suitability for different content types and user needs.

\subsection{CommonMark}
CommonMark is a rigorously specified, unambiguous syntax for Markdown, serving as a foundational standard that many Markdown processors adhere to.\footnote{\url{https://www.markdownguide.org/extended-syntax/}}\footnote{\url{https://www.docstomarkdown.pro/tables-in-markdown/}}\footnote{\url{https://quarto.org/docs/reference/formats/markdown/commonmark.html}}
\begin{itemize}[noitemsep,topsep=0pt]
    \item \textbf{Pros}: Its standardization ensures broad compatibility across various platforms and renderers, making it a safe choice for maximum portability.\footnote{\url{https://quarto.org/docs/reference/formats/markdown/commonmark.html}}\footnote{\url{https://mconverter.eu/convert/markdown/docx/}} It generates simple HTML markup, which is inherently easy for assistive technologies to read.\footnote{\url{https://www.smashingmagazine.com/2021/09/improving-accessibility-of-markdown/}}
    \item \textbf{Cons}: CommonMark's simplicity means it lacks advanced features found in more extended Markdown flavors, such as native support for merged cells in tables.\footnote{\url{https://www.docstomarkdown.pro/tables-in-markdown/}} For complex accessibility requirements, direct HTML embedding or reliance on post-processing tools is often necessary.\footnote{\url{https://www.docstomarkdown.pro/tables-in-markdown/}}
\end{itemize}

\subsection{RMarkdown / Quarto}
RMarkdown and its successor, Quarto, are powerful tools primarily used in academic and technical contexts for creating dynamic documents that integrate code, narrative text, and mathematical expressions.\footnote{\url{https://r-resources.massey.ac.nz/rmarkdown/}}\footnote{\url{https://quarto.org/docs/reference/formats/markdown/commonmark.html}}
\begin{itemize}[noitemsep,topsep=0pt]
    \item \textbf{Pros}: They excel at producing accessible HTML, Word, and even PDF documents (often via Word conversion) from a single source file, streamlining the multi-format output process.\footnote{\url{https://quarto.org/docs/reference/formats/markdown/commonmark.html}}\footnote{\url{https://paulnorthrop.github.io/accessr/}} RMarkdown/Quarto support explicit alternative text (\texttt{fig.alt}) and captions (\texttt{fig.cap}) for figures, crucial for image accessibility.\footnote{\url{https://r-resources.massey.ac.nz/rmarkdown/}}\footnote{\url{https://paulnorthrop.github.io/accessr/}} Integration with MathJax ensures mathematical content is rendered in an accessible format that screen readers can interpret.\footnote{\url{https://create.uw.edu/a11y-in-action/accessible-courses/making-homework-write-up-accessible/}}\footnote{\url{https://r-resources.massey.ac.nz/rmarkdown/}} They also allow for structural adjustments like shifting heading levels.\footnote{\url{https://quarto.org/docs/reference/formats/markdown/commonmark.html}}
    \item \textbf{Cons}: The learning curve can be steeper for users unfamiliar with R or Python environments. While they can produce PDFs, achieving fully accessible PDFs remains challenging and often requires specific configurations or intermediate steps.\footnote{\url{https://create.uw.edu/a11y-in-action/accessible-courses/making-homework-write-up-accessible/}}\footnote{\url{https://quarto.org/docs/reference/formats/markdown/commonmark.html}}
\end{itemize}

\subsection{GitHub Flavored Markdown (GFM)}
GFM is a widely adopted variant of Markdown used extensively on GitHub, extending CommonMark with additional features.\footnote{\url{https://docs.github.com/en/contributing/writing-for-github-docs/using-markdown-and-liquid-in-github-docs}}
\begin{itemize}[noitemsep,topsep=0pt]
    \item \textbf{Pros}: Its widespread use in development communities means many users are familiar with it. GFM supports semantic \texttt{<img>} elements, allowing for the provision of alternative text directly.\footnote{\url{https://docs.github.com/en/contributing/writing-for-github-docs/using-markdown-and-liquid-in-github-docs}}\footnote{\url{https://testpros.com/accessibility/accessibility-in-github-with-git-flavored-markdown/}} It automatically generates IDs for headings, facilitating internal linking and navigation.\footnote{\url{https://docs.gitlab.com/user/markdown/}} GFM also encourages accessible practices for tables, such as avoiding empty cells by suggesting "N/A" or "None".\footnote{\url{https://docs.gitlab.com/user/markdown/}} It is compliant with CommonMark, ensuring a baseline of accessibility.\footnote{\url{https://docs.github.com/en/contributing/writing-for-github-docs/using-markdown-and-liquid-in-github-docs}}
    \item \textbf{Cons}: Despite its extensions, GFM still has limitations for complex tables, lacking native support for features like merged cells.\footnote{\url{https://www.docstomarkdown.pro/tables-in-markdown/}} Its accessibility features are often optimized for rendering within the GitHub platform, and may not translate perfectly to other environments.
\end{itemize}

\subsection{HTML (Direct)}
Direct HTML authoring offers the highest degree of control over document structure and accessibility attributes.
\begin{itemize}[noitemsep,topsep=0pt]
    \item \textbf{Pros}: Provides ultimate control over semantic markup and the inclusion of advanced accessibility attributes like WAI-ARIA roles, \texttt{scope} attributes for tables, and \texttt{id}/\texttt{headers} for complex data relationships.\footnote{\url{https://developer.mozilla.org/en-US/docs/Learn_web_development/Core/Accessibility/HTML}}\footnote{\url{https://universaldesign.ie/communications-digital/web-and-mobile-accessibility/web-accessibility-techniques/developers-introduction-and-index/provide-an-accessible-page-structure-and-layout/do-not-misuse-semantic-markup}}\footnote{\url{https://developer.mozilla.org/en-US/docs/Learn_web_development/Core/Structuring_content/Table_accessibility}} When coded correctly, HTML is highly compatible with screen readers and other assistive technologies.\footnote{\url{https://reciteme.com/news/blind-accessibility/}} It allows for adaptive or responsive layouts and explicit language declarations.\footnote{\url{https://universaldesign.ie/communications-digital/web-and-mobile-accessibility/web-accessibility-techniques/developers-introduction-and-index/provide-an-accessible-page-structure-and-layout/do-not-misuse-semantic-markup}}\footnote{\url{https://quarto.org/docs/reference/formats/markdown/commonmark.html}}
    \item \textbf{Cons}: Requires a higher learning curve compared to Markdown, as authors must understand HTML syntax and accessibility best practices in detail.\footnote{\url{https://universaldesign.ie/communications-digital/web-and-mobile-accessibility/web-accessibility-techniques/developers-introduction-and-index/provide-an-accessible-page-structure-and-layout/do-not-misuse-semantic-markup}} It is also easier to inadvertently introduce accessibility errors if semantic markup is misused or omitted.\footnote{\url{https://universaldesign.ie/communications-digital/web-and-mobile-accessibility/web-accessibility-techniques/developers-introduction-and-index/provide-an-accessible-page-structure-and-layout/do-not-misuse-semantic-markup}}
\end{itemize}

\subsection{LaTeX}
LaTeX is a powerful typesetting system widely used in academia and scientific publishing, particularly for its robust handling of complex mathematical equations.\footnote{\url{https://create.uw.edu/a11y-in-action/accessible-courses/making-homework-write-up-accessible/}}
\begin{itemize}[noitemsep,topsep=0pt]
    \item \textbf{Pros}: It is the de facto standard for scientific documents, offering unparalleled control over typography and layout. It handles complex mathematical content exceptionally well, which can be converted to accessible math formats like MathJax.\footnote{\url{https://create.uw.edu/a11y-in-action/accessible-courses/making-homework-write-up-accessible/}}\footnote{\url{https://researchguides.library.wisc.edu/latex/accessibility}} LaTeX supports the use of sectioning commands (\texttt{\textbackslash section\{\}}) to create a structured document with headings, which are understood by screen readers when converted to PDF.\footnote{\url{https://portal.lancaster.ac.uk/ask/checklist-latex}} It also allows for sans-serif fonts and adjustable line spacing for improved readability.\footnote{\url{https://portal.lancaster.ac.uk/ask/checklist-latex}}
    \item \textbf{Cons}: LaTeX has a steep learning curve, requiring users to learn a complex markup language. Native PDF output from LaTeX is often not fully accessible and may require additional tools like Pandoc or PreTeXt for proper tagging and accessibility features.\footnote{\url{https://create.uw.edu/a11y-in-action/accessible-courses/making-homework-write-up-accessible/}}\footnote{\url{https://researchguides.library.wisc.edu/latex/accessibility}} Using italic text, which is common in academic writing, can be difficult to read for some individuals.\footnote{\url{https://portal.lancaster.ac.uk/ask/checklist-latex}}
\end{itemize}

\subsection{DAISY (Digital Accessible Information System)}
DAISY is a technical standard specifically designed for digital audiobooks, periodicals, and computerized text for people with print disabilities, including blindness, impaired vision, and dyslexia.\footnote{\url{https://en.wikipedia.org/wiki/Digital_Accessible_Information_System}}
\begin{itemize}[noitemsep,topsep=0pt]
    \item \textbf{Pros}: It offers advanced navigation features beyond traditional audiobooks, allowing users to search, place bookmarks, navigate line by line, and control speaking speed without distortion.\footnote{\url{https://en.wikipedia.org/wiki/Digital_Accessible_Information_System}}\footnote{\url{https://snow.idrc.ocadu.ca/4b-0-alternative-formats/what-are-alternative-formats/audio/daisy/}} DAISY supports synchronized text and audio, as well as refreshable braille displays.\footnote{\url{https://en.wikipedia.org/wiki/Digital_Accessible_Information_System}}\footnote{\url{https://snow.idrc.ocadu.ca/4b-0-alternative-formats/what-are-alternative-formats/audio/daisy/}} It is ideal for complex materials like textbooks and encyclopedias due to its ability to provide aurally accessible tables, references, and multi-level navigation.\footnote{\url{https://en.wikipedia.org/wiki/Digital_Accessible_Information_System}}
    \item \textbf{Cons}: DAISY is a specialized format not intended for general document creation. It requires specific production tools and playback software, limiting its widespread adoption for everyday content.\footnote{\url{https://en.wikipedia.org/wiki/Digital_Accessible_Information_System}}\footnote{\url{https://snow.idrc.ocadu.ca/4b-0-alternative-formats/what-are-alternative-formats/audio/daisy/}}
\end{itemize}

\subsection{EPUB 3}
EPUB 3 is the latest version of the open ebook standard, with a strong emphasis on accessibility features.\footnote{\url{https://shop.elsevier.com/flexible-ebook-solutions/epub3}}\footnote{\url{https://cnib-beyondprint.ca/making-an-accessible-epub/}}
\begin{itemize}[noitemsep,topsep=0pt]
    \item \textbf{Pros}: It is an industry standard for digital publications, offering enhanced reading experiences for individuals with print disabilities.\footnote{\url{https://shop.elsevier.com/flexible-ebook-solutions/epub3}} EPUB 3 supports features such as high contrast, enlarged text adjustments, text-to-speech functionality, keyboard-only navigation, chapter and page navigation, full-text search, and properly marked tables.\footnote{\url{https://shop.elsevier.com/flexible-ebook-solutions/epub3}} It allows for content magnification without loss of functionality and adjustable text spacing/margins.\footnote{\url{https://shop.elsevier.com/flexible-ebook-solutions/epub3}}
    \item \textbf{Cons}: Primarily designed for ebooks, it may not be suitable for all types of general document creation. Creating accessible EPUB 3 files requires specific authoring tools and adherence to EPUB specifications, including validation with tools like EPUBCheck.\footnote{\url{https://cnib-beyondprint.ca/making-an-accessible-epub/}}
\end{itemize}
The choice among these options depends heavily on the content's complexity, the target audience's needs, and the author's technical proficiency. While Markdown offers a strong starting point for accessible plain text, more specialized formats like DAISY and EPUB 3 provide comprehensive solutions for complex publications tailored for print disabilities.

\begin{table}[h!]
\caption{Table 2: Comparison of Markdown Flavors and Alternative Formats for Accessibility}
\label{tab:markdown-flavors-comparison}
\centering
\begin{tabular}{|l|l|l|l|}
\toprule
Format/Language & Pros for Accessibility & Cons for Accessibility & Best Use Case \\
\midrule
\textbf{CommonMark} & Standardized, high compatibility, simple HTML output easily read by AT.\footnote{\url{https://www.smashingmagazine.com/2021/09/improving-accessibility-of-markdown/}}\footnote{\url{https://quarto.org/docs/reference/formats/markdown/commonmark.html}} & Basic features, limited native support for complex elements like merged tables.\footnote{\url{https://www.docstomarkdown.pro/tables-in-markdown/}} & Simple documentation, READMEs, web content where basic accessibility is sufficient. \\
\addlinespace
\textbf{RMarkdown / Quarto} & Single-source publishing to accessible HTML, Word, PDF (via Word); excellent for code/math; supports alt text/captions for figures; MathJax for accessible math.\footnote{\url{https://create.uw.edu/a11y-in-action/accessible-courses/making-homework-write-up-accessible/}}\footnote{\url{https://r-resources.massey.ac.nz/rmarkdown/}}\footnote{\url{https://quarto.org/docs/reference/formats/markdown/commonmark.html}} & Steeper learning curve for non-programmers; PDF accessibility can be challenging.\footnote{\url{https://create.uw.edu/a11y-in-action/accessible-courses/making-homework-write-up-accessible/}} & Academic papers, technical reports, data science notebooks with code, math, and figures. \\
\addlinespace
\textbf{GitHub Flavored Markdown (GFM)} & Widely used, extends CommonMark; supports semantic \texttt{<img>} for alt text; automatic heading IDs; encourages accessible table practices.\footnote{\url{https://docs.github.com/en/contributing/writing-for-github-docs/using-markdown-and-liquid-in-github-docs}}\footnote{\url{https://docs.gitlab.com/user/markdown/}}\footnote{\url{https://testpros.com/accessibility/accessibility-in-github-with-git-flavored-markdown/}} & Still limited for complex tables (no merged cells); accessibility often tied to GitHub platform rendering.\footnote{\url{https://www.docstomarkdown.pro/tables-in-markdown/}} & Collaborative documentation, project READMEs on GitHub, simple web content. \\
\addlinespace
\textbf{HTML (Direct)} & Ultimate control over semantic structure and ARIA attributes; highly compatible with screen readers when coded correctly.\footnote{\url{https://developer.mozilla.org/en-US/docs/Learn_web_development/Core/Accessibility/HTML}}\footnote{\url{https://universaldesign.ie/communications-digital/web-and-mobile-accessibility/web-accessibility-techniques/developers-introduction-and-index/provide-an-accessible-page-structure-and-layout/do-not-misuse-semantic-markup}} & Higher learning curve; more verbose; easier to introduce errors if not careful.\footnote{\url{https://universaldesign.ie/communications-digital/web-and-mobile-accessibility/web-accessibility-techniques/developers-introduction-and-index/provide-an-accessible-page-structure-and-layout/do-not-misuse-semantic-markup}} & Highly custom web applications, complex interactive content, situations requiring fine-grained accessibility control. \\
\addlinespace
\textbf{LaTeX} & Standard for scientific publishing; excellent for complex math; structured documents via sectioning commands.\footnote{\url{https://create.uw.edu/a11y-in-action/accessible-courses/making-homework-write-up-accessible/}}\footnote{\url{https://portal.lancaster.ac.uk/ask/checklist-latex}} & Steepest learning curve; native PDF output often not fully accessible; requires additional tools for accessibility.\footnote{\url{https://create.uw.edu/a11y-in-action/accessible-courses/making-homework-write-up-accessible/}}\footnote{\url{https://researchguides.library.wisc.edu/latex/accessibility}} & Scientific papers, academic textbooks, documents with extensive mathematical notation. \\
\addlinespace
\textbf{DAISY} & Designed for print disabilities; advanced navigation (line, page, bookmarks); synchronized text/audio; supports refreshable braille displays.\footnote{\url{https://en.wikipedia.org/wiki/Digital_Accessible_Information_System}}\footnote{\url{https://snow.idrc.ocadu.ca/4b-0-alternative-formats/what-are-alternative-formats/audio/daisy/}} & Specialized format, not for general document creation; requires specific production and playback tools.\footnote{\url{https://en.wikipedia.org/wiki/Digital_Accessible_Information_System}}\footnote{\url{https://snow.idrc.ocadu.ca/4b-0-alternative-formats/what-are-alternative-formats/audio/daisy/}} & Digital talking books, accessible textbooks, complex periodicals for print-disabled users. \\
\addlinespace
\textbf{EPUB 3} & Industry standard for ebooks with strong accessibility features; high contrast, enlarged text, text-to-speech, keyboard navigation, accessible tables.\footnote{\url{https://shop.elsevier.com/flexible-ebook-solutions/epub3}} & Primarily for ebooks; requires specific authoring tools and validation.\footnote{\url{https://cnib-beyondprint.ca/making-an-accessible-epub/}} & Accessible ebooks, digital publications, and long-form content. \\
\bottomrule
\end{tabular}
\end{table}

\section{Conclusions and Recommendations}
\label{sec:conclusions-markdown}

The analysis demonstrates that Markdown, with its inherent simplicity and direct mapping to semantic HTML, serves as an excellent foundation for creating accessible documents for individuals who are blind or visually impaired. Its plain-text nature encourages practices that naturally align with screen reader functionality, such as logical heading structures and properly formatted lists. This makes Markdown a powerful tool for content creators who may not have deep expertise in HTML or accessibility standards, enabling them to produce content that is more readily consumable by assistive technologies from the outset.

However, the report also highlights that Markdown is not a complete solution for all accessibility challenges. Its simplicity introduces limitations, particularly concerning complex elements like tables, where native Markdown syntax lacks the rich semantic attributes (e.g., merged cells, explicit header-cell associations) crucial for comprehensive screen reader interpretation. In such cases, a hybrid approach, involving the embedding of raw HTML within Markdown, or the reliance on robust conversion tools like Pandoc, becomes indispensable. These tools act as vital bridges, preserving and enhancing the semantic fidelity of the document as it transforms into various accessible formats.

To maximize the accessibility of documents created with Markdown, the following recommendations are provided:

\begin{itemize}[noitemsep,topsep=0pt]
    \item \textbf{Prioritize Semantic Markup}: Always use Markdown syntax for its intended semantic purpose. Headings should strictly define document structure, lists should be used for enumerated or bulleted items, and emphasis should convey meaning, not just visual style. Avoid using visual formatting (e.g., bolding without proper heading syntax) to convey structural meaning, as this renders content inaccessible to screen readers.
    \item \textbf{Provide Comprehensive Alternative Text and Captions}: Meticulously add concise and descriptive alternative text for all images. For tables, always include a clear caption that summarizes the table's content, and ensure all header cells are correctly identified. This provides crucial context for users who cannot visually perceive the content.
    \item \textbf{Master Table Accessibility}: For simple tables, ensure proper header rows are defined. For more complex tabular data, be prepared to leverage embedded HTML with attributes such as \texttt{scope}, \texttt{id}, and \texttt{headers}, in addition to the \texttt{<caption>} element. Avoid merged or blank cells, as these significantly hinder linear screen reader navigation.
    \item \textbf{Utilize Robust Conversion Tools}: Employ powerful document converters like Pandoc for reliable and semantically faithful conversion of Markdown files to other accessible formats such as Word, HTML, or intermediate formats required for Braille production. This ensures that the accessibility features embedded in the Markdown source are preserved in the output.
    \item \textbf{Choose Flavors and Editors Wisely}: Select Markdown flavors (e.g., RMarkdown/Quarto for academic or data-driven content, GitHub Flavored Markdown for collaborative development) and accessible Markdown editors that best suit the content creation workflow and offer enhanced accessibility features. Editors with strong screen reader compatibility, customizable keyboards, and visual adjustments are paramount for visually impaired users.
    \item \textbf{Test with Assistive Technologies}: Regular testing of the final rendered documents with various screen readers (e.g., JAWS, NVDA, VoiceOver) is critical. This practical verification helps identify and rectify any accessibility barriers that automated checks might miss, ensuring a truly usable experience.
    \item \textbf{Consider Specialized Formats for Specific Needs}: For highly structured, interactive, or long-form content specifically intended for print-disabled users (e.g., textbooks, complex publications), investigate specialized accessible formats like DAISY (Digital Accessible Information System) or EPUB 3. These formats are designed from the ground up to provide advanced navigation and accessibility features.
    \item \textbf{Embrace Continuous Learning}: Accessibility standards and assistive technologies are constantly evolving. Staying informed about the latest best practices, tool capabilities, and user feedback is crucial for consistently producing high-quality, accessible digital content.
\end{itemize}
