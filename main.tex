\documentclass[12pt,letterpaper,twoside]{extreport}
\usepackage[bottom, hang, multiple]{footmisc}
\usepackage[export]{adjustbox}
\usepackage{geometry}
\usepackage[T1]{fontenc}
\usepackage{amsfonts}
\usepackage{amsmath}
\usepackage{amssymb}
\usepackage{array}
\usepackage{avant}
\usepackage{bbold}
\usepackage{blindtext}
\usepackage{calc}
\usepackage[font=small, labelfont=bf, labelsep=colon, hang, singlelinecheck=off, justification=raggedright]{caption}
\usepackage{changepage}
\usepackage{color}
\usepackage{csquotes}
\usepackage[firstpageonly=true]{draftwatermark}
\DraftwatermarkOptions{color=red, angle=0, scale=.25, text=Technical Draft, vanchor=t, vpos=.05\paperheight}
\usepackage{enumitem}
\usepackage{etoolbox}
\usepackage{extramarks}
\usepackage{fancyhdr}
\usepackage{fontspec}
\usepackage{graphicx}
\usepackage{hyperref}
\usepackage{iftex}
\usepackage{lipsum}
\usepackage{lmodern}
\usepackage{longtable,booktabs,array}
\usepackage{makecell}
\usepackage{mathptmx}
\usepackage{multirow}
\usepackage{pdfpages}
\usepackage{placeins}
\usepackage{ragged2e}
\usepackage{rotating}
\usepackage{stmaryrd}
\usepackage{tcolorbox}
\usepackage{textcomp} 
\usepackage{tikz}
\usepackage{titlesec}
\usepackage{titletoc}
\usepackage[titles]{tocloft}
\usepackage{xcolor}
\usepackage{etoolbox}
\usepackage{lastpage}
\usepackage{colortbl}
\usepackage{lastpage}
\usepackage{catoptions}
\usepackage{float}
\usepackage{parskip}
\usepackage{minitoc}
\setlength{\LTleft}{0pt}
\AtBeginEnvironment{longtable}{\fontsize{10pt}{10pt}\selectfont} 
\usepackage{xcolor}
\usepackage{framed}
\setlist{nosep, topsep=0pt}
\AtBeginEnvironment{quote}{\vspace{-\baselineskip}}
\usepackage{arydshln}
\usepackage[none]{hyphenat}

\renewenvironment{leftbar}[1][\hsize]
{%
    \def\FrameCommand
    {%
        {\color{gray!25}\vrule width 10pt}%
        \hspace{0pt}
        \fboxsep=\FrameSep\colorbox{gray!5}%
    }%
    \MakeFramed{\hsize#1\advance\hsize-\width\FrameRestore}%
}
{\endMakeFramed}

\setlength{\footnotemargin}{.75em}

\newcommand\fnsep{\textsuperscript{,}}

\renewcommand{\footnoterule}{%
  \kern -3pt
  \hrule width \textwidth height .5pt
  \kern 2pt
}

\graphicspath{ {./images/} }
\pagestyle{fancyplain}                 
\renewcommand\plainheadrulewidth{1pt}  
\renewcommand\plainfootrulewidth{1pt}  
\raggedbottom
\raggedright
\setcounter{secnumdepth}{5}
\setlength\parindent{0pt}
\urlstyle{same}
\setlength{\headheight}{25.0pt}
\renewcommand{\headrulewidth}{1pt} 
\renewcommand{\footrulewidth}{1pt}

\setmainfont{Atkinson-Hyperlegible}[
    Path=./fontfiles/,
    Extension=.otf,
    UprightFont=*-Regular-102,
    BoldFont=*-Bold-102,
    ItalicFont=*-Italic-102,
    BoldItalicFont=*-BoldItalic-102
]

\titleformat{\chapter}[block]
  {\normalfont\LARGE\bfseries\raggedright}{Chapter \thechapter\newline}{1em}{}
\titlespacing*{\chapter}{-10pt}{-20pt}{0pt}

\titleformat{\section}[block]
  {\normalfont\Large\bfseries\raggedright}{Section \thesection:}{1em}{}
\titlespacing*{\section}{0pt}{10pt}{10pt}

\titleformat{\subsection}[block]
  {\normalfont\Large\bfseries\raggedright}{Subsection \thesubsection:}{1em}{}
\titlespacing*{\subsection}{0pt}{10pt}{10pt}

\renewcommand\cftsecafterpnum{\vskip1pt}
\setlength{\cftbeforesecskip}{5pt}

\title{\Huge ENHANCING EDUCATIONAL EQUITY: \\ \Large A Comprehensive Exploration of Technology Needs for Students with Visual Impairments}
\author{Michael Ryan Hunsaker, M.Ed., Ph.D.}
\date{\vfill \textit{Last Updated: {\today}}}
\renewcommand\cftsecleader{\cftdotfill{\cftdotsep}}

\begin{document}
\dominitoc
\pagenumbering{roman}
\maketitle
\hypersetup{
	pdfborderstyle={/S/U/W 1},     % underline links instead of boxes
	linkbordercolor=red,           % color of internal links
	citebordercolor=blue,          % color of links to bibliography
	filebordercolor=blue,          % color of file links
	urlbordercolor=blue            % color of external links
}

\setcounter{tocdepth}{1}
\cleardoublepage
\cftpagenumbersoff{chapter}
\tableofcontents
{\listoffigures\let\clearpage\relax\vskip20pt\listoftables}
\newpage{}
\fancyhead{}
\fancyfoot{}


\hypertarget{intro}{}\chapter*{Introduction}\label{intro}
\extramarks{Vision Department Technology Needs}{Introduction}
%\addcontentsline{toc}{chapter}{Introduction}
\addstarredchapter{Introduction}
\pagestyle{fancyplain}
\fancyfoot[C]{\thepage}
In the pursuit of an inclusive and equitable educational landscape, it is imperative to recognize the unique challenges faced by students with visual impairments. The Individuals with Disabilities Education Improvement Act (IDEIA) of 2004\footnote{\href{https://sites.ed.gov/idea/statuteregulations/}{20 U.S.C. § 1400, et.}} underscores the commitment to providing every student with a free and appropriate education, regardless of their abilities. For students with visual impairments, technology plays a pivotal role in dismantling barriers, fostering independence, and unlocking opportunities for academic success.

This document seeks to delve into the critical importance of addressing the technology needs of students with visual impairments within the framework of IDEIA, which mandates that students with disabilities, including those with visual impairments, must be given access to assistive technology to ensure they can participate fully in the curriculum. Screen magnification is one such assistive technology that can help students with visual impairments access their free public education\footnote{\href{https://sites.ed.gov/idea/statuteregulations/}{ibid}}. The overarching goal is to shed light on the essential role that technology plays in not only accommodating these students but empowering them to thrive in educational environments. By understanding and meeting their specific technological requirements, we can bridge the accessibility gap, promote inclusivity, and ensure that visually impaired students receive the education they deserve.

It is evident that technology is not merely an auxiliary tool but a catalyst for educational equality. The integration of appropriate technology is fundamental to providing a level playing field, enabling visually impaired students to engage with educational content, interact with peers, and pursue academic excellence with the same vigor as their sighted counterparts.

Throughout this document, we will delve into the diverse spectrum of technological solutions available, ranging from adaptive devices to assistive software, and explore how these tools contribute to an enriched learning experience. By doing so, we aim to advocate for the comprehensive implementation of technology solutions that align with the principles of IDEIA, ensuring that the promise of a free and appropriate education extends to every student, regardless of their visual abilities\footnote{\href{https://sites.ed.gov/idea/statuteregulations/}{ibid}}.

\pagebreak
\fancyhead[RO]{\lastxmark}
\fancyhead[LE]{\firstxmark}
\fancyfoot[RE, LO]{Page \thepage\  of \pageref{LastPage}}
\fancyfoot[LE, RO]{Last Updated: \today}
\fancyfoot[C]{}
\pagenumbering{arabic}

\hypertarget{vision-assistive-technology-laptop-computer-requirements}{}\chapter[\raggedright Navigating Success: \\ The Indispensable Role of Screen Readers and Magnification Programs for Visually Impaired Students]{Navigating Success: The Indispensable Role of Screen Readers and Magnification Programs for Visually Impaired Students}\label{vision-assistive-technology-laptop-computer-requirements}
\extramarks{Vision Department Technology Needs}{Navigating Success}
\minitoc



In the dynamic landscape of education, technology stands as a powerful enabler, breaking down barriers and creating pathways for inclusivity. Nowhere is this more evident than in the realm of assistive technology designed for visually impaired students. Among the myriad tools at their disposal, screen readers and magnification programs emerge as keystones, indispensable in shaping an environment where success is not just attainable but expected.

For visually impaired students, these tools represent a digital gateway to a world of knowledge, interaction, and independent learning. This chapter endeavors to illuminate the significance of screen readers and magnification programs, highlighting their essential roles in fostering student success. As the educational landscape continues to evolve, it is crucial to recognize that the transformative power of technology is not a luxury but a necessity, especially for those whose access to information is mediated by visual impairments.

Screen readers, with their adept ability to convert digital text into synthesized speech, empower visually impaired students to engage with written content. As we delve into the intricacies of these tools, we will uncover their pivotal role in granting students access to textbooks, online resources, and educational materials that are the bedrock of academic achievement. Simultaneously, magnification programs play a crucial role in enhancing visual content, allowing students to explore images, charts, and diagrams with a level of detail that might otherwise be elusive.

Screen magnification is another crucial tool for students with visual impairments, as it enables them to access text and other visual content in the classroom.  By magnifying the text and images on the screen, students can read and view the content more easily, which can help them keep up with their peers and achieve academic success. In this chapter, we will explore the importance of screen magnification for students with visual impairments and how it can help them access their free public education.

Through the lens of accessibility, this exploration seeks to underscore the imperative nature of these technologies, not as mere tools but as companions on the road to success for visually impaired students navigating the educational landscape.

\pagebreak \hypertarget{software-needs}{}\section{Vision Specific Software Needs}\label{software-needs}
As a student with visual impairments, accessing a free and appropriate public education can be challenging. However, with the help of special software, students can overcome these challenges and achieve academic success. Assistive Technology has improved the lives of students and adults living with visual impairments by providing access, connectedness, and engagement. The use of assistive technology can facilitate a learning environment where students are able to better access their educational program through low or high technology accommodations.

Incorporating special software can help students with visual impairments to access the same educational resources as their peers. It can also help them to learn at their own pace and in their own way. For instance, screen readers and magnification options built into mainstream tablets and smartphones can help students with visual impairments to read and write emails, texts, and documents.

In addition, technology can help students with visual impairments to become active learners. Traditional lectures may not be the optimal learning style for visually impaired students. However, hands-on engagement through technology can help them to better understand and complete assignments and tests2.

By using special software, students with visual impairments can have equal access to educational opportunities and can be better prepared for the real world. It can also help them to acquire the skills required for independent living and getting higher education.

\pagebreak\hypertarget{student-software-needs}{}\subsection{Student Software Needs}\label{student-software-needs}
Table \ref{tab:table1} is a list of software used to access material as well as necessary academic software used by students with visual impairments\footnote{We focus on Windows-based laptops due to the ubiquity of Windows-based software used in schools. MacOS-based laptops are more than adequate to run software along with the built in VoiceOver screenreader  To date there are no additional screenreaders for MacOS, however one is currently in development called \href{https://youtu.be/qTkS-zNzF88?si=3XTXtbyOWD9kvwlk}{VOSH}. There are multiple Linux distributions that are actively working to improve ORCA screenreader accessibility within the GNOME desktop environment. There are also a number of independently developed screenreaders for Linux architectures}\fnsep\footnote{This includes the TVI teaching how to use the software skills. But this primarily refers to programs the students need to access the curriculum}. This information will be used to determine necessary laptop specifications for students using these software to access their schoolwork at the same time as their sighted peers.

\begin{longtable}[]{>{\raggedright\arraybackslash}m{.2\textwidth}>{\raggedright\arraybackslash}m{.2\textwidth}>{\raggedright\arraybackslash}m{.15\textwidth}>{\raggedright\arraybackslash}m{.1\textwidth}>{\raggedright\arraybackslash}m{.15\textwidth}>{\raggedright\arraybackslash}b{.15\textwidth}}
	\toprule
	\textbf{Program}                                                                                                                                                                                                                                                                                                                      & \textbf{Type of Program}                                                                                                                                                                                                             & \textbf{Cost}                                                                                                                                                                                                                                                             & \textbf{Min RAM} & \textbf{Pref RAM} & \textbf{Processor}       \\
	\midrule
	\endhead \hline                                                                                                                                                                                                                                                                                                                                                                                                                                                                                                                                                                                                                                                                                                                                                                                                                                                                                                            \\
	\multicolumn{6}{r}{\textbf{Continued on Next Page}} \endfoot
	\endlastfoot
	JAWS                                                                                                                                                                                                                                                                                                                                  & Screenreader                                                                                                                                                                                                                         & \$95/yr\footnote{typically purchased via through APH quota funds}                                                                                                                                                                                                         & 8GB              & \textgreater16GB  & \textgreater11th Gen i5+ \\[1.0em]
	TypeAbility                                                                                                                                                                                                                                                                                                                           & Typing Instruction\footnote{TypeAbility requires JAWS or Fusion to run as it uses the JAWS voice engine in order to run}                                                                                                             & \$150                                                                                                                                                                                                                                                                     & 8GB              & \textgreater16GB  & \textgreater11th Gen i5+ \\[1.0em]
	Narrator\footnote{Windows Narrator is a built in to Windows 10 and Windows 11}                                                                                                                                                                                                                                                        & Screenreader                                                                                                                                                                                                                         & \$0                                                                                                                                                                                                                                                                       & 4GB              & \textgreater16GB  & \textgreater11th Gen i5  \\[1.0em]
	NVDA                                                                                                                                                                                                                                                                                                                                  & Screenreader                                                                                                                                                                                                                         & \$0\footnote{NVDA is free, but if you want Eloquence, Acapella, or Vocalizer Expressive TTS Voices, they have to be purchased from \href{https://codefactoryglobal.com/nova/eloquence-and-vocalizer-embedded-add-on-for-nvda/}{CodeFactory} for \$70 (perpetual license)} & 2GB              & \textgreater8GB   & \textgreater11th Gen i5  \\[1.0em]
	ZDSR                                                                                                                                                                                                                                                                                                                                  & Screen Reader                                                                                                                                                                                                                        & \$232                                                                                                                                                                                                                                                                     & 2GB              & \textgreater8GB   & \textgreater11th Gen i7+ \\[1.0em]
	Dolphin Screenreader                                                                                                                                                                                                                                                                                                                  & Screenreader                                                                                                                                                                                                                         & \$1,105/yr                                                                                                                                                                                                                                                                & 8GB              & \textgreater32GB  & \textgreater11th Gen i7+ \\[1.0em]
	ZoomText                                                                                                                                                                                                                                                                                                                              & Magnification                                                                                                                                                                                                                        & \$85/yr\footnote{typically purchased via through APH quota funds}                                                                                                                                                                                                         & 16GB             & \textgreater32GB  & \textgreater11th Gen i7+ \\[1.0em]
	Windows Magnifier\footnote{Windows Magnifier is a built in to Windows 10 and Windows 11}                                                                                                                                                                                                                                              & Magnification                                                                                                                                                                                                                        & \$0                                                                                                                                                                                                                                                                       & 16GB             & \textgreater16GB  & \textgreater11th Gen i7+ \\[1.0em]
	Dolphin SuperNova                                                                                                                                                                                                                                                                                                                     & Magnification                                                                                                                                                                                                                        & \$545/yr                                                                                                                                                                                                                                                                  & 16GB             & \textgreater32GB  & \textgreater11th Gen i7+ \\[1.0em]
	Dolphin SuperNova\break +Speech                                                                                                                                                                                                                                                                                                       & Magnification\break \& Speech                                                                                                                                                                                                        & \$825/yr                                                                                                                                                                                                                                                                  & 16GB             & \textgreater32GB  & \textgreater11th Gen i7+ \\[1.0em]
	Fusion                                                                                                                                                                                                                                                                                                                                & Screenreader \break \& Magnification                                                                                                                                                                                                 & \$170/yr\footnote{typically purchased via through APH quota funds}                                                                                                                                                                                                        & 16GB             & \textgreater32GB  & \textgreater11th Gen i7+ \\[1.0em]


	Dolphin Screenreader\break +SuperNova                                                                                                                                                                                                                                                                                                 & Screenreader \break \& Magnification                                                                                                                                                                                                 & \$1,665/yr                                                                                                                                                                                                                                                                & 8GB              & \textgreater32GB  & \textgreater11th Gen i7+ \\[1.0em]
	Java JDK 8\footnote{This JDK is no longer considered up to date but has been designated as receiving long trm support until 2030, however most modern accessibility tools are developed using Java 11, 17, or 21. \textit{Cf}. \href{https://www.oracle.com/java/technologies/java-se-support-roadmap.html}{Java SE Support Roadmap}} & Dependency\footnote{JAWS and NVDA screenreaders often communicate with the Operating System using custom modifications to the JAVA Access Bridge. As such, JAVA is a dependency for most software packages addressing accessibility} & \$0                                                                                                                                                                                                                                                                       & 4GB              & \textgreater8GB   & \textgreater9th Gen i3+  \\ [1.0em]
	Microsoft 365 \footnote{Microsoft is adding OpenAI based tools called \href{https://www.microsoft.com/en-us/microsoft-365/enterprise/microsoft-365-copilot}{Microsoft CoPilot} to their products, which takes an extra 1-3GB of RAM in order to concurrently run Office applications and screenreaders smoothly on my computers.}     & Work Completion                                                                                                                                                                                                                      & \$7/mo                                                                                                                                                                                                                                                                    & 4GB              & \textgreater16GB  & \textgreater11th Gen i5  \\[1.0em]
	Windows 11                                                                                                                                                                                                                                                                                                                            & Operating System                                                                                                                                                                                                                     & Home \$139   \break Pro \$199                                                                                                                                                                                                                                             & 4GB              & \textgreater16GB  & \textgreater11th Gen i7+ \\[1.0em]
	Windows 12\footnote{Current tech reports suggest Windows 12 will require 8GB rather than the 4GB requirement for Windows 10-11}\break (June 2024)                                                                                                                                                                                     & Operating System                                                                                                                                                                                                                     & Home \$139   \break Pro \$199                                                                                                                                                                                                                                             & 8GB              & \textgreater16GB  & \textgreater12th Gen i7+ \\[1.0em]
	Microsoft Teams                                                                                                                                                                                                                                                                                                                       & Web Meeting                                                                                                                                                                                                                          & \$0\footnote{free for a limited set of features}                                                                                                                                                                                                                          & 4GB              & \textgreater16GB  & \textgreater11th Gen i7+ \\[1.0em]
	Zoom                                                                                                                                                                                                                                                                                                                                  & Web Meeting                                                                                                                                                                                                                          & \$0\footnote{free for a limited set of features}                                                                                                                                                                                                                          & 4GB              & \textgreater16GB  & \textgreater11th Gen i7+ \\[1.0em]

	Notepad++                                                                                                                                                                                                                                                                                                                             & Coding\footnote{Notepad++ is accessible with all screenreaders}                                                                                                                                                                      & \$0                                                                                                                                                                                                                                                                       & 512MB            & \textgreater4GB   & \textgreater11th Gen i7+ \\[1.0em]
	Visual Studio Code                                                                                                                                                                                                                                                                                                                    & Coding\footnote{Notepad++ is accessible with all screenreaders}                                                                                                                                                                      & \$0                                                                                                                                                                                                                                                                       & 4GB              & \textgreater8GB   & \textgreater11th Gen i7+ \\[1.0em]
	Python\footnote{This is accessed through the Windows Terminal or Command Line}                                                                                                                                                                                                                                                        & Coding                                                                                                                                                                                                                               & \$0                                                                                                                                                                                                                                                                       & 4GB              & \textgreater8GB   & \textgreater11th Gen i7+ \\[1.0em]


	Adobe Reader                                                                                                                                                                                                                                                                                                                          & PDF Reader                                                                                                                                                                                                                           & \$0                                                                                                                                                                                                                                                                       & 2GB              & \textgreater16GB  & \textgreater11th Gen i7+ \\[1.0em]

	MuseScore                                                                                                                                                                                                                                                                                                                             & Music braille                                                                                                                                                                                                                        & \$0                                                                                                                                                                                                                                                                       & 8GB              & \textgreater32GB  & \textgreater11th Gen i7+ \\[1.0em]
	Sibelius                                                                                                                                                                                                                                                                                                                              & Music braille                                                                                                                                                                                                                        & \$0\footnote{Sibelius ONE is free but very limited in capability}                                                                                                                                                                                                         & 8GB              & \textgreater32GB  & \textgreater11th Gen i7+ \\[1.0em] \hline
	\caption[Software used by Students with Visual Imairments]{Software used by Vision Students to Access and Complete Academic Tasks}\label{tab:table1}
\end{longtable}
\pagebreak \hypertarget{teacher-software-needs}{}\subsection{Teacher Software Needs}\label{teacher-software-needs}
Table \ref{tab:table2} is a list of software used by Teachers of Students with Visual Impairments to generate materials for students with visual impairments\footnote{This includes everything from the above table in order to be able to teach the software}.

These software programs are often memory intensive and sometimes benefit from use of command-line tools originally developed for Linux or MacOS environments but are available in the Windows environment using tools such as the Windows Subsystem for Linux.

\begin{longtable}[]{>{\raggedright\arraybackslash}m{.25\textwidth}>{\raggedright\arraybackslash}m{.2\textwidth}>{\raggedright\arraybackslash}m{.1\textwidth}>{\raggedright\arraybackslash}m{.1\textwidth}>{\raggedright\arraybackslash}m{.15\textwidth}>{\raggedright\arraybackslash}b{.15\textwidth}}
	\toprule
	\textbf{Program}                                                                                                                                                                                                                                                                                                                                           & \textbf{Type of Program}                                                                                                                                                                                                                                                      & \textbf{Cost}                                                                                         & \textbf{Min RAM} & \textbf{Pref RAM}                                                                                                                                          & \textbf{Processor}       \\
	\midrule
	\endhead \hline                                                                                                                                                                                                                                                                                                                                                                                                                                                                                                                                                                                                                                                                                                                                                                                                                                                                                                                                               \\
	\multicolumn{6}{r}{\textbf{Continued on Next Page}} \endfoot
	\endlastfoot

	Duxbury DBT 12.7\footnote{Duxbury is considered the "Gold Standard" print to braille transcription program, largely due to being present in the field since 1969}\fnsep\footnote{NimPro 3.0 must be purchased for \$295 to import and work with NIMAS files from NIMAC}                                                                                    & Braille Transcription                                                                                                                                                                                                                                                         & \$695/yr                                                                                              & not given        & not given                                                                                                                                                  & not given                \\[1.0em]
	Braille2000\footnote{Braille2000 is preferred by braille proofreaders as the "Gold Standard" program for editing brf files in place}\break Basic Edition                                                                                                                                                                                                   & Braille Transcription                                                                                                                                                                                                                                                         & \$21/mo\break\$439/yr\footnote{Braille to Print Interpreter requires an extra \&2/mo or \$49/yr}.     & not given        & not given                                                                                                                                                  & not given                \\[1.0em]

	Braille2000\break Direct Entry Ed.                                                                                                                                                                                                                                                                                                                         & Braille Transcription                                                                                                                                                                                                                                                         & \$21/mo\break\$749/yr\footnote{Children's Braille Grade Relaxer requires an extra \$4/mo or \$149/yr} & not given        & not given                                                                                                                                                  & not given                \\[1.0em]

	Braille2000\break Document Processing Ed.                                                                                                                                                                                                                                                                                                                  & Braille Transcription                                                                                                                                                                                                                                                         & \$32/mo\break\$1149/yr\footnote{required
	to import NIMAS files from NIMAC.}\fnsep\footnote{An extra \$7/mo \$239/yr must be purchased for MathML support}                                                                                                                                                                                                                                           & not given                                                                                                                                                                                                                                                                     & not given                                                                                             & not given                                                                                                                                                                                                \\[1.0em]

	Braille2000\break The Talking Ed.\footnote{This gives built in text to speech for Braille2000 as three are known issues with JAWS and NVDA}                                                                                                                                                                                                                & Braille Transcription                                                                                                                                                                                                                                                         & \$40/mo\break\$1299/yr                                                                                & not given        & not given                                                                                                                                                  & not given                \\[1.0em]


	Dotify                                                                                                                                                                                                                                                                                                                                                     & Braille Transcription                                                                                                                                                                                                                                                         & \$0                                                                                                   & 8GB              & \textgreater16GB                                                                                                                                           & \textgreater11th Gen i7+ \\[1.0em]
	BrailleBlaster\footnote{BrailleBlaster has been developed by APH in order to more readily import and format NIMAS files from NIMAC}\fnsep\footnote{BrailleBlaster has weaknesses in custom braille formatting. Built-in features only allow formatting following the official Braille Association of North America formatting standards published in 2016} & Braille Transcription                                                                                                                                                                                                                                                         & \$0                                                                                                   & 8GB              & \textgreater16GB                                                                                                                                           & \textgreater11th Gen i7+ \\[1.0em]
	Sao Mai Braille                                                                                                                                                                                                                                                                                                                                            & Music Braille\break Braille Transcription                                                                                                                                                                                                                                     & \$0                                                                                                   & 4GB              & \textgreater8GB                                                                                                                                            & \textgreater11th Gen i7+ \\[1.0em]

	Tiger Software Suite\footnote{Requires Microsoft Word for some functions of the software}                                                                                                                                                                                                                                                                  & Tactile Graphics                                                                                                                                                                                                                                                              & \$195/yr\break\$595 perpetual                                                                         & 1GB              & \textgreater4GB                                                                                                                                            & \textgreater11th Gen i7+ \\[1.0em]


	TactileView                                                                                                                                                                                                                                                                                                                                                & Tactile Graphics                                                                                                                                                                                                                                                              & \$484/yr                                                                                              & 4GB              & \textgreater8GB                                                                                                                                            & \textgreater11th Gen i7+ \\[1.0em]

	FireBird                                                                                                                                                                                                                                                                                                                                                   & Tactile Graphics                                                                                                                                                                                                                                                              & \$0                                                                                                   & 4GB              & \textgreater8GB                                                                                                                                            & \textgreater11th Gen i7+ \\[1.0em]
	QuickTac                                                                                                                                                                                                                                                                                                                                                   & Tactile Graphics                                                                                                                                                                                                                                                              & \$0                                                                                                   & 1GB              & \textgreater4GB                                                                                                                                            & \textgreater11th Gen i7+ \\[1.0em]

	Ultimaker Cura                                                                                                                                                                                                                                                                                                                                             & 3D modeline\break 3D Printing                                                                                                                                                                                                                                                 & \$0                                                                                                   & 8GB              & \textgreater16GB                                                                                                                                           & \textgreater11th Gen i7+ \\[1.0em]
	PrusaSlicer                                                                                                                                                                                                                                                                                                                                                & 3D modeline\break 3D Printing                                                                                                                                                                                                                                                 & \$0                                                                                                   & 8GB              & \textgreater16GB                                                                                                                                           & \textgreater11th Gen i7+ \\[1.0em]
	Blender                                                                                                                                                                                                                                                                                                                                                    & 3D modeling                                                                                                                                                                                                                                                                   & \$0                                                                                                   & 8GB              & \textgreater16GB                                                                                                                                           & \textgreater12th Gen i7+ \\[1.0em]

	Docker                                                                                                                                                                                                                                                                                                                                                     & Programming Interface\footnote{This requires Windows Subsystem for Linux and Windows Hyper-V activated for use.}\fnsep\footnote{Docker allows a user to use any Linux-based program locally through a command-line interface. However, this can be rather resource intensive} & \$0                                                                                                   & 8GB              & \textgreater16GB                                                                                                                                           & \textgreater11th Gen i7+ \\[1.0em]
	OpenBook                                                                                                                                                                                                                                                                                                                                                   & Optical Character Recognition                                                                                                                                                                                                                                                 & \$1,000                                                                                               & 8GB              & \textgreater16GB                                                                                                                                           & \textgreater11th Gen i7+ \\[1.0em]
	Adobe Acrobat Pro                                                                                                                                                                                                                                                                                                                                          & Optical Character Recognition                                                                                                                                                                                                                                                 & \$14/mo                                                                                               & 2GB              & \textgreater16GB\footnote{This recommendation comes from \href{https://www.crucial.com/articles/about-memory/how-much-ram-does-my-computer-need}{Crucial}} & \textgreater11th Gen i7+ \\ [1.0em]
	ABBYY FineReader                                                                                                                                                                                                                                                                                                                                           & Optical Character Recognition                                                                                                                                                                                                                                                 & \$177/yr                                                                                              & 8GB              & \textgreater16GB                                                                                                                                           & \textgreater11th Gen i7+ \\ [1.0em]
	Adobe Indesign                                                                                                                                                                                                                                                                                                                                             & Math Transcription \break Typesetting\break ePub Creation                                                                                                                                                                                                                     & \$23/mo                                                                                               & 8GB              & \textgreater16GB                                                                                                                                           & \textgreater11th Gen i7+ \\ [1.0em]
	Scribus                                                                                                                                                                                                                                                                                                                                                    & Typesetting\break ePub Creation                                                                                                                                                                                                                                               & \$0                                                                                                   & 2GB              & \textgreater8GB                                                                                                                                            & \textgreater Pentium III \\ [1.0em]
	Adobe Illustrator                                                                                                                                                                                                                                                                                                                                          & Tactile Graphics                                                                                                                                                                                                                                                              & \$32/mo                                                                                               & 8GB              & \textgreater16GB                                                                                                                                           & \textgreater11th Gen i7+ \\ [1.0em]
	Inkscape                                                                                                                                                                                                                                                                                                                                                   & Tactile Graphics                                                                                                                                                                                                                                                              & \$0                                                                                                   & 8GB              & \textgreater16GB                                                                                                                                           & \textgreater11th Gen i7+ \\ [1.0em]
	Corel Draw                                                                                                                                                                                                                                                                                                                                                 & Tactile Graphics                                                                                                                                                                                                                                                              & \$16/mo                                                                                               & 8GB              & \textgreater16GB                                                                                                                                           & \textgreater12th Gen i7+ \\ [1.0em]
	DAISY Pipeline                                                                                                                                                                                                                                                                                                                                             & ePub Creation                                                                                                                                                                                                                                                                 & \$0                                                                                                   & 4GB              & \textgreater8GB                                                                                                                                            & \textgreater11th Gen i7+ \\ [1.0em]
	TeXStudio                                                                                                                                                                                                                                                                                                                                                  & Math Transcription \break Math Typesetting                                                                                                                                                                                                                                    & \$0                                                                                                   & 4GB              & \textgreater8GB                                                                                                                                            & \textgreater11th Gen i7+ \\ [1.0em] \hline
	\caption[Software used by Teachers of Students with Visual Impairments]{Software used by Teachers of Students with Visual Impairments to transcribe, typeset, and generate materials for students with visual impairments. }\label{tab:table2}
\end{longtable}

\pagebreak \hypertarget{ram-requirements}{}\section{RAM Requirements}\label{ram-requirements}
Having a computer with sufficient RAM and processor speed is crucial for the effective functioning of a screen reader, which serves as a vital assistive technology for individuals with visual impairments. A screen reader relies heavily on processing power and memory to rapidly convert textual information into synthesized speech or refreshable braille displays, allowing users to access and navigate digital content. A computer with inadequate RAM or a slow processor may struggle to process and relay information in real-time, resulting in delayed responses, sluggish navigation, and an overall compromised user experience. Insufficient hardware specifications can significantly hinder the screen reader's ability to provide timely and accurate information, rendering it an inadequate accommodation for individuals with visual impairments. Therefore, ensuring that the computer meets or exceeds the recommended RAM and processor speed is essential to guarantee an optimal and seamless user experience, empowering individuals with visual impairments to access and engage with digital content effectively.

The information in Table \ref{tab:table3} is from Crucial, in an \href{https://www.crucial.com/articles/about-memory/how-much-ram-does-my-computer-need}{article discussing RAM needs for different scenarios}.

\begin{longtable}[]{@{}>{\raggedright\arraybackslash}m{.5\textwidth}>{\raggedright\arraybackslash}b{.5\textwidth}@{}}
	\toprule

	\textbf{If this is how you use your computer}                                                                                                                                                                                                                                                                                                                                                        & \textbf{Here's how much memory we recommend} \\
	\midrule
	\endhead \hline                                                                                                                                                                                                                                                                                                                                                                                                                                     \\
	\multicolumn{2}{r}{\textbf{Continued on Next Page}}                                                                                                                                                                                                                                                                                                                                                                                                 \\
	\endfoot

	\endlastfoot
	\vskip1em\textbf{Casual User} \break \break Internet browsing\break email\break listening to music\break watching videos                                                                                                                                                                                                                                                                             & \emph{At least} 8GB                          \\[2.5em]
	\vskip1em\textbf{Intermediate User} \break \break Internet browsing\break email\break Word Processing\break spreadsheets\break music\break videos or multitasking                                                                                                                                                                                                                                    & \emph{At least 16}GB                         \\[2.5em]
	\vskip1em\textbf{Professional User}\footnote{I place students using screenreaders into this category since they are having to use a resource intensive screenreader/Screen Magnifier while performing all the tasks required of an ``Intermediate User'' in the table above.}\break \break High performance gaming\break multimedia editing\break high-definition video\break intensive multitasking & \emph{At least} 32GB                         \\[1.0em] \hline
	\caption{How Much RAM is Needed?}\label{tab:table3}
\end{longtable}

\pagebreak From the article (\emph{emphasis mine}):
\begin{leftbar} \begin{quote}
		32GB of RAM is the amount of memory we recommend for serious gamers, engineers, scientists, and entry-level multimedia users. This level of RAM allows for these memory-hungry programs to run smoothly, \textbf{\emph{even as your computer ages}}. Therefore, It's not too much, it's just right.
	\end{quote}\end{leftbar}
\hypertarget{current-student-professional-laptops}{}\section{Current Student \& Professional Laptops}\label{current-student-professional-laptops}
Table \ref{tab:table4} lists the laptops students I work with use in classrooms as of {\today}.

\begin{longtable}[]{@{}
	>{\raggedright\arraybackslash}m{.28\textwidth}
	>{\raggedright\arraybackslash}m{.1\textwidth}
	>{\raggedright\arraybackslash}m{.1\textwidth}
	>{\raggedright\arraybackslash}m{.1\textwidth}
	>{\raggedright\arraybackslash}m{.15\textwidth}
	>{\raggedright\arraybackslash}b{.2\textwidth}@{}
	}
	\toprule

	\textbf{Company / Model}                                               & \textbf{Cost} & \textbf{Keyboard}                                & \textbf{RAM}                                                                                                                & \textbf{Screen Size} & \textbf{Processor} \\
	\midrule
	\endhead \hline                                                                                                                                                                                                                                                                                                     \\
	\multicolumn{6}{r}{\textbf{Continued on Next Page}} \endfoot
	\endlastfoot
	\textbf{Students \& Professionals} \break Dell Latitude 3190 Education & \$379         & QWERTY                                           & 4GB\footnote{student laptops have 4GB}\break 8GB\footnote{\emph{some} professional laptops have 4GB, the majority have 8GB}
	                                                                       & 11.6''        & Intel Celeron Silver\break (Intel for Education)                                                                                                                                                                           \\[1.0em]
	\textbf{Professionals} \break Dell Precision 3530                      & \$1751        & QWERTY                                           & 16GB                                                                                                                        & 16.0''               & 8th Gen i7         \\[1.0em]
	\textbf{Professionals} \break Dell Precision 7420                      & \$2,349       & QWERTY                                           & 16GB                                                                                                                        & 16.0''               & 8th Gen i7         \\[1.0em]
	\textbf{My Personal Laptop} \break Microsoft Surface Laptop 3          & \$2500        & QWERTY                                           & 32GB                                                                                                                        & 15.0''               & AMD Ryzen 7        \\ [1.0em] \hline
	\caption{ Current Student and Professional Laptops}\label{tab:table4}
\end{longtable}


\pagebreak 	\hypertarget{current-laptop-performance-measured}{}\section{Current Laptop Performance}\label{current-laptop-performance-measured}
Having a computer with sufficient RAM and an up-to-date processor is crucial for running a screenreader and screen magnifier smoothly as a student with visual impairments in order to receive a free and appropriate public education. Screenreaders and screen magnifiers are software applications that require a significant amount of processing power and memory to function properly. Insufficient RAM can cause the screenreader or screen magnifier to load slowly, which can lead to delays in the user’s workflow. An up-to-date processor is also important because it can help ensure that the software runs smoothly and efficiently. By having a computer with sufficient RAM and an up-to-date processor, students with visual impairments can access the same educational materials as their sighted peers and participate fully in the curriculum. This can help improve their academic performance and ensure that they have the tools they need to succeed in their studies and beyond.

In addition to having sufficient RAM and an up-to-date processor, it is also important to ensure that the computer is running the latest version of the screenreader and screen magnifier software. Software updates often include bug fixes, performance improvements, and new features that can help improve the user experience. By keeping the software up-to-date, students with visual impairments can ensure that they are getting the most out of their assistive technology. It is also important to ensure that the computer is free of malware and other malicious software that can slow down the system and interfere with the operation of the screenreader and screen magnifier. By taking these steps, students with visual impairments can ensure that their computer is running optimally and that they have the tools they need to succeed in their studies and beyond.

\pagebreak \hypertarget{screenreader-loading}{}\section{Screenreader Loading}\label{screenreader-loading}
The latency of a screenreader is the time it takes for the software to load and start functioning. It is important to measure the latency of a screenreader to determine if the laptop has sufficient RAM to run the software properly. Insufficient RAM can cause the screenreader to load slowly, which can lead to delays in the user’s workflow. Measuring the latency of a screenreader can help identify if the laptop has enough RAM to run the software smoothly. This can help users avoid frustration and improve their productivity. In addition to identifying insufficient RAM, measuring the latency of a screenreader can also help users identify if there are other issues with their laptop that may be causing the software to run slowly. For example, if the latency is still high even after upgrading the RAM, it could be an indication of a slow hard drive or outdated drivers. By measuring the latency of a screenreader, users can ensure that their laptop is running optimally and that they are getting the most out of their software. It is recommended to measure the latency of a screenreader periodically to ensure that the laptop is running smoothly and to identify any issues that may arise.

Figure \ref{fig:figure 1} shows a boxplot of the latency to load JAWS measured across the various student and professional computers I had access to. The student laptop generally took \textgreater2 minutes for JAWS to load, a higher spec student laptop took about 1 minute, and the professional laptops took under a minute\footnote{\href{https://github.com/mrhunsaker/MiscResources/raw/main/ComputerRBDisplaySpecsTVIFig1.zip}{Zipped Interactive HTML version of Figure \ref{fig:figure 1}}}.

\begin{figure}[H]
	\centering
	\includegraphics[width=\textwidth]{images/ComputerRBDisplaySpecsTVIFig1.png}

	\caption[Latency to Load JAWS]{Plot showing Latency to Load JAWS while Microsoft Word is open across a typical student laptop (Dell Latitude 3190 with \textit{8GB RAM}), a high quality student laptop (Dell Precision 3530 with \textit{16GB RAM}), a professional laptop (Lenovo ThinkPad E16 with \textit{24GB RAM}), and a high power laptop (Microsoft Surface Laptop 3 with \textit{32GB RAM}).}\label{fig:figure 1}
\end{figure}

\pagebreak
\hypertarget{screenreader-response}{}\section{Screenreader Responsiveness}\label{screenreader-response}
Measuring the latency of a screenreader to respond to key presses is important to determine if the laptop has sufficient RAM to run the software properly. If the laptop has insufficient RAM, the screenreader may take longer to respond to key presses, which can lead to delays in the user’s workflow. Measuring the latency of a screenreader can help identify if the laptop has enough RAM to run the software smoothly. This can help users avoid frustration and improve their productivity. Additionally, measuring the latency of a screenreader can help users identify if there are other issues with their laptop that may be causing the software to run slowly. By measuring the latency of a screenreader, users can ensure that their laptop is running optimally and that they are getting the most out of their software. Table \ref{tab:table5} provides these data for the same computers shown in \ref{fig:figure 1}
\begin{longtable}[]{@{}
	>{\raggedright\arraybackslash}m{.5\textwidth}
	>{\raggedright\arraybackslash}m{.25\textwidth}
	>{\raggedright\arraybackslash}b{.25\textwidth}
	@{}
	}

	\toprule

	\textbf{Computer} \break (Color as Labelled in Figure 1)                                                                                                                                        & \textbf{Loading Time}\break (median [Range])                                                                                                               & \textbf{Response Lag}\break (median [Range])
	\\
	\midrule
	\endhead \hline                                                                                                                                                                                                                                                                                                                                                                                                                                                                                                                                                                                                                                                      \\
	\multicolumn{3}{r}{\textbf{Continued on Next Page}} \endfoot
	\endlastfoot
	\fcolorbox{red}{red}{\rule{0pt}{6pt}\rule{6pt}{0pt}}\qquad $\begin{array}{l}\textbf{Students Laptop}\footnote{Dell Latitude 3190} \\ \text{8GB RAM}\end{array}$                                 & 143 [93-183] \footnote{These are the data plotted in Figure 1 above. The responsiveness data are more clear when presented as a table here than as a plot} & 38 [27-91]\footnote{It is further important to note here that any lag in screenreader responsiveness of \textgreater1 sec means the student is behind their peers and their educational opportunity is limited by the technology not being sufficient (\emph{i.e.}, not an adequate accommodation). } \\[1.0em]
	\fcolorbox{cyan}{cyan}{\rule{0pt}{6pt}\rule{6pt}{0pt}}\qquad $\begin{array}{l}\textbf{Student/Professional Laptop}\fnsep\footnote{Dell Precision 3530} \\ \text{16GB RAM}\end{array}$           & 64 [38-93]                                                                                                                                                 & 9 [4-15]                                                                                                                                                                                                                                                                                              \\[1.0em]
	\fcolorbox{violet}{violet}{\rule{0pt}{6pt}\rule{6pt}{0pt}}\qquad$\begin{array}{l}\textbf{Professional Laptop}\footnote{Lenovo ThinkPad E16--TVI Personal Laptop} \\ \text{24GB RAM}\end{array}$ & 49 [26-65]                                                                                                                                                 & 1 [0.05-2.5]                                                                                                                                                                                                                                                                                          \\[1.0em]
	\fcolorbox{orange}{orange}{\rule{0pt}{6pt}\rule{6pt}{0pt}}\qquad$\begin{array}{l}\textbf{Professional Laptop}\footnote{Microsoft Surface 3--My Personal Laptop} \\ \text{32GB RAM}\end{array}$  & 25 [10-32]                                                                                                                                                 & 0.5 [0.01-1]\footnote{0.01 is for immediate responses that could not be measured}                                                                                                                                                                                                                     \\ [1.0em] \hline \caption{Lag in JAWS Reporting HTML Content}\label{tab:table5} \\
\end{longtable}

\pagebreak \hypertarget{notes-on-future-proofing-laptops}{}\section{Laptop Cost Factors}\label{notes-on-future-proofing-laptops}
Students with visual impairments require laptops that are compatible with assistive technology to receive a free and accessible public education. When selecting a laptop, there are several cost factors that must be taken into account. Firstly, the laptop should have sufficient RAM to run the assistive technology software smoothly. Secondly, the laptop should have a high-quality screen that is large enough to display text and images clearly. Thirdly, the laptop should have a long battery life to ensure that students can use it for extended periods of time without needing to recharge. Fourthly, the laptop should have a durable build quality to withstand daily wear and tear. Finally, the laptop should be lightweight and portable to enable students to carry it around easily. These factors can help ensure that students with visual impairments have access to the technology they need to succeed in their studies.

\hypertarget{future-proofing}{}\section{Laptop Durability}\label{future-proofing}

I asked ChatGPT why we should consider future usage needs of a laptop into my purchasing plans and here is its response:

\begin{leftbar} \begin{quote}
		Buying laptops with high-end RAM and processor specifications can be beneficial for several reasons, especially in terms of longevity and performance. Here are some key considerations:

		\begin{enumerate}[leftmargin=*]
			\def\labelenumi{\arabic{enumi}.}

			\item
			      Future-Proofing:
		\end{enumerate}

		\begin{itemize}

			\item
			      Investing in higher-end components can help future-proof your laptop to some extent. As software and applications become more demanding over time, a laptop with better specifications is likely to remain relevant and capable of handling new software updates and requirements.
		\end{itemize}

		\begin{enumerate}[leftmargin=*]
			\def\labelenumi{\arabic{enumi}.}
			\setcounter{enumi}{1}

			\item
			      Multitasking and Performance:
		\end{enumerate}

		\begin{itemize}

			\item
			      Higher RAM allows for better multitasking capabilities. If you use resource-intensive applications or run multiple programs simultaneously, having ample RAM prevents slowdowns and ensures smooth performance.
			\item
			      A powerful processor can handle complex computations and tasks more efficiently, contributing to faster overall system performance.
		\end{itemize}

		\begin{enumerate}[leftmargin=*]
			\def\labelenumi{\arabic{enumi}.}
			\setcounter{enumi}{2}

			\item
			      Graphics Processing:
		\end{enumerate}

		\begin{itemize}

			\item
			      For tasks such as video editing, gaming, or other graphics-intensive applications, a high-end GPU (Graphics Processing Unit) is crucial. This is separate from RAM and the main processor but is equally important for certain tasks.
		\end{itemize}

		\begin{enumerate}[leftmargin=*]
			\def\labelenumi{\arabic{enumi}.}
			\setcounter{enumi}{3}

			\item
			      Long-Term Usability:
		\end{enumerate}

		\begin{itemize}

			\item
			      Technology advances quickly, and what is considered high-end today might become standard or entry-level in a few years. However, starting with a high-end laptop gives you a more extended period before your system becomes outdated.
		\end{itemize}

		\begin{enumerate}[leftmargin=*]
			\def\labelenumi{\arabic{enumi}.}
			\setcounter{enumi}{4}

			\item
			      Smooth Upgrades:
		\end{enumerate}

		\begin{itemize}

			\item
			      Some laptops allow users to upgrade certain components like RAM and storage. Starting with higher-end specifications provides more headroom for future upgrades, allowing you to extend the useful life of your laptop.
		\end{itemize}

		\begin{enumerate}[leftmargin=*]
			\def\labelenumi{\arabic{enumi}.}
			\setcounter{enumi}{5}

			\item
			      Resource-Intensive Applications:
		\end{enumerate}

		\begin{itemize}

			\item
			      If you use applications that require a lot of processing power, like video editing software, 3D modeling tools, or virtualization software, a high-end laptop can handle these tasks more efficiently, providing a better user experience.
		\end{itemize}

		\begin{enumerate}[leftmargin=*]
			\def\labelenumi{\arabic{enumi}.}
			\setcounter{enumi}{6}

			\item
			      Responsiveness and Speed:
		\end{enumerate}

		\begin{itemize}

			\item
			      A faster processor and more RAM contribute to a more responsive system. Applications load quicker, and the overall user experience is smoother, which can significantly impact productivity.
		\end{itemize}

		\begin{enumerate}[leftmargin=*]
			\def\labelenumi{\arabic{enumi}.}
			\setcounter{enumi}{7}

			\item
			      Enhanced Resale Value:
		\end{enumerate}

		\begin{itemize}

			\item
			      High-end laptops generally retain their value better than lower-end models. If you decide to sell or upgrade your laptop in the future, having top-tier specifications can make it more attractive to potential buyers.
		\end{itemize}

		It's important to note that the specific requirements depend on your usage patterns and needs. Not everyone needs the highest specifications, and sometimes a mid-range laptop can be more than sufficient for everyday tasks. However, for users with demanding workloads or those who want a longer usable life from their device, opting for higher-end specifications can be a wise investment.
	\end{quote}\end{leftbar}

\pagebreak	\hypertarget{optimizing-cost-with-performance}{}\section{Laptop Cost Optimization}\label{optimizing-cost-with-performance}

There is an economic theory based on a Terry Pratchett novel that explains this phenomenon better than we can. It is called the \href{https://en.wikipedia.org/wiki/Boots_theory}{Boots Theory}\footnote{Full Text (emphasis mine):
	\begin{leftbar}
		\begin{quote}The reason that the rich were so rich, Vimes reasoned, was because they managed to spend less money.

			Take boots, for example. He earned thirty-eight dollars a month plus allowances. A really good pair of leather boots cost fifty dollars. But an affordable pair of boots, which were sort of OK for a season or two and then leaked like hell when the cardboard gave out, cost about ten dollars. Those were the kind of boots Vimes always bought, and wore until the soles were so thin that he could tell where he was in Ankh-Morpork on a foggy night by the feel of the cobbles.

			But the thing was that good boots lasted for years and years. \textbf{A man who could afford fifty dollars had a pair of boots that’d still be keeping his feet dry in ten years’ time, while the poor man who could only afford cheap boots would have spent a hundred dollars on boots in the same time and would still have wet feet.}

			Basically, \textbf{we are destined to be stuck in a cycle of perpetually spending more money for inferior products and will, in the end, spend more money than if we just paid for better product in the first place.} -- \textit{Men At Arms}, page 38
		\end{quote}
	\end{leftbar} }

\hfill \break Table \ref{tab:table6} illustrates this theory in terms of student laptop computers (Assuming student has a laptop using a screenreader through 3rd-12th grade).

I put this Table \ref{tab:table6} here to illustrate why we choose to err on the side of spending \$2000-3000 now on a laptop that will last 3-5 years over spending \$1500-2000 on a laptop that will barely make it 1-2 years before becoming obsolete. By the end of 5 years we will have spent more on Low End and Mid Range Laptops than we would have had we purchased a High End Laptop. Importantly; however, we also would have been using laptops that always perform more poorly than a High End laptop would.

\hspace{-1cm} \begin{longtable}[]{@{}
	>{\raggedright\arraybackslash}m{.15\textwidth}
	>{\raggedright\arraybackslash}m{.05\textwidth}
	>{\raggedright\arraybackslash}m{.05\textwidth}
	>{\raggedright\arraybackslash}m{.05\textwidth}
	>{\raggedright\arraybackslash}m{.05\textwidth}
	>{\raggedright\arraybackslash}m{.05\textwidth}
	>{\raggedright\arraybackslash}m{.05\textwidth}
	>{\raggedright\arraybackslash}m{.05\textwidth}
	>{\raggedright\arraybackslash}m{.05\textwidth}
	>{\raggedright\arraybackslash}m{.05\textwidth}
	>{\raggedright\arraybackslash}m{.05\textwidth}
	>{\raggedright\arraybackslash}b{.10\textwidth}@{}
	}
	\toprule                                                                                                                                                                                         &
	\multicolumn{10}{c}{\textbf{Does School Have to Purchase a Replacement Laptop by Year}}                                                                                                          &                                                                                                                                                                             \\[1.0em]
	\cline{2-11}                                                                                                                                                                                                                                                                                                                                                                   \\
	\textbf{RAM} \break Cost                                                                                                                                                                         & \textbf{1}   & \textbf{2}   & \textbf{3}   & \textbf{4}   & \textbf{5}   & \textbf{6}   & \textbf{7}   & \textbf{8}   & \textbf{9}   & \textbf{10}  & \textbf{10 year Cost} \\
	\midrule
	\endhead \hline                                                                                                                                                                                                                                                                                                                                                                \\
	\multicolumn{6}{r}{\textbf{Continued on Next Page}} \endfoot
	\endlastfoot
	\textbf{4GB}\footnote{Dell Latitude 3190 Education}\fnsep\footnote{The 4GB Laptop cannot run JAWS and is included to show price comparison to the other options} \break \$525.04                 & $\checkmark$ & $\checkmark$ & $\checkmark$ & $\checkmark$ & $\checkmark$ & $\checkmark$ & $\checkmark$ & $\checkmark$ & $\checkmark$ & $\checkmark$ & \$5,250               \\[1.0em]
	\textbf{8GB} \break \$1184\footnote{Dell Latitude 3190 Education}                                                                                                                                & $\checkmark$ & $\checkmark$ & $\checkmark$ & $\checkmark$ & $\checkmark$ & $\checkmark$ & $\checkmark$ & $\checkmark$ & $\checkmark$ & $\checkmark$ & \$11,840              \\[1.0em]
	\textbf{16GB} \break \$1751\footnote{Dell Precision 3530 given to TVIs teaching screenreaders}                                                                                                   & $\checkmark$ & -            & $\checkmark$ & -            & $\checkmark$ & -            & $\checkmark$ & -            & $\checkmark$ & -            & \$8,755               \\[1.0em]
	\hdashline[0.5pt/5pt]                                                                                                                                                                                                                                                                                                                                                          \\
	\textbf{32GB}\break \$2824\footnote{Microsoft Surface Laptop 3}\break \textcolor{red}{Best Case}\footnote{This is my personal experience}                                                        & $\checkmark$ & -            & -            & -            & -            & $\checkmark$ & -            & -            & -            & -            & \$5,648               \\[1.0em] \\

	\textbf{32GB}\break \$2824\footnote{Microsoft Surface Laptop 3}\break \textcolor{red}{Cautious}\footnote{This is a conservative estimate to account for potential rough treatment of a computer} & $\checkmark$ & -            & -            & $\checkmark$ & -            & -            & $\checkmark$ & -            & -            & -            & \$8,472               \\[1.0em] \hline


	\caption[Cost of Laptops over Time]{Cost of Laptops Across Time. Notice that the final cost of the 32GB option is comparable to the 4GB over 10 years. However, the 4GB laptop is not capable of running JAWS reliably in the classroom setting.
		\begin{itemize}[leftmargin=6em]
			\item For the \textcolor{red}{Best Case} Scenario, the 32GB laptop is between \$3,107 and \$6,192 \textit{\textbf{cheaper}} over time compared to the 16GB and 8GB laptops, respectively.
			\item For the \textcolor{red}{Cautious} Scenario, the 32GB laptop is between \$283 and \$3,386 \textit{\textbf{cheaper}} over time compared to the 16GB and 8GB laptops, respectively.
		\end{itemize}}\label{tab:table6}
\end{longtable}

\pagebreak
\hypertarget{minimum-laptop-recommendations}{}\section{Recommended Laptop Specifications}\label{minimum-laptop-recommendations}
Table \ref{tab:table7} is a list of recommendations for laptop specifications by use case.

\begin{longtable}[]{@{}
	>{\raggedright\arraybackslash}m{.7\textwidth}
	>{\raggedright\arraybackslash}b{.3\textwidth}@{}
	}
	\toprule

	\textbf{Use Case}                                                                                                                                                                                                                                                                & \textbf{Recommendation}      \\
	\midrule
	\endhead \hline                                                                                                                                                                                                                                                                                                 \\
	\multicolumn{2}{r}{\textbf{Continued on Next Page}} \endfoot
	\endlastfoot
	\multicolumn{2}{c}{\textbf{\large Screenreader Only}\normalfont}                                                                                                                                                                                                                                                \\[2em]
	JAWS Screenreader\footnote{Although NVDA, ZDSR, and Dolphin Screenreader have been widely available for \textgreater10 years, JAWS is considered the industry standard and is the one primarily supported by Industry in the United States as well as Vocational Rehabilitation} & \textgreater\textbf{24-32GB} \\[1.0em]
	NVDA Screenreader                                                                                                                                                                                                                                                                & \textgreater\textbf{24-32GB} \\[1.0em]
	Dolphin Screenreader                                                                                                                                                                                                                                                             & \textgreater\textbf{24-32GB} \\[1.0em]
	ZDSR Screenreader                                                                                                                                                                                                                                                                & \textgreater\textbf{24-32GB} \\[1.0em]
	\multicolumn{2}{c}{\textbf{\large  Screen Magnification Only}\normalfont}                                                                                                                                                                                                                                       \\[2em]
	ZoomText                                                                                                                                                                                                                                                                         & \textgreater\textbf{24-32GB} \\[1.0em]
	Windows Magnifier                                                                                                                                                                                                                                                                & \textgreater\textbf{16GB}    \\[1.0em]
	Dolphin SuperNova                                                                                                                                                                                                                                                                & \textgreater\textbf{24-32GB} \\[1.0em]
	\multicolumn{2}{c}{\textbf{\large Screenreader + Magnification}\normalfont}                                                                                                                                                                                                                                     \\[2em]
	JAWS Screenreader + Windows Magnifier                                                                                                                                                                                                                                            & \textgreater\textbf{24-32GB} \\[1.0em]
	NVDA Screenreader + Windows Magnifier                                                                                                                                                                                                                                            & \textgreater\textbf{24-32GB} \\[1.0em]
	ZDSR Screenreader + Windows Magnifier                                                                                                                                                                                                                                            & \textgreater\textbf{24-32GB} \\[1.0em]
	Dolphin Screenreader + Windows Magnifier                                                                                                                                                                                                                                         & \textgreater\textbf{24-32GB} \\[1.0em]
	SuperNova Screenreader + Magnification                                                                                                                                                                                                                                           & \textgreater\textbf{32-64GB} \\[1.0em]
	Fusion Screenreader + Magnification                                                                                                                                                                                                                                              & \textgreater\textbf{32-64GB} \\[1.0em] \hline
	\caption{Recommended Laptop Specifications}\label{tab:table7}
\end{longtable}

\pagebreak
\hypertarget{laptops-meeting-recommended-specifications}{}\section{Laptops Meeting Specifications}\label{laptops-meeting-recommended-specifications}
Table \ref{tab:table8} is an alphabetical list of laptop computers that meet the specifications defined in Table \ref{tab:table7}.
\begin{longtable}[]{@{}
	>{\raggedright\arraybackslash}m{.3\textwidth}
	>{\raggedright\arraybackslash}m{.12\textwidth}
	>{\raggedright\arraybackslash}m{.12\textwidth}
	>{\raggedright\arraybackslash}m{.12\textwidth}
	>{\raggedright\arraybackslash}m{.12\textwidth}
	>{\raggedright\arraybackslash}b{.2\textwidth}@{}
	}
	\toprule

	\textbf{Company / Model}                                                                                    & \textbf{Cost}                                                                                                                                   & \textbf{Keyboard}      & \textbf{RAM} & \textbf{Screen Size} & \textbf{Processor} \\
	\midrule
	\endhead \hline                                                                                                                                                                                                                                                                                                                                   \\
	\multicolumn{6}{r}{\textbf{Continued on Next Page}} \endfoot
	\endlastfoot
	Acer Predator Helios 16                                                                                     & \$2,499                                                                                                                                         & QWERTY                 & 32GB         & 16.0''               & 13th Gen i9        \\[1.0em]
	Acer Predator Triton                                                                                        & \$3,799                                                                                                                                         & QWERTY                 & 64GB         & 17.0''               & 13th Gen i9        \\[1.0em]
	Alienware m16                                                                                               & \$3,499                                                                                                                                         & QWERTY                 & 32GB         & 16.0''               & 13th Gen 19        \\[1.0em]
	Alienware x14                                                                                               & \$1,999                                                                                                                                         & QWERTY                 & 32GB         & 14.0''               & 13th Gen i7        \\[1.0em]
	Asus ProArt Studiobook                                                                                      & \$2,999                                                                                                                                         & QWERTY                 & 32GB         & 16.0''               & 13th Gen i9        \\[1.0em]
	Asus Zenbook Pro 16X                                                                                        & \$2,599                                                                                                                                         & QWERTY                 & 32GB         & 16.0''               & 13th Gen i9        \\[1.0em]
	Dell G16 Gaming Laptop                                                                                      & \$1,999                                                                                                                                         & QWERTY                 & 32GB         & 16.0''               & 13th Gen i7        \\[1.0em]
	Dell Inspiron 16 Plus                                                                                       & \$1,499                                                                                                                                         & QWERTY                 & 32GB         & 16.0''               & 13th Gen i7        \\[1.0em]
	Dell Precision 3480                                                                                         & \$3,205                                                                                                                                         & QWERTY                 & 32GB         & 14.0''               & 13th Gen i7        \\[1.0em]
	Dell Precision 3581                                                                                         & \$3,854                                                                                                                                         & QWERTY                 & 32GB         & 15.6''               & 13th Gen i7        \\[1.0em]
	Dell Precision 5480                                                                                         & \$4,354                                                                                                                                         & QWERTY                 & 32GB         & 14.0''               & 13th Gen i7        \\[1.0em]
	Dell Precision 5770                                                                                         & \$2,789                                                                                                                                         & QWERTY                 & 32GB         & 17''                 & 12th Gen i7        \\[1.0em]
	Dell XPS 13 Plus                                                                                            & \$2,009                                                                                                                                         & QWERTY                 & 32GB         & 13.4''               & 13th Gen i7        \\[1.0em]
	Dell XPS 15                                                                                                 & \$2,099                                                                                                                                         & QWERTY                 & 32GB         & 15.6''               & 12th Gen i7        \\[1.0em]
	Dell XPS 15                                                                                                 & \$2,349                                                                                                                                         & QWERTY                 & 32GB         & 15.6''               & 12th Gen i9        \\[1.0em]
	Dell XPS 15                                                                                                 & \$2,999                                                                                                                                         & QWERTY                 & 32GB         & 15.6''               & 13th Gen i9        \\[1.0em]
	Dell XPS 17                                                                                                 & \$3,349                                                                                                                                         & QWERTY                 & 32GB         & 15.6''               & 13th Gen i7        \\[1.0em]
	Dell XPS 17                                                                                                 & \$3,549                                                                                                                                         & QWERTY                 & 32GB         & 15.6''               & 13th Gen i9        \\[1.0em]
	Dell latitude 7440                                                                                          & \$3,615                                                                                                                                         & QWERTY                 & 32GB         & 14.0''               & 13th Gen i7        \\[1.0em]
	Dell precision 5680                                                                                         & \$5,597                                                                                                                                         & QWERTY                 & 32GB         & 16.0''               & 13th Gen i9        \\[1.0em]
	Dell precision 7680                                                                                         & \$7,225                                                                                                                                         & QWERTY                 & 32GB         & 16.0''               & 13th Gen i9        \\[1.0em]
	Framework 13                                                                                                & \$1,102                                                                                                                                         & QWERTY                 & 32GB         & 13.5''               & 13th Gen i7        \\[1.0em]
	Framework 13                                                                                                & \$1,732                                                                                                                                         & QWERTY                 & 32GB         & 13.5''               & AMD Ryzen 7        \\[1.0em]
	Framework 13                                                                                                & \$1,892                                                                                                                                         & QWERTY                 & 64GB         & 13.5''               & AMD Ryzen 7        \\[1.0em]
	Framework 13                                                                                                & \$2,222                                                                                                                                         & QWERTY                 & 64GB         & 13.5''               & 13th Gen i7        \\[1.0em]
	Framework 16                                                                                                & \$2,239                                                                                                                                         & QWERTY\break (+NUMPAD) & 32GB         & 16.0''               & AMD Ryzen 9        \\[1.0em]
	Framework 16                                                                                                & \$2,399                                                                                                                                         & QWERTY\break (+NUMPAD) & 64GB         & 16.0''               & AMD Ryzen 9        \\[1.0em]
	Framework 16\break (+ AMD GPU)                                                                              & \$2,639                                                                                                                                         & QWERTY\break (+NUMPAD) & 32GB         & 16.0''               & AMD Ryzen 9        \\[1.0em]
	Framework 16\break (+ AMD GPU)                                                                              & \$2,799                                                                                                                                         & QWERTY\break (+NUMPAD) & 64GB         & 16.0''               & AMD Ryzen 9        \\[1.0em]
	HP Dragonfly Pro                                                                                            & \$1,549                                                                                                                                         & QWERTY                 & 32GB         & 14.0''               & AMD Ryzen 7        \\[1.0em]
	HP Envy                                                                                                     & \$1,749                                                                                                                                         & QWERTY                 & 32GB         & 17.3''               & 13th Gen i7        \\[1.0em]
	Legion Pro 7i                                                                                               & \$3,599                                                                                                                                         & QWERTY                 & 32GB         & 16.0''               & 13th Gen i9        \\[1.0em]
	Lenovo ThinkPad P16 Gen 2                                                                                   & \$2,039                                                                                                                                         & QWERTY                 & 32GB         & 16.0''               & AMD Ryzen 7        \\[1.0em]
	Lenovo ThinkPad P16 Gen 2                                                                                   & \$2,829                                                                                                                                         & QWERTY                 & 64GB         & 16.0''               & AMD Ryzen 7        \\[1.0em]
	Lenovo ThinkPad P16 Gen 2                                                                                   & \$3,239                                                                                                                                         & QWERTY                 & 32GB         & 16.0''               & 13th Gen i7        \\[1.0em]
	Lenovo ThinkPad P16 Gen 2                                                                                   & \$4,189                                                                                                                                         & QWERTY                 & 64GB         & 16.0''               & 13th Gen i7        \\[1.0em]
	Lenovo ThinkPad P16 Gen 2                                                                                   & \$4,829                                                                                                                                         & QWERTY                 & 64GB         & 16.0''               & 13th Gen i7        \\[1.0em]
	Lenovo ThinkPad P16v                                                                                        & \$3,339                                                                                                                                         & QWERTY                 & 32GB         & 16.0''               & 13th Gen i7        \\[1.0em]
	Lenovo ThinkPad P16v                                                                                        & \$4,929                                                                                                                                         & QWERTY                 & 64GB         & 16.0''               & 13th Gen i7        \\[1.0em]
	Lenovo Thinkpad P14s                                                                                        & \$2,199                                                                                                                                         & QWERTY                 & 32GB         & 14.0''               & AND Ryzen 7        \\[1.0em]
	Lenovo Thinkpad P14s                                                                                        & \$2,509                                                                                                                                         & QWERTY                 & 64GB         & 14.0''               & AND Ryzen 7        \\[1.0em]
	Lenovo Thinkpad X1 Yoga                                                                                     & \$3,719                                                                                                                                         & QWERTY                 & 32GB         & 14.0''               & 13th Gen i7        \\[1.0em]
	MSI Vector                                                                                                  & \$1,999                                                                                                                                         & QWERTY                 & 64GB         & 14.0''               & 13th Gen i9        \\[1.0em]
	Microsoft Surface Laptop 5                                                                                  & \$2824                                                                                                                                          & QWERTY                 & 32GB         & 15.0''               & 12th Gen i7        \\[1.0em]
	Notey the Notetaker                                                                                         & \textasciitilde\$750+\footnote{Self build \href{https://notey-project.com/2023/03/07/notey-user-manual-v1-0-2/}{Specs for Notey the NoteTaker}} & QWERTY \break Perkins  & 8GB-16GB     & none                 & Celeron N5105      \\ [1.0em]
	Orbit Optima\footnote{This is a Windows Computer without a monitor, substituting a 40 cell braille display} & \$6,500                                                                                                                                         & QWERTY                 & 32GB-64GB    & none                 & 13th Gen i7        \\ [1.0em]
	Seika Studio\footnote{This is a Windows Computer without a monitor, substituting a 40 cell braille display} & \$6,500                                                                                                                                         & QWERTY                 & 8GB-16GB     & none                 & 12th Gen i7        \\ [1.0em]\hline
	\caption{Laptop Computers Meeting Recommended Specifications}\label{tab:table8}
\end{longtable}

\pagebreak \hypertarget{ios-devices}{}\chapter[\raggedright Transformative Tablets: \\Pioneering Success for Visually Impaired Students Through Innovative Apps]{Transformative Tablets: Pioneering Success for Visually Impaired Students Through Innovative Apps}\label{ios-devices}
\minitoc
\extramarks{Vision Department Technology Needs}{Transformative Tablets}
In an era where technology shapes the landscape of education, tablets have emerged as transformative tools, providing visually impaired students with unprecedented access to knowledge and fostering independence in their academic journeys\footnote{I am omitting iPhones and Android phones from this document as the purchase of student phones is beyond the purview of a scool district}. Within the realm of tablets, both iPad and Android devices stand as beacons of innovation, offering not only user-friendly interfaces but also a diverse array of applications specifically tailored to bridge the accessibility gap. This chapter embarks on a compelling exploration of how tablets, in tandem with purpose-built apps, are not just tools but catalysts for success in the educational odyssey of visually impaired students.

The tactile elegance of tablets goes beyond mere convenience; it represents a paradigm shift in the way students interact with information. For visually impaired learners, tablets serve as dynamic portals, offering a multi-sensory approach to engagement. Through the lens of this chapter, we will unravel the unique functionalities of both iPad and Android tablets, delving into their respective strengths and contributions to an inclusive educational experience.

Apps, the lifeblood of these devices, play an instrumental role in transforming tablets into personalized learning companions. From screen readers that convert text to speech with remarkable precision to magnification apps that enhance visual content, the ecosystem of applications available empowers visually impaired students to navigate the digital realm with confidence. Tablets, coupled with innovative apps, are not mere gadgets; they represent a dynamic force propelling visually impaired students toward success; underscoring the indispensable role these tools play in shaping an educational landscape where every student, regardless of visual abilities, can seize the opportunities that lie ahead.

\hypertarget{tablet-considerations}{}\section{Tablet Considerations}\label{tab:tablelet-considerations}

When selecting a tablet for students with visual impairments to access their schoolwork, careful consideration must be given to the device's accessibility features to ensure an inclusive and conducive learning environment. Essential considerations include the tablet's compatibility with screen readers and magnification tools, ensuring that these assistive technologies seamlessly integrate with the device's operating system. Additionally, evaluating the availability and effectiveness of built-in accessibility features such as VoiceOver for iOS or TalkBack for Android is crucial\footnote{Traditionally, assistive technology for the blind has focused on the iPad line since the Android line had historically lagged behind the Apple products for accessibility features prior to 2020 so accessibility apps have favored the iOS/iPadOS architecture. However, groups are emerging that specifically instruct users of Android devices how to access and use accessibility settings \textit{cf.}, \href{https://www.youtube.com/channel/UCvEM-SmpwElNALldhp8hG1g}{Blind Android Users}}. The tablet's tactile features, size, and weight should also be taken into account to accommodate students' specific needs. High contrast and customizable color settings, as well as text-to-speech functionalities, are vital components that enhance readability. Furthermore, the tablet's compatibility with a variety of educational apps designed with accessibility in mind is paramount. By prioritizing these considerations, educators and administrators can empower students with visual impairments to engage with their schoolwork independently and efficiently, fostering a more inclusive learning experience.

One frequently overlooked challenge in using tablets for individuals with visual impairments is the potential for visual fatigue. Recent research, such as the study by Pakdee and Sengsoon (2021)\footnote{\textit{cf}., \href{https://www.researchgate.net/publication/352764109_Immediate_Effects_of_Different_Screen_Sizes_on_Visual_Fatigue_in_Video_Display_Terminal_Users}{Pakdee, S., \& Sengsoon, P. (2021). Immediate Effects of Different Screen Sizes on Visual Fatigue in Video Display Terminal Users. \textit{Iranian Rehabilitation Journal, 19(2)}, 137-1461. DOI:10.32598/irj.19.2.1108.2}}, reveals that opting for a slightly larger device can mitigate visual fatigue, particularly for those engaged in visually demanding tasks. This consideration becomes even more pertinent for individuals with visual impairments.

While the iPad Pro2 and Samsung Galaxy Tab 9 tablets are often lauded for their increased brightness, it's crucial not to prioritize brightness as a major factor. Research suggests that boosting brightness can exacerbate visual fatigue. Instead, emphasis should be placed on the larger screen's enhanced resolution and expanded visual area, facilitating efficient use of Zoom functions. This becomes especially significant when aiming to teach students to adeptly navigate assistive technology without relying solely on pinch zooming, a feature that may not consistently function within all applications.

Another critical consideration involves contrast ratios. For students with photophobia, adjusting luminance levels to lower settings can significantly enhance clarity of text and images. This nuanced approach to tablet selection is pivotal in creating an accessible and comfortable learning environment for students with visual impairments.

For individuals with visual impairments, the efficacy of these devices relies heavily on factors that go beyond mere functionality. One crucial aspect that significantly impacts the accessibility of tablets for visually impaired students is the contrast ratio. The contrast ratio, representing the difference in luminance between the brightest and darkest elements on a screen, plays a pivotal role in ensuring that individuals with visual impairments can effectively engage with educational content. In a school setting, where tablets are increasingly utilized for various learning activities, understanding and prioritizing contrast ratio becomes paramount in fostering an inclusive and enriching educational environment for all students, regardless of their visual abilities.


\pagebreak	\hypertarget{tablet-options}{}\section{Tablet Options}\label{tab:tablelet-options}
Table \ref{tab:table9} describes current tablet computers that are vailable for students with visual impairments.

\begin{longtable}[]{@{}
	>{\raggedright\arraybackslash}m{.3\textwidth}
	>{\raggedright\arraybackslash}m{.1\textwidth}
	>{\raggedright\arraybackslash}m{.1\textwidth}
	>{\raggedright\arraybackslash}m{.2\textwidth}@{}
	>{\raggedright\arraybackslash}b{.3\textwidth}@{}
	}
	\toprule

	\textbf{Tablet}             & \textbf{Cost} & \textbf{Screen}                                                                                                                                                                                                                                          & \textbf{Contrast Ratio}                                                                                                                                                                                                                    \\
	\midrule
	\endhead \hline                                                                                                                                                                                                                                                                                                                                                                                                                                                                                                                                     \\
	\multicolumn{3}{r}{\textbf{Continued on Next Page}} \endfoot
	\endlastfoot
	iPad                        & \$599         & 10.9''                                                                                                                                                                                                                                                   & 1,417:1                                                                                                                                                                                                                                    \\[1.0em]
	iPad Air                    & \$749         & 10.9''                                                                                                                                                                                                                                                   & 1,417:1                                                                                                                                                                                                                                    \\[1.0em]
	iPad Pro                    & \$1,099       & 12.9''\footnote{The larger size of the iPad Pro2 screen provides students with access to using Zoom functions that allow student to use the Window Zoom function to magnify portions of the screen while maintaining the overall view}                   & 1,000,000:1\footnote{The higher contrast ratio allows students to pinch-zoom material to a larger overall size prior to having the material pixelate or blur (note, using the Zoom function re-flows fonts and prevents blurring of fonts} \\[1.0em]
	Samsung Galaxy Tab S9       & \$799         & 11''                                                                                                                                                                                                                                                     & 1,417:1                                                                                                                                                                                                                                    \\[1.0em]
	Samsung Galaxy Tab S9 Plus  & \$999         & 11''                                                                                                                                                                                                                                                     & 1,500:1                                                                                                                                                                                                                                    \\
	Samsung Galaxy Tab S9 Ultra & \$1,099       & 12.4''\footnote{The larger size of the Samsung Galaxy Tab S9 Ultra screen provides students with access to using Zoom functions that allow student to use the Window Zoom function to magnify portions of the screen while maintaining the overall view} & 1,000,000 to 1                                                                                                                                                                                                                             \\[1.0em]
	OnePlus Pad                 & \$480         & 11.6''                                                                                                                                                                                                                                                   & 1195:1                                                                                                                                                                                                                                     \\[1.0em]
	Microsoft Surface Go 3      & \$450         & 10.5''                                                                                                                                                                                                                                                   & 961:1                                                                                                                                                                                                                                      \\[1.0em]
	Google Pixel Tablet         & \$499         & 10.95''                                                                                                                                                                                                                                                  & 1500:1                                                                                                                                                                                                                                     \\[1.0em]
	Lenovo Yoga Tab 13          & \$1,165       & 13''                                                                                                                                                                                                                                                     & 1000:1                                                                                                                                                                                                                                     \\[1.0em]\hline
	\caption[Recommended Tablet Specifications]{Recommended Tablet Specifications}\label{tab:table9}
\end{longtable}


\pagebreak
\hypertarget{tablet-apps}{}\section{Mobile Applications}\label{tab:tablelet-apps}
Mobile apps run on tablets are becoming increasingly important for students with visual impairments to access a free and appropriate public education. These apps can provide students with access to digital content, assistive technology, and other tools that can help them succeed in their studies. High-quality mobile apps can help students with visual impairments access the same educational materials as their sighted peers and participate fully in the curriculum. They can also help improve literacy skills, comprehension, and productivity. In this section, we will explore the importance of high-quality mobile apps for students with visual impairments and discuss some of the best apps available on the market today. Table \ref{tab:table10} gives a list of current apps available for use with students with visual impairments.
\begin{longtable}[]{@{}
	>{\raggedright\arraybackslash}m{.3\textwidth}
	>{\raggedright\arraybackslash}m{.1\textwidth}
	>{\raggedright\arraybackslash}m{.3\textwidth}@{}
	>{\raggedright\arraybackslash}b{.3\textwidth}@{}
	}
	\toprule

	\textbf{App}                               & \textbf{Cost}                                                                                & \textbf{Function}                                              & \textbf{OS}                     \\
	\midrule
	\endhead \hline                                                                                                                                                                                                                              \\
	\multicolumn{3}{r}{\textbf{Continued on Next Page}} \endfoot
	\endlastfoot
	\multicolumn{4}{c}{\textbf{\large Accessibility Training\normalfont}}                                                                                                                                                                        \\[1em]
	Ballyland Rotor                            & \$2.99                                                                                       & Train VoiceOver rotor                                          & iOS/iPadOS                      \\[1.0em]
	Ballyland Magic Plus                       & \$3.99                                                                                       & Train VoiceOver Gestures                                       & iOS/iPadOS                      \\[1.0em]
	VO Lab                                     & \$4.99                                                                                       & Train VoiceOver Gestures\break (meant for adolescents)         & iOS/iPadOS                      \\[1.0em]
	\multicolumn{4}{c}{\textbf{\large Cortical Vision Impairment\normalfont}}                                                                                                                                                                    \\[1em]
	Sensory Light Box                          & free                                                                                         & Gaze training                                                  & iOS/iPadOS                      \\[1.0em]
	CVI Connect Pro                            & free                                                                                         & CVI-based Vision Training                                      & iOS/iPadOS                      \\[1.0em]
	\multicolumn{4}{c}{\textbf{\large Audiobook/Reading\normalfont}}                                                                                                                                                                             \\[1em]
	Dolphin EZ Reader                          & free                                                                                         & DAISY Reader                                                   & iOS/iPadOS\break AndroidOS 13+  \\[1.0em]
	VoiceDream Reader                          & free\footnote{requires \$79.99/yr subscription and additional \$4.99 for each premium voice} & DAISY Reader                                                   & iOS/iPadOS                      \\[1.0em]
	Bookshare Reader                           & free                                                                                         & DAISY Reader                                                   & iOS/iPadOS                      \\[1.0em]
	Kindle                                     & free\footnote{requires books to be purchased}                                                & e-Book                                                         & iOS/iPadOS\break AndroidOS 13+  \\[1.0em]
	Libby                                      & free\footnote{requires a library membership}                                                 & Audiobook                                                      & iOS/iPadOS\break AndroidOS 13+  \\[1.0em]
	Audible                                    & free\footnote{requires books to be purchased}                                                & Audiobook\                                                     & iOS/iPadOS\break AndroidOS 13+  \\[1.0em]
	BARD Mobile                                & free\footnote{requires account with local affiliate State Library for the Blind}             & e-Book                                                         & iOS/iPadOS\break AndroidOS 13+  \\[1.0em]
	KNFB Reader\break(rebranded OneStepReader) & \$99.99                                                                                      & OCR/Reading                                                    & iOS/iPadOS\break AndroidOS 13+  \\[1.0em]
	Audible                                    & free\footnote{requires books to be purchased}                                                & Audiobook                                                      & iOS/iPadOS\break AndroidOS 13+  \\[1.0em]
	\multicolumn{4}{c}{\textbf{\large Productivity/Schoolwork\normalfont}}                                                                                                                                                                       \\[1em]
	Notability                                 & free\footnote{with in app puchases}                                                          & Scan \& Markup Documents                                       & iOS/iPadOS \break AndroidOS 13+ \\[1.0em]
	GoodNotes                                  & free\footnote{with in app puchases}                                                          & Scan \& Markup Documents                                       & iOS/iPadOS \break AndroidOS 13+ \\[1.0em]
	My Board Buddy                             & free                                                                                         & all the things                                                 & iOS/iPadOS                      \\[1.0em]
	Desmos Graphing Calculator                 & free                                                                                         & all the things                                                 & iOS/iPadOS \break AndroidOS 13+ \\[1.0em]
	Desmos Scientific Calculator               & free                                                                                         & all the things                                                 & iOS/iPadOS \break AndroidOS 13+ \\[1.0em]
	\multicolumn{4}{c}{\textbf{\large Optical Character Recognition\normalfont}}                                                                                                                                                                 \\[1em]
	KNFB Reader\break(rebranded OneStepReader) & \$99.99                                                                                      & OCR/Reading                                                    & iOS/iPadOS\break AndroidOS 13+  \\[1.0em]
	SeeingAI                                   & free                                                                                         & Talking Camera\break OCR                                       & iOS/iPadOS\break AndroidOS 13+  \\[1.0em]
	TapTapSee                                  & free                                                                                         & Talking Camera\break OCR                                       & iOS/iPadOS\break AndroidOS 13+  \\[1.0em]
	\multicolumn{4}{c}{\textbf{\large Orientation \& Mobility / Navigation\normalfont}}                                                                                                                                                          \\[1em]
	HapticNav                                  & free\footnote{requires in app puchases}                                                      & Haptic GPS navigation                                          & iOS/iPadOS\break AndroidOS 13+  \\[1.0em]
	BlindSquare                                & \$39.99                                                                                      & GPS Navigation                                                 & iOS/iPadOS                      \\[1.0em]
	XploreNinja                                & \$39.99                                                                                      & GPS Navigation                                                 & AndroidOS 13+                   \\[1.0em]
	WeWalk                                     & free\footnote{requires in app puchases}                                                      & GPS Navigation                                                 & iOS/iPadOS                      \\[1.0em]
	Google Maps                                & free                                                                                         & Turn by Turn GPS Navigation                                    & iOS/iPadOS\break AndroidOS 13+  \\[1.0em]
	Apple Maps                                 & free                                                                                         & Turn by Turn GPS Navigation                                    & iOS/iPadOS                      \\[1.0em]
	Moovit                                     & free\footnote{requires in app puchases}                                                      & Local Public Transit                                           & iOS/iPadOS\break AndroidOS 13+  \\[1.0em]
	Waymap                                     & free                                                                                         & Turn by Turn GPS Navigation\break Focuses on indoor navigation & iOS/iPadOS\break AndroidOS 13+  \\[1.0em]
	Seeing Eye GPS                             & free\footnote{requires in app puchases}                                                      & Turn by Turn GPS Navigation                                    & iOS/iPadOS\break AndroidOS 13+  \\[1.0em]\hline
	\caption[Mobile/Tablet Apps]{Mobile/Tablet Apps}\label{tab:table10}
\end{longtable}

\pagebreak
\hypertarget{braille-first-devices}{}\chapter[\raggedright Bridging Literacy: \\The Crucial Role of Refreshable Braille Displays in Empowering Visually Impaired Students]{Bridging Literacy: The Crucial Role of Refreshable Braille Displays in Empowering Visually Impaired Students}\label{braille-first-devices}
\minitoc
\extramarks{Vision Department Technology Needs}{Bridging Literacy}
In the intricate tapestry of education, the pursuit of literacy is a fundamental thread, weaving through the academic journey of every student. For visually impaired learners, the path to literacy takes on a unique character, one in which the tactile elegance of braille becomes a vital conduit to knowledge. Within this narrative, refreshable braille displays emerge as indispensable companions, unlocking the doors to literacy, fostering engagement, and propelling students toward academic success. This chapter embarks on a compelling exploration of how refreshable braille displays are not merely tools but keystones in the quest for literacy and educational achievement among visually impaired students.

At the heart of this exploration lies the transformative power of refreshable braille displays—a technological marvel that seamlessly integrates the tactile richness of braille with the dynamic capabilities of digital communication. This chapter delves into the ergonomic design and sophisticated functionalities of these devices, spotlighting their pivotal role in ensuring that visually impaired students not only read but actively participate in the discourse of knowledge acquisition.

Refreshable braille displays play a dual role in the educational narrative of visually impaired students. Firstly, they serve as conduits for accessing textual content, enabling the exploration of literature, textbooks, and diverse educational materials in a format that aligns with the tactile language of braille. Secondly, and perhaps more profoundly, these devices empower students to actively contribute to the discourse, facilitating note-taking, writing, and engaging in classroom discussions with the same spontaneity and fluency as their sighted peers.

By providing visually impaired students with the means to interact with written information independently and dynamically, these devices emerge not just as tools but as instruments of empowerment, fostering a sense of agency and paving the way for academic success in the rich landscape of education.

\pagebreak
\hypertarget{braille-notetakers-and-braille-laptop-computers}{}\section{Braille Notetakers and Laptops}\label{braille-notetakers-and-braille-laptop-computers}

Braille notetakers such as the BrailleSense6 and BrailleNote Touch Plus are essential tools for students with visual impairments to access their schoolwork and receive a free and accessible public education. These devices are small and portable, allowing students to take notes in class using either braille or standard (QWERTY) keyboard, or both. They can also be used to read books, write class assignments, find directions, record lectures, and listen to podcasts 1. The notes written on these devices can be transferred to a computer for storage or printed in either braille or print formats. Many note-taking devices have word processors, appointment calendars, calculators or clocks, and can do almost everything a computer can do. Some note-taking devices have a speech program with braille input. Many newer models are Bluetooth accessible which allows them to be used with iPads, iPhones and other Bluetooth devices as well as Wi-Fi access Braille notetakers are useful not only for note taking in class, but also for composing and printing essays, writing notes, sending e-mails, or browsing the Internet These devices can give students who are blind or have low vision support in all academic areas as well as in expanded core curriculum. By providing students with visual impairments access to braille notetakers, we can help ensure that they have the tools they need to succeed in their studies and beyond. Table \ref{tab:table11} gives the specs for currently available braille notetakers.

\pagebreak\begin{longtable}[]{@{}
	>{\raggedright\arraybackslash}m{.25\textwidth}
	>{\raggedright\arraybackslash}m{.15\textwidth}
	>{\raggedright\arraybackslash}m{.15\textwidth}
	>{\raggedright\arraybackslash}m{.15\textwidth}
	>{\raggedright\arraybackslash}m{.15\textwidth}
	>{\raggedright\arraybackslash}b{.15\textwidth}@{}
	}
	\toprule

	\textbf{Display}                                                                                                                                                                                                                                             & \textbf{Cost}                                                                                                             & \textbf{Battery} & \textbf{Keyboard} & \textbf{Manufacturer} & \textbf{OS}                                                                                                                                                                                                                                                                                                                                                                                       \\
	\midrule
	\endhead \hline                                                                                                                                                                                                                                                                                                                                                                                                                                                                                                                                                                                                                                                                                                                                                                                                                                             \\
	\multicolumn{6}{r}{\textbf{Continued on Next Page}} \endfoot
	\endlastfoot
	Orbit Optima\footnote{This is a Windows Computer without a monitor that runs on any screenreader, but it has settings to function like a braille notetaker}                                                                                                  & \$6,000                                                                                                                   & TBD              & QWERTY            & Orbit Research        & Windows 11                                                                                                                                                                                                                                                                                                                                                                                        \\[1.0em]
	ElBraille 40\footnote{This is a Windows Computer without a monitor that runs on JAWS, but it has settings to function like a braille notetaker}\fnsep\footnote{This is not included above as a laptop option since it has only 4GB of RAM}                   & \$6,000\footnote{This price is for the ElBraille unit itself as well as a Focus 40 that docks into the unit as a display} & TBD              & QWERTY            & Elita                 & Windows 10\footnote{Windows 11 is not yet officially supported, but users are updating to Windows 11 without issue}                                                                                                                                                                                                                                                                               \\[1.0em]
	Seika Studio\footnote{This is a Windows Computer without a monitor, but it has settings to function like a braille notetaker}                                                                                                                                & \$6,500                                                                                                                   & TBD              & QWERTY            & Nippon Telesoft       & Windows 10                                                                                                                                                                                                                                                                                                                                                                                        \\[1.0em]

	BrailleNote Touch+\footnote{For both the BrailleNote Touch+ and BrailleSense 6, there is an emerging issue with outdated operating systems, WiFi connectivity inconsistencies, and incompatibility with Google applications. \begin{itemize}[leftmargin=2em]
			                                                                                                                                                                                                                             \item
			                                                                                                                                                                                                                                   \href{https://perkins.org/braillenote-touch-outdated-os/}{Link to article from Perkins.org regarding the BrailleNote Touch Plus}
			                                                                                                                                                                                                                             \item
			                                                                                                                                                                                                                                   \href{https://endoflife.date/android}{Continually Updated List for End of Life for all flavors of AndroidOS}
		                                                                                                                                                                                                                             \end{itemize}} & \$5,795                                                                                                                   & 12h              & Perkins           & Humanware             & Android 8\footnote{Android 8 `Oreo' Security Support Ended (\emph{i.e.}, End of Life) 2017-12-05}                                                                                                                                                                                                                                                                                                                  \\[1.0em]
	BrailleSense 6                                                                                                                                                                                                                                               & \$5,795                                                                                                                   & 12h              & Perkins           & HIMS                  & Android 10\footnote{Android 10 `Queen Cake' Security Support Ended (\emph{i.e.}, End of Life) 2023-03-06}\fnsep\footnote{The Braillesense has a firmware update v2.0 released 2023-11-28 which updates the operating system to Android 12, but this is currently buggy and causing system overheating. (\emph{note:} Android 12 will only receive updates until \textasciitilde October of 2024)} \\[1.0em]
	InsideONE                                                                                                                                                                                                                                                    & \$5,499                                                                                                                   & 6h               & Perkins           & InsideVision          & Windows 11                                                                                                                                                                                                                                                                                                                                                                                        \\[1.0em]
	Nattiq Note                                                                                                                                                                                                                                                  & \$5,200                                                                                                                   & 12h              & QWERTY            & Nattiq                & Windows 11                                                                                                                                                                                                                                                                                                                                                                                        \\[1.0em]
	Notey the Notetaker                                                                                                                                                                                                                                          & \textasciitilde\$750+\footnote{Self build
	\href{https://notey-project.com/2023/03/07/notey-user-manual-v1-0-2/}{Specs for Notey the NoteTaker}}                                                                                                                                                        &                                                                                                                           & QWERTY Perkins   & Miscs             & Windows 11                                                                                                                                                                                                                                                                                                                                                                                                                \\[1.0em]
	b.note                                                                                                                                                                                                                                                       & \$4,360                                                                                                                   & 15h              & Perkins           & Eurobraille           & Proprietary                                                                                                                                                                                                                                                                                                                                                                                       \\[1.0em]
	Canute Console                                                                                                                                                                                                                                               & \$6,890\footnote{\$3,995 for the Canute Console+\$2,895 for the Canute Display}                                           & 15h              & QWERTY            & Bristol Braille       & Rasperian 12\footnote{Debian 12 Linux for Raspberry Pi}\fnsep\footnote{This is a console/terminal driven operating system that requires knowledge of Linux and using bash commands}                                                                                                                                                                                                               \\[1.0em]\hline
	\caption{ Braille NoteTakers and Laptops }\label{tab:table11}
\end{longtable}
\pagebreak
\hypertarget{refreshable-braille-displays}{}\section{Refreshable Braille
  Displays}\label{refreshable-braille-displays}

Refreshable braille displays are essential tools for students with visual impairments to access digital content. The number of braille cells in a display is an important factor to consider when selecting a device. Displays with 32-40 cells are generally better than those with 14-20 cells for several reasons. Firstly, they provide more space for displaying text, which can help reduce the need for scrolling and improve reading speed. Secondly, they allow for more complex formatting, such as tables and graphs, which can be important for STEM subjects. Thirdly, they provide more context for the user, which can help improve comprehension and reduce errors. Fourthly, they are more versatile and can be used for a wider range of tasks, such as taking notes, writing essays, and browsing the internet. Finally, they are more future-proof, as they are more likely to be compatible with new technologies and software updates. While 14-20 cell displays may be more affordable, investing in a 32-40 cell display can provide significant benefits for students with visual impairments in the long run.

\pagebreak
\hypertarget{few-cell-refreshable-braille-displays}{}\subsection{14-20 cell Refreshable Braille
	Displays}\label{few-cell-refreshable-braille-displays}
There are some situations where 14-20 cell displays may be more appropriate. For example, if the student only needs to read short messages or simple documents, a smaller display may be sufficient. Additionally, smaller displays are more portable and can be easier to carry around. They may also be more affordable, which can be important for students on a tight budget. Finally, smaller displays may be more appropriate for younger students who are just learning braille and may not need as much space for displaying text. While 14-20 cell displays may not be as versatile as larger displays, they can still provide significant benefits for students with visual impairments in certain situations. Table \ref{tab:table12} lists current available display options.
\begin{flushleft} \begin{longtable}[]{@{}
		>{\raggedright\arraybackslash}m{.25\textwidth}
		>{\raggedright\arraybackslash}m{.15\textwidth}
		>{\raggedright\arraybackslash}m{.15\textwidth}
		>{\raggedright\arraybackslash}m{.15\textwidth}
		>{\raggedright\arraybackslash}b{.15\textwidth}@{}
		}
		\toprule

		\textbf{Display}                                                                                             & \textbf{Cost} & \textbf{Battery} & \textbf{Keyboard} & \textbf{Manufacturer} \\
		\midrule
		\endhead \hline                                                                                                                                                                             \\
		\multicolumn{5}{r}{\textbf{Continued on next page}}
		\endfoot	\endlastfoot
		Brailliant BI20x                                                                                             & \$2,199       & 14               & Perkins           & Humanware             \\[1.0em]
		Focus 14 Blue                                                                                                & \$1,545       & 18               & Perkins           & Vispero               \\[1.0em]
		Chameleon 20                                                                                                 & \$1,715       & 14               & Perkins           & APH                   \\[1.0em]
		Orbit Reader 20+                                                                                             & \$799         & 20               & Perkins           & Orbit Research        \\[1.0em]
		Orbit Speak\footnote{This device has no braille output, but uses braille input and returns aauditory output} & TBD           & 20               & Perkins           & Orbit Research        \\[1.0em]
		Actilino                                                                                                     & \$2,795       & 16               & Perkins           & Help Tech             \\[1.0em]
		Basic Braille 20                                                                                             & \$2,295       & 16               & none              & Help Tech             \\[1.0em]
		VarioUltra 20                                                                                                & \$4,340       & 12               & Perkins           & VisioBraille          \\[1.0em]
		Seika Mini Plus                                                                                              & \$2,795       & 20               & none              & Nippon Telesoft       \\[1.0em]
		Seika 24                                                                                                     & \$2,395       & 20               & none              & Nippon Telesoft       \\[1.0em]
		b.note 20                                                                                                    & \$2,695       & 15               & Perkins           & Eurobraille           \\[1.0em] \hline
		\caption[ 14-20 cell Single Line Refreshable Braille Displays]{14-20 cell Single Line Refreshable Braille Displays}\label{tab:table12}
	\end{longtable}  \end{flushleft}

\pagebreak
\hypertarget{cell-refreshable-braille-displays}{}\subsection{32-40 cell Refreshable Braille
	Displays}\label{cell-refreshable-braille-displays}
Refreshable braille displays are essential tools for students with visual impairments to access digital content. The number of braille cells in a display is an important factor to consider when selecting a device. Displays with 32-40 cells are generally better than those with 14-20 cells for several reasons. Firstly, they provide more space for displaying text, which can help reduce the need for scrolling and improve reading speed. Secondly, they allow for more complex formatting, such as tables and graphs, which can be important for STEM subjects. Thirdly, they provide more context for the user, which can help improve comprehension and reduce errors. Fourthly, they are more versatile and can be used for a wider range of tasks, such as taking notes, writing essays, and browsing the internet. Finally, they are more future-proof, as they are more likely to be compatible with new technologies and software updates. While 14-20 cell displays may be more affordable, investing in a 32-40 cell display can provide significant benefits for students with visual impairments in the long run. Table \ref{tab:table13} lists current available display options.
\begin{longtable}[]{@{}
	>{\raggedright\arraybackslash}m{.25\textwidth}
	>{\raggedright\arraybackslash}m{.15\textwidth}
	>{\raggedright\arraybackslash}m{.15\textwidth}
	>{\raggedright\arraybackslash}m{.15\textwidth}
	>{\raggedright\arraybackslash}b{.15\textwidth}@{}
	}
	\toprule

	\textbf{Display}   & \textbf{Cost} & \textbf{Battery} & \textbf{Keyboard} & \textbf{Manufacturer} \\
	\midrule
	\endhead \hline                                                                                   \\
	\multicolumn{5}{r}{\textbf{Continued on Next Page}} \endfoot
	\endlastfoot
	Brailliant BI40x   & \$3,195       & 14               & Perkins           & Humanware             \\[1.0em]
	Focus 40 Blue      & \$3,145       & 18               & Perkins           & Vispero               \\[1.0em]
	Mantis Q40         & \$2,495       & 14               & QWERTY            & APH                   \\[1.0em]
	Orbit Reader 40    & \$1,399       & 20               & Perkins           & Orbit Research        \\[1.0em]
	QBraille XL        & \$3,195       & 16               & Perkins           & HIMS                  \\[1.0em]
	Basic Braille Plus & \$3,295       & 12               & Perkins           & Help Tech             \\[1.0em]
	Active Star        & \$6,795       & 40               & Perkins           & Help Tech             \\[1.0em]
	Active Braille     & \$6,495       & 20               & Perkins           & Help Tech             \\[1.0em]
	Activator          & \$6,495       & 40               & Perkins           & Help Tech             \\[1.0em]
	Vario Ultra 40     & \$7,643       & 12               & Perkins           & VisioBraille          \\[1.0em]
	Vario 340          & \$5,138       & 20               & none              & VisioBraille          \\[1.0em]
	Vario 440          & \$4,550       & 20               & none              & VisioBraille          \\[1.0em]
	Seika V5           & \$2,495       & 20               & none              & Nippon Telesoft       \\[1.0em]
	Alva 640 Comfort   & \$3,046       & 10               & Perkins           & Optelec               \\[1.0em]
	Alva 640 USB       & \$3837        & n/a              & none              & Optelec               \\[1.0em]
	Alva BC 640        & \$2,087       & 10               & none              & Alva                  \\[1.0em]
	b.note             & \$3,565       & 15               & Perkins           & Eurobraille           \\[1.0em] \hline
	\caption{ 32-40 cell Single Line Refreshable Braille Displays }\label{tab:table13}
\end{longtable}

\pagebreak
\hypertarget{multiple-line-refreshable-braille-displaystablets}{}\section{Multiple Line Braille Displays/Tablets}\label{multiple-line-refreshable-braille-displaystablets}
Multiple line braille displays are better than single line refreshable braille displays for students with visual impairments for several reasons. Firstly, they provide more space for displaying text, which can help reduce the need for scrolling and improve reading speed. Secondly, they allow for more complex formatting, such as tables and graphs, which can be important for STEM subjects. Thirdly, they provide more context for the user, which can help improve comprehension and reduce errors. Fourthly, they are more versatile and can be used for a wider range of tasks, such as taking notes, writing essays, and browsing the internet. Finally, they are more future-proof, as they are more likely to be compatible with new technologies and software updates. While single line refreshable braille displays may be more affordable, investing in a multiple line display can provide significant benefits for students with visual impairments in the long run. Table \ref{tab:table14} lists current available display options.


\pagebreak\begin{longtable}[]{@{}
	>{\raggedright\arraybackslash}m{.2\textwidth}
	>{\raggedright\arraybackslash}m{.1\textwidth}
	>{\raggedright\arraybackslash}m{.1\textwidth}
	>{\raggedright\arraybackslash}m{.15\textwidth}
	>{\raggedright\arraybackslash}m{.15\textwidth}
	>{\raggedright\arraybackslash}b{.15\textwidth}@{}
	}
	\toprule

	\textbf{Display} & \textbf{Cost}            & \textbf{Battery} & \textbf{Braille Lines}                 & \textbf{Keyboard} & \textbf{Manufacturer}              \\
	\midrule
	\endhead \hline                                                                                                                                                  \\
	\multicolumn{6}{r}{\textbf{Continued on Next Page}} \endfoot
	\endlastfoot
	Orbit Slate 520  & \$3,495                  & 20-22            & 5 row x 20 cell                        & Perkins           & Orbit Research                     \\[1.0em]
	Orbit Slate 340  & \$3,995                  & 20-22            & 5 row x 20 cell                        & Perkins           & Orbit Research                     \\[1.0em]
	Graphiti         & \textasciitilde\$15,000  & 20-22            & 60 row x 40 cell                       & Perkins           & Orbit Research                     \\[1.0em]
	Graphiti Plus    & \textasciitilde\$15,000  & 20-22            & 60 row x 40 cell + 40 cell line        & Perkins           & Orbit Research                     \\[1.0em]
	TACTIS Table     & \textasciitilde\$15,,000 & Req AC           & 25 row x 40 cell                       & none              & Tactisplay                         \\[1.0em]
	TACTIS Walk      & \textasciitilde\$7,000   & Req AC           & 10 row x 25 cell                       & none              & Tactisplay                         \\[1.0em]
	TACTIS 100       & \textasciitilde\$5,000   & Req AC           & 4 row x 25 cell                        & none              & Tactisplay                         \\[1.0em]
	Canute 360       & \$2,895                  & Req AC           & 9 row x 40 cell                        & none              & Bristol Braille                    \\[1.0em]
	DotPad           & \textasciitilde\$15,000  & 11 hr            & 10 row x 32 cell + 20 cell line        & Touch interface   & Dot Inc.                           \\[1.0em]
	APH Monarch      & \textasciitilde\$15,000  & 11 hr            & 10 row x 32 cell line                  & Perkins           & Dot Inc.\break Humanware\break APH \\[1.0em]
	Blitab           & \$500                    & TBD              & 14 row x 23 cell                       & Touch Interface   & Blitab                             \\[1.0em]
	Tactile Pro      & TBD                      & TBD              & TBD                                    & Perkins           & PCT                                \\[1.0em]
	NewHaptics (?)   & TBD                      & TBD              & TBD                                    & TBD               & New Haptics                        \\[1.0em]
	InsideONE        & \$5,499                  & 6h               & 128GB                                  & Perkins           & InsideVision                       \\[1.0em]
	BraillePad       & \$4,390                  & ?                & 50 row x 40 cells                      & none              & 4Blind                             \\[1.0em]
	Cadence          & TBD                      & ?                & 6 row x 8 cells stack to 24 x 16       & Perkins           & Tactile Engineering                \\[1.0em]
	Tactonom Pro     & \textasciitilde\$15,000  & Requries AC      & 119 col x 89 lines 10,500 touch points & N/A               & Tactonom                           \\[1.0em]\hline
	\caption{ Multiple Line Refreshable Braille Devices }\label{tab:table14}
\end{longtable}
\pagebreak
\hypertarget{learning-tools}{}\section{Braille Education Devices}\label{learning-tools}
In many cases, students do not learn braille as efficiently as their sighted peers learn print. One potential explanation is that there is limited time that a student has access to a teacher trained in braille. One solution is to provide devices that can be used to reinforce or train a student in braille skills without the need for a braille-fluent adult present. This is analogous to the Lexia, Prodigy, or other academic learning systems that allow for self-paced learning.  In the last 5 years, a number of teaching tools have been developed, primarily by groups in India and South Korea to address these needs.

The pioneering companies Thinkerbell Labs in India and OHFA Tech in South Korea, have independently suggested that the low braille literacy rates world wide can be mitigated by providing braille instruction using this type of approach\footnote{\href{https://www.thinkerbelllabs.com/static/decks/tl_deck.pdf}{PDF Slide Deck}}. Both groups have also demonstrated that the use of their products reduces the time to learn braille by upwards of 50\%\footnote{\href{https://www.taptilo.com/}{Taptilo}}. Their reasoning is that the current, more traditional system of braille instruction requires:
\begin{itemize}
	\item Hand-holding and constant supervision of a special educator
	\item Primitive teaching methods resulting in slow learning
	\item Non-interactive content keeps the students uninterested and demotivated
\end{itemize}
Table \ref{tab:table15} lists current available options for braille instructional devices.

\pagebreak\begin{flushleft} \begin{longtable}[]{@{}
		>{\raggedright\arraybackslash}m{.3\textwidth}
		>{\raggedright\arraybackslash}m{.1\textwidth}
		>{\raggedright\arraybackslash}b{.6\textwidth}@{}
		}
		\toprule

		\textbf{Equipment}                                        & \textbf{Cost}                                                         & \textbf{Manufacturer}       \\
		\midrule
		\endhead \hline                                                                                                                                                 \\
		\multicolumn{3}{r}{\textbf{Continued on next page}}
		\endfoot	\endlastfoot

		Polly\footnote{Called ``Annie" outside the Unites States} & \$999                                                                 & APH \break THinkerbell Labs \\[1.0em]
		Taptilo                                                   & \$1,349                                                               & HIMS\break OHFA Tech        \\[1.0em]
		Braille Doodle                                            & \$85                                                                  & Touchpad Pro Fundation      \\[1.0em]
		Braille Teach                                             & \$150                                                                 & Braille Teach               \\[1.0em]
		BrailleCoach                                              & \$1,095                                                               & Logan Tech                  \\[1.0em]
		Read Read                                                 & \$645                                                                 & T-var EdVar Tech            \\[1.0em]
		Feelif Pro                                                & \$3,595                                                               & Feelif Technology           \\[1.0em]
		Feelif Creator                                            & \$2,200                                                               & Feelif Technology           \\[1.0em]
		BrailleBlox                                               & \$85 \footnote{Req LeapFrog Fridge Phonics base, \textasciitilde\$20} & BrailleBot                  \\[1.0em]
		BrailleBuzz                                               & \$99                                                                  & APH                         \\[1.0em]
		Mountbatten Braille Tutor                                 & \$5,495                                                               & Harpo                       \\[1.0em]
		SMART Brailler                                            & \$2,195                                                               & Perkins                     \\[1.0em]\hline
		\caption[Braille Education Device]{Braille Education Device}\label{tab:table15}
	\end{longtable}  \end{flushleft}

\pagebreak
\hypertarget{generation}{}\chapter[\raggedright Empowering Minds: \\The Crucial Role of High-Quality Braille Embossers in Unlocking STEM Literacy for Visually Impaired Students]{Empowering Minds: The Crucial Role of High-Quality Braille Embossers in Unlocking STEM Literacy for Visually Impaired Students}\label{generation}
\minitoc
\extramarks{Vision Department Technology Needs}{Empowering Minds}
In the ever-evolving realms of Science, Technology, Engineering, and Mathematics (STEM), the pursuit of literacy takes on a particularly intricate form. For visually impaired students, the challenges are multifaceted, but with the advent of high-quality braille embossers, a transformative bridge has been constructed. This chapter embarks on a profound exploration of the indispensable role that high-quality braille embossers play in shaping the educational narrative of visually impaired students, especially in the critical domains of Math and STEM. These sophisticated devices, with their ability to translate complex symbols and notations into tangible braille and tactile graphics, stand as pioneers in fostering literacy, comprehension, and success in STEM fields.

The crux of this exploration lies in recognizing the nuanced requirements of visually impaired students pursuing education in Math and STEM disciplines. Traditional print materials, laden with intricate diagrams, mathematical symbols, and graphs, pose formidable challenges for learners with visual impairments. High-quality braille embossers emerge as transformative tools that bridge this gap, converting abstract mathematical concepts and scientific data into tangible formats, empowering students to actively engage with and comprehend the intricacies of STEM subjects.

Embossed tactile graphics break down the barriers to understanding complex mathematical equations, graphical representations, and scientific concepts, ultimately fostering a sense of autonomy and empowerment among visually impaired students. Assessing the landscape of inclusive STEM education, it becomes evident that high-quality braille embossers are not merely tools but enablers of success. By providing access to the visual nuances inherent in STEM fields, these devices pave the way for literacy, comprehension, and active participation, ensuring that visually impaired students can unlock the full spectrum of opportunities in the exciting and dynamic world of Math and STEM disciplines.


\hypertarget{embossers}{}\section{Braille Embossers}\label{embossers}
Having access to a high-quality braille embosser is essential for students with visual impairments to receive a free and appropriate public education. Braille embossers are printers that produce braille text and tactile graphics on paper. They are used to create braille copies of textbooks, worksheets, and other educational materials. High-quality embossers produce sharp, clear braille that is easy to read and tactile graphics that are easy to interpret. This is important because it allows students with visual impairments to access the same educational materials as their sighted peers. Braille embossers also allow students to create their own braille notes and written work, which can help improve their literacy skills and independence. By providing students with visual impairments access to high-quality braille embossers, we can help ensure that they have the tools they need to succeed in their studies and beyond. Table \ref{tab:table16} lists current available embossers.
\begin{longtable}[]{@{}
	>{\raggedright\arraybackslash}m{.2\textwidth}
	>{\raggedright\arraybackslash}m{.1\textwidth}
	>{\raggedright\arraybackslash}m{.4\textwidth}
	>{\raggedright\arraybackslash}b{.15\textwidth}@{}
	}
	\toprule

	\textbf{Machine}                                                                                                                                & \textbf{Cost}                                                                                                                                                    & \textbf{Capability}                  & \textbf{Company}        \\
	\midrule
	\endhead \hline                                                                                                                                                                                                                                                                                                                                                                     \\
	\multicolumn{4}{r}{\textbf{Continued on next page}}
	\endfoot	\endlastfoot
	ViewPlus Rogue\break (used to be Max)\footnote{I am only focusing on 11x11.5'' braille paper size as US Letter size is impractical for braille} & \$5,995                                                                                                                                                          & Complex Graphics, Interpoint Braille & ViewPlus                \\[1.0em]
	ViewPlus Columbia                                                                                                                               & \$3,495                                                                                                                                                          & Complex Graphics, Interpoint Braille & ViewPlus                \\[1.0em]
	APH PageBlaster\break (old Model Index-D V4)                                                                                                    & \$4,295\footnote{Req free \href{https://www.aph.org/app/uploads/2020/07/Firebird_signed_V31.zip}{Firebird Tactile Graphics} Software for graphics functionality} & Simple Graphics, Interpoint Braille  & APH\break Index Braille \\[1.0em]
	Basic-D V5                                                                                                                                      & \$3,695\footnote{Req \href{https://tactileview.com/}{TactileView} license for full graphics functionality (\$\$484 for 1 year license)}                          & Simple Graphics, Interpoint Braille  & Index Braille           \\[1.0em]
	Juliet 120                                                                                                                                      & \$4,195\footnote{Req free \href{https://www.aph.org/app/uploads/2020/07/Firebird_signed_V31.zip}{Firebird Tactile Graphics} Software for graphics functionality} & Simple Graphics, Interpoint Braille  & ETS\break Humanware     \\[1.0em]
	BrailleTrac 120                                                                                                                                 & \$3,595                                                                                                                                                          & Simple Graphics, Interpoint Braille  & Irie-AT                 \\[1.0em]\hline
	\caption[ Braille Embossers focusing on Text]{ Braille Embossers focusing on Text}\label{tab:table16}
\end{longtable}

\pagebreak
\hypertarget{tactile-graphics-high-resolution-complex-graphics}{}\section{High Resolution Tactile Graphics}\label{tactile-graphics-high-resolution-complex-graphics}
There are some historical challenges that have befallen blind students that rely on tactile graphics and braille.
\begin{itemize}[leftmargin=*]
	\item "Historically, by the time students with visual impairments enter school, they have not received enough instruction in the development and use of their tactile skills or had enough opportunities to touch and explore their world."\footnote{\href{https://www.tsbvi.edu/tx-senseabilities/issues/fall-winter-2016/the-development-of-tactile-skills}{Adkins, A., Sewell, D., \& Cleveland, J. (2016). The Development of Tactile Skills. TX \textit{SenseAbilities, Fall/Winter}.}}
	\item Tactile Graphicacy requires the ability to access, comprehend, and produce tactile graphics or raised line drawings. This requires:\begin{itemize}
		      \item Fine motor sensitivity and dexterity
		      \item Efficient use of carefully constructed knowledge
		      \item Variety of tactile-cognitive strategies
	      \end{itemize}
	\item Students have to develop a perception that there are different kinds of symbolic information on a page with different kinds of meaning
	\item Students have to develop an ability to discriminate between different tactile surfaces and to draw meaning from them
	\item These are \textbf{not} inherant or natural for braille readers as they require:
	      \begin{itemize}
		      \item Explicit attention
		      \item Education
		      \item Careful, systematic building of tactile exploratory and interpretive skills
	      \end{itemize}

\end{itemize}

There are a number of benefits to having access to accessible tactile graphics in the classroom. These include:
\begin{itemize}[leftmargin=*]
	\item Provides a focus for attention and perception
	\item Builds pathways to retain and memorize information
	\item Natural destination for conversation and social interaction
	\item Pictures invite and motivate a learner's curiosity and engagement
\end{itemize}
Table \ref{tab:table17} lists current available embossers and other devices for creation of high resolution tactile graphics.
\pagebreak \begin{longtable}[]{@{}
	>{\raggedright\arraybackslash}m{.5\textwidth}
	>{\raggedright\arraybackslash}m{.3\textwidth}
	>{\raggedright\arraybackslash}b{.15\textwidth}@{}
	}
	\toprule

	\textbf{Machine}                                    & \textbf{Cost}                                                                                                                                                                          & \textbf{Company}    \\
	\midrule
	\endhead \hline                                                                                                                                                                                                                                                    \\
	\multicolumn{3}{r}{\textbf{Continued on Next Page}} \endfoot
	\endlastfoot
	PIAF tactile embosser                               & \$1,745                                                                                                                                                                                & Humanware           \\[1.0em]
	Swell Form Machine                                  & \$1,400.00                                                                                                                                                                             & American Thermoform \\[1.0em]
	EZ-Form Brailon Duplicator                          & \$3,899                                                                                                                                                                                & American thermoform \\[1.0em]
	ViewPlus Rogue \break (used to be ViewPlus Max)     & \$5,995\footnote{Req \href{https://viewplus.com/product/tiger-software-suite8/}{Tiger Software Suite} for full functionality (\$195 for 1 year license; \$595 for perpetual license)}  & ViewPlus            \\[1.0em]
	ViewPlus Columbia                                   & \$3,495\footnote{Req \href{https://viewplus.com/product/tiger-software-suite8/}{Tiger Software Suite} for full functionality (\$195 for 1 year license; \$595 for perpetual license)}  & ViewPlus            \\[1.0em]
	APH PixBlaster \break (old Model ViewPlus Columbia) & \$4,295\footnote{Req \href{https://viewplus.com/product/tiger-software-suite8/}{Tiger Software Suite} for full functionality (\$195 for 1 year license; \$595 for perpetual license)}  & APH\break ViewPlus  \\[1.0em]
	ViewPlus Delta                                      & \$4,195\footnote{Req \href{https://viewplus.com/product/tiger-software-suite8/}{Tiger Software Suite} for full functionality (\$195 for 1 year license; \$595 for perpetual license)}  & ViewPlus            \\[1.0em]
	ViewPlus Premier                                    & \$9,995\footnote{Req \href{https://viewplus.com/product/tiger-software-suite8/}{Tiger Software Suite} for full functionality (\$195 for 1 year license; \$595 for perpetual license)}  & ViewPlus            \\[1.0em]
	ViewPlus Elite                                      & \$14,995\footnote{Req \href{https://viewplus.com/product/tiger-software-suite8/}{Tiger Software Suite} for full functionality (\$195 for 1 year license; \$595 for perpetual license)} & ViewPlus            \\[1.0em]
	Basic-D V5                                          & \$3,695\footnote{Req \href{https://tactileview.com/}{TactileView} license for full graphics functionality (\$484 for 1 year license)}                                                  & Index Braille       \\[1.0em]\hline
	\caption{ Equipment for Tactile Graphics Generation}\label{tab:table17}
\end{longtable}

\pagebreak
\hypertarget{tactile-paper}{}\section{Tactile Graphic Supplies}\label{tactile-paper}
Table \ref{tab:table18} lists materials needed to use with the graphics devices shown in Table \ref{tab:table17}.
\begin{longtable}[]{@{}
	>{\raggedright\arraybackslash}m{.5\textwidth}
	>{\raggedright\arraybackslash}m{.3\textwidth}
	>{\raggedright\arraybackslash}b{.15\textwidth}@{}
	}
	\toprule
	\textbf{Paper / Medium}                                                    & \textbf{Cost}                                            & \textbf{Company}    \\
	\midrule
	\endhead \hline                                                                                                                                             \\
	\multicolumn{3}{r}{\textbf{Continued on Next Page}} \endfoot
	\endlastfoot
	Tangible Magic capsule Paper (11x11.5'')\break(for PIAF tactile embosser)  & \$220 for 100 pages                                      & Humanware           \\[1.0em]
	Swell Touch Paper \break (for Swell Form Machine)                          & \$165 for 100 pages                                      & American Thermoform \\[1.0em]
	Tractor-Feed Braille Transcribing Paper: (11x11.5'')\break (for embossers) & \$79.47 for 1000 pages                                   & APH                 \\[1.0em]
	Brailon Thermoform Paper\break (for EZ-Form Duplicator)                    & \$42.99 (100 sheets heavy), \$75.99 (500 sheets regular) & American Thermoform \\[1.0em]\hline
	\caption{ Paper supplies for Tactile Graphics Generation }\label{tab:table18}
\end{longtable}

\pagebreak
\hypertarget{d-printers}{}\chapter[\raggedright Shaping Knowledge: \\The Imperative Role of 3D Printed Materials in Fostering Hands On Literacy for Visually Impaired Students]{Shaping Knowledge: The Imperative Role of 3D Printed Materials in Fostering Hands On Literacy for Visually Impaired Students}\label{d-printers}
\minitoc
\extramarks{Vision Department Technology Needs}{Shaping Knowledge}
In the realm of education, the power of hands-on experience is unparalleled. For visually impaired students, the journey toward literacy and comprehension takes on a unique dimension—one that is enriched and transformed through the tactile exploration of 3D printed materials. This chapter embarks on a captivating exploration of the indispensable role that 3D printed materials play in providing a tangible, tactile bridge to knowledge. These innovative creations not only facilitate hands-on engagement with concepts but stand as catalysts for literacy, fostering success for visually impaired students across a diverse spectrum of subjects.

The need for tangible exploration is paramount, especially when conceptualizing abstract ideas or interacting with physical entities is integral to the learning process. Traditional educational materials often rely on visual cues that pose challenges for students with visual impairments. Enter 3D printed materials—a technological marvel that transcends the limitations of traditional teaching tools. This chapter delves into the transformative impact of these creations, spotlighting their role in enhancing literacy by providing a multisensory gateway to understanding.

From intricate historical artifacts to complex mathematical models, 3D printed materials have the power to transform abstract concepts into tangible, touchable entities. Through this chapter, we will explore how these creations transcend the boundaries of traditional education, offering visually impaired students the opportunity to feel, explore, and internalize knowledge in a manner that aligns with their unique learning styles.

The immersive nature of hands-on learning with 3D printed materials not only fosters comprehension but also instills a sense of empowerment and curiosity. Given the diverse applications of these innovative tools, it is clear that they are not merely educational aids but agents of transformation, democratizing access to knowledge and enhancing the educational journey for visually impaired students. This chapter seeks to underscore the necessity of 3D printed materials in shaping the hands-on literacy experience, ensuring that visually impaired students can grasp the intricacies of the world around them with confidence, curiosity, and a sense of empowerment. Table \ref{tab:table19} lists current available 3D printers.


\hypertarget{d-print-equipment}{}	\section{3D Printers}\label{d-print-equipment}

\begin{longtable}[]{@{}
	>{\raggedright\arraybackslash}m{.2\textwidth}
	>{\raggedright\arraybackslash}m{.08\textwidth}
	>{\raggedright\arraybackslash}m{.2\textwidth}
	>{\raggedright\arraybackslash}m{.2\textwidth}
	>{\raggedright\arraybackslash}b{.2\textwidth}@{}
	}
	\toprule

	\textbf{Model}  & \textbf{Cost} & P\textbf{Print Bed Size} & \textbf{Filament Size} & \textbf{Manufacturer} \\
	\midrule
	\endhead \hline                                                                                             \\
	\multicolumn{5}{r}{\textbf{Continued on Next Page}} \endfoot
	\endlastfoot
	Neptune 4 Pro   & \$330         & 225x225x265mm            & 1.75mm                 & Elegoo                \\[1.0em]
	Neptune 3 Max   & \$470         & 420x420x500mm            & 1.75mm                 & Elegoo                \\[1.0em]
	M5C             & \$399         & 220x220x250mm            & 1.75mm                 & AnkerMake             \\[1.0em]
	Kobra Plus      & \$499         & 300x300x350mm            & 1.75mm                 & Anycubic              \\[1.0em]
	Kobra Max       & \$569         & 450x400x400mm            & 1.75mm                 & Anycubic              \\[1.0em]
	Ender 3 Max Neo & \$359         & 300x300x320mm            & 1.75mm                 & Creality              \\[1.0em]
	Ender 5 Plus    & \$579         & 350x350x400mm            & 1.75mm                 & Creality              \\[1.0em]
	Mini+           & \$459         & 180x180x180mm            & 1.75mm                 & Prusa                 \\[1.0em]
	Sidewinder X2   & \$399         & 300x300x396mm            & 1.75mm                 & Artillery             \\[1.0em]
	P1P 3D Printer  & \$699         & 256×256×256mm            & 1.75mm                 & Bambu                 \\[1.0em]
	P1S 3D Printer  & \$949         & 256×256×256mm            & 1.75mm                 & Bambu                 \\[1.0em]\hline
	\caption{ 3D Printers }\label{tab:table19}
\end{longtable}

\pagebreak
\hypertarget{d-printer-materials}{}\section{3D Printer Materials}\label{d-printer-materials}
Table \ref{tab:table19} lists materials needed in order to use the 3d printers shown in Table \ref{tab:table20}.
\begin{longtable}[]{@{}
	>{\raggedright\arraybackslash}m{.5\textwidth}
	>{\raggedright\arraybackslash}m{.2\textwidth}
	>{\raggedright\arraybackslash}b{.3\textwidth}@{}
	}
	\toprule
	\textbf{Item}                     & \textbf{Cost}             & \textbf{Vendor} \\
	\midrule
	\endhead \hline                                                                 \\
	\multicolumn{3}{r}{\textbf{Continued on Next Page}} \endfoot
	\endlastfoot
	1.75mm filament                   & \textasciitilde\$12.00/kg & Elegoo          \\[1.0em]
	Assorted Sandpaper (48 pcs)       & \$7.00                    & Vicien          \\[1.0em]
	Glue Sticks (30 pack)             & \$10.00                   & Amazon Basics   \\[1.0em]
	Painter's Tape (2" width 12 Pack) & \$43.00                   & Amazon          \\[1.0em]
	3D Print Tool Kit                 & \$58.00                   & HIJIRH          \\[1.0em]\hline
	\caption{ 3D Printer Materials }\label{tab:table20}
\end{longtable}

\pagebreak
\hypertarget{low-vision}{}\chapter[\raggedright Amplifying Vision: \\The Vital Role of Video Magnification Products in Fostering Literacy and Success for Visually Impaired Students]{Amplifying Vision: The Vital Role of Video Magnification Products in Fostering Literacy and Success for Visually Impaired Students}\label{low-vision}
\minitoc
\extramarks{Vision Department Technology Needs}{Amplifying Vision}
In the realm of visual impairment, the quest for literacy and academic success is a journey characterized by innovation and adaptability. For visually impaired students, the challenge of accessing printed materials, charts, and visual content is met with a powerful solution—video magnification products. The indispensable role that video magnification plays in providing enhanced visual access, breaking down barriers to literacy, and empowering students to navigate the educational landscape with confidence cannot be overstated.

The significance of video magnification products lies in their ability to transform the visual experience for students with visual impairments. As we navigate this chapter, we will unravel the sophisticated functionalities of these devices, showcasing how they go beyond traditional magnification methods to provide an immersive and dynamic visual experience. Whether exploring the pages of a textbook, deciphering intricate diagrams, or engaging with digital content, video magnification stands as a technological ally, ensuring that every student can access and interpret visual information with ease.

In the pursuit of literacy, the role of video magnification becomes increasingly pivotal, particularly in subjects where visual content is integral to comprehension. This chapter will delve into how these products facilitate not only enhanced readability but also active participation in classroom discussions, visual learning activities, and the overall educational experience. By providing visually impaired students with a clear and magnified view of the visual world, video magnification products serve as gateways to knowledge, fostering a sense of inclusion and leveling the playing field in academic settings.

It is evident that these tools are not mere aids; they are essential components in the arsenal of resources necessary for the success of visually impaired students. Video magnification imperatively contributes to shaping a learning environment where visual content is accessible to all, ensuring that literacy and success are attainable goals for every student, regardless of their visual abilities.


\hypertarget{video-magnification-devices}{}\section{Video Magnification
  Devices}\label{video-magnification-devices}
Table \ref{tab:table21} lists current available video magnification devices for students with visual impairments.
\begin{longtable}[]{@{}
	>{\raggedright\arraybackslash}m{.25\textwidth}
	>{\raggedright\arraybackslash}m{.15\textwidth}
	>{\raggedright\arraybackslash}m{.25\textwidth}
	>{\raggedright\arraybackslash}b{.2\textwidth}@{}
	}
	\toprule

	\textbf{Model}             & \textbf{Cost}     & \textbf{Deployment}                                             & \textbf{Company}   \\
	\midrule
	\endhead \hline                                                                                                                       \\
	\multicolumn{4}{r}{\textbf{Continued on Next Page}} \endfoot
	\endlastfoot
	Tactonum Pro               & £10,000           & Desktop\break Not Readily Mobile\break Requires mobile App      & Tactonum           \\[1.0em]
	Taxtonum Reader            & £3,795.00         & Desktop\break Not Readily Mobile                                & Tactonum           \\[1.0em]
	MATT Connect v2            & \$3,895           & Desktop\break (Camera + Android Tablet)\break Mobile, but heavy & APH                \\[1.0em]
	Juno                       & \$1,392           & Hand-Held (7'' screen)                                          & APH                \\[1.0em]
	Jupiter Portable Magnifier & \$3,599.00        & Desktop\break Mobile, but heavy                                 & APH                \\[1.0em]
	ONYX Desk set HD           & \$3,330           & Desktop                                                         & Freedom Scientific \\[1.0em]
	ONYX OCR                   & \$4,520           & Desktop                                                         & Freedom Scientific \\[1.0em]
	RUBY                       & \$600.60          & Hand-Held (4.3'' screen)                                        & Freedom Scientific \\[1.0em]
	RUBY 10                    & \$1,640           & Hand-Held (10'' Screen)                                         & Freedom Scientific \\[1.0em]
	RUBY 7 HD                  & \$1,317.75        & Hand-Held (7'' Screen)                                          & Freedom Scientific \\[1.0em]
	RUBY HD                    & \$710.85          & Hand-Held (4.3'' screen)                                        & Freedom Scientific \\[1.0em]
	RUBY XL HD                 & \$987.00          & Hand-Held (5'' screen)                                          & Freedom Scientific \\[1.0em]
	TOPAZ XL HD                & \$4,045.00        & Desktop                                                         & Freedom Scientific \\[1.0em]
	TOPAZ OCR                  & \$4,640.00        & Desktop                                                         & Freedom Scientific \\[1.0em]
	TOPAZ EZ HD                & \$3,081.75        & Desktop                                                         & Freedom Scientific \\[1.0em]
	MAGNA 5                    & \$249             & Hand-Held (5'' screen)                                          & Orbit Research     \\[1.0em]
	MAGNA 4                    & \$199             & Hand-Held (4.3'' screen)                                        & Orbit Research     \\[1.0em]
	MAGNA 3                    & \$149             & Hand-Held (3.5'' screen)                                        & Orbit Research     \\[1.0em]
	I-See 22''                 & \$2,095           & Desktop                                                         & Irie AT            \\[1.0em]
	MagniLink Vision           & \$3,190 - \$4,250 & Desktop                                                         & Irie AT            \\[1.0em]
	MagniLink One              & \$2,395           & Desktop                                                         & Irie AT            \\[1.0em]
	Acuity 22                  & \$2,695           & Desktop                                                         & Irie AT            \\[1.0em]
	Acuity 22 Speech           & \$3,695           & Desktop                                                         & Irie AT            \\[1.0em]
	Snow 12                    & \$1,395           & Desktop\break Mobile                                            & Zoomax             \\[1.0em]
	Luna 6                     & \$795             & Hand-Held (6'' screen)                                          & Zoomax             \\[1.0em]
	Luna S                     & \$385             & Hand-Held (4.3'' screen)                                        & Zoomax             \\[1.0em]
	Luna 8                     & \$895             & Hand-Held (8'' screen)                                          & Zoomax             \\[1.0em]
	Luna HD Pro                & \$2,995           & Desktop                                                         & Zoomax             \\[1.0em]
	Panda HD                   & \$2,098           & Desktop                                                         & Zoomax             \\[1.0em]
	Acesight                   & \$4,295           & E-Glasses                                                       & Zoomax             \\[1.0em]
	Acesight 8                 & \$2,995           & E-Glasses                                                       & Zoomax             \\[1.0em]
	AceSight VR                & \$2,695           & VR Headset                                                      & Zoomax             \\[1.0em]
	Luna Eye                   & TBD               & Hand-Held                                                       & Zoomax             \\[1.0em]
	Snow Pad                   & TBD               & Hand-Held                                                       & Zoomax             \\[1.0em]
	Distance Camera            & TBD               & Hand-Held                                                       & Zoomax             \\[1.0em]
	explore 5                  & \$845             & Hand-Held (5'' screen)                                          & Humanware          \\[1.0em]
	explore 8                  & \$1,275           & Hand-Held (8'' screen)                                          & Humanware          \\[1.0em]
	explore 12                 & \$1,895           & Desktop                                                         & Humanware          \\[1.0em]
	Reveal 16                  & \$3,295           & Desktop                                                         & Humanware          \\[1.0em]
	Reveal 16 (XY table)       & \$3,995           & Desktop                                                         & Humanware          \\[1.0em]
	Reveal 16i                 & \$4,295           & Desktop                                                         & Humanware          \\[1.0em]
	Reveal 16i (XY table)      & \$4,995           & Desktop                                                         & Humanware          \\[1.0em]
	Connect 12 (25x)           & \$2,895           & Desktop\break Mobile                                            & Humanware          \\[1.0em]
	Connect 12 (10x)           & \$2,845           & Desktop\break Mobile                                            & Humanware          \\[1.0em]
	Connect 12                 & \$2,695           & Desktop\break Mobile                                            & Humanware          \\[1.0em]\hline
	\caption{ Video Magnification Devices}\label{tab:table21}
\end{longtable}

\pagebreak
\hypertarget{audio}{}\chapter[\raggedright Beyond Boundaries: \\Audiobook and DAISY Readers as Catalysts for Literacy and Success in Visual Impairment Education]{Beyond Boundaries: Audiobook and DAISY Readers as Catalysts for Literacy and Success in Visual Impairment Education}\label{audio}
\minitoc
\extramarks{Vision Department Technology Needs}{Beyond Boundaries}
In the evolving landscape of education, the pursuit of literacy is a journey marked by innovation and inclusivity. For visually impaired students, the traditional pathways to literacy take on a distinctive form, guided by the transformative power of audiobook and DAISY (Digital Accessible Information System) readers. This chapter embarks on a profound exploration of the indispensable role that these tools play in breaking down barriers to literacy, ensuring access to a rich tapestry of knowledge, and propelling visually impaired students towards academic success.

The quest for literacy is deeply entwined with the ability to access and engage with textual content. Audiobook and DAISY readers emerge as champions in this narrative, providing a dynamic bridge that transcends the limitations posed by traditional print materials. As we delve into this chapter, we will unravel the nuanced functionalities of these tools, showcasing how they offer not just access to literature but a pathway to immersive and engaging learning experiences.

In subjects ranging from literature to science, where the written word is the gateway to understanding, Audiobook and DAISY readers become indispensable companions for visually impaired students. This chapter will explore how these technologies facilitate not only independent reading but also active participation in classroom discussions and assignments. By providing a platform that transforms written content into spoken words or navigable digital formats, these tools empower students to explore the vast realms of knowledge with autonomy and confidence.

The significance of Audiobook and DAISY readers extends beyond mere accessibility; they are key enablers of success. It is evident that these tools are not just aids for the visually impaired; they are essential components of an inclusive education landscape. Audiobook and DAISY readers ensure that every student, regardless of their visual abilities, can embark on a literary journey that is both enriching and empowering.

\hypertarget{text-to-speech-music-podcast}{}\section{DAISY Readers}\label{text-to-speech-music-podcast}
DAISY (Digital Accessible Information System) is a standard format for digital audio books, magazines, and computerized text. DAISY-encoded educational content is an essential tool for students with visual impairments to receive a free and appropriate public education. DAISY books can be read with specialized software that allows the user to navigate through the book using bookmarks, headings, and other navigational aids. This allows students with visual impairments to access the same educational materials as their sighted peers. DAISY books can also be read aloud using text-to-speech software, which can help improve literacy skills and comprehension. Additionally, DAISY books can include tactile graphics, which can help students with visual impairments better understand complex concepts. By providing students with visual impairments access to DAISY-encoded educational content, we can help ensure that they have the tools they need to succeed in their studies and beyond. Table \ref{tab:table22} lists current available DAISY readers.
\begin{longtable}[]{@{}
	>{\raggedright\arraybackslash}m{.25\textwidth}
	>{\raggedright\arraybackslash}m{.15\textwidth}
	>{\raggedright\arraybackslash}m{.25\textwidth}
	>{\raggedright\arraybackslash}b{.2\textwidth}@{}
	}
	\toprule

	\textbf{Model}                  & \textbf{Cost} & \textbf{Function}                                               & \textbf{Company} \\
	\midrule
	\endhead \hline                                                                                                                      \\
	\multicolumn{4}{r}{\textbf{Continued on Next Page}} \endfoot
	\endlastfoot
	Victor Reader Stream            & \$550         & Digital Audio Player                                            & Humanware        \\[1.0em]
	Victor Reader Stratus           & \$495         & DAISY Reader\break Digital Audio Player\break Not very portable & Humanware        \\[1.0em]
	Victor Reader Trek(GPS + media) & \$975         & GPS\break Digital Audio Player                                  & Humanware        \\[1.0em]
	Reizen DAISY Digital Recorder   & \$219         & DAISY Reader\break Digital Audio Player                         & Reizen           \\[1.0em]
	Milestone 212 Ace Book Reader   & \$380         & DAISY Reader\break Digital Audio Player                         & Bones            \\[1.0em]
	PlexTalk Pocket                 & \$275         & DAISY Reader\break Digital Audio Player                         & PlexTalk         \\[1.0em]
	PlexTalk PTN2                   & \$375         & DAISY Reader\break CD Player                                    & PlexTalk         \\[1.0em]\hline
	\caption{ DAISY/Audiobook/Podcast Devices }\label{tab:table22}
\end{longtable}

\hypertarget{text-to-speech}{}\section{Text-to-Speech}\label{text-to-speech}
Text-to-speech software is an essential tool for students with visual impairments to receive a free and appropriate public education. This software allows the computer to “read” digital text to the student in a digitized voice. Some programs will highlight words as they are read, allowing students to follow along. Refreshable braille displays can be connected to the digital text source, providing students with the option to read the text tactually 1. Text-to-speech technology offers an alternative way for students of all ages and learning abilities to access a variety of texts with ease. Using it correctly, the text-to-speech benefits students who struggle with text in written form 2. By providing students with visual impairments access to text-to-speech software, we can help ensure that they have the tools they need to succeed in their studies and beyond. Table \ref{tab:table23} lists current available text-to-speech devices.
\begin{longtable}[]{@{}
	>{\raggedright\arraybackslash}m{.25\textwidth}
	>{\raggedright\arraybackslash}m{.15\textwidth}
	>{\raggedright\arraybackslash}m{.25\textwidth}
	>{\raggedright\arraybackslash}b{.2\textwidth}@{}
	}
	\toprule

	\textbf{Model}                  & \textbf{Cost} & \textbf{Function}                       & \textbf{Company} \\
	\midrule
	\endhead \hline                                                                                              \\
	\multicolumn{4}{r}{\textbf{Continued on Next Page}} \endfoot
	\endlastfoot
	c-Pen2                          & \$399         & Pen Scanner\break Text-to-Speech Reader & c-Pen            \\[1.0em]
	Scanmarker Air Wireless Digital & \$150         & Hand-held Text to Speech                & Scanmarker       \\[1.0em]
	Read 3                          & \$2,490       & Hand-held Text to Speech                & OrCam            \\[1.0em]
	MyEye Pro                       & \$4,250       & Glasses Mounted\break Text to Speech    & OrCam            \\[1.0em]\hline
	\caption{ Text-to-Speech Devices}\label{tab:table23}
\end{longtable}


\pagebreak \hypertarget{accessible-gps-mapping}{}\chapter[\raggedright Navigating Independence: \\The Essential Role of Accessible GPS Equipment in Empowering Visually Impaired Students for Success and Safety]{Navigating Independence: The Essential Role of Accessible GPS Equipment in Empowering Visually Impaired Students for Success and Safety}\label{accessible-gps-mapping}
\minitoc
\extramarks{Vision Department Technology Needs}{Navigating Independence}
In the pursuit of independence and safety, orientation and mobility training holds a pivotal place in the educational journey of visually impaired students. In this dynamic landscape, accessible GPS equipment emerges as a technological beacon, offering a transformative bridge to mobility, autonomy, and enhanced safety. This chapter embarks on a compelling exploration of the indispensable role that accessible GPS tools play in empowering visually impaired students for success, ensuring safe navigation through the world, and fostering a sense of confidence in their daily lives.

The quest for independence is intricately tied to the ability to navigate and explore the surrounding environment. For visually impaired students, this journey is often met with challenges that extend beyond the typical obstacles encountered in education. Accessible GPS equipment becomes a critical ally, providing not only the means to explore the world independently but also enhancing safety through reliable navigational assistance.

As we delve into this chapter, we will unravel the functionalities of accessible GPS devices tailored to the unique needs of visually impaired users. From real-time audible directions to haptic feedback systems, these tools extend beyond standard navigation, creating a multi-sensory experience that empowers students to traverse their surroundings confidently. The importance of this technology is accentuated during orientation and mobility training, where students learn not only to navigate physical spaces but also to develop crucial skills for safety and situational awareness.

Beyond the practicalities of navigation, the impact of accessible GPS equipment on student success cannot be overstated. This chapter will explore how these tools contribute to broader educational goals by fostering a sense of independence, reducing reliance on external assistance, and instilling a foundational skill set for safe and self-assured mobility.

Through this exploration, it becomes clear that accessible GPS equipment is not merely a tool for navigation; it is a catalyst for empowerment and safety. Through orientation and mobility training, we ensure that visually impaired students can embark on their educational journeys with a sense of autonomy, confidence, and, above all, safety.
Table \ref{tab:table24} lists current available accessible GPS hardware devices.

\hypertarget{accessible-gps-mapping-hardware}{}\section{Accessible GPS Hardware}\label{accessible-gps-mapping-hardware}

\begin{longtable}[]{@{}
	>{\raggedright\arraybackslash}m{.25\textwidth}
	>{\raggedright\arraybackslash}m{.15\textwidth}
	>{\raggedright\arraybackslash}m{.25\textwidth}
	>{\raggedright\arraybackslash}b{.2\textwidth}@{}
	}
	\toprule

	\textbf{Model}     & \textbf{Cost} & \textbf{Function}          & \textbf{Company} \\
	\midrule
	\endhead \hline                                                                    \\
	\multicolumn{4}{r}{\textbf{Continued on Next Page}} \endfoot
	\endlastfoot
	Victor Reader Trek & \$975         & GPS + Digital Audio Player & Humanware        \\[1.0em]
	Stellar Trek       & \$1,595       & GPS                        & Humanware        \\[1.0em]
	Wayband            & \$250         & GPS (Haptic Output)        & WearWorks        \\[1.0em]\hline
	\caption{Accessible GPS Mapping / Navigation}\label{tab:table24}
\end{longtable}

\hypertarget{conclusion}{}\chapter{Conclusion}\label{conclusion}
\extramarks{Vision Department Technology Needs}{Conclusion}
In conclusion, the Individuals with Disabilities Education Act (IDEIA) mandates that students with disabilities, including those with visual impairments, must be given access to assistive technology to ensure they can participate fully in the curriculum\footnote{\href{https://sites.ed.gov/idea/statuteregulations/}{20 U.S.C. § 1400, et.}}. Screen magnification is one such assistive technology that can help students with visual impairments access their free public education. The overarching goal of this document is to shed light on the essential role that technology plays in not only accommodating these students but empowering them to thrive in educational environments. By providing students with visual impairments access to the technology they need, we can help ensure that they have the tools they need to succeed in their studies and beyond.

Assistive technology is a critical component of ensuring that students with visual impairments receive a free and appropriate public education. The technology needs of these students must be addressed within the framework of IDEIA, which mandates that students with disabilities must be given access to assistive technology to ensure they can participate fully in the curriculum. Screen magnification is one such assistive technology that can help students with visual impairments access their free public education. By providing students with visual impairments access to the technology they need, we can help ensure that they have the tools they need to succeed in their studies and beyond.

The use of assistive technology is essential for students with visual impairments to access the same educational materials as their sighted peers. Assistive technology can help students with visual impairments access text information across all curricular areas and participate fully in instruction that is often rich with visual content. The use of assistive technology also helps prepare students for independent living, vocational pursuits, or higher education following graduation from high school. By providing students with visual impairments access to the technology they need, we can help ensure that they have the tools they need to succeed in their studies and beyond.

In conclusion, the use of assistive technology is critical for students with visual impairments to receive a free and appropriate public education. The technology needs of these students must be addressed within the framework of IDEIA\footnote{\href{https://sites.ed.gov/idea/statuteregulations/}{ibid}}, which mandates that students with disabilities must be given access to assistive technology to ensure they can participate fully in the curriculum. Screen magnification is one such assistive technology that can help students with visual impairments access their free public education. By providing students with visual impairments access to the technology they need, we can help ensure that they have the tools they need to succeed in their studies and beyond.
\end{document}
